\subsection{for-loop}
繰り返す操作はloopでおこなう.もっとも単純なfor-loop.
\begin{MapleInput}
> for i from 1 to 3 do 
    i; 
  end do;
\end{MapleInput}
\begin{MapleOutputGather}
1 \notag \\
2 \notag \\
3 \notag
\end{MapleOutputGather}
初期値や増減を調整したfor-loop
\begin{MapleInput}
> for i from 10 by -2 to 0 do
    i; 
  end do;
\end{MapleInput}
\begin{MapleOutputGather}
10 \notag \\
8 \notag \\
6 \notag \\
4 \notag \\
2 \notag \\
0 \notag
\end{MapleOutputGather}
loop回数が少ないときは,loopの中身も出力される.これを止めるには,end do;の最後のセミコロンをコロンに変える.

\subsection{二重ループ}
i,jという二つの変数を使って2重化したループ.
\begin{MapleInput}
> for i from 1 to 3 do 
    for j from 1 to 3 do 
      print(i,j); 
    end do; 
  end do;
\end{MapleInput}
\begin{MapleOutputGather}
1, 1 \notag \\
1, 2 \notag \\
1, 3 \notag \\
2, 1 \notag \\
2, 2 \notag \\
2, 3 \notag \\
3, 1 \notag \\
3, 2 \notag \\
3, 3 \notag
\end{MapleOutputGather}
while-loop も同じように使える.
\begin{MapleInput}
> i:=0; 
  while i<5 do 
    i:=i+1;
  end do;
\end{MapleInput}
\begin{MapleOutputGather}
0 \notag \\
1 \notag \\
2 \notag \\
3 \notag \\
4 \notag \\
5 \notag
\end{MapleOutputGather}
