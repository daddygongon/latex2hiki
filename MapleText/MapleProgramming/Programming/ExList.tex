\begin{enumerate}
\item 1から100までの整数のうち5個をランダムに含んだ配列を生成せよ.

1から6までのランダムな数を生成する関数は,
\begin{MapleInput}
> roll:=rand(1..6):
\end{MapleInput}
として作ることができる.実行は次の通り.
\begin{MapleInput}
> seq(roll(),i=1..10);
\end{MapleInput}
\begin{MapleOutput}
5, 2, 5, 6, 2, 3, 4, 4, 6, 5
\end{MapleOutput}
\item さいころを100回振って,出た目1から6が何回出たかを表示せよ.
\item コイン6枚を一度に投げて,表向きの枚数を数えるプログラムを書け.
\item 0から9までの整数5個から5桁の整数を作れ.(1桁目が0になっても気にするな)
\item 小数点以下8桁のそれぞれの桁の数を配列に格納せよ.8桁の少数は以下のようにして作られる.
\item 255以下の10進数をランダムに生成して,8桁の2進数へ変換せよ.

整数の割り算には irem(余り) と iquo(商) がある. 使用法は以下の通り.
\begin{MapleInput}
> irem(7,3); #res: 1
> iquo(7,3); #res: 2
\end{MapleInput}
\end{enumerate}