配列は変数を入れる箱が沢山用意されていると考えればよい.配列を使うときは,箱を指す数(示数,index)をいじっているのか,箱の中身(要素)をいじっているのかを区別すれば,動作を理解しやすい.Mapleにはいくつかの配列構造が用意されている.もっとも,頻繁に使うlistを示す.

\subsection{基本}
リスト構造は,中に入れる要素を[]でくくる.
\begin{MapleInput}
> restart; list1:=[1,3,5,7];
\end{MapleInput}
\begin{MapleOutput}
{\it list1}\, := \,[1,3,5,7]
\end{MapleOutput}

要素にアクセスするには,以下のようにインデックスを指定する.
\begin{MapleInput}
> list1[2]; list1[-1]; list1[2..4];
\end{MapleInput}
\begin{MapleOutputGather}
3 \notag \\
7 \notag \\
[3, 5, 7] \notag
\end{MapleOutputGather}
-1,-2等は後ろから1番目,2番目を指す.C言語と違い0番目はない.

\begin{MapleInput}
> list1[0];
\end{MapleInput}
\begin{MapleError}
Error, invalid subscript selector
\end{MapleError}

ひとつの要素を書き換えるには,以下のようにする.
\begin{MapleInput}
> list1[3]:=x: list1;
\end{MapleInput}
\begin{MapleOutput}
[1, 3, x, 7]
\end{MapleOutput}

要素の数,および要素の中身を取り出すには以下のようにする.
\begin{MapleInput}
> nops(list1); 
> op(list1);
\end{MapleInput}
\begin{MapleOutputGather}
4 \notag \\
1, 3, x, 7 \notag
\end{MapleOutputGather}

\subsection{for-loopの省略形}
for-loopを省略するのによく使う手を二つ.
(\verb|#|より後ろはコメント文です)

\paragraph{配列の生成(seq)} 
\begin{MapleInput}
> aa:=[]; #空で初期化 
  for i from 1 to 3 do 
    aa:=[op(aa),i]; #付け足していく
  end do:
  print(aa);
\end{MapleInput}
\begin{MapleOutputGather}
{\it aa}\, := \,[] \notag \\
[1, 2, 3] \notag
\end{MapleOutputGather}
同じことをseqを使って短く書くことができる.
\begin{MapleInput}
> aa :=[seq(i,i=1..3)];
\end{MapleInput}
\begin{MapleOutput}
{\it aa}\, := \,[1,2,3]
\end{MapleOutput}

\paragraph{配列の和(sum)} 
\begin{MapleInput}
> n:=nops(aa): 
  total:=0: 
  for i from 1 to n do 
    total:=total+aa[i]; 
  end do:
  print(total):
\end{MapleInput}
\begin{MapleOutput}
6
\end{MapleOutput}

同じことをsumを使って短く書くことができる.
\begin{MapleInput}
> sum(aa[i],i=1..nops(aa));
\end{MapleInput}
\begin{MapleError}
Error, invalid subscript selector
\end{MapleError}

sumやseqを使っていると,このようなエラーがよくでる.これは,for-loopをまわすときにiに値が代入されているため引っかかる.変数を換えるか,iを初期化すればよい.
\begin{MapleInput}
> i;
\end{MapleInput}
\begin{MapleOutput}
4
\end{MapleOutput}

                                      
\begin{MapleInput}
> sum(aa[j],j=1..nops(aa));
\end{MapleInput}
\begin{MapleOutput}
6
\end{MapleOutput}

\subsection{リストへの付け足し(append, prepend)}
opを用いると,リストに新たな要素を前後,あるいは途中に付け足すことができる.
\begin{MapleInput}
> list1:=[op(list1),9];
\end{MapleInput}
\begin{MapleOutput}
{\it list1}\, := \,[1,3,x,7,9]
\end{MapleOutput}

\subsection{2つの要素の入れ替え}
要素の3,4番目の入れ替えは以下の通り.
\begin{MapleInput}
> tmp:=list1[3]: 
  list1[3]:=list1[4]: 
  list1[4]:=tmp: 
  list1;
\end{MapleInput}
\begin{MapleOutput}
[1, 3, 7, x, 9]
\end{MapleOutput}

\subsection{2次元配列(listlist)}
[ ] を二重化することで 2 次元の配列を作ることも可能で,リストのリスト (listlist) と呼ばれる.
\begin{MapleInput}
> l2:=[[1,2,3,4],[1,3,5,7]];
\end{MapleInput}
\begin{MapleOutput}
{\it l2}\, := \,[[1,2,3,4],[1,3,5,7]]
\end{MapleOutput}

                         
要素へのアクセスは以下の通り.
\begin{MapleInput}
> l2[2]; l2[2,3]; l2[2][3];
\end{MapleInput}
\begin{MapleOutputGather}
[1, 3, 5, 7] \notag \\
5 \notag \\
5 \notag
\end{MapleOutputGather}

\subsection{listの表示(listplot)}
listに入っている数値を視覚化するのにはlistplotが便利.
\begin{MapleInput}
> la:=[1,2,3,4,3,2,1]; 
  with(plots): 
  listplot(la);
\end{MapleInput}
\begin{MapleOutput}
[1, 2, 3, 4, 3, 2, 1]
\end{MapleOutput}
\MaplePlot{50mm}{./figures/MapleProgrammingplot2d1.eps}


