\subsection{基本}
複雑な手続きや,何度も繰り返すルーチンはprocで作る.
procは以下のようにして作る.
\begin{verbatim}
ユーザ関数名:=proc(仮引数)
  動作
end proc;
\end{verbatim}
\begin{MapleInput}
> test1:=proc(a) 
    print(a); 
  end proc:
\end{MapleInput}
procの呼び出しは,以下のようになる.
\begin{MapleInput}
> test1(13);
\end{MapleInput}
\begin{MapleOutput}
13
\end{MapleOutput}
仮引数としてはどんな型(変数や配列)でもよい.複数指定するときにはコンマで区切る.仮引数をprocの中で変更することはできない.下で示すglobalで取り込むか,local変数にコピーして使う.

\subsection{戻り値}
procの戻り値はreturnで指定される.return文がないときは,最後の動作結果が戻り値となる.
\begin{MapleInput}
> test2:=proc(a) 
    return a+1; 
  end proc:
\end{MapleInput}
\begin{MapleInput}
> test2(13);
\end{MapleInput}
\begin{MapleOutput}
14
\end{MapleOutput}

\subsection{グローバル(大域),ローカル(局所)変数}
procの内部だけで使われるのがlocal,外部を参照するのがglobal.global,localを省略してもMapleが適当に判断してくれるが,あまり信用せず,明示的に宣言した方が良い.宣言の仕方は以下の通り.
\begin{verbatim}
変数名:=proc(引数)
  local 変数,変数...;
  global 変数,変数...;
  動作の記述
end proc;
\end{verbatim}