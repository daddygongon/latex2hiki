\documentclass[10pt,a4j]{jbook}
\usepackage[dvips]{graphicx,color}
\usepackage{verbatim}
\usepackage{amsmath,amsthm,amssymb}
\topmargin -15mm\oddsidemargin -4mm\evensidemargin\oddsidemargin
\textwidth 170mm\textheight 257mm\columnsep 7mm
\setlength{\fboxrule}{0.2ex}
\setlength{\fboxsep}{0.6ex}

\pagestyle{empty}

\newcommand{\MaplePlot}[2]{{\begin{center}
    \includegraphics[width=#1,clip]{#2}
                     \end{center}
%
} }

\newenvironment{MapleInput}{%
    \color{red}\verbatim
}{%
    \endverbatim
}

\newenvironment{MapleError}{%
    \color{blue}\verbatim
}{%
    \endverbatim
}

\newenvironment{MapleOutput}{%
    \color{blue}\begin{equation*}
}{%
    \end{equation*}
}

\newenvironment{MapleOutputGather}{%
    \color{blue}\gather
}{%
    \endgather
}

\newcommand{\ChartElement}[1]{{
	\color{magenta}\begin{flushleft}$\left[\left[\left[\textbf{\large #1}\right]\right]\right]$
	\end{flushleft}\vspace{-10mm}
} }

\newcommand{\ChartElementTwo}[1]{{
	\color{magenta}\begin{flushleft}$\left[\left[\left[\textbf{\large #1}\right]\right]\right]$
	\end{flushleft}
} }

\newif\ifHIKI
\HIKItrue % TRUEの設定                                                                                                                             
%\HIKIfalse % FALSEの設定                                                                                                                            
\begin{document}
\chapter{Programming}
\section{代入と出力(Variables, printf)}
\subsection{解説}
\subsection{値の変数への代入(:=)}

Mapleは変数の初期設定で型宣言をする必要がない.数式処理の章で示したとおり,変数への代入は:=を使う.変数a,bにそれぞれ10,3を代入し,a+bの結果をc
に代入するというプログラムは以下の通り.
\begin{MapleInput}
> a:=10: b:=3: c:=a+b;
\end{MapleInput}
\begin{MapleOutput}
c\, := \,13
\end{MapleOutput}

\subsection{整数と浮動小数点数}
浮動小数点数から整数に直すにはいくつかの関数がある.
\begin{itemize}
\item trunc:数値から数直線で 0 に向って最も近い整数
\item round:数値の四捨五入
\item floor:数値より小さな最も大きな整数
\item ceil:数値より大きな最も小さな整数
\end{itemize}
負の値の時に floor と trunc は違った値を返す.
\MaplePlot{80mm}{./figures/MapleProgrammingimage0.eps}

小数点以下を取りだすにはfrac が用意されている.
\begin{MapleInput}
> frac(1.7);
\end{MapleInput}
\begin{MapleOutput}
0.7
\end{MapleOutput}

整数の割り算はirem(余り)とiquo(商).
\begin{MapleInput}
> irem(7,3); #res: 1
> iquo(7,3); #res: 2
\end{MapleInput}

\subsection{出力(print, printf)}
Mapleではデフォルトで結果が出力される.これを抑えるには行末の”;”を”:”に変える必要がある.出力を明示的におこなうにはprintを使う.デバッグの時に便利.
\begin{MapleInput}
> x:=1: print(x); #res: 1
\end{MapleInput}
さらに,出力を整えるのに便利なprintf関数がある.これはC言語と同じ構文で,
\begin{MapleInput}
> printf("Hello world!!\n");
\end{MapleInput}
\begin{MapleError}
Hello world!!
\end{MapleError}
と打ち込んでenterを押せば,出力が即座に表示される.値を表示するときには,
\begin{MapleInput}
> i:=3: printf("%3d\n",i);
\end{MapleInput}
\begin{MapleOutput}
3
\end{MapleOutput}
となる.これは

「変数iに入っている値を,3桁の整数形式で打ち出した後,改行せよ」

と言う意味.\%3dが出力の形式,\verb|\n|が改行を意味する.OSによっては,\verb|\|は¥と画面あるいはキーボードで表示されているかもしれない.実数の出力指定は\%10.5fで,全部で10桁,小数点以下5桁で浮動小数点数を表示.複数の変数の出力は
\begin{MapleInput}
> printf("%3d : %10.5f \n",i,a);
\end{MapleInput}
などとなる.

\begin{table}[htbp]
\caption{printfの書式指定}
\begin{center}
\begin{tabular}{|l|l|}
\hline
\%指定 & 意味\\ \hline
\%o & 整数を8進数で表示.\\
\%d & 整数を10進数で表示.\\
\%x,%X & 整数を16進数で表示.xは小文字,Xは大文字を使用.\\
\%f & 浮動小数点数として表示.\\
\%e,%E & 指数形式で表示.eは小文字,Eは大文字を使用.\\
\%s & 文字列を出力.\\ 
\hline
\end{tabular}
\end{center}
\label{default}
\end{table}%

 

\pagebreak
\section{ループ(Loop)}

\subsection{解説}
\subsection{for-loop}
繰り返す操作はloopでおこなう.もっとも単純なfor-loop.
\begin{MapleInput}
> for i from 1 to 3 do 
    i; 
  end do;
\end{MapleInput}
\begin{MapleOutputGather}
1 \notag \\
2 \notag \\
3 \notag
\end{MapleOutputGather}
初期値や増減を調整したfor-loop
\begin{MapleInput}
> for i from 10 by -2 to 0 do
    i; 
  end do;
\end{MapleInput}
\begin{MapleOutputGather}
10 \notag \\
8 \notag \\
6 \notag \\
4 \notag \\
2 \notag \\
0 \notag
\end{MapleOutputGather}
loop回数が少ないときは,loopの中身も出力される.これを止めるには,end do;の最後のセミコロンをコロンに変える.

\subsection{二重ループ}
i,jという二つの変数を使って2重化したループ.
\begin{MapleInput}
> for i from 1 to 3 do 
    for j from 1 to 3 do 
      print(i,j); 
    end do; 
  end do;
\end{MapleInput}
\begin{MapleOutputGather}
1, 1 \notag \\
1, 2 \notag \\
1, 3 \notag \\
2, 1 \notag \\
2, 2 \notag \\
2, 3 \notag \\
3, 1 \notag \\
3, 2 \notag \\
3, 3 \notag
\end{MapleOutputGather}
while-loop も同じように使える.
\begin{MapleInput}
> i:=0; 
  while i<5 do 
    i:=i+1;
  end do;
\end{MapleInput}
\begin{MapleOutputGather}
0 \notag \\
1 \notag \\
2 \notag \\
3 \notag \\
4 \notag \\
5 \notag
\end{MapleOutputGather}
 
\subsection{課題}
\begin{enumerate}
\item printfを使って次のように表示せよ.

i) Hello world. ii) 1+1=2
\item 次の数を順に表示せよ.

i) 1から5までの整数.ii) 5から1までの整数 .iii) 1から10にある偶数.

\item 9x9表を作れ.
\item 1 から 5 までの和を求めよ.
\item nを5にして,$n!=n \times (n-1) \times (n-1) \cdots 3 \times 2 \times 1$を求めよ.
\end{enumerate}
 
\subsection{解答例}
\begin{enumerate}
\item

\begin{MapleInput}
> printf("Hello world!!\n");
\end{MapleInput}
\begin{MapleError}
Hello world!!
\end{MapleError}
\begin{MapleInput}
> i:=1; 
> printf("%d+%d=%d\n",i,i,i+i);
\end{MapleInput}
\begin{MapleOutputGather}
 1 \notag \\
1+1=2 \notag
\end{MapleOutputGather}

\item
i)
\begin{MapleInput}
> for i from 1 to 5 do
    i;
  end do;
\end{MapleInput}
\begin{MapleOutputGather}
 1 \notag \\
 2 \notag \\
 3 \notag \\
 4 \notag \\
 5 \notag
\end{MapleOutputGather}
ii)
\begin{MapleInput}
> for i from 5 to 1 by -1 do
    i; 
  end do;
\end{MapleInput}
\begin{MapleOutputGather}
 5 \notag \\
 4 \notag \\
 3 \notag \\
 2 \notag \\
 1 \notag
\end{MapleOutputGather}
iii)
\begin{MapleInput}
> for i from 2 to 10 by 2 do
    i; 
  end do;
\end{MapleInput}
\begin{MapleOutputGather}
 2 \notag \\
 4 \notag \\
 6 \notag \\
 8 \notag \\
 10 \notag
\end{MapleOutputGather}

\item
\begin{MapleInput}
> for i from 1 to 9 do 
    for j from 1 to 9 do
      printf("%4d",i*j); 
    end do;
    printf("\n"); 
  end do;
\end{MapleInput}
\begin{MapleError}
   1   2   3   4   5   6   7   8   9
   2   4   6   8  10  12  14  16  18
   3   6   9  12  15  18  21  24  27
   4   8  12  16  20  24  28  32  36
   5  10  15  20  25  30  35  40  45
   6  12  18  24  30  36  42  48  54
   7  14  21  28  35  42  49  56  63
   8  16  24  32  40  48  56  64  72
   9  18  27  36  45  54  63  72  81
\end{MapleError}

\begin{MapleInput}
> sum1:=0; for i from 1 to 5 do 
    sum1:=sum1+i; 
  end do;
\end{MapleInput}
\begin{MapleOutputGather}
 0 \notag \\
 1 \notag \\
 3 \notag \\
 6 \notag \\
 10 \notag \\
 15 \notag
\end{MapleOutputGather}

\item
\begin{MapleInput}
> n:=5: 
  total1:=1:
  for i from 1 to n do 
    total1:=total1*i; 
  end do;
\end{MapleInput}
\begin{MapleOutputGather}
 1 \notag \\
 2 \notag \\
 6 \notag \\
 24 \notag \\
120 \notag
\end{MapleOutputGather}
\end{enumerate} 


\pagebreak
\section{配列(List)}
\subsection{解説}
配列は変数を入れる箱が沢山用意されていると考えればよい.配列を使うときは,箱を指す数(示数,index)をいじっているのか,箱の中身(要素)をいじっているのかを区別すれば,動作を理解しやすい.Mapleにはいくつかの配列構造が用意されている.もっとも,頻繁に使うlistを示す.

\subsection{基本}
リスト構造は,中に入れる要素を[]でくくる.
\begin{MapleInput}
> restart; list1:=[1,3,5,7];
\end{MapleInput}
\begin{MapleOutput}
{\it list1}\, := \,[1,3,5,7]
\end{MapleOutput}

要素にアクセスするには,以下のようにインデックスを指定する.
\begin{MapleInput}
> list1[2]; list1[-1]; list1[2..4];
\end{MapleInput}
\begin{MapleOutputGather}
3 \notag \\
7 \notag \\
[3, 5, 7] \notag
\end{MapleOutputGather}
-1,-2等は後ろから1番目,2番目を指す.C言語と違い0番目はない.

\begin{MapleInput}
> list1[0];
\end{MapleInput}
\begin{MapleError}
Error, invalid subscript selector
\end{MapleError}

ひとつの要素を書き換えるには,以下のようにする.
\begin{MapleInput}
> list1[3]:=x: list1;
\end{MapleInput}
\begin{MapleOutput}
[1, 3, x, 7]
\end{MapleOutput}

要素の数,および要素の中身を取り出すには以下のようにする.
\begin{MapleInput}
> nops(list1); 
> op(list1);
\end{MapleInput}
\begin{MapleOutputGather}
4 \notag \\
1, 3, x, 7 \notag
\end{MapleOutputGather}

\subsection{for-loopの省略形}
for-loopを省略するのによく使う手を二つ.
(\verb|#|より後ろはコメント文です)

\paragraph{配列の生成(seq)} 
\begin{MapleInput}
> aa:=[]; #空で初期化 
  for i from 1 to 3 do 
    aa:=[op(aa),i]; #付け足していく
  end do:
  print(aa);
\end{MapleInput}
\begin{MapleOutputGather}
{\it aa}\, := \,[] \notag \\
[1, 2, 3] \notag
\end{MapleOutputGather}
同じことをseqを使って短く書くことができる.
\begin{MapleInput}
> aa :=[seq(i,i=1..3)];
\end{MapleInput}
\begin{MapleOutput}
{\it aa}\, := \,[1,2,3]
\end{MapleOutput}

\paragraph{配列の和(sum)} 
\begin{MapleInput}
> n:=nops(aa): 
  total:=0: 
  for i from 1 to n do 
    total:=total+aa[i]; 
  end do:
  print(total):
\end{MapleInput}
\begin{MapleOutput}
6
\end{MapleOutput}

同じことをsumを使って短く書くことができる.
\begin{MapleInput}
> sum(aa[i],i=1..nops(aa));
\end{MapleInput}
\begin{MapleError}
Error, invalid subscript selector
\end{MapleError}

sumやseqを使っていると,このようなエラーがよくでる.これは,for-loopをまわすときにiに値が代入されているため引っかかる.変数を換えるか,iを初期化すればよい.
\begin{MapleInput}
> i;
\end{MapleInput}
\begin{MapleOutput}
4
\end{MapleOutput}

                                      
\begin{MapleInput}
> sum(aa[j],j=1..nops(aa));
\end{MapleInput}
\begin{MapleOutput}
6
\end{MapleOutput}

\subsection{リストへの付け足し(append, prepend)}
opを用いると,リストに新たな要素を前後,あるいは途中に付け足すことができる.
\begin{MapleInput}
> list1:=[op(list1),9];
\end{MapleInput}
\begin{MapleOutput}
{\it list1}\, := \,[1,3,x,7,9]
\end{MapleOutput}

\subsection{2つの要素の入れ替え}
要素の3,4番目の入れ替えは以下の通り.
\begin{MapleInput}
> tmp:=list1[3]: 
  list1[3]:=list1[4]: 
  list1[4]:=tmp: 
  list1;
\end{MapleInput}
\begin{MapleOutput}
[1, 3, 7, x, 9]
\end{MapleOutput}

\subsection{2次元配列(listlist)}
[ ] を二重化することで 2 次元の配列を作ることも可能で,リストのリスト (listlist) と呼ばれる.
\begin{MapleInput}
> l2:=[[1,2,3,4],[1,3,5,7]];
\end{MapleInput}
\begin{MapleOutput}
{\it l2}\, := \,[[1,2,3,4],[1,3,5,7]]
\end{MapleOutput}

                         
要素へのアクセスは以下の通り.
\begin{MapleInput}
> l2[2]; l2[2,3]; l2[2][3];
\end{MapleInput}
\begin{MapleOutputGather}
[1, 3, 5, 7] \notag \\
5 \notag \\
5 \notag
\end{MapleOutputGather}

\subsection{listの表示(listplot)}
listに入っている数値を視覚化するのにはlistplotが便利.
\begin{MapleInput}
> la:=[1,2,3,4,3,2,1]; 
  with(plots): 
  listplot(la);
\end{MapleInput}
\begin{MapleOutput}
[1, 2, 3, 4, 3, 2, 1]
\end{MapleOutput}
\MaplePlot{50mm}{./figures/MapleProgrammingplot2d1.eps}


 
\subsection{課題}
\input{ExList.tex} 
\subsection{解答例}
\begin{enumerate}
\item
\begin{MapleInput}
> roll:=rand(1..100): 
  [seq(roll(),i=1..5)];
\end{MapleInput}
\begin{MapleOutput}
[27, 96, 17, 90, 34]
\end{MapleOutput}

\item
\begin{MapleInput}
> roll:=rand(1..6):
> A:=[seq(0,i=1..6)]; \end{MapleInput}
\begin{MapleOutput}
[0, 0, 0, 0, 0, 0]
\end{MapleOutput}
\begin{MapleInput}
> for i from 1 to 100 do 
    i1:=roll(); 
    A[i1]:=A[i1]+1; 
  end do:
  A;
\end{MapleInput}
\begin{MapleOutput}
[16, 18, 21, 18, 18, 9]
\end{MapleOutput}

\item
\begin{MapleInput}
> toss:=rand(0..1): 
  n:=6: 
  up:=0: 
  for i from 1 to n do 
    up:=up+toss(); 
  end do:
  up;
\end{MapleInput}
\begin{MapleOutput}
3
\end{MapleOutput}

\item
\begin{MapleInput}
> roll:=rand(0..9): 
  n:=5: 
  A:=[seq(roll(),i=1..n)];
\end{MapleInput}
\begin{MapleOutput}
[5, 7, 3, 7, 6]
\end{MapleOutput}

\begin{MapleInput}
> sum1:=0:
  for i from 1 to n do 
    sum1:=sum1*10+A[i]; 
  end do: 
  sum1;
\end{MapleInput}
\begin{MapleOutput}
57376
\end{MapleOutput}

\item
\begin{MapleInput}
> restart; 
  n:=8:
  roll:=rand(10^(n-1)..10^n): 
  B:=evalf(roll()/10^n,8); 
  A:=[]:
\end{MapleInput}
\begin{MapleOutput}
0.19550684
\end{MapleOutput}

\begin{MapleInput}
> B:=10*B; 
  for i from 1 to n do 
    A:=[op(A),floor(B)]; 
    B:=(B-A[i])*10; 
  end do: 
  A;
\end{MapleInput}
\begin{MapleOutputGather}
1.95506840 \notag \\
[1, 9, 5, 5, 0, 6, 8, 4] \notag
\end{MapleOutputGather}
                          
\item
\begin{MapleInput}
> n:=8: 
  roll:=rand(0..2^n-1): 
  B:=roll();
\end{MapleInput}
\begin{MapleOutput}
246
\end{MapleOutput}

\begin{MapleInput}
> A:=[seq(0,j=1..n)]: 
  for i from 1 to n do 
    A[n-i+1]:=irem(B,2); 
    B:=iquo(B,2);
  end do: 
  A;
\end{MapleInput}
\begin{MapleOutput}
[1, 1, 1, 1, 0, 1, 1, 0]
\end{MapleOutput}

\end{enumerate} 

\pagebreak
\section{交通整理(If)}
\subsection{解説}
\subsection{if}
もっとも簡単なif文の例.
\begin{MapleInput}
> x:=-4:
  if (x<0) then
   y:=-x; 
  end if;
\end{MapleInput}
\begin{MapleOutput}
 4
\end{MapleOutput}

例外付き.
\begin{MapleInput}
> x:=3:
  if (x<0) then 
    y:=-x; 
  else 
    y:=x; 
  end if;
\end{MapleInput}
\begin{MapleOutput}
3
\end{MapleOutput}
2個の条件がある例
\begin{MapleInput}
> x:=3:
  if (x<0) then 
    y:=-x; 
  elif (x>5) then 
    y:=x; 
  else
    y:=2*x; 
  end if;
\end{MapleInput}
\begin{MapleOutput}
6
\end{MapleOutput}

\paragraph{条件文に使える式と意味}
関係演算子は\verb|<, <=, >, >=, =, <>|で表記される.論理演算子にはand, or, xor, notがある.その他にもブール値を返す関数としてimplies, evalb, type などいくつかあり,条件分岐に使える.

\begin{table}[htbp]
\caption{条件分岐のいくつかの例}
\begin{center}
\begin{tabular}{|l|l|}
\hline
xとyの値が一致 & (x=y) \\
xとyの値が一致しない & (x\verb|<>|y)\\ \hline
条件文を複数つなぐ & ((x\verb|>|0) and (x\verb|<|4)) \\
&  ((x\verb|<|0) or (x\verb|>|4)) \\
&  not (x=0) \\ \hline
\end{tabular}
\end{center}
\label{default}
\end{table}%

\subsection{nextとbreak}
do-loopの途中で流れを変更するための命令.nextはdo-loop を一回スキップ.breakはそこで do-loop
を一つ抜ける.以下のコードの出力結果を参照.
\begin{MapleInput}
> for i from 1 to 5 do 
    if (i=3) then 
	  next; 
    end if; 
    print(i); 
  end do:
\end{MapleInput}
\begin{MapleError}
#res: 1 2 4 5
\end{MapleError}

\begin{MapleInput}
> for i from 1 to 5 do
    if (i=3) then
      break; 
    end if; 
    print(i); 
  end do:
\end{MapleInput}
\begin{MapleError}
#res: 1 2
\end{MapleError}

 
\subsection{課題}
\input{ExIf.tex} 
\subsection{解答例}
\begin{enumerate}
\item
\begin{MapleInput}
> year:=1890; 
  if year<1868 then printf("明治より前です.\n"); 
  elif year<1912 then printf("明治%d年です.\n",year-1868+1); 
  elif year<1926 then printf("大正%d年です.\n",year-1912+1); 
  elif year<1989 then printf("昭和%d年です.\n",year-1926+1); 
  elif year<2011 then printf("平成%d年です.\n",year-1989+1); 
  else printf("今年より後です.\n"); 
  end;
\end{MapleInput}
\begin{MapleError}
明治23年です.
\end{MapleError}

\item
\begin{MapleInput}
> n:=10:
  for i from 1 to n do 
    if (isprime(i)) then 
      print(i); 
    end if; 
  end do;
\end{MapleInput}
\begin{MapleError}
#res: 2 3 5 7
\end{MapleError}

\item
\begin{MapleInput}
> for i from 10 to 100-2 do 
    if (isprime(i) and isprime(i+2)) then 
      print(i,i+2);
    end if; 
  end do;
\end{MapleInput}
\begin{MapleError}
                                    11, 13
                                    17, 19
                                    29, 31
                                    41, 43
                                    59, 61
                                    71, 73
\end{MapleError}

\item
\begin{MapleInput}
> n:=12: 
  banpei:=0:
  for i from 2 to n-1 do 
    residue:=irem(n,i); 
    # print(n,residue): 
    if residue=0 then 
      banpei:=1; 
      break; 
    end if; 
  end do: 
  if banpei=1 then 
    printf("%d is not prime number.\n",n); 
  else 
    printf("%d is prime number.\n",n); 
  end if;
\end{MapleInput}
\begin{MapleError}
12 is not prime number.
\end{MapleError}

\item
\begin{MapleInput}
> year:=[2010,1984,2004,1800,1900,1600,2000]: 
  for i from 1 to nops(year) do 
    if (irem(year[i],400)=0) then 
      printf("%d is a leap year.\n",year[i]); 
    elif (irem(year[i],4)=0) and (irem(year[i],100)<>0) then
      printf("%d is a leap year.\n",year[i]); 
    else printf("%d is not a leap year.\n",year[i]); 
    end if;
  end do;
\end{MapleInput}
\begin{MapleError}
2010 is not a leap year. 
1984 is a leap year. 
2004 is a leap year. 
1800 is not a leap year. 
1900 is not a leap year. 
1600 is a leap year. 
2000 is a leap year.
\end{MapleError}

別解
\begin{MapleInput}
> for i from 1 to nops(year) do 
    if (irem(year[i],4)=0) and ((irem(year[i],100)<>0) or (irem(year[i],400)=0)) then 
      printf("%d is a leap year.\n",year[i]); 
    else 
      printf("%d is not a leap year.\n",year[i]); 
    end if;
  end do;
\end{MapleInput}
\begin{MapleError}
略
\end{MapleError}

\item
\begin{MapleInput}
> prime1:=[]; 
  for i from 1 to 100 do 
    if isprime(i) then 
      prime1:=[op(prime1),i];
    end if; 
  end do; 
  prime1;
\end{MapleInput}
\begin{MapleError}
[2, 3, 5, 7, 11, 13, 17, 19, 23, 29, 31, 37, 41, 43, 47, 53, 59, 61, 67, 71, 
73, 79, 83, 89, 97]
\end{MapleError}
\begin{MapleInput}
> nops(prime1);
\end{MapleInput}
\begin{MapleOutput}
25
\end{MapleOutput}

\begin{MapleInput}
> for i from 6 to 100 by 2 do 
    for j1 from 1 to nops(prime1) do 
      for j2 from 1 to nops(prime1) do 
        if i=(prime1[j1]+prime1[j2]) then
          print(i,prime1[j1],prime1[j2]); 
          break;
        end if 
      end do; 
      if j2<=nops(prime1) then
        break; 
      end if; 
    end do; 
  end do;
\end{MapleInput}
\begin{MapleError}
6, 3, 3
8, 3, 5
10, 3, 7
中略
98, 19, 79
100, 3, 97
\end{MapleError}
                                  
\end{enumerate} 

\pagebreak
\section{手続き関数(Procedure)}
\subsection{解説}
\subsection{基本}
複雑な手続きや,何度も繰り返すルーチンはprocで作る.
procは以下のようにして作る.
\begin{verbatim}
ユーザ関数名:=proc(仮引数)
  動作
end proc;
\end{verbatim}
\begin{MapleInput}
> test1:=proc(a) 
    print(a); 
  end proc:
\end{MapleInput}
procの呼び出しは,以下のようになる.
\begin{MapleInput}
> test1(13);
\end{MapleInput}
\begin{MapleOutput}
13
\end{MapleOutput}
仮引数としてはどんな型(変数や配列)でもよい.複数指定するときにはコンマで区切る.仮引数をprocの中で変更することはできない.下で示すglobalで取り込むか,local変数にコピーして使う.

\subsection{戻り値}
procの戻り値はreturnで指定される.return文がないときは,最後の動作結果が戻り値となる.
\begin{MapleInput}
> test2:=proc(a) 
    return a+1; 
  end proc:
\end{MapleInput}
\begin{MapleInput}
> test2(13);
\end{MapleInput}
\begin{MapleOutput}
14
\end{MapleOutput}

\subsection{グローバル(大域),ローカル(局所)変数}
procの内部だけで使われるのがlocal,外部を参照するのがglobal.global,localを省略してもMapleが適当に判断してくれるが,あまり信用せず,明示的に宣言した方が良い.宣言の仕方は以下の通り.
\begin{verbatim}
変数名:=proc(引数)
  local 変数,変数...;
  global 変数,変数...;
  動作の記述
end proc;
\end{verbatim} 
\subsection{課題}
\input{ExProcedure.tex} 
\subsection{解答例}
\begin{enumerate}
\item
\begin{MapleInput}
> area:=proc(base,height) 
    base*height/2; 
  end proc:
\end{MapleInput}
\begin{MapleInput}
> area(3,4); #res: 6
\end{MapleInput}

\item
\begin{MapleInput}
> restart; 
  n:=19: 
  banpei:=0; 
  for i from 2 to n-1 do 
    amari:=irem(n,i);
    print(amari): 
    if amari=0 then 
      banpei:=1; 
      break;
    end if; 
  end do: 
  if banpei=1 then 
    print(n," is not prime number."); 
  else 
    print(n," is prime number."); 
  end if;
\end{MapleInput}
\begin{MapleOutputGather}
0 \notag \\
1 \notag \\
1 \notag \\
1 \notag \\
19, "\, is\, prime\, number." \notag
\end{MapleOutputGather}

\begin{MapleInput}
> MyIsprime:=proc(n) 
    local i,amari; 
    for i from 2 to evalf(sqrt(n)) do
      amari:=irem(n,i); 
      if amari=0 then 
        return false; 
      end if; 
    end do:
    return true;
  end proc:
\end{MapleInput}

\begin{MapleInput}
> MyIsprime(104729);
\end{MapleInput}
\begin{MapleOutput}
true
\end{MapleOutput}

\item 
\begin{MapleInput}
> restart; x1:=[0.0, 0.0]: x2:=[1.0, 1.0]:
\end{MapleInput}

\begin{MapleInput}
> MyDistance:=proc(x1,x2) 
    local dx,dy; 
    dx:=(x1[1]-x2[1]); 
    dy:=(x1[2]-x2[2]);
    sqrt(dx^2+dy^2); 
  end proc:
\end{MapleInput}
\begin{MapleInput}
> MyDistance(x1,x2);
\end{MapleInput}
\begin{MapleOutput}
1.414213562
\end{MapleOutput}
\begin{MapleInput}
> x[1]:=[0.0, 0.0]: x[2]:=[1.0, 1.0]: x[3]:=[1.0, 0.0]: x[4]:=[0.0, 1.0]: x[5]:=x[1]: 
  sum(MyDistance(x[i],x[i+1]),i=1..4);
\end{MapleInput}
\begin{MapleOutput}
4.828427124
\end{MapleOutput}

\item
\begin{MapleInput}
> MyMax:=proc(A) 
    local imax,i; 
    imax:=0; 
    for i from 1 to nops(A) do 
      if A[i]>imax then 
          imax:=A[i];
      end if 
    end do; 
    return imax; 
  end proc:
\end{MapleInput}
\begin{MapleInput}
> MyMax(A);
\end{MapleInput}
\begin{MapleOutput}
100
\end{MapleOutput}

\end{enumerate} 

\end{document}
