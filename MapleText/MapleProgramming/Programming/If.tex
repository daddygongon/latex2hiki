\subsection{if}
もっとも簡単なif文の例.
\begin{MapleInput}
> x:=-4:
  if (x<0) then
   y:=-x; 
  end if;
\end{MapleInput}
\begin{MapleOutput}
 4
\end{MapleOutput}

例外付き.
\begin{MapleInput}
> x:=3:
  if (x<0) then 
    y:=-x; 
  else 
    y:=x; 
  end if;
\end{MapleInput}
\begin{MapleOutput}
3
\end{MapleOutput}
2個の条件がある例
\begin{MapleInput}
> x:=3:
  if (x<0) then 
    y:=-x; 
  elif (x>5) then 
    y:=x; 
  else
    y:=2*x; 
  end if;
\end{MapleInput}
\begin{MapleOutput}
6
\end{MapleOutput}

\paragraph{条件文に使える式と意味}
関係演算子は\verb|<, <=, >, >=, =, <>|で表記される.論理演算子にはand, or, xor, notがある.その他にもブール値を返す関数としてimplies, evalb, type などいくつかあり,条件分岐に使える.

\begin{table}[htbp]
\caption{条件分岐のいくつかの例}
\begin{center}
\begin{tabular}{|l|l|}
\hline
xとyの値が一致 & (x=y) \\
xとyの値が一致しない & (x\verb|<>|y)\\ \hline
条件文を複数つなぐ & ((x\verb|>|0) and (x\verb|<|4)) \\
&  ((x\verb|<|0) or (x\verb|>|4)) \\
&  not (x=0) \\ \hline
\end{tabular}
\end{center}
\label{default}
\end{table}%

\subsection{nextとbreak}
do-loopの途中で流れを変更するための命令.nextはdo-loop を一回スキップ.breakはそこで do-loop
を一つ抜ける.以下のコードの出力結果を参照.
\begin{MapleInput}
> for i from 1 to 5 do 
    if (i=3) then 
	  next; 
    end if; 
    print(i); 
  end do:
\end{MapleInput}
\begin{MapleError}
#res: 1 2 4 5
\end{MapleError}

\begin{MapleInput}
> for i from 1 to 5 do
    if (i=3) then
      break; 
    end if; 
    print(i); 
  end do:
\end{MapleInput}
\begin{MapleError}
#res: 1 2
\end{MapleError}

