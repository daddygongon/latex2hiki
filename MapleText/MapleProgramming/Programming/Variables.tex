\subsection{値の変数への代入(:=)}

Mapleは変数の初期設定で型宣言をする必要がない.数式処理の章で示したとおり,変数への代入は:=を使う.変数a,bにそれぞれ10,3を代入し,a+bの結果をc
に代入するというプログラムは以下の通り.
\begin{MapleInput}
> a:=10: b:=3: c:=a+b;
\end{MapleInput}
\begin{MapleOutput}
c\, := \,13
\end{MapleOutput}

\subsection{整数と浮動小数点数}
浮動小数点数から整数に直すにはいくつかの関数がある.
\begin{itemize}
\item trunc:数値から数直線で 0 に向って最も近い整数
\item round:数値の四捨五入
\item floor:数値より小さな最も大きな整数
\item ceil:数値より大きな最も小さな整数
\end{itemize}
負の値の時に floor と trunc は違った値を返す.
\MaplePlot{80mm}{./figures/MapleProgrammingimage0.eps}

小数点以下を取りだすにはfrac が用意されている.
\begin{MapleInput}
> frac(1.7);
\end{MapleInput}
\begin{MapleOutput}
0.7
\end{MapleOutput}

整数の割り算はirem(余り)とiquo(商).
\begin{MapleInput}
> irem(7,3); #res: 1
> iquo(7,3); #res: 2
\end{MapleInput}

\subsection{出力(print, printf)}
Mapleではデフォルトで結果が出力される.これを抑えるには行末の”;”を”:”に変える必要がある.出力を明示的におこなうにはprintを使う.デバッグの時に便利.
\begin{MapleInput}
> x:=1: print(x); #res: 1
\end{MapleInput}
さらに,出力を整えるのに便利なprintf関数がある.これはC言語と同じ構文で,
\begin{MapleInput}
> printf("Hello world!!\n");
\end{MapleInput}
\begin{MapleError}
Hello world!!
\end{MapleError}
と打ち込んでenterを押せば,出力が即座に表示される.値を表示するときには,
\begin{MapleInput}
> i:=3: printf("%3d\n",i);
\end{MapleInput}
\begin{MapleOutput}
3
\end{MapleOutput}
となる.これは

「変数iに入っている値を,3桁の整数形式で打ち出した後,改行せよ」

と言う意味.\%3dが出力の形式,\verb|\n|が改行を意味する.OSによっては,\verb|\|は¥と画面あるいはキーボードで表示されているかもしれない.実数の出力指定は\%10.5fで,全部で10桁,小数点以下5桁で浮動小数点数を表示.複数の変数の出力は
\begin{MapleInput}
> printf("%3d : %10.5f \n",i,a);
\end{MapleInput}
などとなる.

\begin{table}[htbp]
\caption{printfの書式指定}
\begin{center}
\begin{tabular}{|l|l|}
\hline
\%指定 & 意味\\ \hline
\%o & 整数を8進数で表示.\\
\%d & 整数を10進数で表示.\\
\%x,%X & 整数を16進数で表示.xは小文字,Xは大文字を使用.\\
\%f & 浮動小数点数として表示.\\
\%e,%E & 指数形式で表示.eは小文字,Eは大文字を使用.\\
\%s & 文字列を出力.\\ 
\hline
\end{tabular}
\end{center}
\label{default}
\end{table}%

