\begin{enumerate}
\item 
行列$A= \left[ \begin {array}{cc} 1&2\\ 3&4\end {array} \right]$,
$B= \left[ \begin {array}{cc} 2&3\\ 4&5\end {array} \right]$,
およびベクトル$v= \left[ \begin {array}{c} 1\\ 2\end {array} \right] $を作り,
以下の計算を行い結果を観察せよ.

i) $A+3B$, 
ii) $A-B$, 
iii) $A+E$, 
iv) $A.B$, 
v) $B.A$, 
vi) $A.v$, 
vii) $v.A$,
viii) $v$の転置(Transpose)を
$A$に左側から掛けよ, 
ix) $A^3$

\item 2次元平面上で原点の周りの角度tの回転行列は
\begin{MapleInput}
> Ar:=t->Matrix([[cos(t),-sin(t)],[sin(t),cos(t)]]);
\end{MapleInput}
で定義できる.

i) Pi/6 回転させる行列を作り,単位ベクトル(1,0),(0,1)がどの点に移動するか確認せよ.

ii) Pi/6 回転させた後,続けてPi/4 回転させる操作を続けて行う回転行列を求めよ.また,角度を直接入力して要素を比較せよ.

\item
行列$A= \left[ \begin {array}{ccc} 1&2&3\\ 4&5&6\\ 7&8&9\end {array} \right]$
について
$A+A^t$, 
$A-A^t$
を求めて交代行列,対称行列を作れ.
\end{enumerate}
