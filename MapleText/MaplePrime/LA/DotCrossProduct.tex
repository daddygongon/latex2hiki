線形代数の計算にはあらかじめ関数パッケージ(LinearAlgebra)を呼び出しておく.
\begin{MapleInput}
> with(LinearAlgebra):
\end{MapleInput}

\subsection{スカラーとのかけ算}

\begin{MapleInput}
> v1:=Vector([x, y]): 3*v1;
\end{MapleInput}
\begin{MapleOutput}
\left[ \begin {array}{c} 3\,x\\ 3\,y\end {array} \right]
\end{MapleOutput}

\subsection{行列,ベクトルの足し算,引き算}
\begin{MapleInput}
> LL1 := [[1, 2], [3, 4]]: A1 := Matrix(LL1): A2 := Matrix([[x, x], [y, y]]):
> 3*A1-4*A2;
\end{MapleInput}
\begin{MapleOutput}
\left[ \begin {array}{cc} 3-4\,x&6-4\,x\\ 9-4\,y&12-4\,y\end {array} \right] 
\end{MapleOutput}

\subsection{内積(DotProduct, `.`)}
\begin{MapleInput}
> v1:=Vector([1,1,3]): v2:=Vector([1,2,-1]): v1.v2;
\end{MapleInput}
\begin{MapleOutput}
                                      0
\end{MapleOutput}
\subsection{外積(CrossProduct, `\&x`)}
\begin{MapleInput}
> CrossProduct(v1, v2); v1 &x v2:
\end{MapleInput}
\begin{MapleOutput}
\left[ \begin {array}{c} -7\\ 4\\ 1\end {array} \right] 
\end{MapleOutput}

\subsection{スカラー3重積}
\begin{MapleInput}
> v3 := Vector([-1,2,1]); CrossProduct(v1,v2).v3;
\end{MapleInput}
\begin{MapleOutputGather}
{\it v3}\, := \, \left[ \begin {array}{c} -1\\ 2\\ 1\end {array} \right] \notag \\
16 \notag
\end{MapleOutputGather}

\subsection{転置(Transpose, `\&T`)}は,行列Aのij成分a[i,j]をa[j,i]にする.
\begin{MapleInput}
> Transpose(A1);
\end{MapleInput}
\begin{MapleOutput}
\left[ \begin {array}{cc} 1&3\\ 2&4\end {array} \right]
\end{MapleOutput}
また横ベクトルを縦ベクトル(あるいはその逆)にするのも同じ.
\begin{MapleInput}
> Transpose(v1);
\end{MapleInput}
\begin{MapleOutput}
\left[ \begin {array}{ccc} 1&1&3\end {array} \right]
\end{MapleOutput}
