\begin{enumerate}
\item
\begin{enumerate}
\item
\begin{MapleInput}
> Matrix([3,3,3],[3,3,3]]);
> Matrix(<3,3|3,3|3,3>);
\end{MapleInput}
\begin{MapleOutput}
\displaystyle  \left[ \begin {array}{ccc} 3&3&3\\ 3&3&3\end {array} \right]
\end{MapleOutput}
そのまま直打ちしてもいいが,少し賢い生成法も記しておく.詳しくはヘルプ参照.
\begin{MapleInput}
> Matrix(2,3,3);
\end{MapleInput}
\begin{MapleOutput}
\displaystyle  \left[ \begin {array}{ccc} 3&3&3\\ 3&3&3\end {array} \right]
\end{MapleOutput}

\item
\begin{MapleInput}
> Matrix(2,3,shape=identity);
\end{MapleInput}
\begin{MapleOutput}
\displaystyle  \left[ \begin {array}{ccc} 1&0&0\\ 0&1&0\end {array} \right]
\end{MapleOutput}

\item
\begin{MapleInput}
> Vector[row]([1,2,3]);
> Vector(<1|2|3>):
\end{MapleInput}
\begin{MapleOutput}
\displaystyle  \left[ \begin {array}{ccc} 1&2&3\end {array} \right]
\end{MapleOutput}

\item
\begin{MapleInput}
> with(LinearAlgebra):
> V:=Vector[row]([1,2,3]);
> DiagonalMatrix(V);
\end{MapleInput}
\begin{MapleOutputGather}
\displaystyle V\, := \, \left[ \begin {array}{ccc} 1&2&3\end {array} \right] \notag \\
\displaystyle  \left[ \begin {array}{ccc} 1&0&0\\ 0&2&0\\ 0&0&3\end {array} \right] \notag
\end{MapleOutputGather}

\item
\begin{MapleInput}
> f:= (i,j) -> x^(i+j-1):
> Matrix(2,f);
\end{MapleInput}
\begin{MapleOutput}
\displaystyle  \left[ \begin {array}{cc} x&{x}^{2}\\ {x}^{2}&{x}^{3}\end {array} \right] 
\end{MapleOutput}


\item
\begin{MapleInput}
> Matrix(3,[seq(i,i=1..9)]);
\end{MapleInput}
\begin{MapleOutput}
\displaystyle  \left[ \begin {array}{ccc} 1&2&3\\ 4&5&6\\ 7&8&9\end {array} \right]
\end{MapleOutput}

\begin{MapleInput}
> n:=3:
> f:=(i,j)->(i-1)*n+j;
> Matrix(3,3,f);
\end{MapleInput}
\begin{MapleOutputGather}
\displaystyle f\, := \,( {i,j} )\mapsto  \left( i-1 \right) n+j  \notag \\
\displaystyle  \left[ \begin {array}{ccc} 1&2&3\\ 4&5&6\\ 7&8&9\end {array} \right] \notag
\end{MapleOutputGather}

\end{enumerate}
\end{enumerate}