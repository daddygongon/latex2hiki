線形代数の計算にはあらかじめ関数パッケージ(LinearAlgebra)を呼び出しておく.
\begin{MapleInput}
> with(LinearAlgebra):
\end{MapleInput}

\subsection{ベクトルの生成(Vector)}
ベクトルの生成は,
\begin{MapleInput}
> v1 := Vector([x, y]);
\end{MapleInput}
\begin{MapleOutput}
\displaystyle {\it v1}\, := \, \left[ \begin {array}{c} x\\ y\end {array} \right] 
\end{MapleOutput}
通常の方法では,縦(列)ベクトル(column)ができることに注意.横(行)ベクトル(row)を作るには,明示する必要あり.
\begin{MapleInput}
> v2 := (Vector[row])([x, y, z]);
\end{MapleInput}
\begin{MapleOutput}
\displaystyle {\it v2}\, := \, \left[ \begin {array}{ccc} x&y&z\end {array} \right] 
\end{MapleOutput}
新聞の囲み記事(列)がcolumn,劇場の座席(行)はrow.

\subsection{行列の生成(Matrix)}
標準的な行列(Matrix)の生成は,
\begin{MapleInput}
> A0 := Matrix([[1, 2, 3], [4, 5, 6]]); #res: 省略
\end{MapleInput}
リストリストからの変換は,
\begin{MapleInput}
> LL1 := [[1, 2], [3, 4]]:
> A1 := Matrix(LL1);
\end{MapleInput}
\begin{MapleOutput}
\displaystyle {\it A1}\, := \, \left[ \begin {array}{cc} 1&2\\ 3&4\end {array} \right] 
\end{MapleOutput}
単位行列の生成は,
\begin{MapleInput}
> E := IdentityMatrix(2);
\end{MapleInput}
\begin{MapleOutput}
\displaystyle E\, := \, \left[ \begin {array}{cc} 1&0\\ 0&1\end {array} \right]
\end{MapleOutput}
対角行列を生成するDiagonalMatrixもある.同じことは以下のようにしても生成が可能.
\begin{MapleInput}
> Matrix(2,2,shape=identity);
\end{MapleInput}
\begin{MapleOutput}
\displaystyle  \left[ \begin {array}{cc} 1&0\\ 0&1\end {array} \right] 
\end{MapleOutput}

\subsection{縦横ベクトル・行列の簡易作成法(MVShortcut)}
かぎかっこ($<\cdots>$)を使って,ベクトルあるいは行列を直感的に作ることが可能.カンマで区切ると縦に積み,縦棒で区切ると横に積む.セミコロンで区切るとそこで次の行へ.
\begin{MapleInput}
> v1:=<x,y>; #縦ベクトル,列
> v2:=<x|y|z>; #横ベクトル,行
> A1:=<1,2;3,4>; #2x2行列
> <A1|v1>; #2x3行列(拡大係数行列などの作成)
> ?MVShortcut; #res:参照
\end{MapleInput}
\begin{MapleOutputGather}
\displaystyle {\it v1}\, := \, \left[ \begin {array}{c} x\\ y\end {array} \right] \notag \\
\displaystyle {\it v2}\, := \, \left[ \begin {array}{ccc} x&y&z\end {array} \right] \notag \\
\displaystyle {\it A1}\, := \, \left[ \begin {array}{cc} 1&2\\ 3&4\end {array} \right]  \notag \\
\displaystyle  \left[ \begin {array}{ccc} 1&2&x\\ 3&4&y\end {array} \right] \notag 
\end{MapleOutputGather}

\subsection{行列,ベクトルの成分の抽出(MVextraction)}
行列A1の1行2列の成分を取り出すには,
\begin{MapleInput}
> A1[1,2]; #res: 2
\end{MapleInput}
行列の一部を行列として取り出すには
\begin{MapleInput}
> A1[1..2,1..2];
\end{MapleInput}
\begin{MapleOutput}
\displaystyle  \left[ \begin {array}{cc} 1&2\\ 3&4\end {array} \right]
\end{MapleOutput}
2x2行列の2列目(行の長さに関係なく)でつくるベクトルは
\begin{MapleInput}
> A1[..,2..2];
\end{MapleInput}
\begin{MapleOutput}
\displaystyle  \left[ \begin {array}{c} 2\\ 4\end {array} \right] 
\end{MapleOutput}
同じことが,行(Row)あるいは列(Column)抽出関数でもできる.使い方は次の通り.
\begin{MapleInput}
> Column(A1,2);
\end{MapleInput}
\begin{MapleOutput}
\displaystyle  \left[ \begin {array}{c} 2\\ 4\end {array} \right]
\end{MapleOutput}
