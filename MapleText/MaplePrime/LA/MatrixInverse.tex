線形代数の計算にはあらかじめ関数パッケージ(LinearAlgebra)を呼び出しておく.
\begin{MapleInput}
> with(LinearAlgebra):
\end{MapleInput}

\subsection{行列式(Determinant)}
\begin{MapleInput}
> A0 := Matrix([[x,y],[z,u]]); Determinant(A0);
\end{MapleInput}
\begin{MapleOutputGather}
{\it A0}\, := \, \left[ \begin {array}{cc} x&y\\  z&u\end {array} \right] \notag \\
xu-yz \notag
\end{MapleOutputGather}

\subsection{逆行列(MatrixInverse)}
\begin{MapleInput}
> A2:=MatrixInverse(A0); simplify(A0.A2);
\end{MapleInput}
\begin{MapleOutputGather}
{\it A2}\, := \, \left[ \begin {array}{cc} {\frac {u}{xu-yz}}&-{\frac {y}{xu-yz}}\\  -{\frac {z}{xu-yz}}&{\frac {x}{xu-yz}}\end {array} \right]\notag \\
\left[ \begin {array}{cc} 1&0\\  0&1\end {array} \right]\notag
\end{MapleOutputGather}

\subsection{その他の演算}
随伴(Adjoint)などもコマンドだけで求まる.詳しくはヘルプ参照.
