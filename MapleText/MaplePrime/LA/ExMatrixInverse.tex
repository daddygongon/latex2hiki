\begin{enumerate}
\item
次の連立方程式の係数行列の行列式を求めよ.

(i)
\begin{equation*}
\left\{
\begin{array}{cc}x +y -z &=2 \\
2 x -3 y +z &=4  \\
4 x -y +3 z &=1  
\end{array} \right.
\end{equation*}

(ii)
\begin{equation*}
\left\{
\begin{array}{cl}2 x +4 y -3 z &=1   \\
3 x -8 y +6 z &=58   \\
x -2 y -9 z &=23   \\
\end{array} \right.
\end{equation*}

(iii)
\begin{equation*}
\left\{
\begin{array}{cl}
1 x -10 y -3 z -7 u &=2   \\
2 x -4 y +3 z +4 u &=-3   \\
x -2 y +6 z +5 u &=-1   \\
x +8 y +9 z +3 u &=5
\end{array} \right.
\end{equation*}
(iv)
\begin{equation*}
\left\{\begin{array}{cl}x +y +z &=a +b +c    \\
ax+by+cz &=ab +bc +ca   \\
bc\,x +ca\,y + ab\,z &=3\,abc 
\end{array} \right.
\end{equation*}

\item
上の連立方程式の係数行列の逆行列を求めよ.またベクトルbに作用して解を求めよ.
\end{enumerate}
