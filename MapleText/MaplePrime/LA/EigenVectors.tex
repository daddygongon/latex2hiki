線形代数の計算にはあらかじめ関数パッケージ(LinearAlgebra)を呼び出しておく.
\begin{MapleInput}
> with(LinearAlgebra):
\end{MapleInput}

\subsection{固有値(EigenVectors)}
固有値(Eigenvalues)と固有ベクトルを共に求めるにはEigenvectorsを使う.下の例では,固有値と固有ベクトルを変数l,vに代入している.
\begin{MapleInput}
> A0 := Matrix(2, 2, [[1,2], [2,1]]); l,v:=Eigenvectors(A0);
\end{MapleInput}
\begin{MapleOutputGather}
{\it A0}\, := \, \left[ \begin {array}{cc} 1&2\\  2&1\end {array} \right] \notag \\
l,\,v\, := \, \left[ \begin {array}{c} -1\\  3\end {array} \right] ,\, \left[ \begin {array}{cc} -1&1\\  1&1\end {array} \right] \notag
\end{MapleOutputGather}
\subsection{固有ベクトルの取り出し(Column)}
行列の列を要素とするベクトル生成Columnを使って,一番目の固有値に対応する固有ベクトルを取り出す.
\begin{MapleInput}
> Column(v,1);
\end{MapleInput}
\begin{MapleOutput}
\left[ \begin {array}{c} -1\\  1\end {array} \right]
\end{MapleOutput}
これを使って,固有値(l)と固有ベクトル(v)の関係
\begin{equation*}
A_0.v=\lambda.v
\end{equation*}
が確認できる.
\begin{MapleInput}
> A0.Column(v,1); l[1]*Column(v,1);
\end{MapleInput}
\begin{MapleOutputGather}
\left[ \begin {array}{c} 1\\  -1\end {array} \right] \notag \\
\left[ \begin {array}{c} 1\\  -1\end {array} \right] \notag
\end{MapleOutputGather}

\subsection{固有ベクトルの規格化(Normalize)}
用意されているコマンドが確かめられる.
\begin{MapleInput}
> ?Normalize;
\end{MapleInput}
一般的な内積を使って長さを規格化するには,以下のコマンドを使う.
\begin{MapleInput}
> Normalize(Column(v,1),Euclidean);
\end{MapleInput}
\begin{MapleOutput}
\left[ \begin {array}{c} -1/2\,\sqrt {2}\\  1/2\,\sqrt {2}\end {array} \right] 
\end{MapleOutput}

\subsection{対角化}
固有ベクトルを用いて,次のとおり行列は対角化される.
\begin{MapleInput}
> MatrixInverse(v).A0.v;
\end{MapleInput}
\begin{MapleOutput}
\left[ \begin {array}{cc} -1&0\\  0&3\end {array} \right]
\end{MapleOutput}
一見対角化されてない場合でも,simplifyを掛けて整理すると対角化されているのが確認できる.

\subsection{その他の演算}
対角和(Trace),ジョルダン標準形(JordanForm)などもコマンドだけで求まる.詳しくはヘルプ参照.
