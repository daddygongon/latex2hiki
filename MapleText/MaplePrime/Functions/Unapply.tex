\subsection{ユーザー定義関数(unapply)}
初等関数やその他の関数を組み合わせてユーザー定義関数を作ることができる.

関数$f(x) = 2 x - 3$とおくとする場合,Mapleでは,
\begin{MapleInput}
> f:=x->2*x-3;
\end{MapleInput}
\begin{MapleOutput}
f:= x \rightarrow 2 x - 3
\end{MapleOutput}
と,矢印で書く.これが関数としてちゃんと定義されているかは,いくつかの数値や変数を$f(x)$に代入して確認する.
\begin{MapleInput}
> f(3);               #res: 3 (以降出力を省略する場合はこのように表記)
  f(a);               #res: 2 a - 3
  plot(f(x),x=-2..2);
\end{MapleInput}
\MaplePlot{50mm}{./figures/Functionsplot2d1.eps}
もう一つ関数定義のコマンドとして次のunapplyも同じ意味である.
\begin{MapleInput}
> f:=unapply(2*x-3,x);
\end{MapleInput}
\begin{MapleOutput}
f:= x \rightarrow 2 x - 3
\end{MapleOutput}
ただし,矢印での定義ではときどき変な振る舞いになるので,unapplyを常に使うようにこころがけたほうが安全.

