\subsection{単純な微分(diff)}
単純な一変数関数の一次微分は,以下の通り.
\begin{MapleInput}
> diff(x^2-3*x+2,x);	#res: 2x-3
\end{MapleInput}
高次の微分は,微分変数を必要なだけ並べる.
\begin{MapleInput}
> diff(sin(x),x,x);	#res: -sin(x)
\end{MapleInput}
さらに高次では次のように\$を使った記法が便利.これはxについての3次微分を表わす.
\begin{MapleInput}
> diff(x^4,x$3);	#res: 24x
\end{MapleInput}

\paragraph{偏微分(PartialDiff)}
複数の変数を持つ多変数の関数では,微分する変数を明示すれば偏微分が求められる.
\begin{MapleInput}
> eq1:=(x+y)/(x*y);
> diff(eq1,x);
\end{MapleInput}
\begin{MapleOutput}
eq1 := \,{\frac {x+y}{xy}}
\end{MapleOutput}
\begin{MapleOutput}
{\frac {1}{xy}}-{\frac {x+y}{{x}^{2}y}}
\end{MapleOutput}