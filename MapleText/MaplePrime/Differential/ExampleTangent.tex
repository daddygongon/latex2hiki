\subsection{例題:接線(Tangent)}
次の関数の$\displaystyle x=3$での接線を求め,2つの関数を同時にプロットせよ.
\begin{equation*}
y={x}^{3}-2\,{x}^{2}-35\,x
\end{equation*}

\paragraph{解答例}
与関数をf0と定義.
\begin{MapleInput}
> f0:=unapply(x^3 - 2*x^2 - 35*x,x);
\end{MapleInput}
\begin{MapleOutput}
{\it f0}\, := \,x\mapsto {x}^{3}-2\,{x}^{2}-35\,x
\end{MapleOutput}
微分関数をdfと定義
\begin{MapleInput}
> df:=unapply(diff(f0(x),x),x);
\end{MapleInput}
\begin{MapleOutput}
{\it df}\, := \,x\mapsto 3\,{x}^{2}-4\,x-35
\end{MapleOutput}
接点(x0,f0(x0))で傾きdf(x0)の直線をf1と定義.
\begin{MapleInput}
> x0:=3;
> eq1:=df(x0)*(x-x0)+f0(x0);
> f1:=unapply(eq1,x);
\end{MapleInput}
\begin{MapleOutputGather}
{\it x0}\, := \,3 \notag \\
{\it eq1}\, := \,-20\,x-36 \notag \\
{\it f1}\, := \,x\mapsto -20\,x-36 \notag
\end{MapleOutputGather}
2つの関数を同時にプロット.
\begin{MapleInput}
> plot([f0(x),f1(x)],x=-5..5);
\end{MapleInput}
\MaplePlot{30mm}{./figures/Diffplot2d2.eps}
