\subsection{全微分(D)}
全微分を計算するときは,Dを用いる.
\begin{MapleInput}
> f:=unapply(x^4*exp(-y^2),(x,y));
> D(f(x,y));
> (D@@2)(f(x,y));
\end{MapleInput}
\begin{MapleOutputGather}
f\, := \,( {x,y} )\mapsto {x}^{4}\exp(-{y}^{2}) \notag \\
4\, {D} \left( x \right) {x}^{3}\exp(-{y}^{2})+{x}^{4} {D} \left( \exp(-{y}^{2}) \right) \notag \\
4\, \left( D^{ \left( 2 \right) } \right)  \left( x \right) {x}^{3}\exp(-{y}^{2})+12\, \left(  {D} \left( x \right)  \right) ^{2}{x}^{2}\exp(-{y}^{2})+8\, {D} \left( x \right) {x}^{3} {D} \left( \exp(-{y}^{2}) \right) +{x}^{4} \left( D^{ \left( 2 \right) } \right)  \left( \exp(-{y}^{2}) \right) \notag
\end{MapleOutputGather}

ここで,D(x)などはxの全微分を表わす.これは,x,yを変数としているので
\begin{MapleInput}
> diff(x,x);
> diff(exp(-y^2),y);
\end{MapleInput}
\begin{MapleOutputGather}
1 \notag \\
 -2\,y\exp(-{y}^{2}) \notag
\end{MapleOutputGather}
であるがMapleには分からない.そこで全微分の最終形を得るには,あらかじめD(x)などの結果を求めておき,subsで明示的に代入する必要がある.
\begin{MapleInput}
> dd:=D(f(x,y)):
> eqs:={D(x)=diff(x,x),D(exp(-y^2))=diff(exp(-y^2),y)};
> subs(eqs,dd);
\end{MapleInput}
\begin{MapleOutputGather}
 {\it eqs}\, := \, \left\{  {D} \left( x \right) =1, {D} \left( \exp(-{y}^{2}) \right) =-2\,y\exp(-{y}^{2}) \right\}  \notag \\
 4\,{x}^{3}\exp(-{y}^{2})-2\,{x}^{4}y\exp(-{y}^{2}) \notag
\end{MapleOutputGather}
