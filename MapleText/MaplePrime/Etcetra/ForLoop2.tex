\ifHIKI
{{toc_here}}
\else
\fi
\subsection{ランダムな配列の生成}
1から100までの整数5個からなる配列の生成.
\begin{MapleInput}
> restart: 
  roll:=rand(1..100): 
  n:=5: 
  A:=[seq(roll(),i=1..n)];
\end{MapleInput}
\begin{MapleError}
                             [93, 45, 96, 6, 98]
\end{MapleError}

\subsection{要素数の取り出し}
for-loopで配列を使うときには,配列の大きさ(要素数)がfor-loopの終了条件になることが多い.
リスト構造では単純にnopsとすればよい.
\begin{MapleInput}
> nops(A);
\end{MapleInput}
\begin{MapleError}
                                      5
\end{MapleError}

\subsection{すべての要素の表示}
配列はおなじ箱が沢山用意されていると考えればよい.配列をfor-loopで使うときは,箱を指す数(示数,index)をいじっているのか,箱の中身(要素)をいじっているのかを意識すれば,動作を理解しやすい.
\begin{MapleInput}
> for i from 1 to n do 
    print(i,A[i]); 
  end do;
\end{MapleInput}
\begin{MapleError}
                                    1, 93
                                    2, 45
                                    3, 96
                                     4, 6
                                    5, 98
\end{MapleError}
逆順の表示
\begin{MapleInput}
> for i from n by -1 to 1 do
    print(i,A[i]); 
  end do;
\end{MapleInput}
\begin{MapleError}
                                    5, 98
                                     4, 6
                                    3, 96
                                    2, 45
                                    1, 93
\end{MapleError}
逆順の表示2
\begin{MapleInput}
> for i from 1 to n do
    print(n-i+1,A[n-i+1]); 
  end do;
\end{MapleInput}
\begin{MapleError}
                                    5, 98
                                     4, 6
                                    3, 96
                                    2, 45
                                    1, 93
\end{MapleError}
\subsection{和}
\begin{MapleInput}
> sum1:=0: 
  for i from 1 to n do 
    sum1:=sum1+A[i]; 
  end do: 
  sum1;
\end{MapleInput}
\begin{MapleError}
                                     338
\end{MapleError}
\paragraph{課題:積を求めよ.}
\subsection{値の代入}
\begin{MapleInput}
> k:=64: 
  for i from 1 to n do 
    A[i]:=A[i]/k; 
  end do: 
  A;
\end{MapleInput}
\begin{MapleError}
                        [93/64, 45/64, 3/2, 3/32, 49/32]
\end{MapleError}
\paragraph{課題:先の和と組み合わせて,全要素の和が1になるように規格化せよ.}
\paragraph{課題:配列Bへ逆順に代入せよ.}

\subsection{一桁の整数5個から5桁の整数を作る}
まず,一桁の整数でできるランダムな配列を作成する.
\begin{MapleInput}
> roll:=rand(0..9): n:=5: A:=[seq(roll(),i=1..n)];
\end{MapleInput}
\begin{MapleError}
                            A := [3, 5, 4, 0, 7]
\end{MapleError}
\begin{MapleInput}
> sum1:=0; 
  for i from 1 to n do 
    sum1:=sum1*10+A[i]; 
  end do: 
  sum1;
\end{MapleInput}
\begin{MapleError}
                                      0
                                    35407
\end{MapleError}
\paragraph{課題:上記と同様にして,10桁の2進数を10進数へ変換せよ}


\subsection{255以下の10進数をランダムに生成して,8桁の2進数へ変換せよ.}
\begin{MapleInput}
> n:=8: 2^n;
\end{MapleInput}
\begin{MapleError}
                                     256
\end{MapleError}
\begin{MapleInput}
> roll:=rand(0..255):
  B:=roll();
\end{MapleInput}
\begin{MapleError}
                                     161
\end{MapleError}
ちょっとカンニング.
\begin{MapleInput}
> convert(B,binary);
\end{MapleInput}
\begin{MapleError}
                            10100001
\end{MapleError}

\begin{MapleInput}
> A:=[]:
  for i from 1 to n do
    A:=[irem(B,2),op(A)];
    B:=iquo(B,2);
  end do:
  A;
\end{MapleInput}
\begin{MapleOutput}
[1, 0, 1, 0, 0, 0, 0, 1]
\end{MapleOutput}

\paragraph{課題:8桁の整数のそれぞれの桁の値を配列に格納せよ.}
8桁の整数は以下のようにして作られる.
\begin{MapleInput}
> n:=8; 
  roll:=rand(10^(n-1)..10^n): 
  B:=roll();
\end{MapleInput}
\begin{MapleError}
                                      8
                                   17914675
\end{MapleError}
\subsection{小数点以下8桁のそれぞれの桁の数を配列に格納せよ}
\begin{MapleInput}
> n:=8: 
  roll:=rand(10^(n-1)..10^n): 
  B:=evalf(roll()/10^n);
\end{MapleInput}
\begin{MapleError}
                                 0.6308447100
\end{MapleError}
\begin{MapleInput}
> B:=10*B:
  A:=[]:
  for i from 1 to n do
    A:=[op(A),floor(B)];
    B:=(B-A[i])*10;
  end do:
  A;
\end{MapleInput}
\begin{MapleOutput}
[6, 3, 0, 8, 4, 4, 7, 1]
\end{MapleOutput}

\subsection{最大数}
\begin{MapleInput}
> roll:=rand(1..100): 
  n:=5: 
  A:=[seq(roll(),i=1..n)]; 
  i_max:=A[1]: 
  for i from 2 to n do 
    if (A[i]>i_max) then 
      i_max:=A[i]; 
    end if; 
  end do: 
  i_max;
\end{MapleInput}
\begin{MapleError}
                                      64
\end{MapleError}
\paragraph{課題:最小値を求めよ.}

\subsection{ある値の上下で分けた個数}
\begin{MapleInput}
> roll:=rand(1..100): 
  n:=5: A:=[seq(roll(),i=1..n)];
  i_div:=50:i_low:=0:i_high:=0: 
  for i from 1 to n do 
    if (A[i]>i_div) then
      i_high:=i_high+1; 
    else 
      i_low:=i_low+1; 
    end if 
  end do; 
  print(i_low,i_high);
\end{MapleInput}
\begin{MapleError}
                                     2, 3
\end{MapleError}

\subsection{素数かどうかの判定}
\begin{MapleInput}
> n:=10; 
  for i from 1 to n do 
    if (isprime(i)) then 
      print(i); 
    end if; 
  end do;
\end{MapleInput}

\subsection{2つの要素の入れ替え}
\begin{MapleInput}
> roll:=rand(1..100): n:=5: A:=[seq(roll(),i=1..n)]; sel:=rand(1..n):
  isel:=sel(); 
  jsel:=sel(); 
  a:=A[isel]; b:=A[jsel]; A[isel]:=b; A[jsel]:=a; 
  A;
\end{MapleInput}
\begin{MapleError}
                      [60, 93, 14, 50, 47]
                               2
                               4
                               93
                               50
                               50
                               93
                      [60, 50, 14, 93, 47]
\end{MapleError}
より短くするには,
\begin{MapleInput}
> roll:=rand(1..100):
  n:=5:
  A:=[seq(roll(),i=1..n)];
  sel:=rand(1..n):
  isel:=sel();
  jsel:=sel();
  a:=A[isel];
  A[isel]:=A[jsel];
  A[jsel]:=a;
  A;
\end{MapleInput}
\begin{MapleError}
                       [9, 77, 59, 16, 1]
                               5
                               4
                               1
                               16
                               1
                       [9, 77, 59, 1, 16]
\end{MapleError}

\subsection{コインの表向きの枚数}
\begin{MapleInput}
> roll:=rand(0..1):
  n:=10:
  up:=0:
  for i from 1 to n do
    trial:=roll();
    if (trial=1) then 
      up:=up+1;
    end if;
  end do:
  up;
\end{MapleInput}
\begin{MapleError}
                                      5
\end{MapleError}

\paragraph{課題:1..6のサイコロを20回振って,出た目を記録せよ.}
記録には,要素が0の配列を最初に用意し,出た目を示数にして配列の要素をひとつずつ増やす.
\subsection{2次元配列}
2次元配列に対しても同様の操作ができる.ここでは列に対する規格化を示す.
\begin{MapleInput}
> roll:=rand(1..5):
  n:=3:
  A:=[seq([seq(roll(),i=1..n)],j=1..n)];
\end{MapleInput}
\begin{MapleError}
                     A := [[5, 2, 2], [2, 3, 2], [4, 2, 1]]
\end{MapleError}
\begin{MapleInput}
> roll:=rand(1..5):
  n:=3:
  A:=[seq([seq(roll(),i=1..n)],j=1..n)];
\end{MapleInput}
\begin{MapleError}
                            1, 1, 5
                            1, 2, 2
                            1, 3, 2
                            2, 1, 2
                            2, 2, 3
                            2, 3, 2
                            3, 1, 4
                            3, 2, 2
                            3, 3, 1
\end{MapleError}
i,jの順序に注意.
\begin{MapleInput}
> for j from 1 to n do
    tmp:=0;
    for i from 1 to n do
      tmp:=tmp+A[i,j];
    end do;
    for i from 1 to n do
      A[i,j]:=A[i,j]/tmp;
    end do;
  end do:
  A;
\end{MapleInput}
\begin{MapleError}
         [[5/11, 2/7, 2/5], [2/11, 3/7, 2/5], [4/11, 2/7, 1/5]]
\end{MapleError}
