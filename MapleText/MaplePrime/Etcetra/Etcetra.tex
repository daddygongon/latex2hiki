\documentclass[10pt,a4j]{jreport}
\usepackage[dvips]{graphicx,color}
\usepackage{tabularx}
\usepackage{verbatim}
\usepackage{amsmath,amsthm,amssymb}
\topmargin -15mm\oddsidemargin -4mm\evensidemargin\oddsidemargin
\textwidth 170mm\textheight 257mm\columnsep 7mm
\setlength{\fboxrule}{0.2ex}
\setlength{\fboxsep}{0.6ex}

\pagestyle{empty}

\newcommand{\MaplePlot}[2]{{\begin{center}
    \includegraphics[width=#1,clip]{#2}
                     \end{center}
%
} }

\newenvironment{MapleInput}{%
    \color{red}\verbatim
}{%
    \endverbatim
}

\newenvironment{MapleError}{%
    \color{blue}\verbatim
}{%
    \endverbatim
}

\newenvironment{MapleOutput}{%
    \color{blue}\begin{equation*}
}{%
    \end{equation*}
}

\newenvironment{MapleOutputGather}{%
    \color{blue}\gather
}{%
    \endgather
}

\newcommand{\ChartElement}[1]{{
\color{magenta}\begin{flushleft}$\left[\left[\left[\textbf{\large #1}\right]\right]\right]$
\end{flushleft}\vspace{-10mm}
} }

\newcommand{\ChartElementTwo}[1]{{
\color{magenta}\begin{flushleft}$\left[\left[\left[\textbf{\large #1}\right]\right]\right]$
\end{flushleft}
} }

\newcommand{\ChartElementThree}[2]{{
\color{magenta}\begin{flushleft}$\left[\left[\left[\textbf{\large #2}\right]\right]\right]$
\end{flushleft}\vspace{#1}
} }

\newif\ifHIKI
%\HIKItrue % TRUEの設定
\HIKIfalse % FALSEの設定
\begin{document}
\chapter{その他(Etcetra)}
\section{ファイルの入出力(InputOutput)}
\ChartElementTwo{解説}
%% Created by Maple 15.01, Mac OS X
%% Source Worksheet: InputOutput.mw
%% Generated: Sat Sep 01 07:24:50 JST 2012
\documentclass{article}
\usepackage{maplestd2e}
\def\emptyline{\vspace{12pt}}
\begin{document}
\pagestyle{empty}
\DefineParaStyle{Maple Heading 1}
\DefineParaStyle{Maple Text Output}
\DefineParaStyle{Maple Dash Item}
\DefineParaStyle{Maple Bullet Item}
\DefineParaStyle{Maple Normal}
\DefineParaStyle{Maple Heading 4}
\DefineParaStyle{Maple Heading 3}
\DefineParaStyle{Maple Heading 2}
\DefineParaStyle{Maple Warning}
\DefineParaStyle{Maple Title}
\DefineParaStyle{Maple Error}
\DefineCharStyle{Maple Hyperlink}
\DefineCharStyle{Maple 2D Math}
\DefineCharStyle{Maple Maple Input}
\DefineCharStyle{Maple 2D Output}
\DefineCharStyle{Maple 2D Input}
\section{\begin{center}
\textbf{�t�@�C�����o��}\end{center}
}
\begin{maplelatex}\begin{flushright}
Copyright @2006 by Shigeto R. Nishitani\end{flushright}
\end{maplelatex}
\begin{maplelatex}�����animation��gif�t�@�C���Ƃ��ĕۑ�������C����f�[�^�Ȃǂ�ǂݍ���ŕ\�������y�ȕ��@������D���̂��߂ɂ̓t�@�C���Ƃ̂��Ƃ������K�v������܂��D\end{maplelatex}
\subsection{\textbf{�f�[�^�̓��o��}}
\subsubsection{\textbf{\textit{�t�@�C�����̎擾}}}
\begin{maplegroup}
\begin{Maple Normal}{
�t�@�C�����̎擾�́CJava�̕W���֐����g����Maplet�p�b�P�[�W��GetFile�֐����g���DGetFile�֐����Ăт����ĊJ�����t�@�C���I���E�B���h�E�Ńt�@�C�����w�肷��ƃt�@�C���̃p�X��file1�ɓ���D}\end{Maple Normal}

\textbf{restart;}\textbf{with(Maplets[Examples]):}\textbf{file1:=GetFile();}\mapleresult
\begin{maplelatex}
\mapleinline{inert}{2d}{file1 := "/Users/bob/MapleTest/data1.txt"}{\[\displaystyle {\it file1}\, := \,``/Users/bob/MapleTest/data1.txt''\]}
\end{maplelatex}
\end{maplegroup}
\begin{maplegroup}
\begin{Maple Normal}{
Windows�ł�"\"��"/"�ɕϊ�����K�v������D���{��̃t�@�C�����͕����������Ďg���Ȃ��D}\end{Maple Normal}

\textbf{with(StringTools):}\textbf{file2:=SubstituteAll(file1,"\\\\","/");}\mapleresult
\begin{maplelatex}
\mapleinline{inert}{2d}{file2 := "/Users/bob/MapleTest/data1.txt"}{\[\displaystyle {\it file2}\, := \,``/Users/bob/MapleTest/data1.txt''\]}
\end{maplelatex}
\end{maplegroup}
\begin{maplegroup}
\begin{Maple Normal}{
�t�@�C�����̕ύX�͎�ł�邩�C���邢��Substitute�֐����g���D}\end{Maple Normal}

\textbf{with(StringTools):}\textbf{file2:=Substitute(file1,"data1","data2");}\mapleresult
\begin{maplelatex}
\mapleinline{inert}{2d}{file2 := "/Users/bob/MapleTest/data2.txt"}{\[\displaystyle {\it file2}\, := \,``/Users/bob/MapleTest/data2.txt''\]}
\end{maplelatex}
\end{maplegroup}
\subsubsection{\textbf{\textit{�ȒP�ȃf�[�^�̂��Ƃ�}}}
\begin{maplegroup}
\begin{Maple Normal}{
�t�@�C���Ƃ̒P���ȃf�[�^�̂��Ƃ��writedata,readdata�֐����g���D�Ⴆ�΁C�ȉ��̂悤�ȃf�[�^��������Ƃ���D������t�@�C���֏��������ɂ́C�ȉ��̂悤����D}\end{Maple Normal}

\textbf{f1:=t->subs(\{a=10,b=40000,c=380,d=128\},a+b/(c+(t-d)\symbol{94}2) ):}\textbf{T:=[seq(f1(i)*(0.6+0.8*evalf(rand()/10\symbol{94}12)),i=1..256)]:}\textbf{writedata(file1,T);}\end{maplegroup}
\begin{maplegroup}
\begin{Maple Normal}{
�����悤�ɂ��ēǂݍ���ŕ\�������Ă݂�D}\end{Maple Normal}

\textbf{T:=readdata(file1,1):}\textbf{with(plots):}\textbf{listplot(T);}\mapleresult
\mapleplot{InputOutputplot2d1.eps}
\end{maplegroup}
\subsubsection{\textbf{\textit{�������x�ȃf�[�^�̂��Ƃ�}}}
\begin{maplegroup}
\begin{Maple Normal}{
writeto�֐��ŏo�͂��O���t�@�C���֐؂�ւ��邱�Ƃ��”\�D}\end{Maple Normal}

\textbf{interface(quiet=true);}\textbf{writeto(file2);}\textbf{for i from 1 to 10 do}\textbf{s1:=data||i;}\textbf{printf("\%10.5f \%s\symbol{92}n",evalf(f1(i)),s1);}\textbf{end do:}\textbf{writeto(terminal):}\textbf{interface(quiet=false);}\mapleresult
\begin{maplelatex}
\mapleinline{inert}{2d}{false}{\[\displaystyle {\it false}\]}
\end{maplelatex}
\mapleresult
\begin{maplelatex}
\mapleinline{inert}{2d}{true}{\[\displaystyle {\it true}\]}
\end{maplelatex}
\end{maplegroup}
\begin{maplegroup}
\begin{Maple Normal}{
C����̕W���I�ȃf�[�^�ǂݍ��݂Ɏ������������ł���D�ȉ���fopen, readline, sscanf, fclose���g�����f�[�^�̓��́D}\end{Maple Normal}

\textbf{fd:=fopen(file2,READ);}\textbf{for i from 1 to 2 do}\textbf{l1:=readline(fd);}\textbf{d:=sscanf(l1,"\%f \%s");}\textbf{end do;}\textbf{fclose(fd):}\mapleresult
\begin{maplelatex}
\mapleinline{inert}{2d}{fd := 1}{\[\displaystyle {\it fd}\, := \,1\]}
\end{maplelatex}
\mapleresult
\begin{maplelatex}
\mapleinline{inert}{2d}{l1 := "  12.42292 data1"}{\[\displaystyle {\it l1}\, := \,``  12.42292 data1''\]}
\end{maplelatex}
\mapleresult
\begin{maplelatex}
\mapleinline{inert}{2d}{d := [12.42292, "data1"]}{\[\displaystyle d\, := \,[ 12.42292,``data1'']\]}
\end{maplelatex}
\mapleresult
\begin{maplelatex}
\mapleinline{inert}{2d}{l1 := "  12.46063 data2"}{\[\displaystyle {\it l1}\, := \,``  12.46063 data2''\]}
\end{maplelatex}
\mapleresult
\begin{maplelatex}
\mapleinline{inert}{2d}{d := [12.46063, "data2"]}{\[\displaystyle d\, := \,[ 12.46063,``data2'']\]}
\end{maplelatex}
\end{maplegroup}
\begin{maplegroup}
\begin{Maple Normal}{
fd�Ƀt�@�C�����ʎq(file descripter)�������Ă����Creadline��1�s���“ǂ܂���D�����sscanf��format�ɂ���������l1�Ɋi�[���Ă����Dl1�ɂ͎����I�ɓK�؂Ȍ`���ŕϐ������Ă����D}\end{Maple Normal}

\textbf{d[1];}\textbf{whattype(d[1]);}\mapleresult
\begin{maplelatex}
\mapleinline{inert}{2d}{12.46063}{\[\displaystyle  12.46063\]}
\end{maplelatex}
\mapleresult
\begin{maplelatex}
\mapleinline{inert}{2d}{float}{\[\displaystyle {\it float}\]}
\end{maplelatex}
\end{maplegroup}
\subsubsection{\textbf{\textit{animation�̏o��}}}
\begin{maplegroup}
\begin{Maple Normal}{
animation�Ȃǂ�gif�`����plot���O���t�@�C���֏o�͂��ĕ\��������ɂ́C�ȉ��̈�A�̃R�}���h�̂悤�ɂ���D}\end{Maple Normal}

\textbf{plotsetup(gif,plotoutput=file2):}\textbf{display(tmp,insequence=true);}\textbf{plotsetup(default):}\end{maplegroup}
\begin{maplegroup}
\begin{Maple Normal}{
�����‚�quicktime�ȂǂɐH�킹��΁CMaple�ȊO�̃\�t�g�œ���\�����”\�ƂȂ�D3�����}�`�̕W���K�i�ł���vrml�������悤�ɂ��č쐬���邱�Ƃ��”\�ł�(?vrml;�Q��)�D}\end{Maple Normal}

\end{maplegroup}
\section{\textbf{linux�ł̃t�B���^�[�Ƃ��Ă̗��p�@}}
\begin{maplegroup}
\begin{Maple Normal}{
linux�łł͕����x�[�X��maple���g���āCfilter�Ƃ��č��x�ȍ�Ƃ������邱�Ƃ��o���܂��D�X�N���v�g�̒��ɊO���t�@�C���Ƃ̓��o�͂�g�ݍ��߂΁C���܂܂ŏЉ�Ă������G�ȓ�����u���b�N�{�b�N�X�̓��������Ƃ��Ă��̂܂܎g���܂��D}\end{Maple Normal}

\end{maplegroup}
\begin{maplegroup}
\begin{Maple Normal}{
[bob@asura0 \symbol{126}/test]\$ cat test.txt}\end{Maple Normal}

\begin{Maple Normal}{
T:=readdata("./data101");}\end{Maple Normal}

\begin{Maple Normal}{
interface(quiet=true);}\end{Maple Normal}

\begin{Maple Normal}{
writeto("./result");}\end{Maple Normal}

\begin{Maple Normal}{
print(T[1]);}\end{Maple Normal}

\begin{Maple Normal}{
writeto(terminal);}\end{Maple Normal}

\begin{Maple Normal}{
interface(quiet=false);}\end{Maple Normal}

\begin{Maple Normal}{
�Ƃ���΁Cdata101����ǂݍ��񂾃f�[�^�ɉ��炩�̏������{�������ʂ�result�ɑł��o�����Ƃ��”\�Dinterface(quiet=true)�ŗ]�v�ȏo�͂�}�~���Ă��܂��D�����maple�ɐH�킹���}\end{Maple Normal}

\begin{Maple Normal}{
[bob@asura0 \symbol{126}/test]\$ /usr/local/maple9.5/bin/maple < test.txt}\end{Maple Normal}

\begin{Maple Normal}{
|\\symbol{94}/|     Maple 9.5 (IBM INTEL LINUX)}\end{Maple Normal}

\begin{Maple Normal}{
.\_|\|   |/|\_. Copyright (c) Maplesoft, a division of Waterloo Maple Inc. 2004}\end{Maple Normal}

\begin{Maple Normal}{
\  MAPLE  /  All rights reserved. Maple is a trademark of}\end{Maple Normal}

\begin{Maple Normal}{
<\_\_\_\_ \_\_\_\_>  Waterloo Maple Inc.}\end{Maple Normal}

\begin{Maple Normal}{
|       Type ? for help.}\end{Maple Normal}

\begin{Maple Normal}{
> T:=readdata("./data101");}\end{Maple Normal}

\begin{Maple Normal}{
T := [1.23, 2.35]}\end{Maple Normal}

\begin{Maple Normal}{
> interface(quiet=true);}\end{Maple Normal}

\begin{Maple Normal}{
false}\end{Maple Normal}

\begin{Maple Normal}{
true}\end{Maple Normal}

\begin{Maple Normal}{
> quit}\end{Maple Normal}

\begin{Maple Normal}{
bytes used=211000, alloc=262096, time=0.00}\end{Maple Normal}

\begin{Maple Normal}{
�߂ł����o�͂���Ă���͂��D}\end{Maple Normal}

\begin{Maple Normal}{
[bob@asura0 \symbol{126}/test]\$ cat result}\end{Maple Normal}

\begin{Maple Normal}{
1.23}\end{Maple Normal}
\end{maplegroup}
\section{\textbf{���K}}
\begin{maplegroup}
\begin{Maple Normal}{
}\end{Maple Normal}
\end{maplegroup}
\begin{maplegroup}
\begin{mapleinput}
\mapleinline{active}{1d}{}{}
\end{mapleinput}
\end{maplegroup}
\end{document}

\section{for-loopの基本技(for-loop2)}
\ChartElementTwo{解説}
\ifHIKI
{{toc_here}}
\else
\fi
\subsection{ランダムな配列の生成}
1から100までの整数5個からなる配列の生成.
\begin{MapleInput}
> restart: 
  roll:=rand(1..100): 
  n:=5: 
  A:=[seq(roll(),i=1..n)];
\end{MapleInput}
\begin{MapleError}
                             [93, 45, 96, 6, 98]
\end{MapleError}

\subsection{要素数の取り出し}
for-loopで配列を使うときには,配列の大きさ(要素数)がfor-loopの終了条件になることが多い.
リスト構造では単純にnopsとすればよい.
\begin{MapleInput}
> nops(A);
\end{MapleInput}
\begin{MapleError}
                                      5
\end{MapleError}

\subsection{すべての要素の表示}
配列はおなじ箱が沢山用意されていると考えればよい.配列をfor-loopで使うときは,箱を指す数(示数,index)をいじっているのか,箱の中身(要素)をいじっているのかを意識すれば,動作を理解しやすい.
\begin{MapleInput}
> for i from 1 to n do 
    print(i,A[i]); 
  end do;
\end{MapleInput}
\begin{MapleError}
                                    1, 93
                                    2, 45
                                    3, 96
                                     4, 6
                                    5, 98
\end{MapleError}
逆順の表示
\begin{MapleInput}
> for i from n by -1 to 1 do
    print(i,A[i]); 
  end do;
\end{MapleInput}
\begin{MapleError}
                                    5, 98
                                     4, 6
                                    3, 96
                                    2, 45
                                    1, 93
\end{MapleError}
逆順の表示2
\begin{MapleInput}
> for i from 1 to n do
    print(n-i+1,A[n-i+1]); 
  end do;
\end{MapleInput}
\begin{MapleError}
                                    5, 98
                                     4, 6
                                    3, 96
                                    2, 45
                                    1, 93
\end{MapleError}
\subsection{和}
\begin{MapleInput}
> sum1:=0: 
  for i from 1 to n do 
    sum1:=sum1+A[i]; 
  end do: 
  sum1;
\end{MapleInput}
\begin{MapleError}
                                     338
\end{MapleError}
\paragraph{課題:積を求めよ.}
\subsection{値の代入}
\begin{MapleInput}
> k:=64: 
  for i from 1 to n do 
    A[i]:=A[i]/k; 
  end do: 
  A;
\end{MapleInput}
\begin{MapleError}
                        [93/64, 45/64, 3/2, 3/32, 49/32]
\end{MapleError}
\paragraph{課題:先の和と組み合わせて,全要素の和が1になるように規格化せよ.}
\paragraph{課題:配列Bへ逆順に代入せよ.}

\subsection{一桁の整数5個から5桁の整数を作る}
まず,一桁の整数でできるランダムな配列を作成する.
\begin{MapleInput}
> roll:=rand(0..9): n:=5: A:=[seq(roll(),i=1..n)];
\end{MapleInput}
\begin{MapleError}
                            A := [3, 5, 4, 0, 7]
\end{MapleError}
\begin{MapleInput}
> sum1:=0; 
  for i from 1 to n do 
    sum1:=sum1*10+A[i]; 
  end do: 
  sum1;
\end{MapleInput}
\begin{MapleError}
                                      0
                                    35407
\end{MapleError}
\paragraph{課題:上記と同様にして,10桁の2進数を10進数へ変換せよ}


\subsection{255以下の10進数をランダムに生成して,8桁の2進数へ変換せよ.}
\begin{MapleInput}
> n:=8: 2^n;
\end{MapleInput}
\begin{MapleError}
                                     256
\end{MapleError}
\begin{MapleInput}
> roll:=rand(0..255):
  B:=roll();
\end{MapleInput}
\begin{MapleError}
                                     161
\end{MapleError}
ちょっとカンニング.
\begin{MapleInput}
> convert(B,binary);
\end{MapleInput}
\begin{MapleError}
                            10100001
\end{MapleError}

\begin{MapleInput}
> A:=[]:
  for i from 1 to n do
    A:=[irem(B,2),op(A)];
    B:=iquo(B,2);
  end do:
  A;
\end{MapleInput}
\begin{MapleOutput}
[1, 0, 1, 0, 0, 0, 0, 1]
\end{MapleOutput}

\paragraph{課題:8桁の整数のそれぞれの桁の値を配列に格納せよ.}
8桁の整数は以下のようにして作られる.
\begin{MapleInput}
> n:=8; 
  roll:=rand(10^(n-1)..10^n): 
  B:=roll();
\end{MapleInput}
\begin{MapleError}
                                      8
                                   17914675
\end{MapleError}
\subsection{小数点以下8桁のそれぞれの桁の数を配列に格納せよ}
\begin{MapleInput}
> n:=8: 
  roll:=rand(10^(n-1)..10^n): 
  B:=evalf(roll()/10^n);
\end{MapleInput}
\begin{MapleError}
                                 0.6308447100
\end{MapleError}
\begin{MapleInput}
> B:=10*B:
  A:=[]:
  for i from 1 to n do
    A:=[op(A),floor(B)];
    B:=(B-A[i])*10;
  end do:
  A;
\end{MapleInput}
\begin{MapleOutput}
[6, 3, 0, 8, 4, 4, 7, 1]
\end{MapleOutput}

\subsection{最大数}
\begin{MapleInput}
> roll:=rand(1..100): 
  n:=5: 
  A:=[seq(roll(),i=1..n)]; 
  i_max:=A[1]: 
  for i from 2 to n do 
    if (A[i]>i_max) then 
      i_max:=A[i]; 
    end if; 
  end do: 
  i_max;
\end{MapleInput}
\begin{MapleError}
                                      64
\end{MapleError}
\paragraph{課題:最小値を求めよ.}

\subsection{ある値の上下で分けた個数}
\begin{MapleInput}
> roll:=rand(1..100): 
  n:=5: A:=[seq(roll(),i=1..n)];
  i_div:=50:i_low:=0:i_high:=0: 
  for i from 1 to n do 
    if (A[i]>i_div) then
      i_high:=i_high+1; 
    else 
      i_low:=i_low+1; 
    end if 
  end do; 
  print(i_low,i_high);
\end{MapleInput}
\begin{MapleError}
                                     2, 3
\end{MapleError}

\subsection{素数かどうかの判定}
\begin{MapleInput}
> n:=10; 
  for i from 1 to n do 
    if (isprime(i)) then 
      print(i); 
    end if; 
  end do;
\end{MapleInput}

\subsection{2つの要素の入れ替え}
\begin{MapleInput}
> roll:=rand(1..100): n:=5: A:=[seq(roll(),i=1..n)]; sel:=rand(1..n):
  isel:=sel(); 
  jsel:=sel(); 
  a:=A[isel]; b:=A[jsel]; A[isel]:=b; A[jsel]:=a; 
  A;
\end{MapleInput}
\begin{MapleError}
                      [60, 93, 14, 50, 47]
                               2
                               4
                               93
                               50
                               50
                               93
                      [60, 50, 14, 93, 47]
\end{MapleError}
より短くするには,
\begin{MapleInput}
> roll:=rand(1..100):
  n:=5:
  A:=[seq(roll(),i=1..n)];
  sel:=rand(1..n):
  isel:=sel();
  jsel:=sel();
  a:=A[isel];
  A[isel]:=A[jsel];
  A[jsel]:=a;
  A;
\end{MapleInput}
\begin{MapleError}
                       [9, 77, 59, 16, 1]
                               5
                               4
                               1
                               16
                               1
                       [9, 77, 59, 1, 16]
\end{MapleError}

\subsection{コインの表向きの枚数}
\begin{MapleInput}
> roll:=rand(0..1):
  n:=10:
  up:=0:
  for i from 1 to n do
    trial:=roll();
    if (trial=1) then 
      up:=up+1;
    end if;
  end do:
  up;
\end{MapleInput}
\begin{MapleError}
                                      5
\end{MapleError}

\paragraph{課題:1..6のサイコロを20回振って,出た目を記録せよ.}
記録には,要素が0の配列を最初に用意し,出た目を示数にして配列の要素をひとつずつ増やす.
\subsection{2次元配列}
2次元配列に対しても同様の操作ができる.ここでは列に対する規格化を示す.
\begin{MapleInput}
> roll:=rand(1..5):
  n:=3:
  A:=[seq([seq(roll(),i=1..n)],j=1..n)];
\end{MapleInput}
\begin{MapleError}
                     A := [[5, 2, 2], [2, 3, 2], [4, 2, 1]]
\end{MapleError}
\begin{MapleInput}
> roll:=rand(1..5):
  n:=3:
  A:=[seq([seq(roll(),i=1..n)],j=1..n)];
\end{MapleInput}
\begin{MapleError}
                            1, 1, 5
                            1, 2, 2
                            1, 3, 2
                            2, 1, 2
                            2, 2, 3
                            2, 3, 2
                            3, 1, 4
                            3, 2, 2
                            3, 3, 1
\end{MapleError}
i,jの順序に注意.
\begin{MapleInput}
> for j from 1 to n do
    tmp:=0;
    for i from 1 to n do
      tmp:=tmp+A[i,j];
    end do;
    for i from 1 to n do
      A[i,j]:=A[i,j]/tmp;
    end do;
  end do:
  A;
\end{MapleInput}
\begin{MapleError}
         [[5/11, 2/7, 2/5], [2/11, 3/7, 2/5], [4/11, 2/7, 1/5]]
\end{MapleError}



\end{document}
