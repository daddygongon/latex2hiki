% TexShopではいまのところうまく通らず.2016/04/08
% teminalでdvipdfmx *.dviで変換.
\documentclass[10pt,a4j]{jbook}
\usepackage[dvipdfmx]{graphicx,color}

\usepackage[dvipdfmx,colorlinks,linkcolor=blue,bookmarksopenlevel=4]{hyperref}
% texで日本語のしおり付きpdfを作る方法
%% pdf の しおり をEUCから unicode へ変換
\AtBeginDvi{\special{pdf:tounicode EUC-UCS2}}

\usepackage{tabularx}
\usepackage{verbatim}
\usepackage{amsmath,amsthm,amssymb}
\usepackage{bm}
\topmargin -15mm\oddsidemargin -4mm\evensidemargin\oddsidemargin
\textwidth 170mm\textheight 257mm\columnsep 7mm
\setlength{\fboxrule}{0.2ex}
\setlength{\fboxsep}{0.6ex}

%\pagestyle{empty}

\newcommand{\MaplePlot}[2]{{\begin{center}
    \includegraphics[width=#1,clip]{#2}
                     \end{center}
%
} }

\newenvironment{MapleInput}{%
    \color{red}\verbatim
}{%
    \endverbatim
}

\newenvironment{MapleError}{%
    \color{blue}\verbatim
}{%
    \endverbatim
}

\newenvironment{MapleOutput}{%
    \color{blue}\begin{equation*}
}{%
    \end{equation*}
}

\newenvironment{MapleOutputGather}{%
    \color{blue}\gather
}{%
    \endgather
}

\newcommand{\ChartElement}[1]{{
	\color{magenta}\begin{flushleft}$\left[\left[\left[\textbf{\large #1}\right]\right]\right]$
	\end{flushleft}\vspace{-10mm}
} }

\newcommand{\ChartElementTwo}[1]{{
	\color{magenta}\begin{flushleft}$\left[\left[\left[\textbf{\large #1}\right]\right]\right]$
	\end{flushleft}
} }

\newcommand{\ChartElementThree}[2]{{
	\color{magenta}\begin{flushleft}$\left[\left[\left[\textbf{\large #2}\right]\right]\right]$
	\end{flushleft}\vspace{#1}
} }

\newif\ifHIKI
%\HIKItrue % TRUEの設定
\HIKIfalse % FALSEの設定

\begin{document}
\title{Maple入門}
\author{西谷@関西学院大・理工}
\date{\today}
\maketitle
\tableofcontents

\chapter{基本操作}
\section{最初の一葉(FirstLeaf):文法とヘルプとプロット}

\ChartElement{解説}
\subsection{入力領域と注意点(ShiftEnter)}
Mapleを起動すると赤いプロンプトがともっている.ここに命令(コマンド)を打ち込んでMapleの計算部に『こちらの意志』を伝えて動作させる.例えば,
\begin{MapleInput}
> plot(sin(x),x); 
\end{MapleInput}
と入力し,
\ifHIKI
[ Enter ]
\else
\fbox{enter}
\fi
を入力してみよ.sin関数がプロットされる.
\MaplePlot{50mm}{./figures/FirstLeafplot2d1.eps}
\begin{enumerate}
\item 赤い領域のどこにカーソルがあっても
\ifHIKI
[ Enter ]
\else
\fbox{enter}
\fi
を入れれば,そのブロックごとMapleに命令として渡される.テキストでは
\ifHIKI
[ Enter ]
\else
\fbox{enter}
\fi
を省略している.
\item テキストの修正は普通のワープロソフトと同じ.
\item 命令の入力ではなく,\textbf{改行だけをいれたいときはshift+enterを入れる.}
\item 命令は,enterを入れた順に解釈されるのであって,テキストの上下とは関係ない.
\end{enumerate}
 
\subsection{命令コマンドの基本形(command();)}
命令コマンドは全て次のような構造を取る.
\begin{MapleInput}
> command(引数1,引数2,...);
\end{MapleInput}
あるいは
\begin{MapleInput}
> command(引数1,オプション1,オプション2,...);
\end{MapleInput}
となる.

\begin{enumerate}
\item ()の中の引数やオプションの間はコンマで区切る.
\item 最後の;(セミコロン)は次のコマンドとの区切り記号.
\item セミコロン(;)をコロン(:)に替えるとMapleからの返答が出力されなくなるが,Mapleへの入力は行われている.
\item C言語などの手続き型プログラミング言語の標準的なフォーマットと同じ.
\end{enumerate}

命令コマンドを,英語の命令文と解釈してもよい.たとえば,
\begin{MapleError}
sin(x)をxについて0からpiまでplotせよ.
\end{MapleError}
という日本語を英語に訳すと,
\begin{MapleError}
plot sin(x) with x from 0 to Pi.
\end{MapleError}
となる.この英語をMaple語に訳して
\begin{MapleInput}
> plot(sin(x),x=0..Pi);
\end{MapleInput}
となったとみなせる.英語文法のVerb (動詞), Object (目的語)を当てはめると,Mapleへの命令は,
\begin{MapleInput}
> Verb(Object, その他の修飾);
\end{MapleInput}
である.英文でピリオドを忘れるなと中学時代に言われたのと同じく,Mapleでセミコロンを忘れぬように.

 
\subsection{ヘルプ(?)}
ヘルプは少し違った構文で,例えば先程のコマンドplotのヘルプを参照するときには,
\begin{MapleInput}
> ?plot;
\end{MapleInput}
である.

ヘルプ画面は,左側に操作アイコン,検索ウィンドウ,関連リストが表示され,右側にヘルプの本文がある.本文は,簡単な意味と使い方,説明,例,参照で構成される.ほとん
どが日本語に訳されているが,古いテキストやあまり使わないコマンドは英語のまま.英語が分からなくても例を参考にすればだいたい予測できる.と言うより,日本語訳を読ん
でも初めはチンプンカンプン.Mapleコマンドのコンセプトに慣れるまでは使用例をまねるのが一番の早道でしょう.
 

\ChartElement{課題}
\subsubsection{plotに関する課題}
\begin{enumerate}
\item plotに関するhelpを開けよ.
\item 例をいくつかcopyして実行せよ.円をパラメータプロットする方法を確認せよ.
\item 2つの関数,sin(x)とcos(x),をx=-Pi..Piで同時にプロットせよ.
\item 上記の2つの関数の表示で,オプションにcolor=[red,blue]およびstyle=[point,line]を加えて,結果を観察せよ.plot[options]に関するhelpを開け,そのほかのオプションを試してみよ.
\end{enumerate}
 
\subsubsection{plot3dに関する課題}
\begin{enumerate} 
\item x,yの2次元平面を定義域とする関数のプロットには,3次元で表示するコマンドplot3dが必要となる.2変数関数sin(x)*cos(y)をx=-Pi..Pi,
y=-Pi..Piでプロットせよ.
\item プロットをつまんで回してみよ.また,メニューバーにある表面あるいは軸のアイコンを操作して,等高線にしたり,軸を変更してみよ.
\end{enumerate}
 

\ChartElement{解答例}
\subsubsection{plot}
\begin{enumerate}
\item
\begin{MapleInput}
> ?plot;
\end{MapleInput}

\begin{MapleInput}
> plot(cos(x)+sin(x), x = 0 .. Pi);
\end{MapleInput}
\MaplePlot{30mm}{./figures/FirstLeafplot2d4.eps}
\item
\begin{MapleInput}
> plot([sin(t), cos(t), t = -Pi .. Pi]);
\end{MapleInput}
\MaplePlot{30mm}{./figures/FirstLeafplot2d5.eps}
\item
\begin{MapleInput}
> plot([sin(x),cos(x)],x=-Pi..Pi);
\end{MapleInput}
\MaplePlot{30mm}{./figures/FirstLeafplot2d6.eps}
\item
\begin{MapleInput}
> plot([sin(x),cos(x)],x=-Pi..Pi,color=[red,blue],style=[point,line]);
\end{MapleInput}
\MaplePlot{30mm}{./figures/FirstLeafplot2d7.eps}
\end{enumerate} 
\subsubsection{plot3d}
plot図に対するいくつかの操作は,plotを選んだときにwindowの上部に表示されるリボンのアイコンにあります.同じ操作は,plot3dに与えるoptionsによっても可能です.以下でいくつかのoptionを示します.詳しくは\texttt{?plot3d[option];}を参照.

\begin{enumerate}
\item
\begin{MapleInput}
> plot3d(sin(x)*cos(y),x=-Pi..Pi,y=-Pi..Pi);}
\end{MapleInput}
\MaplePlot{30mm}{./figures/FirstLeafplot3d8.eps}
\item
等高線は
\begin{MapleInput}
> plot3d(sin(x)*cos(y),x=-Pi..Pi,y=-Pi..Pi,style=contour);
\end{MapleInput}
\MaplePlot{30mm}{./figures/FirstLeafplot3d9.eps}

軸の変更は
\begin{MapleInput}
> plot3d(sin(x)*cos(y),x=-Pi..Pi,y=-Pi..Pi,axes=boxed);
\end{MapleInput}
\MaplePlot{30mm}{./figures/FirstLeafplot3d10.eps}

\item
デフォルトの角度もplot3dのoptionで変更することが可能です.
\begin{MapleInput}
> plot3d(sin(x)*cos(y),x=-Pi..Pi,y=-Pi..Pi,orientation=[45,80]);
\end{MapleInput}
\MaplePlot{30mm}{./figures/FirstLeafplot3d11.eps}
\end{enumerate} 

\ChartElement{補足}
\subsubsection{間違い(Error)}
いくつかの典型的な間違い.
先ずは,左右の括弧の数が合ってないとき.
\begin{MapleInput}
> plot(sin(x,x=-Pi..Pi);
\end{MapleInput}
\begin{MapleError}
Error, `;` unexpected
\end{MapleError}
正しくは,
\begin{MapleInput}
> plot(sin(x),x=-Pi..Pi);
\end{MapleInput}
です.関数の中に変数が残ったままplotしようとしたとき.
\begin{MapleInput}
> plot(sin(a*x),x=-Pi..Pi);
\end{MapleInput}
\begin{MapleError}
Warning, unable to evaluate the function to numeric values in the region; see
the plotting command's help page to ensure the calling sequence is correct
\end{MapleError}
\MaplePlot{30mm}{./figures/FirstLeafplot2d2.eps}
何も表示されない.xに1を代入しても,sin(a*x)からは数値ではなく記号で答えが返って来ている.plotは関数が数値を返したときしか表示できない.以下のように,変数aのかわりに数値を入れる.
\begin{MapleInput}
> plot(sin(2*x),x=-Pi..Pi);
\end{MapleInput}
\MaplePlot{50mm}{./figures/FirstLeafplot2d3.eps}


 

\section{初等関数とそのほかの関数(Functions)}
\ChartElement{解説}
\subsection{初等関数(ElementaryFunctions)}
\paragraph{四則演算とevalf}
四則演算は"+-*/".割り切れない割り算は分数のまま表示される.
\begin{MapleInput}
> 3/4;
\end{MapleInput}
\begin{MapleOutput}
\frac{3}{4}
\end{MapleOutput}
強制的に数値(浮動小数点数)で出力するにはevalfを用いる.
\begin{MapleInput}
> evalf(3/4);
\end{MapleInput}
\begin{MapleOutput}
0.7500000000
\end{MapleOutput}
\paragraph{多項式関数(polynom)}
かけ算も省略せずに打ち込む必要がある.またベキ乗は\verb="^"=である.
\begin{MapleInput}
> 3*x^2-4*x+3;
\end{MapleInput}
\begin{MapleOutput}
3x^2-4x+3
\end{MapleOutput}
\paragraph{平方根(sqrt)}
平方根はsquare rootを略したsqrtを使う.
\begin{MapleInput}
> sqrt(2);
\end{MapleInput}
\begin{MapleOutput}
\sqrt{2}
\end{MapleOutput}

\paragraph{三角関数(trigonal)}
sin, cosなどの三角関数はラジアンで入力する.ただし, $\sin^2x$などは
\begin{MapleInput}
> sin^2 x;
\end{MapleInput}
\begin{MapleError}
Error, missing operator or `;`
\end{MapleError}
ではだめで,
\begin{MapleInput}
> sin(x)^2;
\end{MapleInput}
\begin{MapleOutput}
\sin^2x
\end{MapleOutput}
と省略せずに打ち込まねばならない.三角関数でよく使う定数$\pi$は"Pi"と入力する.Mapleは大文字と小文字を区別するので注意.

ラジアン(radian)に度(degree)から変換するには以下のようにする.
\begin{MapleInput}
> convert(90*degrees, radians);
  convert(1/6*Pi,degrees);
\end{MapleInput}
\begin{MapleOutput}
\frac{1}{2}\pi
\end{MapleOutput}
\begin{MapleOutput}
30\,\, degrees
\end{MapleOutput}
\paragraph{その他の関数(inifnc)}
その他の初等関数やよく使われる超越関数など,Mapleの起動時に用意されている関数のリストは,
\begin{MapleInput}
> ?inifnc;
\end{MapleInput}
で得られる.
 
\subsection{ユーザー定義関数(unapply)}
初等関数やその他の関数を組み合わせてユーザー定義関数を作ることができる.

関数$f(x) = 2 x - 3$とおくとする場合,Mapleでは,
\begin{MapleInput}
> f:=x->2*x-3;
\end{MapleInput}
\begin{MapleOutput}
f:= x \rightarrow 2 x - 3
\end{MapleOutput}
と,矢印で書く.これが関数としてちゃんと定義されているかは,いくつかの数値や変数を$f(x)$に代入して確認する.
\begin{MapleInput}
> f(3);               #res: 3 (以降出力を省略する場合はこのように表記)
  f(a);               #res: 2 a - 3
  plot(f(x),x=-2..2);
\end{MapleInput}
\MaplePlot{50mm}{./figures/Functionsplot2d1.eps}
もう一つ関数定義のコマンドとして次のunapplyも同じ意味である.
\begin{MapleInput}
> f:=unapply(2*x-3,x);
\end{MapleInput}
\begin{MapleOutput}
f:= x \rightarrow 2 x - 3
\end{MapleOutput}
ただし,矢印での定義ではときどき変な振る舞いになるので,unapplyを常に使うようにこころがけたほうが安全.

 
\subsection{packageの呼び出し(with)}
Mapleが提供する膨大な数の関数から,目的とするものを捜し出すにはhelpを使う.普段は使わない関数は,使う前に明示的に呼び出す必要がある.例えば,線形代数によく使われる関数群は,
\begin{MapleInput}
> with(LinearAlgebra):
\end{MapleInput}
としておく必要がある.この他にもいくつもの有益な関数パッケージが用意されている.
\begin{MapleInput}
> ?index[package];
\end{MapleInput}
で用意されているすべてのpackageが表示される.

 

\ChartElement{課題}
\subsubsection{関数についての課題}
\begin{enumerate}
\item evalfのヘルプを参照して,Piを1000桁まで表示せよ.
\item 正接関数(tan)とその逆関数arctanをx=-Pi/2..Pi/2,y=-Pi..Pi,scaling=constrainedで同時にプロットせよ.
\item 対数関数はln(x)で与えられる.ヘルプを参照しながら次の値を求めよ.
\begin{equation*}
\log_{10}1000, \log_{2}\frac{1}{16}, \log_{\sqrt{5}}125
\end{equation*}
\item 次の関数は$y=2^x$とどのような位置関係にあるかx=-5..5,y=-5..5で同時にプロットして観察せよ.
\begin{equation*}
y = -2^x, y = (1/2)^x,y = -(1/2)^x
\end{equation*}
\item 指数関数はexpで与えられる.$\mbox{e}^x$と$\log x$関数および$y = x$を同時にx=-5..5,y=-5..5でplotせよ.またそれらの関数の位置関係を述べよ.
\item 階乗関数factorialに3を代入して何を求める関数か予測せよ.ヘルプを参照し,よりなじみの深い表記を試してみよ.
\end{enumerate}
 

\ChartElement{解答例}
\subsubsection{Functions}
\begin{enumerate}
\item evalfのヘルプを参照して,Piを1000桁まで表示せよ.
\begin{MapleInput}
> ?evalf;
> evalf[1000](Pi);      #省略
\end{MapleInput}

\item 正接関数(tan)とその逆関数arctanをx=-Pi/2..Pi/2,y=-Pi..Pi,scaling=constrainedで同時にプロットせよ.
\begin{MapleInput}
> plot([tan(x),arctan(x)],x=-Pi/2..Pi/2,y=-Pi..Pi,scaling=constrained);
\end{MapleInput}
\MaplePlot{50mm}{./figures/Functionsplot2d2.eps}

\item 対数関数はln(x)で与えられる.ヘルプを参照しながら次の値を求めよ.
\begin{equation*}
\log_{10}1000, \log_{2}\frac{1}{16}, \log_{\sqrt{5}}125
\end{equation*}

\begin{MapleInput}
> ?ln;
\end{MapleInput}

\begin{MapleInput}
> log10(1000);
\end{MapleInput}
\begin{MapleOutput}
3
\end{MapleOutput}
\begin{MapleInput}
> log[2](1/16);
\end{MapleInput}
\begin{MapleOutput}
-4
\end{MapleOutput}
\begin{MapleInput}
> log[sqrt(5)](125);
\end{MapleInput}
\begin{MapleOutput}
6
\end{MapleOutput}

\item 次の関数は$y=2^x$とどのような位置関係にあるかx=-5..5,y=-5..5で同時にプロットして観察せよ.
\begin{equation*}
y = -2^x, y = (1/2)^x,y = -(1/2)^x
\end{equation*}
\begin{MapleInput}
> plot([2^x,-2^x,(1/2)^x,-(1/2)^x],x=-5..5,y=-5..5);
\end{MapleInput}
注目している関数以外を消せばはっきりするが,i) x軸に対称,ii) y軸に対称, iii) 原点に対称.
\MaplePlot{50mm}{./figures/Functionsplot2d3.eps}

\item 指数関数はexpで与えられる.$\mbox{e}^x$と$\log x$関数および$y = x$を同時にx=-5..5,y=-5..5でplotせよ.またそれらの関数の位置関係を述べよ.
\begin{MapleInput}
> plot([exp(x),log(x),x],x=-5..5,y=-5..5);
\end{MapleInput}
\MaplePlot{30mm}{./figures/Functionsplot2d4.eps}

\item 階乗関数factorialに3を代入して何を求める関数か予測せよ.ヘルプを参照し,よりなじみの深い表記を試してみよ.
\begin{MapleInput}
> factorial(3)
\end{MapleInput}
\begin{MapleOutput}
6
\end{MapleOutput}
\begin{MapleInput}
> ?factorial;
> 3!;
\end{MapleInput}

\end{enumerate}
 

\section{等号(Equals)}
\ChartElementThree{-6mm}{基本事項}
\paragraph{等号の意味}
等号は,数学でいろいろな意味を持つことを中学校で学ぶ.それぞれの状況による意味の違いを人間は適当に判断できるが,プログラムであるMapleでは無理.Mapleでは,それぞれ違った記号や操作として用意され,人間がMapleに指示する必要がある.

\subsection{変数への代入:= (colonequal)}
変数に値を代入する時には:= (colonequal)を使う.例えば,
\begin{MapleError}
a=3, b=2のとき,a+bはいくらか?
\end{MapleError}
という問題を,Mapleで解かす時には,
\begin{MapleError}
aに3, bに2を代入したとき,a+bはいくらか?
\end{MapleError}
と読み直し,
\begin{MapleInput}
> a:=3; #res: 3
> b:=2; #res: 2
> a+b;  #res: 5
\end{MapleInput}
式の定義も同様.以下は$ax+b=cx^2+dx+e$という式をeq1と定義している.
\begin{MapleInput}
> eq1:=a*x+b=c*x^2+d*x+e;
\end{MapleInput}
\begin{MapleOutput}
3x+2=cx^2+dx+e
\end{MapleOutput}
a,bに値が代入されていることに注意.

\subsection{変数の初期化(restart)}
一度何かを代入した変数を何も入れていない状態に戻す操作を変数の初期化という.すべての変数を一度に初期化するには,
\begin{MapleInput}
> restart;
\end{MapleInput}
とする.なにか新たなひとまとまりの作業をするときには,このコマンドを冒頭に入れることを習慣づけるように.

作業の途中でひとつの変数だけを初期化するには,シングルクォート’でくくる.
\begin{MapleInput}
> a:='a';
\end{MapleInput}
\begin{MapleOutput}
a
\end{MapleOutput}
一時的代入にsubsがある. 
\subsection{方程式の解(solve)}
\begin{MapleError}
3x=2を満たすxをもとめよ.
\end{MapleError}
という問題は,
\begin{MapleInput}
> solve(3*x=2,x);
\end{MapleInput}
\begin{MapleOutput}
\frac{2}{3}
\end{MapleOutput}
連立方程式は以下のとおり.
\begin{MapleInput}
> solve({x+y=1,x-y=2},{x,y});
\end{MapleInput}
\begin{MapleOutput}
\left\{x = \frac{3}{2}, y = -\frac{1}{2} \right\}
\end{MapleOutput}
ただし,solveだけでは,x,yに値は代入されない.
\begin{MapleInput}
> sol1:=solve({x+y=1,x-y=2},{x,y});
> assign(sol1);
\end{MapleInput}
\begin{MapleOutput}
sol1 := \left\{x = \frac{3}{2}, y = -\frac{1}{2} \right\}
\end{MapleOutput}
とする必要がある.確認してみると
\begin{MapleInput}
> x,y;
\end{MapleInput}
\begin{MapleOutput}
\frac{3}{2},-\frac{1}{2}
\end{MapleOutput}
となり,値が代入されていることがわかる.

\paragraph{方程式の数値解(fsolve)}
解析的に解けない場合は,数値的に解を求めるfsolveを使う.上でxにassignしているので,xを初期化している.
\begin{MapleInput}
> x:='x';
> fsolve(log(x)-exp(-x),x);
\end{MapleInput}
\begin{MapleOutputGather}
x := x \notag \\
1.309799586 \notag
\end{MapleOutputGather}

 
\subsection{恒等式(Identity)}
式の変形にも等号が使われる.例えば,
\begin{equation*}
(x-2)^2=x^2-4x+4
\end{equation*}
というのが等号で結ばれている.式の変形とは,変数$x$がどんな値であっても成り立つ恒等的な変形である.

この式変形も,問題としては,
\begin{MapleError}
(x-2)^2を展開(expand)せよ
\end{MapleError}
と与えられるので,そのままMapleコマンドに読み替えて
\begin{MapleInput}
> expand( (x-2)^2 );
\end{MapleInput}
\begin{MapleOutput}
x^2-4x+4
\end{MapleOutput}
とすればよい.因数分解(factor)や微分(diff)・積分(int)も同様に等号で結ばれるが,Mapleには操作を指示する必要がある.詳しくは他の単元で.
 
\ChartElement{課題}
\subsubsection{等号についての課題}
\begin{enumerate}
\item a=3, b=4としてa,bの四則演算をおこなえ.また,べき乗$a^b$を求めよ.
\item eq1=3*x+4=2*x-2, a=2とした場合のeq1/a,eq1+aを試し,両辺を観察せよ.
\item 3点(1,2),(-3,4),(-1,1)を通る2次方程式を求めよ.
\item 方程式$\sin(x+1)-x^2=0$の2つの解をfsolveのヘルプを参照して求めよ.
\end{enumerate}
 
%\item 関数$y=\exp(x)$の逆関数を求めよ.2つの関数およびy=xを同時にx=-5..5,y=-5..5でplotし,y=xに対して2つの関数が対称であることを確認せよ.

先ず,与関数をf(x)として定義しておく.
\begin{MapleInput}
> f:=x->exp(x);
\end{MapleInput}
\begin{MapleOutput}
f\, := \,x\mapsto {\exp(x)}
\end{MapleOutput}
y=f(x)として,xについて解いてみる.
\begin{MapleInput}
> eq1:=y=f(x);
> solve(eq1,x);
\end{MapleInput}
\begin{MapleOutputGather}
 {\it eq1}\, := \,y=exp(x)  \notag \\
 \ln  \left( y \right) \notag
\end{MapleOutputGather}
うまくyの関数として解けている.これをea2と定義し直して,逆関数として定義する.
\begin{MapleInput}
> eq2:=solve(eq1,x);
> invf:=unapply(eq2,y);
\end{MapleInput}
\begin{MapleOutputGather}
 {\it eq2}\, := \,\ln  \left( y \right)  \notag\\
 {\it invf}\, := \,y\mapsto \ln  \left( y \right) \notag
\end{MapleOutputGather} 
同時にプロットしてみる.
\begin{MapleInput}
> plot([f(x),invf(x),x],x=-5..5,y=-5..5);
\end{MapleInput}
\MaplePlot{30mm}{./figures/Equalsplot2d2.eps}
y=xを軸として,2つの関数が対称であることが確認できる.

 
\ChartElement{解答例}
\subsubsection{Equals}
\begin{enumerate}
\item a=3, b=4としてa,bの四則演算をおこなえ.また,べき乗$a^b$を求めよ.
\begin{MapleInput}
> a:=3:
> b:=4:
> a+b;a-b;a*b;a/b;a^b;      #省略
\end{MapleInput}

\item eq1=3*x+4=2*x-2, a=2とした場合のeq1/a,eq1+aを試し,両辺を観察せよ.
\begin{MapleInput}
> eq1:=3*x+4=2*x-2;
> a:=2;
> eq1/a;
> eq1+2;
\end{MapleInput}
\begin{MapleOutputGather}
eq1:=3x+4=2x-2 \notag \\
a:=2 \notag \\
\frac{3}{2}x+2 = x-1 \notag \\
3x+6=2x
\end{MapleOutputGather}

\item 3点(1,2),(-3,4),(-1,1)を通る2次方程式を求めよ.

まず2次関数を定義する.
\begin{MapleInput}
> restart;
> f:=x->a*x^2+b*x+c;
\end{MapleInput}
\begin{MapleOutput}
f := x \mapsto ax^2+bx+c
\end{MapleOutput}
(1,2)を通ることから,f(1)=2が成立.これをeq1として保存.
\begin{MapleInput}
> eq1:=f(1)=2;
\end{MapleInput}
\begin{MapleOutput}
eq1 := a+b+c=2
\end{MapleOutput}
他の点も同様に
\begin{MapleInput}
> eq2:=f(-3)=4;
> eq3:=f(-1)=1;
\end{MapleInput}
\begin{MapleOutputGather}
eq2\, := \,9\,a-3\,b+c=4 \notag\\
eq3\, := \,a-b+c=1 \notag
\end{MapleOutputGather}

この3個の連立方程式から,a,b,cを求めれば解となる.
\begin{MapleInput}
> solve({eq1,eq2,eq3},{a,b,c});
\end{MapleInput}
\begin{MapleOutput}
\left\{ a=1/2,b=1/2,c=1 \right\}
\end{MapleOutput}

\item 方程式$\sin(x+1)-x^2=0$の2つの解をfsolveのヘルプを参照して求めよ.

まず,2つの関数とみなしてプロット.
\begin{MapleInput}
> plot([sin(x+1),x^2],x=-1..1);
\end{MapleInput}
\MaplePlot{30mm}{./figures/Equalsplot2d1.eps}
解が2つあることに注意.与えられた関数値が0となる方程式として定義し,これをsolveでとく.
\begin{MapleInput}
> eq1:=sin(x+1)-x^2=0;
> solve(eq1,x);
\end{MapleInput}
\begin{MapleOutputGather}
eq1 := \,\sin \left( x+1 \right) -{x}^{2}=0 \notag\\
-1+{\it RootOf} \left( -\sin \left( {\it \_Z} \right) +1-2\,{\it \_Z}+{{\it \_Z}}^{2} \right) \notag
\end{MapleOutputGather}
これでは解を求めてくれないので,fsolveで数値解を求める.
\begin{MapleInput}
> fsolve(eq1,x);
\end{MapleInput}
\begin{MapleOutput}
0.9615690350
\end{MapleOutput}
これではxの負にあるもう一つの解がでない.これを解決するには,fsolveでxに初期値を入れて実行する.
\begin{MapleInput}
> fsolve(eq1,x=-1..0);
\end{MapleInput}
\begin{MapleOutput}
-.6137631294
\end{MapleOutput}

\end{enumerate}
 

\chapter{微積分}
\section{微分(Diff)-I}
\ChartElement{解説}
\subsection{\textbf{�ۑ�}}
\subsubsection{\textbf{\textit{1. ���̊֐����������D}}}
\begin{maplegroup}
\begin{Maple Normal}{
i)\mapleinline{inert}{2d}{x*log*x}{$\displaystyle {x}^{2} \left( {\it log} \right) $}
�Cii)\mapleinline{inert}{2d}{1/(1+x)^3}{$\displaystyle  \left( 1+x \right) ^{-3}$}
�Ciii)\mapleinline{inert}{2d}{sqrt(4*x+3)}{$\displaystyle  \sqrt{4\,x+3}$}
�Civ)\mapleinline{inert}{2d}{1/(a^2+(x-x[0])^2)}{$\displaystyle  \left( {a}^{2}+ \left( x-x_{{0}} \right) ^{2} \right) ^{-1}$}
}\end{Maple Normal}

\end{maplegroup}
\subsubsection{\textbf{\textit{2. ���̊֐���1������5�����֐������߂�D}}}
\begin{maplegroup}
\begin{Maple Normal}{
i)\mapleinline{inert}{2d}{sin^2*x}{$\displaystyle {\sin}^{2}x$}
�Cii)\mapleinline{inert}{2d}{e^x}{$\displaystyle {e}^{x}$}
}\end{Maple Normal}

\end{maplegroup}
\subsubsection{\textbf{\textit{3. ���̊֐��Ƃ���1�����֐��𓯎��Ƀv���b�g���T�`���m�F���C����ɑ����\�����߂�D}}}
\begin{maplegroup}
\begin{Maple Normal}{
\mapleinline{inert}{2d}{x/(x^2-2*x+4)}{\[\displaystyle {\frac {x}{{x}^{2}-2\,x+4}}\]}
}\end{Maple Normal}
\end{maplegroup}
\subsubsection{\textbf{\textit{4.���̊֐���}}\mapleinline{inert}{2d}{x = 3}{$\displaystyle x=3$}
\textbf{\textit{�ł̐ڐ������߁C2�‚̊֐��𓯎��Ƀv���b�g����D}}}
\begin{maplegroup}
\begin{Maple Normal}{
\mapleinline{inert}{2d}{y = x^3-2*x^2-35*x}{\[\displaystyle y={x}^{3}-2\,{x}^{2}-35\,x\]}
}\end{Maple Normal}
\end{maplegroup}
\subsubsection{\textbf{\textit{5.�ȉ��̊֐���x0�܂��łR���܂Ńe�C���[�W�J���C����ꂽ�֐��Ƃ��Ƃ̊֐����v���b�g����D�����5���܂œW�J�����ꍇ�͂ǂ��ω����邩�D}}}
\begin{maplegroup}
\begin{Maple Normal}{
i) y=sin(x), x0=0�Cii) y=cos(x), x0=Pi/2}\end{Maple Normal}

\end{maplegroup}
\subsubsection{\textbf{\textit{6.(���W�ۑ�j}}\mapleinline{inert}{2d}{f(x, y) = e^x*log(1+y)}{$\displaystyle f \left( x,y \right) ={e}^{x}{\it log} \left( 1+y \right) $}
\textbf{\textit{��}}\mapleinline{inert}{2d}{x = 0, y = 0}{$\displaystyle x=0,\,y=0$}
\textbf{\textit{�̂܂���3���܂œW�J����D}}}
\begin{maplegroup}
\newpage
\end{maplegroup}
\subsection{\textbf{�𓚗�}}
\subsubsection{\textbf{\textit{1.}}}
\begin{maplegroup}
\begin{mapleinput}
\mapleinline{active}{1d}{diff(x*log(x),x);
diff(1/(1 + x)\symbol{94}3,x);
diff(sqrt(4*x + 3),x);
diff(1/(a\symbol{94}2+(x-x0)\symbol{94}2),x);
}{}
\end{mapleinput}
\mapleresult
\begin{maplelatex}
\mapleinline{inert}{2d}{ln(x)+1}{\[\displaystyle \ln  \left( x \right) +1\]}
\end{maplelatex}
\mapleresult
\begin{maplelatex}
\mapleinline{inert}{2d}{-3/(1+x)^4}{\[\displaystyle -3\, \left( 1+x \right) ^{-4}\]}
\end{maplelatex}
\mapleresult
\begin{maplelatex}
\mapleinline{inert}{2d}{2/sqrt(4*x+3)}{\[\displaystyle 2\, \left(  \sqrt{4\,x+3} \right) ^{-1}\]}
\end{maplelatex}
\mapleresult
\begin{maplelatex}
\mapleinline{inert}{2d}{-(2*x-2*x0)/(a^2+(x-x0)^2)^2}{\[\displaystyle -{\frac {2\,x-2\,{\it x0}}{ \left( {a}^{2}+ \left( x-{\it x0} \right) ^{2} \right) ^{2}}}\]}
\end{maplelatex}
\end{maplegroup}
\subsubsection{\textbf{\textit{2.}}}
\begin{maplegroup}
\begin{mapleinput}
\mapleinline{active}{1d}{diff(sin(x)\symbol{94}2,x);
diff(sin(x)\symbol{94}2,x$2);}{}
\end{mapleinput}
\mapleresult
\begin{maplelatex}
\mapleinline{inert}{2d}{2*sin(x)*cos(x)}{\[\displaystyle 2\,\sin \left( x \right) \cos \left( x \right) \]}
\end{maplelatex}
\mapleresult
\begin{maplelatex}
\mapleinline{inert}{2d}{2*cos(x)^2-2*sin(x)^2}{\[\displaystyle 2\, \left( \cos \left( x \right)  \right) ^{2}-2\, \left( \sin \left( x \right)  \right) ^{2}\]}
\end{maplelatex}
\end{maplegroup}
\begin{maplegroup}
\begin{mapleinput}
\mapleinline{active}{1d}{diff(exp(x),x);
}{}
\end{mapleinput}
\mapleresult
\begin{maplelatex}
\mapleinline{inert}{2d}{exp(x)}{\[\displaystyle {{\rm e}^{x}}\]}
\end{maplelatex}
\end{maplegroup}
\begin{maplegroup}
\begin{Maple Normal}{
�ȉ���}\end{Maple Normal}

\end{maplegroup}
\subsubsection{\textbf{\textit{3.}}}
\begin{maplegroup}
\begin{mapleinput}
\mapleinline{active}{1d}{f0:=unapply(x/(x\symbol{94}2-2*x+4),x):
df:=unapply(diff(f0(x),x),x);
plot([f0(x),df(x)],x);}{}
\end{mapleinput}
\mapleresult
\begin{maplelatex}
\mapleinline{inert}{2d}{df := proc (x) options operator, arrow; 1/(x^2-2*x+4)-x*(2*x-2)/(x^2-2*x+4)^2 end proc}{\[\displaystyle {\it df}\, := \,x\mapsto  \left( {x}^{2}-2\,x+4 \right) ^{-1}-{\frac {x \left( 2\,x-2 \right) }{ \left( {x}^{2}-2\,x+4 \right) ^{2}}}\]}
\end{maplelatex}
\mapleresult
\mapleplot{Diffplot2d1.eps}
\end{maplegroup}
\subsubsection{\textbf{\textit{4.}}}
\begin{maplegroup}
\begin{Maple Normal}{
�^�֐���f0�ƒ�`�D}\end{Maple Normal}

\textbf{f0:=unapply(x\symbol{94}3  - 2*x\symbol{94}2  - 35*x,x);}\mapleresult
\begin{maplelatex}
\mapleinline{inert}{2d}{f0 := proc (x) options operator, arrow; x^3-2*x^2-35*x end proc}{\[\displaystyle {\it f0}\, := \,x\mapsto {x}^{3}-2\,{x}^{2}-35\,x\]}
\end{maplelatex}
\end{maplegroup}
\begin{maplegroup}
\begin{Maple Normal}{
�����֐���df�ƒ�`}\end{Maple Normal}

\textbf{df:=unapply(diff(f0(x),x),x);}\mapleresult
\begin{maplelatex}
\mapleinline{inert}{2d}{df := proc (x) options operator, arrow; 3*x^2-4*x-35 end proc}{\[\displaystyle {\it df}\, := \,x\mapsto 3\,{x}^{2}-4\,x-35\]}
\end{maplelatex}
\end{maplegroup}
\begin{maplegroup}
\begin{Maple Normal}{
�ړ_(x0,f0(x0))�ŌX��df(x0)�̒�����f1�ƒ�`�D}\end{Maple Normal}

\textbf{x0:=3;}
\textbf{eq1:=df(x0)*(x-x0)+f0(x0);}
\textbf{f1:=unapply(eq1,x);}\mapleresult
\begin{maplelatex}
\mapleinline{inert}{2d}{x0 := 3}{\[\displaystyle {\it x0}\, := \,3\]}
\end{maplelatex}
\mapleresult
\begin{maplelatex}
\mapleinline{inert}{2d}{eq1 := -20*x-36}{\[\displaystyle {\it eq1}\, := \,-20\,x-36\]}
\end{maplelatex}
\mapleresult
\begin{maplelatex}
\mapleinline{inert}{2d}{f1 := proc (x) options operator, arrow; -20*x-36 end proc}{\[\displaystyle {\it f1}\, := \,x\mapsto -20\,x-36\]}
\end{maplelatex}
\end{maplegroup}
\begin{maplegroup}
\begin{Maple Normal}{
2�‚̊֐��𓯎��Ƀv���b�g�D}\end{Maple Normal}

\textbf{plot([f0(x),f1(x)],x=-5..5);}\mapleresult
\mapleplot{Diffplot2d2.eps}
\end{maplegroup}
\begin{maplegroup}
\end{maplegroup}
\subsubsection{\textbf{\textit{5. i)}}}
\begin{maplegroup}
\begin{Maple Normal}{
�悸�^�֐���f0�ƒ�`}\end{Maple Normal}

\textbf{f0:=unapply(sin(x),x);}\mapleresult
\begin{maplelatex}
\mapleinline{inert}{2d}{f0 := proc (x) options operator, arrow; sin(x) end proc}{\[\displaystyle {\it f0}\, := \,x\mapsto \sin \left( x \right) \]}
\end{maplelatex}
\end{maplegroup}
\begin{maplegroup}
\begin{Maple Normal}{
�e�C���[�W�J�������ʂ�eq1�Ƃ���D�֐��Ƃ��Ē�`���邽�߂�eq1�𑽍����ɕϊ���(convert)�Cunapply��������D}\end{Maple Normal}

\textbf{eq1:=series(f0(x),x=0,3);}
\textbf{f1:=unapply(convert(eq1,polynom),x);}\mapleresult
\begin{maplelatex}
\mapleinline{inert}{2d}{eq1 := x+O(x^3)}{\[\displaystyle {\it eq1}\, := \,x+O \left( {x}^{3} \right) \]}
\end{maplelatex}
\mapleresult
\begin{maplelatex}
\mapleinline{inert}{2d}{f1 := proc (x) options operator, arrow; x end proc}{\[\displaystyle {\it f1}\, := \,x\mapsto x\]}
\end{maplelatex}
\end{maplegroup}
\begin{maplegroup}
\begin{Maple Normal}{
�T���ɂ‚��Ă����l}\end{Maple Normal}

\textbf{eq2:=series(f0(x),x=0,5);}
\textbf{f2:=unapply(convert(eq2,polynom),x);}\mapleresult
\begin{maplelatex}
\mapleinline{inert}{2d}{eq2 := x-(1/6)*x^3+O(x^5)}{\[\displaystyle {\it eq2}\, := \,x-1/6\,{x}^{3}+O \left( {x}^{5} \right) \]}
\end{maplelatex}
\mapleresult
\begin{maplelatex}
\mapleinline{inert}{2d}{f2 := proc (x) options operator, arrow; x-(1/6)*x^3 end proc}{\[\displaystyle {\it f2}\, := \,x\mapsto x-1/6\,{x}^{3}\]}
\end{maplelatex}
\end{maplegroup}
\begin{maplegroup}
\begin{Maple Normal}{
3�‚̊֐��𓯎��v���b�g}\end{Maple Normal}

\textbf{plot([f0(x),f1(x),f2(x)],x=-Pi..Pi);}\mapleresult
\mapleplot{Diffplot2d3.eps}
\end{maplegroup}
\subsubsection{\textbf{\textit{5. ii)}}}
\begin{maplegroup}
\begin{mapleinput}
\mapleinline{active}{1d}{series(f0(x),x=Pi/2,3)
\begin{Maple Normal}{
�ȊO�͑O��Ƃ��Ȃ��D}\end{Maple Normal}
}{}
\end{mapleinput}
\end{maplegroup}
\subsubsection{\textbf{\textit{6.}}}
\begin{maplegroup}
\begin{mapleinput}
\mapleinline{active}{1d}{f:=unapply(exp(x)*log(1+y),(x,y));
}{}
\end{mapleinput}
\mapleresult
\begin{maplelatex}
\mapleinline{inert}{2d}{f := proc (x, y) options operator, arrow; exp(x)*ln(y+1) end proc}{\[\displaystyle f\, := \,( {x,y} )\mapsto {{\rm e}^{x}}\ln  \left( 1+y \right) \]}
\end{maplelatex}
\end{maplegroup}
\begin{maplegroup}
\begin{mapleinput}
\mapleinline{active}{1d}{eq1:=series(series(f(x,y),x=0,3),y=0,3);
}{}
\end{mapleinput}
\mapleresult
\begin{maplelatex}
\mapleinline{inert}{2d}{eq1 := O(x^3)+(1+(1/2)*x^2+x)*y+(-(1/2)*x-1/2-(1/4)*x^2)*y^2+O(y^3)}{\[\displaystyle {\it eq1}\, := \,O \left( {x}^{3} \right) + \left( 1+1/2\,{x}^{2}+x \right) y+ \left( -1/2\,x-1/2-1/4\,{x}^{2} \right) {y}^{2}\\
\mbox{}+O \left( {y}^{3} \right) \]}
\end{maplelatex}
\end{maplegroup}
\begin{maplegroup}
\begin{mapleinput}
\mapleinline{active}{1d}{g:=unapply(convert(convert(eq1,polynom),polynom),(x,y));
}{}
\end{mapleinput}
\mapleresult
\begin{maplelatex}
\mapleinline{inert}{2d}{g := proc (x, y) options operator, arrow; (1+(1/2)*x^2+x)*y+(-(1/2)*x-1/2-(1/4)*x^2)*y^2 end proc}{\[\displaystyle g\, := \,( {x,y} )\mapsto  \left( 1+1/2\,{x}^{2}+x \right) y+ \left( -1/2\,x-1/2-1/4\,{x}^{2} \right) {y}^{2}\]}
\end{maplelatex}
\end{maplegroup}
\begin{maplegroup}
\begin{mapleinput}
\mapleinline{active}{1d}{plot3d([f(x,y),g(x,y)],x=-1..1,y=-1..1,axes=box);
}{}
\end{mapleinput}
\mapleresult
\mapleplot{Diffplot3d4.eps}
\end{maplegroup}

 
\ChartElement{例題}
\subsection{例題:関数の微分と増減表}
次の関数とその1次導関数を同時にプロットし概形を確認し,さらに増減表を求めよ.
\begin{equation*}
\frac {x}{{x}^{2}-2x+4}
\end{equation*}

\paragraph{解答例}
\begin{MapleInput}
> f0:=unapply(x/(x^2-2*x+4),x):
> df:=unapply(diff(f0(x),x),x);
> plot([f0(x),df(x)],x);
\end{MapleInput}

\begin{MapleOutputGather}
{\it df}\, := \,x\mapsto  \left( {x}^{2}-2\,x+4 \right) ^{-1}-{\frac {x \left( 2\,x-2 \right) }{ \left( {x}^{2}-2\,x+4 \right) ^{2}}}
\end{MapleOutputGather}

\MaplePlot{60mm}{./figures/Diffplot2d1.eps}
 
\subsection{例題:接線(Tangent)}
次の関数の$\displaystyle x=3$での接線を求め,2つの関数を同時にプロットせよ.
\begin{equation*}
y={x}^{3}-2\,{x}^{2}-35\,x
\end{equation*}

\paragraph{解答例}
与関数をf0と定義.
\begin{MapleInput}
> f0:=unapply(x^3 - 2*x^2 - 35*x,x);
\end{MapleInput}
\begin{MapleOutput}
{\it f0}\, := \,x\mapsto {x}^{3}-2\,{x}^{2}-35\,x
\end{MapleOutput}
微分関数をdfと定義
\begin{MapleInput}
> df:=unapply(diff(f0(x),x),x);
\end{MapleInput}
\begin{MapleOutput}
{\it df}\, := \,x\mapsto 3\,{x}^{2}-4\,x-35
\end{MapleOutput}
接点(x0,f0(x0))で傾きdf(x0)の直線をf1と定義.
\begin{MapleInput}
> x0:=3;
> eq1:=df(x0)*(x-x0)+f0(x0);
> f1:=unapply(eq1,x);
\end{MapleInput}
\begin{MapleOutputGather}
{\it x0}\, := \,3 \notag \\
{\it eq1}\, := \,-20\,x-36 \notag \\
{\it f1}\, := \,x\mapsto -20\,x-36 \notag
\end{MapleOutputGather}
2つの関数を同時にプロット.
\begin{MapleInput}
> plot([f0(x),f1(x)],x=-5..5);
\end{MapleInput}
\MaplePlot{30mm}{./figures/Diffplot2d2.eps}
 
\pagebreak
\section{微分(Diff)-II}
\ChartElement{解説}
\subsection{級数展開(series)}
Taylor級数は以下のようにして,中心点(x=a),次数(4次)を指定する.
\begin{MapleInput}
> t1:=series(sin(x),x=a,4);
\end{MapleInput}
\begin{MapleOutput}
t1 := \sin(a)+\cos(a)(x-a)-\frac{1}{2}\sin(a)(x-a)^2-\frac{1}{6}\cos(a)(x-a)^3+O((x-a)^4)
\end{MapleOutput}
誤差の次数を示す$O((x-a)^4)$があるため,このままでは関数として使えない.unapplyで定義関数として使うためには,convertを使って多項式(polynomial)に変換しておく.\begin{MapleInput}
> e1:=convert(t1,polynom);
> f1:=unapply(e1,x);
\end{MapleInput}
\begin{MapleOutputGather}
t1 := \sin(a)+\cos(a)(x-a)-\frac{1}{2}\sin(a)(x-a)^2-\frac{1}{6}\cos(a)(x-a)^3 \notag \\
 {\it f1}\, := \,x\mapsto 
 \sin(a)+\cos(a)(x-a)-\frac{1}{2}\sin(a)(x-a)^2-\frac{1}{6}\cos(a)(x-a)^3 \notag
\end{MapleOutputGather}
 
\subsection{全微分(D)}
全微分を計算するときは,Dを用いる.
\begin{MapleInput}
> f:=unapply(x^4*exp(-y^2),(x,y));
> D(f(x,y));
> (D@@2)(f(x,y));
\end{MapleInput}
\begin{MapleOutputGather}
f\, := \,( {x,y} )\mapsto {x}^{4}\exp(-{y}^{2}) \notag \\
4\, {D} \left( x \right) {x}^{3}\exp(-{y}^{2})+{x}^{4} {D} \left( \exp(-{y}^{2}) \right) \notag \\
4\, \left( D^{ \left( 2 \right) } \right)  \left( x \right) {x}^{3}\exp(-{y}^{2})+12\, \left(  {D} \left( x \right)  \right) ^{2}{x}^{2}\exp(-{y}^{2})+8\, {D} \left( x \right) {x}^{3} {D} \left( \exp(-{y}^{2}) \right) +{x}^{4} \left( D^{ \left( 2 \right) } \right)  \left( \exp(-{y}^{2}) \right) \notag
\end{MapleOutputGather}

ここで,D(x)などはxの全微分を表わす.これは,x,yを変数としているので
\begin{MapleInput}
> diff(x,x);
> diff(exp(-y^2),y);
\end{MapleInput}
\begin{MapleOutputGather}
1 \notag \\
 -2\,y\exp(-{y}^{2}) \notag
\end{MapleOutputGather}
であるがMapleには分からない.そこで全微分の最終形を得るには,あらかじめD(x)などの結果を求めておき,subsで明示的に代入する必要がある.
\begin{MapleInput}
> dd:=D(f(x,y)):
> eqs:={D(x)=diff(x,x),D(exp(-y^2))=diff(exp(-y^2),y)};
> subs(eqs,dd);
\end{MapleInput}
\begin{MapleOutputGather}
 {\it eqs}\, := \, \left\{  {D} \left( x \right) =1, {D} \left( \exp(-{y}^{2}) \right) =-2\,y\exp(-{y}^{2}) \right\}  \notag \\
 4\,{x}^{3}\exp(-{y}^{2})-2\,{x}^{4}y\exp(-{y}^{2}) \notag
\end{MapleOutputGather}
 
\subsection{複合関数の微分}
\begin{MapleInput}
> diff(f(x)*g(x),x);
> diff(f(g(x)),x);
\end{MapleInput}
\begin{MapleOutputGather}
\left( {\frac {d}{dx}}f \left( x \right)  \right) g \left( x \right) +f \left( x \right) {\frac {d}{dx}}g \left( x \right) \notag \\
\mbox {D} \left( f \right)  \left( g \left( x \right)  \right) {\frac {d}{dx}}g \left( x \right) \notag
\end{MapleOutputGather}

\begin{MapleInput}
> f:=x->exp(x);
> g:=x->cos(x);
> diff(f(x)*g(x),x);
> diff(f(g(x)),x);
\end{MapleInput}
\begin{MapleOutputGather}
 f\, := \,x\mapsto \exp(x) \notag \\
 g\, := \,x\mapsto \cos \left( x \right)  \notag \\
  \exp(x)\cos \left( x \right) -\exp(x)\sin \left( x \right)  \notag \\
  -\sin \left( x \right) \exp(\cos x ) \notag
\end{MapleOutputGather} 
\ChartElement{課題}
\subsubsection{微分に関する課題}
\begin{enumerate}
\item
次の関数を微分せよ.

i)$ {x} \log x$, 
ii) $\displaystyle \frac{1}{  \left( 1+x \right) ^{3}}$, 
iii) $ \sqrt{4\,x+3}$, 
iv) $\displaystyle \frac{1}{ a^2+ \left( x-x_0 \right)^2 }$

\item
次の関数の1次から5次導関数を求めよ.

i) $\sin^2 x$, 
ii) ${e}^{x}$

\item 以下の関数をx0まわりで3次までテイラー展開し,得られた関数ともとの関数をプロットせよ.さらに5次まで展開した場合はどう変化するか.

i) $ y=\sin x, x_0=0 $, 
ii) $\displaystyle y=\cos x, x_0=\frac{\pi}{2}$

\item
(発展課題)$f \left( x,y \right) ={e}^{x}{\it log} \left( 1+y \right) $
を$\displaystyle x=0,\,y=0$のまわりで3次まで展開せよ.

\end{enumerate}
 
\ChartElement{解答例}
\subsubsection{Diff}
\begin{enumerate}
\item 
\begin{MapleInput}
> diff(x*log(x),x);
> diff(1/(1 + x)^3,x);
> diff(sqrt(4*x + 3),x);
>diff(1/(a^2+(x-x0)^2),x);
\end{MapleInput}
\begin{MapleOutputGather}
\ln  \left( x \right) +1 \notag \\
-3\, \left( 1+x \right) ^{-4} \notag \\
 2\, \left(  \sqrt{4\,x+3} \right) ^{-1} \notag \\
 -{\frac {2\,x-2\,{\it x0}}{ \left( {a}^{2}+ \left( x-{\it x0} \right) ^{2} \right) ^{2}}} \notag
\end{MapleOutputGather}

\item
\begin{MapleInput}
> diff(sin(x)^2,x);
> diff(sin(x)^2,x$2);
\end{MapleInput}

\begin{MapleOutputGather}
2\,\sin \left( x \right) \cos \left( x \right) \notag \\
 2\, \left( \cos \left( x \right)  \right) ^{2}-2\, \left( \sin \left( x \right)  \right) ^{2}\notag
\end{MapleOutputGather}
\begin{MapleError}
以下略
\end{MapleError}


\item
先ず与関数をf0と定義
\begin{MapleInput}
> f0:=unapply(sin(x),x);
\end{MapleInput}
\begin{MapleOutput}
 {\it f0}\, := \,x\mapsto \sin \left( x \right)
\end{MapleOutput}
テイラー展開した結果をeq1とする.関数として定義するためにeq1を多項式に変換し(convert),unapplyをかける.
\begin{MapleInput}
> eq1:=series(f0(x),x=0,3);
> f1:=unapply(convert(eq1,polynom),x);
\end{MapleInput}
\begin{MapleOutputGather}
{\it eq1}\, := \,x+O \left( {x}^{3} \right) \notag \\
{\it f1}\, := \,x\mapsto x \notag
\end{MapleOutputGather}
5次についても同様
\begin{MapleInput}
> eq2:=series(f0(x),x=0,5);
> f2:=unapply(convert(eq2,polynom),x);
\end{MapleInput}
\begin{MapleOutputGather}
 {\it eq2}\, := \,x-1/6\,{x}^{3}+O \left( {x}^{5} \right) \notag \\
 {\it f2}\, := \,x\mapsto x-1/6\,{x}^{3} \notag
\end{MapleOutputGather}
3つの関数を同時プロット
\begin{MapleInput}
> plot([f0(x),f1(x),f2(x)],x=-Pi..Pi);
\end{MapleInput}
\MaplePlot{40mm}{./figures/Diffplot2d3.eps}

\begin{MapleInput}
> series(f0(x),x=Pi/2,3)
\end{MapleInput}
以外は前問とおなじ.

\item
\begin{MapleInput}
f:=unapply(exp(x)*log(1+y),(x,y));
\end{MapleInput}
\begin{MapleOutput}
f\, := \,( {x,y} )\mapsto \exp(x)\ln  \left( 1+y \right)
\end{MapleOutput}
\begin{MapleInput}
eq1:=series(series(f(x,y),x=0,3),y=0,3);
\end{MapleInput}
\begin{MapleOutput}
{\it eq1}\, := \,O \left( {x}^{3} \right) + \left( 1+1/2\,{x}^{2}+x \right) y+ \left( -1/2\,x-1/2-1/4\,{x}^{2} \right) {y}^{2}\\
\mbox{}+O \left( {y}^{3} \right)
\end{MapleOutput}

\begin{MapleInput}
> g:=unapply(convert(convert(eq1,polynom),polynom),(x,y));
\end{MapleInput}
\begin{MapleOutput}
g\, := \,( {x,y} )\mapsto  \left( 1+1/2\,{x}^{2}+x \right) y+ \left( -1/2\,x-1/2-1/4\,{x}^{2} \right) {y}^{2}
\end{MapleOutput}
\begin{MapleInput}
> plot3d([f(x,y),g(x,y)],x=-1..1,y=-1..1,axes=box);
\end{MapleInput}
\MaplePlot{60mm}{./figures/Diffplot3d4.eps}
\end{enumerate} 
\subsubsection{きれいな表示}

以下のようにすると表示がきれい.
\begin{MapleInput}
> f:=unapply(x^4*exp(-y^2),(x,y));
> d:=Diff(f(x,y),x);
> d=value(d);
\end{MapleInput}
\begin{MapleOutputGather}
f\, := \,( {x,y} )\mapsto {x}^{4}\exp(-{y}^{2}) \notag \\
d\, := \,{\frac {\partial }{\partial x}} \left( {x}^{4}\exp(-{y}^{2}) \right) \notag \\
{\frac {\partial }{\partial x}} \left( {x}^{4}\exp(-{y}^{2}) \right) =4\,{x}^{3}\exp(-{y}^{2})\notag
\end{MapleOutputGather}
 

\pagebreak
\section{積分(int)}
\ChartElement{解説}
\subsection{単純な積分(int)}
単純な不定積分.
\begin{MapleInput}
> int(ln(x),x);	#res: x ln(x) - x
\end{MapleInput}
定積分を実行するには,積分変数の範囲を指定する.
\begin{MapleInput}
> int(sin(x),x=-Pi..0);	#res: -2
\end{MapleInput}
特異点をもつ場合にも適切に積分結果を求めてくれる.
\begin{MapleInput}
> int(1/sqrt(x*(2-x)),x=0..2); #res: pi
\end{MapleInput}
無限区間(infinity)における定積分も同様に計算してくれる.
\begin{MapleInput}
> int(1/(x^2+4),x=-infinity..infinity); #res: 1/2 pi
\end{MapleInput}
部分積分法や置換積分法を用いる必要のある複雑な積分も一発で求まる.
\begin{MapleInput}
> eq:=sqrt(4-x^2);int(eq,x);
\end{MapleInput}
\begin{MapleOutputGather}
 {\it eq}\, := \, \sqrt{4-{x}^{2}} \notag \\
 \frac{1}{2}\,x \sqrt{4-{x}^{2}}+2\,\arcsin \left( 1/2\,x \right) \notag
\end{MapleOutputGather}
数学の公式集に載っているような積分も同じコマンドで求まる.
\begin{MapleInput}
> eq2:=exp(-x^2);int(eq2,x=0..zz);
\end{MapleInput}
\begin{MapleOutputGather}
{\it eq2}\, := \,\exp({-{x}^{2}}) \notag \\
\frac{1}{2}\, \sqrt{\pi }\, \mbox{erf} \left(zz\right) \notag
\end{MapleOutputGather} 
\subsection{studentパッケージいろいろ}
ちょっとぐらい難しい積分も,Mapleは単純にintコマンドだけで実行してくれる.しかし,時には,途中の計算法である部分積分,置換積分,部分分数展開が必要になる.このような計算はstudentパッケージに用意さている.
\begin{MapleInput}
> with(student):
\end{MapleInput}
\paragraph{部分積分(integration by parts)} 
\begin{MapleInput}
> intparts(Int(x*exp(x),x),x);
\end{MapleInput}
\begin{MapleOutput}
x\exp(x)-\int \exp(x){dx}
\end{MapleOutput}
\paragraph{置換(change of variables)による積分} 
\begin{MapleInput}
> Int((cos(x)+1)^3*sin(x), x);
> changevar(cos(x)+1=u, Int((cos(x)+1)^3*sin(x), x=a..b), u);
> changevar(cos(x)+1=u, int((cos(x)+1)^3*sin(x), x), u);
\end{MapleInput}
\begin{MapleOutputGather}
\int \left( \cos \left( x \right) +1 \right) ^{3}\sin \left( x \right) {dx} \notag \\
\int _{\cos \left( a \right) +1}^{\cos \left( b \right) +1}-{u}^{3}{du} \notag \\
 -\frac{1}{4}\,{u}^{4} \notag \\
\end{MapleOutputGather}
\paragraph{部分分数(partial fraction)展開による積分}
部分分数(partial fraction)展開による積分では,convertコマンドを用いる.
\begin{MapleInput}
> pf1:=convert(1/(1+x^3),parfrac,x);
  int(pf1,x);
\end{MapleInput}
\begin{MapleOutputGather}
 {\it pf1}\, := \frac{1}{3}\,{\frac {-x+2}{{x}^{2}-x+1}}+ \frac{1}{3\,\left( x+1 \right) } \notag \\
 -\frac{1}{6}\,\ln  \left( {x}^{2}-x+1 \right) +\frac{1}{3}\, \sqrt{3}\arctan \left( 1/3\, \left( 2\,x-1 \right)  \sqrt{3} \right) +\frac{1}{3}\,\ln  \left( x+1 \right) \notag
\end{MapleOutputGather}

 
\ChartElementTwo{課題}
\begin{enumerate}
\item 不定積分:次の不定積分を求めよ.

i) $\int 4\,x+3{dx}$
,ii)$\displaystyle \int  \frac{1}{ 1+\mbox{e}^{x} }{dx}$
,iii)$\displaystyle \int  \frac{1}{ \mbox{e}^{-x}+\mbox{e}^{x} }{dx}$
,iv)$\displaystyle \int  \sqrt{1-{x}^{2}}{dx}$


\item 定積分:次の定積分を求めよ.

i)$\displaystyle \int _{0}^{\pi } \sin x{dx}$
,ii)$\displaystyle \int _{0}^{1} \arctan x{dx}$
,iii)$\displaystyle \int _{-2}^{2} \frac{1}{ \sqrt{4-{x}^{2}} }{dx}$
,iv)$\displaystyle \int _{0}^{1} \frac{1}{ {x}^{2}+x+1 }{dx}$

\item (発展課題,重積分)次の2重積分を求めよ.
\begin{equation*}
\int \int_{D} \sqrt{x^2+y^2}dxdy\,\, D:0\leq y \leq x \leq 1
\end{equation*}
\end{enumerate} 
\ChartElementTwo{解答例}
\begin{enumerate}

\item 
\begin{MapleInput}
> int(4*x+3,x);
> int( 1/(1+exp(x)),x);
> int(1/(exp(-x)+exp(x)),x);
> int(sqrt(1-x^2),x);
\end{MapleInput}
\begin{MapleOutputGather}
\displaystyle 2\,{x}^{2}+3\,x  \notag \\
\displaystyle -\ln  \left( 1+ \mbox{e}^x \right) +\ln  \left( \mbox{e}^x \right)  \notag \\
\displaystyle \arctan \left( \mbox{e}^x \right)  \notag \\
\displaystyle \frac{1}{2}\,x \sqrt{1-{x}^{2}}+\frac{1}{2}\,\arcsin \left( x \right)  \notag
\end{MapleOutputGather}

\item
\begin{MapleInput}
> int(sin(x),x=0..Pi);
> int(arctan(x),x=0..1);
> int(1/(sqrt(4-x^(2))),x=-2..2);
> int(1/(x^2+x+1),x=0..1);
\end{MapleInput}
\begin{MapleOutputGather}
\displaystyle 2   \notag \\
\displaystyle \frac{1}{4}\,\pi -\frac{1}{2}\,\ln  \left( 2 \right)   \notag \\
\displaystyle \pi   \notag \\
\displaystyle \frac{1}{9}\,\pi \, \sqrt{3}  \notag
\end{MapleOutputGather}

\item
\begin{MapleInput}
> with(plots):
> inequal({x-y>=0,x>=0,x<=1,y>=0},x=-0.5..1.5,y=-0.5..1.5,optionsexcluded=(color=white));
\end{MapleInput}

\MaplePlot{40mm}{./figures/Intplot2d1.eps}

\begin{MapleInput}
> f:=unapply(sqrt(x^2+y^2),(x,y)):
> plot3d(f(x,y),x=0..1,y=0..1,axes=box);
\end{MapleInput}

\MaplePlot{40mm}{./figures/Intplot3d2.eps}
\begin{MapleInput}
> int(int(f(x,y),y=0..x),x=0..1);
\end{MapleInput}
\begin{MapleOutput}
\displaystyle \frac{1}{6} \sqrt{2}+\frac{1}{6}\,\ln  \left( 1+ \sqrt{2} \right) 
\end{MapleOutput}

\end{enumerate} 

\chapter{線形代数}
\section{行列・ベクトル生成(MatrixVector)}
\ChartElementTwo{解説}
線形代数の計算にはあらかじめ関数パッケージ(LinearAlgebra)を呼び出しておく.
\begin{MapleInput}
> with(LinearAlgebra):
\end{MapleInput}

\subsection{ベクトルの生成(Vector)}
ベクトルの生成は,
\begin{MapleInput}
> v1 := Vector([x, y]);
\end{MapleInput}
\begin{MapleOutput}
\displaystyle {\it v1}\, := \, \left[ \begin {array}{c} x\\ y\end {array} \right] 
\end{MapleOutput}
通常の方法では,縦(列)ベクトル(column)ができることに注意.横(行)ベクトル(row)を作るには,明示する必要あり.
\begin{MapleInput}
> v2 := (Vector[row])([x, y, z]);
\end{MapleInput}
\begin{MapleOutput}
\displaystyle {\it v2}\, := \, \left[ \begin {array}{ccc} x&y&z\end {array} \right] 
\end{MapleOutput}
新聞の囲み記事(列)がcolumn,劇場の座席(行)はrow.

\subsection{行列の生成(Matrix)}
標準的な行列(Matrix)の生成は,
\begin{MapleInput}
> A0 := Matrix([[1, 2, 3], [4, 5, 6]]); #res: 省略
\end{MapleInput}
リストリストからの変換は,
\begin{MapleInput}
> LL1 := [[1, 2], [3, 4]]:
> A1 := Matrix(LL1);
\end{MapleInput}
\begin{MapleOutput}
\displaystyle {\it A1}\, := \, \left[ \begin {array}{cc} 1&2\\ 3&4\end {array} \right] 
\end{MapleOutput}
単位行列の生成は,
\begin{MapleInput}
> E := IdentityMatrix(2);
\end{MapleInput}
\begin{MapleOutput}
\displaystyle E\, := \, \left[ \begin {array}{cc} 1&0\\ 0&1\end {array} \right]
\end{MapleOutput}
対角行列を生成するDiagonalMatrixもある.同じことは以下のようにしても生成が可能.
\begin{MapleInput}
> Matrix(2,2,shape=identity);
\end{MapleInput}
\begin{MapleOutput}
\displaystyle  \left[ \begin {array}{cc} 1&0\\ 0&1\end {array} \right] 
\end{MapleOutput}

\subsection{縦横ベクトル・行列の簡易作成法(MVShortcut)}
かぎかっこ($<\cdots>$)を使って,ベクトルあるいは行列を直感的に作ることが可能.カンマで区切ると縦に積み,縦棒で区切ると横に積む.セミコロンで区切るとそこで次の行へ.
\begin{MapleInput}
> v1:=<x,y>; #縦ベクトル,列
> v2:=<x|y|z>; #横ベクトル,行
> A1:=<1,2;3,4>; #2x2行列
> <A1|v1>; #2x3行列(拡大係数行列などの作成)
> ?MVShortcut; #res:参照
\end{MapleInput}
\begin{MapleOutputGather}
\displaystyle {\it v1}\, := \, \left[ \begin {array}{c} x\\ y\end {array} \right] \notag \\
\displaystyle {\it v2}\, := \, \left[ \begin {array}{ccc} x&y&z\end {array} \right] \notag \\
\displaystyle {\it A1}\, := \, \left[ \begin {array}{cc} 1&2\\ 3&4\end {array} \right]  \notag \\
\displaystyle  \left[ \begin {array}{ccc} 1&2&x\\ 3&4&y\end {array} \right] \notag 
\end{MapleOutputGather}

\subsection{行列,ベクトルの成分の抽出(MVextraction)}
行列A1の1行2列の成分を取り出すには,
\begin{MapleInput}
> A1[1,2]; #res: 2
\end{MapleInput}
行列の一部を行列として取り出すには
\begin{MapleInput}
> A1[1..2,1..2];
\end{MapleInput}
\begin{MapleOutput}
\displaystyle  \left[ \begin {array}{cc} 1&2\\ 3&4\end {array} \right]
\end{MapleOutput}
2x2行列の2列目(行の長さに関係なく)でつくるベクトルは
\begin{MapleInput}
> A1[..,2..2];
\end{MapleInput}
\begin{MapleOutput}
\displaystyle  \left[ \begin {array}{c} 2\\ 4\end {array} \right] 
\end{MapleOutput}
同じことが,行(Row)あるいは列(Column)抽出関数でもできる.使い方は次の通り.
\begin{MapleInput}
> Column(A1,2);
\end{MapleInput}
\begin{MapleOutput}
\displaystyle  \left[ \begin {array}{c} 2\\ 4\end {array} \right]
\end{MapleOutput}
 
\ChartElementTwo{課題}
\begin{enumerate}
\item 次の行列,ベクトルを作れ.

(a)$\displaystyle  \left[ \begin {array}{ccc} 3&3&3\\ 3&3&3\end {array} \right] $,
(b)$\displaystyle  \left[ \begin {array}{ccc} 1&0&0\\ 0&1&0\end {array} \right] $,
(c)$\displaystyle  \left[ \begin {array}{ccc} 1&2&3\end {array} \right] $,
(d)$\displaystyle  \left[ \begin {array}{ccc} 1&0&0\\ 0&2&0\\ 0&0&3\end {array} \right] $,
(e)$\displaystyle  \left[ \begin {array}{cc} x&{x}^{2}\\ {x}^{2}&{x}^{3}\end {array} \right] $,
(f)$\displaystyle  \left[ \begin {array}{ccc} 1&2&3\\ 4&5&6\\ 7&8&9\end {array} \right] $
\end{enumerate}

 
\ChartElementTwo{解答例}
\begin{enumerate}
\item
\begin{enumerate}
\item
\begin{MapleInput}
> Matrix([3,3,3],[3,3,3]]);
> Matrix(<3,3|3,3|3,3>);
\end{MapleInput}
\begin{MapleOutput}
\displaystyle  \left[ \begin {array}{ccc} 3&3&3\\ 3&3&3\end {array} \right]
\end{MapleOutput}
そのまま直打ちしてもいいが,少し賢い生成法も記しておく.詳しくはヘルプ参照.
\begin{MapleInput}
> Matrix(2,3,3);
\end{MapleInput}
\begin{MapleOutput}
\displaystyle  \left[ \begin {array}{ccc} 3&3&3\\ 3&3&3\end {array} \right]
\end{MapleOutput}

\item
\begin{MapleInput}
> Matrix(2,3,shape=identity);
\end{MapleInput}
\begin{MapleOutput}
\displaystyle  \left[ \begin {array}{ccc} 1&0&0\\ 0&1&0\end {array} \right]
\end{MapleOutput}

\item
\begin{MapleInput}
> Vector[row]([1,2,3]);
> Vector(<1|2|3>):
\end{MapleInput}
\begin{MapleOutput}
\displaystyle  \left[ \begin {array}{ccc} 1&2&3\end {array} \right]
\end{MapleOutput}

\item
\begin{MapleInput}
> with(LinearAlgebra):
> V:=Vector[row]([1,2,3]);
> DiagonalMatrix(V);
\end{MapleInput}
\begin{MapleOutputGather}
\displaystyle V\, := \, \left[ \begin {array}{ccc} 1&2&3\end {array} \right] \notag \\
\displaystyle  \left[ \begin {array}{ccc} 1&0&0\\ 0&2&0\\ 0&0&3\end {array} \right] \notag
\end{MapleOutputGather}

\item
\begin{MapleInput}
> f:= (i,j) -> x^(i+j-1):
> Matrix(2,f);
\end{MapleInput}
\begin{MapleOutput}
\displaystyle  \left[ \begin {array}{cc} x&{x}^{2}\\ {x}^{2}&{x}^{3}\end {array} \right] 
\end{MapleOutput}


\item
\begin{MapleInput}
> Matrix(3,[seq(i,i=1..9)]);
\end{MapleInput}
\begin{MapleOutput}
\displaystyle  \left[ \begin {array}{ccc} 1&2&3\\ 4&5&6\\ 7&8&9\end {array} \right]
\end{MapleOutput}

\begin{MapleInput}
> n:=3:
> f:=(i,j)->(i-1)*n+j;
> Matrix(3,3,f);
\end{MapleInput}
\begin{MapleOutputGather}
\displaystyle f\, := \,( {i,j} )\mapsto  \left( i-1 \right) n+j  \notag \\
\displaystyle  \left[ \begin {array}{ccc} 1&2&3\\ 4&5&6\\ 7&8&9\end {array} \right] \notag
\end{MapleOutputGather}

\end{enumerate}
\end{enumerate} 

\pagebreak
\section{内積外積(DotProduct,CrossProduct)}
\ChartElementTwo{解説}
線形代数の計算にはあらかじめ関数パッケージ(LinearAlgebra)を呼び出しておく.
\begin{MapleInput}
> with(LinearAlgebra):
\end{MapleInput}

\subsection{スカラーとのかけ算}

\begin{MapleInput}
> v1:=Vector([x, y]): 3*v1;
\end{MapleInput}
\begin{MapleOutput}
\left[ \begin {array}{c} 3\,x\\ 3\,y\end {array} \right]
\end{MapleOutput}

\subsection{行列,ベクトルの足し算,引き算}
\begin{MapleInput}
> LL1 := [[1, 2], [3, 4]]: A1 := Matrix(LL1): A2 := Matrix([[x, x], [y, y]]):
> 3*A1-4*A2;
\end{MapleInput}
\begin{MapleOutput}
\left[ \begin {array}{cc} 3-4\,x&6-4\,x\\ 9-4\,y&12-4\,y\end {array} \right] 
\end{MapleOutput}

\subsection{内積(DotProduct, `.`)}
\begin{MapleInput}
> v1:=Vector([1,1,3]): v2:=Vector([1,2,-1]): v1.v2;
\end{MapleInput}
\begin{MapleOutput}
                                      0
\end{MapleOutput}
\subsection{外積(CrossProduct, `\&x`)}
\begin{MapleInput}
> CrossProduct(v1, v2); v1 &x v2:
\end{MapleInput}
\begin{MapleOutput}
\left[ \begin {array}{c} -7\\ 4\\ 1\end {array} \right] 
\end{MapleOutput}

\subsection{スカラー3重積}
\begin{MapleInput}
> v3 := Vector([-1,2,1]); CrossProduct(v1,v2).v3;
\end{MapleInput}
\begin{MapleOutputGather}
{\it v3}\, := \, \left[ \begin {array}{c} -1\\ 2\\ 1\end {array} \right] \notag \\
16 \notag
\end{MapleOutputGather}

\subsection{転置(Transpose, `\&T`)}は,行列Aのij成分a[i,j]をa[j,i]にする.
\begin{MapleInput}
> Transpose(A1);
\end{MapleInput}
\begin{MapleOutput}
\left[ \begin {array}{cc} 1&3\\ 2&4\end {array} \right]
\end{MapleOutput}
また横ベクトルを縦ベクトル(あるいはその逆)にするのも同じ.
\begin{MapleInput}
> Transpose(v1);
\end{MapleInput}
\begin{MapleOutput}
\left[ \begin {array}{ccc} 1&1&3\end {array} \right]
\end{MapleOutput}
 
\ChartElementTwo{課題}
\begin{enumerate}
\item 
行列$A= \left[ \begin {array}{cc} 1&2\\ 3&4\end {array} \right]$,
$B= \left[ \begin {array}{cc} 2&3\\ 4&5\end {array} \right]$,
およびベクトル$v= \left[ \begin {array}{c} 1\\ 2\end {array} \right] $を作り,
以下の計算を行い結果を観察せよ.

i) $A+3B$, 
ii) $A-B$, 
iii) $A+E$, 
iv) $A.B$, 
v) $B.A$, 
vi) $A.v$, 
vii) $v.A$,
viii) $v$の転置(Transpose)を
$A$に左側から掛けよ, 
ix) $A^3$

\item 2次元平面上で原点の周りの角度tの回転行列は
\begin{MapleInput}
> Ar:=t->Matrix([[cos(t),-sin(t)],[sin(t),cos(t)]]);
\end{MapleInput}
で定義できる.

i) Pi/6 回転させる行列を作り,単位ベクトル(1,0),(0,1)がどの点に移動するか確認せよ.

ii) Pi/6 回転させた後,続けてPi/4 回転させる操作を続けて行う回転行列を求めよ.また,角度を直接入力して要素を比較せよ.

\item
行列$A= \left[ \begin {array}{ccc} 1&2&3\\ 4&5&6\\ 7&8&9\end {array} \right]$
について
$A+A^t$, 
$A-A^t$
を求めて交代行列,対称行列を作れ.
\end{enumerate}
 
\ChartElementTwo{解答例}
\begin{enumerate}
\item

\begin{MapleInput}
> with(LinearAlgebra): A:=Matrix([[1,2],[3,4]]); B:=Matrix([[2,3],[4,5]]);
> v:=Vector([1,2]); E:=IdentityMatrix(2);
\end{MapleInput}
\begin{MapleOutputGather}
A\, := \, \left[ \begin {array}{cc} 1&2\\ 3&4\end {array} \right] \notag \\
B\, := \, \left[ \begin {array}{cc} 2&3\\ 4&5\end {array} \right] \notag \\
v\, := \, \left[ \begin {array}{c} 1\\ 2\end {array} \right] \notag \\
E\, := \, \left[ \begin {array}{cc} 1&0\\ 0&1\end {array} \right] \notag
\end{MapleOutputGather}
i)--vi)
\begin{MapleInput}
> A+3*B; A-B; A+E; A.B; B.A; A.v;
\end{MapleInput}
\begin{MapleOutputGather}
\left[ \begin {array}{cc} 7&11\\ 15&19\end {array} \right] \notag \\
\left[ \begin {array}{cc} -1&-1\\ -1&-1\end {array} \right] \notag \\
\left[ \begin {array}{cc} 2&2\\ 3&5\end {array} \right] \notag \\
\left[ \begin {array}{cc} 10&13\\ 22&29\end {array} \right] \notag \\
\left[ \begin {array}{cc} 11&16\\ 19&28\end {array} \right] \notag \\
\left[ \begin {array}{c} 5\\ 11\end {array} \right] \notag
\end{MapleOutputGather}
vii)
\begin{MapleInput}
> v.A;
\end{MapleInput}
\begin{MapleError}
Error, (in LinearAlgebra:-VectorMatrixMultiply) invalid input:
LinearAlgebra:-VectorMatrixMultiply expects its 1st argument, v, to be of type
Vector[row] but received Vector(2, {(1) = 1, (2) = 2})
\end{MapleError}
v.Aは次元が合わないので計算できない.次元を合わすためには,vに転置(Transpose)をかけて横ベクトルにしておく必要がある.

viii)
\begin{MapleInput}
> Transpose(v).A;
\end{MapleInput}
\begin{MapleOutput}
\left[ \begin {array}{cc} 7&10\end {array} \right]
\end{MapleOutput}

ix)
\begin{MapleInput}
> A^3;
\end{MapleInput}
\begin{MapleOutput}
\left[ \begin {array}{cc} 37&54\\ 81&118\end {array} \right] 
\end{MapleOutput}
\item
i)
\begin{MapleInput}
> with(LinearAlgebra): e1:=Vector([1,0]); e2:=Vector([0,1]);
> Ar:=t->Matrix([[cos(t),-sin(t)],[sin(t),cos(t)]]); Ar(Pi/6).e1; Ar(Pi/6).e2;
\end{MapleInput}
\begin{MapleOutputGather}
{\it e1}\, := \, \left[ \begin {array}{c} 1\\ 0\end {array} \right] \notag \\
{\it e2}\, := \, \left[ \begin {array}{c} 0\\ 1\end {array} \right]  \notag \\
{\it Ar}\, := \,t\mapsto  \left[ \begin {array}{cc} \cos \left( t \right) &-\sin \left( t \right) \\ \sin \left( t \right) &\cos \left( t \right) \end {array} \right]  \notag \\
\left[ \begin {array}{c} 1/2\,\sqrt {3}\\ 1/2\end {array} \right] \notag \\
\left[ \begin {array}{c} -1/2\\ 1/2\,\sqrt {3}\end {array} \right]  \notag
\end{MapleOutputGather}

ii.) 2つの関数を別々に計算.
\begin{MapleInput}
> Ar(Pi/4).Ar(Pi/6); 
> Ar(Pi/6+Pi/4);
\end{MapleInput}
\begin{MapleOutputGather}
\left[ \begin {array}{cc} 1/4\,\sqrt {2}\sqrt {3}-1/4\,\sqrt {2}&-1/4\,\sqrt {2}-1/4\,\sqrt {2}\sqrt {3}\\ 1/4\,\sqrt {2}\sqrt {3}+1/4\,\sqrt {2}&1/4\,\sqrt {2}\sqrt {3}-1/4\,\sqrt {2}\end {array} \right] \notag \\
\left[ \begin {array}{cc} \cos \left( {\frac {5}{12}}\,\pi  \right) &-\sin \left( {\frac {5}{12}}\,\pi  \right) \\ \sin \left( {\frac {5}{12}}\,\pi  \right) &\cos \left( {\frac {5}{12}}\,\pi  \right) \end {array} \right] \notag
\end{MapleOutputGather}

2つの操作の差のevalfをとるとほぼ0,つまり一致していることが確認できる.
\begin{MapleInput}
> evalf(Ar(Pi/6+Pi/4)-Ar(Pi/6).Ar(Pi/4));
\end{MapleInput}
\begin{MapleOutput}
\left[ \begin {array}{cc} - 0.0000000002000000000& 0.0\\  0.0&- 0.0000000002000000000\end {array} \right]
\end{MapleOutput}

\item

\begin{MapleInput}
> A:=Matrix([[1,2,3],[4,5,6],[7,8,9]]); As:=A+Transpose(A); Aa:=A-Transpose(A);
\end{MapleInput}
\begin{MapleOutputGather}
A\, := \, \left[ \begin {array}{ccc} 1&2&3\\ 4&5&6\\ 7&8&9\end {array} \right] \notag \\
{\it As}\, := \, \left[ \begin {array}{ccc} 2&6&10\\ 6&10&14\\ 10&14&18\end {array} \right] \notag \\
{\it Aa}\, := \, \left[ \begin {array}{ccc} 0&-2&-4\\ 2&0&-2\\ 4&2&0\end {array} \right] \notag 
\end{MapleOutputGather}
\end{enumerate}
 

\pagebreak
\section{行列の基本操作,掃き出し(LUDecomposition)}
\ChartElementTwo{解説}
係数行列$A$とベクトル$b$を足して作られる行列は拡大係数行列と呼ばれます.Mapleでは,これは
\begin{MapleInput}
> <A|b>;
\end{MapleInput}
\begin{MapleOutput}
\left[ \begin {array}{ccc} 2&5&7\\ 4&1&5\end {array} \right]
\end{MapleOutput}
として作られます.ここから行列の掃き出し操作をおこなうには,LUDecompositionというコマンドを使います.
\begin{MapleInput}
> P,L,U:=LUDecomposition(<A|b>);
\end{MapleInput}
\begin{MapleOutput}
P,\,L,\,U\, := \, \left[ \begin {array}{cc} 1&0\\ 0&1\end {array} \right] ,\, \left[ \begin {array}{cc} 1&0\\ 2&1\end {array} \right] ,\, \left[ \begin {array}{ccc} 2&5&7\\ 0&-9&-9\end {array} \right]
\end{MapleOutput}
これは,下三角行列(Lower Triangle Matrix)と上三角行列(Upper Triangle Matrix)に分解(decompose)するコマンドです.$P$行列は置換(permutation)行列を意味します.LUDecompositionだけでは,前進消去が終わっただけの状態です.そこで,後退代入までおこなうには,optionにoutput='R'をつけます.そうすると出力は,
\begin{MapleInput}
> LUDecomposition(<A|b>,output='R');
\end{MapleInput}
\begin{MapleOutput}
\left[ \begin {array}{ccc} 1&0&1\\ 0&1&1\end {array} \right]
\end{MapleOutput}
で,$b$ベクトルの部分が解になっています.
 
\ChartElementTwo{課題}
\begin{enumerate}
\item 行列$A= \left[ \begin {array}{ccc} 1&2&3\\  4&5&6\\  7&8&9\end {array} \right]$について,
RowOperationのヘルプを参照して,次の行基本操作をおこない,階数を求め,コマンドLUDecomposition, Rankの結果と比べよ.

i) 2行目から1行目の4倍を引く.

ii) 3行目から1行目の7倍を引く.

iii) 2行目を-1/3倍する.

iv) 3行目に2行目の6倍を足す.

RowOperationのコツは,最初はinplace=falseでやってみて,うまくいけばtrueにかえる.

\item 
次の連立方程式の解を掃き出し法で求めよ.GenerateMatrixを使えば連立方程式から拡大係数行列を直接生成することも可能.

(i)
\begin{equation*}
\left\{
\begin{array}{cc}x +y -z &=2 \\
2 x -3 y +z &=4  \\
4 x -y +3 z &=1  
\end{array} \right.
\end{equation*}

(ii)
\begin{equation*}
\left\{
\begin{array}{cl}2 x +4 y -3 z &=1   \\
3 x -8 y +6 z &=58   \\
x -2 y -9 z &=23   \\
\end{array} \right.
\end{equation*}

(iii)
\begin{equation*}
\left\{
\begin{array}{cl}
1 x -10 y -3 z -7 u &=2   \\
2 x -4 y +3 z +4 u &=-3   \\
x -2 y +6 z +5 u &=-1   \\
x +8 y +9 z +3 u &=5
\end{array} \right.
\end{equation*}
(iv)
\begin{equation*}
\left\{\begin{array}{cl}x +y +z &=a +b +c    \\
ax+by+cz &=ab +bc +ca   \\
bc\,x +ca\,y + ab\,z &=3\,abc 
\end{array} \right.
\end{equation*}

\item 
次の連立方程式 
\begin{equation*}
\left\{
\begin{array}{cc}
x _{1}+2 x _{2}-x _{3} & =0\\
x _{1}+x _{2}+3 x _{4}&=0    \\
x _{1}+5 x _{2}-2 x _{3}+3 x _{4}&=0  \\
x _{1}+3 x _{2}-2 x _{3}- 3 x _{4}&=0 
\end{array}
 \right.
\end{equation*}
をsolveを使って,$x_1, x_2, x_3, x_4$について解け.
次にGenerateMatrixを使って,拡大係数行列にした後,LUDecompositionを用いて掃き出しを行い結果を比較せよ.
\end{enumerate} 
\ChartElementTwo{解答例}
\begin{enumerate}
\item
\begin{MapleInput}
> A:=Matrix([[1,2,3],[4,5,6],[7,8,9]]); with(LinearAlgebra): ?RowOperation;
\end{MapleInput}
\begin{MapleOutput}
A\, := \, \left[ \begin {array}{ccc} 1&2&3\\  4&5&6\\  7&8&9\end {array} \right]
\end{MapleOutput}
ヘルプに書かれてある例を見本にして,コマンドを記述.
\begin{MapleInput}
> RowOperation(A,[2,1],-4);
\end{MapleInput}
\begin{MapleOutput}
\left[ \begin {array}{ccc} 1&2&3\\  0&-3&-6\\  7&8&9\end {array} \right]
\end{MapleOutput}
結果を最初の引数(A)に上書きするoption(inplace=true)をつける.最初からではなく,うまくいったのを確認してからつけるのがコツ.
\begin{MapleInput}
> RowOperation(A,[3,1],-7,inplace=true);
\end{MapleInput}
\begin{MapleOutput}
\left[ \begin {array}{ccc} 1&2&3\\  0&-3&-6\\  0&-6&-12\end {array} \right] 
\end{MapleOutput}
\begin{MapleInput}
> RowOperation(A,2,-1/3,inplace=true);
\end{MapleInput}
\begin{MapleOutput}
\left[ \begin {array}{ccc} 1&2&3\\  0&1&2\\  0&-6&-12\end {array} \right]
\end{MapleOutput}
\begin{MapleInput}
> RowOperation(A,[3,2],6);
\end{MapleInput}
\begin{MapleOutput}
\left[ \begin {array}{ccc} 1&2&3\\  0&1&2\\  0&0&0\end {array} \right]
\end{MapleOutput}
LUDecompositionによる結果と見比べる.
\begin{MapleInput}
> A0:=Matrix([[1,2,3],[4,5,6],[7,8,9]]):
> LUDecomposition(A0);
\end{MapleInput}
\begin{MapleOutput}
\left[ \begin {array}{ccc} 1&0&0\\  0&1&0\\  0&0&1\end {array} \right] ,\, \left[ \begin {array}{ccc} 1&0&0\\  4&1&0\\  7&2&1\end {array} \right] ,\, \left[ \begin {array}{ccc} 1&2&3\\  0&-3&-6\\  0&0&0\end {array} \right] 
\end{MapleOutput}
最後の行がすべて0になっているので,階数は2となる.Rankにより確認.
\begin{MapleInput}
> Rank(A);
\end{MapleInput}
\begin{MapleOutput}
2
\end{MapleOutput}

\item
i) GenerateMatrixによる係数行列と右辺のベクトルを生成する方法は以下のとおり.
\begin{MapleInput}
> eqs:={x+y-z=2,2*x-3*y+z=4,4*x-y+3*z=1}; GenerateMatrix(eqs,{x,y,z});
\end{MapleInput}
\begin{MapleOutputGather}
{\it eqs}\, := \, \left\{ x+y-z=2,2\,x-3\,y+z=4,4\,x-y+3\,z=1 \right\} \notag \\
\left[ \begin {array}{ccc} 1&1&-1\\  2&-3&1\\  4&-1&3\end {array} \right] ,\, \left[ \begin {array}{c} 2\\  4\\  1\end {array} \right] \notag
\end{MapleOutputGather}

i)
\begin{MapleInput}
> A:= Matrix([[1,1,-1],[2,-3,1],[4,-1,3]]); b:=<2,4,1>;
> LUDecomposition(<A|b>,output='R');
\end{MapleInput}
\begin{MapleOutputGather}
A\, := \, \left[ \begin {array}{ccc} 1&1&-1\\  2&-3&1\\  4&-1&3\end {array} \right]  \notag \\
b\, := \, \left[ \begin {array}{c} 2\\  4\\  1\end {array} \right]  \notag \\
\left[ \begin {array}{cccc} 1&0&0&{\frac {13}{10}}\\  0&1&0&-{\frac {21}{20}}\\  0&0&1&-7/4\end {array} \right]  \notag
\end{MapleOutputGather}

ii)
\begin{MapleInput}
> A:= Matrix([[2,4,-3],[3,-8,6],[8,-2,-9]]); b:=<1,5,-23>;
> LUDecomposition(<A|b>,output='R');
\end{MapleInput}
\begin{MapleOutputGather}
A\, := \, \left[ \begin {array}{ccc} 2&4&-3\\  3&-8&6\\  8&-2&-9\end {array} \right]  \notag \\
b\, := \, \left[ \begin {array}{c} 1\\  5\\  -23\end {array} \right]  \notag \\
\left[ \begin {array}{cccc} 1&0&0&1\\  0&1&0&2\\  0&0&1&3\end {array} \right]  \notag
\end{MapleOutputGather}

iii)
\begin{MapleInput}
> A:= Matrix([[1,-10,-3,-7],[2,-4,3,4],[3,-2,6,5],[1,8,9,3]]); b:=<2,-3,-1,5>;
> LUDecomposition(<A|b>,output='R');
\end{MapleInput}
\begin{MapleOutputGather}
A\, := \, \left[ \begin {array}{cccc} 1&-10&-3&-7\\  2&-4&3&4\\  3&-2&6&5\\  1&8&9&3\end {array} \right]  \notag \\
b\, := \, \left[ \begin {array}{c} 2\\  -3\\  -1\\  5\end {array} \right]   \notag \\
\left[ \begin {array}{ccccc} 1&0&0&0&1\\  0&1&0&0&1/2\\  0&0&1&0&1/3\\  0&0&0&1&-1\end {array} \right]  \notag
\end{MapleOutputGather}

iv)
\begin{MapleInput}
> restart; with(LinearAlgebra): A:= Matrix([[1,1,1],[a,b,c],[b*c,c*a,a*b]]);
> bb:=<a+b+c,a*b+b*c+c*a,3*a*b*c>; RR:=LUDecomposition(<A|bb>,output='R');
\end{MapleInput}
\begin{MapleOutputGather}
A\, := \, \left[ \begin {array}{ccc} 1&1&1\\  a&b&c\\  bc&ca&ab\end {array} \right]   \notag \\
{\it bb}\, := \, \left[ \begin {array}{c} a+b+c\\  ab+bc+ca\\  3\,abc\end {array} \right]   \notag \\
{\it RR}\, := \, \left[ \begin {array}{cccc} 1&0&0&-{\frac {a \left( {b}^{2}-2\,bc+{c}^{2} \right) }{ \left( a-b \right)  \left( a-c \right) }}\\  0&1&0&{\frac { \left( {a}^{2}-2\,ca+{c}^{2} \right) b}{ \left( b-c \right)  \left( a-b \right) }}\\  0&0&1&-{\frac {c \left( -2\,ab+{b}^{2}+{a}^{2} \right) }{ab-bc+{c}^{2}-ca}}\end {array} \right]   \notag
\end{MapleOutputGather}

\begin{MapleInput}
> factor(Column(RR,4)[1]);
\end{MapleInput}
などとすればさらに見やすく,変形される
\begin{MapleOutput}
-{\frac {a \left( b-c \right) ^{2}}{ \left( a-b \right)  \left( a-c \right) }}
\end{MapleOutput}
3.
\begin{MapleInput}
> eqs:={x1+2*x2-x3=0,x1+x2+3*x4=0,3*x1+5*x2-2*x3+3*x4=0,x1+3*x2-2*x3-3*x4=0};
> solve(eqs,{x1,x2,x3,x4}); 
\end{MapleInput}
\begin{MapleOutputGather}
{\it eqs}\, := \, \left\{ {\it x1}+{\it x2}+3\,{\it x4}=0,{\it x1}+2\,{\it x2}-{\it x3}=0, \right. \notag \\
\left. {\it x1}+3\,{\it x2}-2\,{\it x3}-3\,{\it x4}=0,3\,{\it x1}+5\,{\it x2}-2\,{\it x3}+3\,{\it x4}=0 \right\} \notag \\
\left\{ {\it x1}=-6\,{\it x4}-{\it x3},{\it x2}=3\,{\it x4}+{\it x3},{\it x3}={\it x3},{\it x4}={\it x4} \right\} \notag
\end{MapleOutputGather}

\begin{MapleInput}
> A1,b:=GenerateMatrix(eqs,[x1,x2,x3,x4]);
> LUDecomposition(<A1|b>,output='R');
\end{MapleInput}
\begin{MapleOutputGather}
{\it A1},\,b\, := \, \left[ \begin {array}{cccc} 1&1&0&3\\  1&2&-1&0\\  1&3&-2&-3\\  3&5&-2&3\end {array} \right] ,\, \left[ \begin {array}{c} 0\\  0\\  0\\  0\end {array} \right]   \notag \\
\left[ \begin {array}{ccccc} 1&0&1&6&0\\  0&1&-1&-3&0\\  0&0&0&0&0\\  0&0&0&0&0\end {array} \right]   \notag
\end{MapleOutputGather}
\end{enumerate} 

\pagebreak
\section{逆行列(MatrixInverse)}
\ChartElementTwo{解説}
線形代数の計算にはあらかじめ関数パッケージ(LinearAlgebra)を呼び出しておく.
\begin{MapleInput}
> with(LinearAlgebra):
\end{MapleInput}

\subsection{行列式(Determinant)}
\begin{MapleInput}
> A0 := Matrix([[x,y],[z,u]]); Determinant(A0);
\end{MapleInput}
\begin{MapleOutputGather}
{\it A0}\, := \, \left[ \begin {array}{cc} x&y\\  z&u\end {array} \right] \notag \\
xu-yz \notag
\end{MapleOutputGather}

\subsection{逆行列(MatrixInverse)}
\begin{MapleInput}
> A2:=MatrixInverse(A0); simplify(A0.A2);
\end{MapleInput}
\begin{MapleOutputGather}
{\it A2}\, := \, \left[ \begin {array}{cc} {\frac {u}{xu-yz}}&-{\frac {y}{xu-yz}}\\  -{\frac {z}{xu-yz}}&{\frac {x}{xu-yz}}\end {array} \right]\notag \\
\left[ \begin {array}{cc} 1&0\\  0&1\end {array} \right]\notag
\end{MapleOutputGather}

\subsection{その他の演算}
随伴(Adjoint)などもコマンドだけで求まる.詳しくはヘルプ参照.
 
\ChartElementTwo{課題}
\begin{enumerate}
\item
次の連立方程式の係数行列の行列式を求めよ.

(i)
\begin{equation*}
\left\{
\begin{array}{cc}x +y -z &=2 \\
2 x -3 y +z &=4  \\
4 x -y +3 z &=1  
\end{array} \right.
\end{equation*}

(ii)
\begin{equation*}
\left\{
\begin{array}{cl}2 x +4 y -3 z &=1   \\
3 x -8 y +6 z &=58   \\
x -2 y -9 z &=23   \\
\end{array} \right.
\end{equation*}

(iii)
\begin{equation*}
\left\{
\begin{array}{cl}
1 x -10 y -3 z -7 u &=2   \\
2 x -4 y +3 z +4 u &=-3   \\
x -2 y +6 z +5 u &=-1   \\
x +8 y +9 z +3 u &=5
\end{array} \right.
\end{equation*}
(iv)
\begin{equation*}
\left\{\begin{array}{cl}x +y +z &=a +b +c    \\
ax+by+cz &=ab +bc +ca   \\
bc\,x +ca\,y + ab\,z &=3\,abc 
\end{array} \right.
\end{equation*}

\item
上の連立方程式の係数行列の逆行列を求めよ.またベクトルbに作用して解を求めよ.
\end{enumerate}
 
\ChartElementTwo{解答例}
\begin{enumerate}
\item
\begin{MapleInput}
> with(LinearAlgebra): eqs:={x+y-z=2,2*x-3*y+z=4,4*x-y+3*z=1};
> A,b:=GenerateMatrix(eqs,{x,y,z}); Determinant(A);
\end{MapleInput}
\begin{MapleOutputGather}
{\it eqs}\, := \, \left\{ x+y-z=2,2\,x-3\,y+z=4,4\,x-y+3\,z=1 \right\}  \notag \\
A,\,b\, := \, \left[ \begin {array}{ccc} 1&1&-1\\  2&-3&1\\  4&-1&3\end {array} \right] ,\, \left[ \begin {array}{c} 2\\  4\\  1\end {array} \right]\notag \\
-20\notag 
\end{MapleOutputGather}

\item
\begin{MapleInput}
> MatrixInverse(A); simplify(MatrixInverse(A).b);
\end{MapleInput}
\begin{MapleOutputGather}
\left[ \begin {array}{ccc} 2/5&1/10&1/10\\  1/10&-{\frac {7}{20}}&{\frac {3}{20}}\\  -1/2&-1/4&1/4\end {array} \right] \notag \\
\left[ \begin {array}{c} {\frac {13}{10}}\\  -{\frac {21}{20}}\\  -7/4\end {array} \right]\notag 
\end{MapleOutputGather}
\end{enumerate} 

\pagebreak
\section{固有値(EigenVectors)}
\ChartElementTwo{解説}
線形代数の計算にはあらかじめ関数パッケージ(LinearAlgebra)を呼び出しておく.
\begin{MapleInput}
> with(LinearAlgebra):
\end{MapleInput}

\subsection{固有値(EigenVectors)}
固有値(Eigenvalues)と固有ベクトルを共に求めるにはEigenvectorsを使う.下の例では,固有値と固有ベクトルを変数l,vに代入している.
\begin{MapleInput}
> A0 := Matrix(2, 2, [[1,2], [2,1]]); l,v:=Eigenvectors(A0);
\end{MapleInput}
\begin{MapleOutputGather}
{\it A0}\, := \, \left[ \begin {array}{cc} 1&2\\  2&1\end {array} \right] \notag \\
l,\,v\, := \, \left[ \begin {array}{c} -1\\  3\end {array} \right] ,\, \left[ \begin {array}{cc} -1&1\\  1&1\end {array} \right] \notag
\end{MapleOutputGather}
\subsection{固有ベクトルの取り出し(Column)}
行列の列を要素とするベクトル生成Columnを使って,一番目の固有値に対応する固有ベクトルを取り出す.
\begin{MapleInput}
> Column(v,1);
\end{MapleInput}
\begin{MapleOutput}
\left[ \begin {array}{c} -1\\  1\end {array} \right]
\end{MapleOutput}
これを使って,固有値(l)と固有ベクトル(v)の関係
\begin{equation*}
A_0.v=\lambda.v
\end{equation*}
が確認できる.
\begin{MapleInput}
> A0.Column(v,1); l[1]*Column(v,1);
\end{MapleInput}
\begin{MapleOutputGather}
\left[ \begin {array}{c} 1\\  -1\end {array} \right] \notag \\
\left[ \begin {array}{c} 1\\  -1\end {array} \right] \notag
\end{MapleOutputGather}

\subsection{固有ベクトルの規格化(Normalize)}
用意されているコマンドが確かめられる.
\begin{MapleInput}
> ?Normalize;
\end{MapleInput}
一般的な内積を使って長さを規格化するには,以下のコマンドを使う.
\begin{MapleInput}
> Normalize(Column(v,1),Euclidean);
\end{MapleInput}
\begin{MapleOutput}
\left[ \begin {array}{c} -1/2\,\sqrt {2}\\  1/2\,\sqrt {2}\end {array} \right] 
\end{MapleOutput}

\subsection{対角化}
固有ベクトルを用いて,次のとおり行列は対角化される.
\begin{MapleInput}
> MatrixInverse(v).A0.v;
\end{MapleInput}
\begin{MapleOutput}
\left[ \begin {array}{cc} -1&0\\  0&3\end {array} \right]
\end{MapleOutput}
一見対角化されてない場合でも,simplifyを掛けて整理すると対角化されているのが確認できる.

\subsection{その他の演算}
対角和(Trace),ジョルダン標準形(JordanForm)などもコマンドだけで求まる.詳しくはヘルプ参照.
 
\ChartElementTwo{課題}
\begin{enumerate}
\item
行列
\begin{equation*}
A= \left[ \begin {array}{ccc} 1&-2&1\\  -1&2&1\\  1&2&1\end {array} \right]
\end{equation*}
の固有値を固有方程式
\begin{equation*}
\left| A-\lambda E \right| =0
\end{equation*}
を解いて求めよ.EigenVectorsを用いて固有値と固有ベクトルを求めよ.固有値,固有ベクトルの関係
\begin{equation*}
A.v  =\lambda v
\end{equation*}
を確認せよ.さらに,固有ベクトルを長さ1に規格化せよ.

\item
行列
\begin{equation*}
A= \left[ \begin {array}{ccc} 2&0&1\\  0&3&0\\  1&0&2\end {array} \right]
\end{equation*}
を対角化する変換行列Pを求め,対角化せよ.
\end{enumerate} 
\ChartElementTwo{解答例}
\begin{enumerate}
\item
\begin{MapleInput}
> A:=Matrix([[1,-2,1],[-1,2,1],[1,2,1]]); E:=Matrix(3,3,shape=identity):
> eq:=Determinant(A-x*E); solve(eq,x);
\end{MapleInput}
\begin{MapleOutputGather}
A\, := \, \left[ \begin {array}{ccc} 1&-2&1\\  -1&2&1\\  1&2&1\end {array} \right] \notag \\
{\it eq}\, := \,4\,{x}^{2}-{x}^{3}-8 \notag \\
2,\,1+ \sqrt{5},\,1- \sqrt{5} \notag
\end{MapleOutputGather}
\begin{MapleInput}
> l,v:=Eigenvectors(A); v1:=Column(v,3); evalf(A.v1); evalf(l[3].v1);
> Normalize(v1,Euclidean); evalf(Normalize(v1,Euclidean));
\end{MapleInput}
\begin{MapleOutputGather}
l,\,v\, := \, \left[ \begin {array}{c} \sqrt {5}+1\\  1-\sqrt {5}\\  2\end {array} \right] ,\, \left[ \begin {array}{ccc} {\frac { \left( \sqrt {5}-3 \right) \sqrt {5}}{-5+3\,\sqrt {5}}}&-{\frac { \left( -3-\sqrt {5} \right) \sqrt {5}}{-5-3\,\sqrt {5}}}&1\\  -{\frac {-5+\sqrt {5}}{-5+3\,\sqrt {5}}}&-{\frac {-5-\sqrt {5}}{-5-3\,\sqrt {5}}}&0\\  1&1&1\end {array} \right] \notag \\
{\it v1}\, := \, \left[ \begin {array}{c} 1\\  0\\  1\end {array} \right] \notag \\
\left[ \begin {array}{c}  2.0\\   0.0\\   2.0\end {array} \right] \notag \\
\left[ \begin {array}{c}  2.0\\   0.0\\   2.0\end {array} \right] \notag \\
\left[ \begin {array}{c} 1/2\,\sqrt {2}\\  0\\  1/2\,\sqrt {2}\end {array} \right] \notag \\
\left[ \begin {array}{c}  0.7071067810\\   0.0\\   0.7071067810\end {array} \right] \notag \\
\end{MapleOutputGather}

\item
\begin{MapleInput}
> A:=Matrix([[2,0,1],[0,3,0],[1,0,2]]); l,v:=Eigenvectors(A);
> MatrixInverse(v).A.v;
\end{MapleInput}
\begin{MapleOutputGather}
A\, := \, \left[ \begin {array}{ccc} 2&0&1\\  0&3&0\\  1&0&2\end {array} \right] \notag \\
l,\,v\, := \, \left[ \begin {array}{c} 3\\  3\\  1\end {array} \right] ,\, \left[ \begin {array}{ccc} 1&0&-1\\  0&1&0\\  1&0&1\end {array} \right] \notag \\
\left[ \begin {array}{ccc} 3&0&0\\  0&3&0\\  0&0&1\end {array} \right] \notag
\end{MapleOutputGather}
\end{enumerate} 

\chapter{式変形}
\section{数式処理コマンドの分類(EquationManipulationTable)}
\ChartElementTwo{解説}
数式の変形は,手で直すほうが圧倒的に早くきれいになる場合が多い.しかし,テイラー展開や,複雑な積分公式,三角関数とexp関数の変換などの手間がかかるところを,Mapleは間違いなく変形してくれる.ここで示すコマンドを全て覚える必要は全くない.というか忘れるもの.ここでは,できるだけコンパクトにまとめて,悩んだときに参照できるようにする.初めての人は,ざっと眺めた後,鉄則からじっくりフォローせよ.

\begin{table}[htbp]
\caption{数式処理で頻繁に使うコマンド.}
\begin{center}
\begin{tabular}{llll}
\hline
式の変形&式の分割抽出&省略操作,その他 &代入,置換,仮定\\
\hline
simplify:簡単化 &lhs, rhs:左辺,右辺	&||:連結作用素 &subs:一時的代入	\\
expand:展開		&numer, denom:分子,分母&seq:for-loopの簡易表記&restart,a:='a':初期化\\
factor:因数分解	&coeff:係数				&map:関数の要素への適用&assume:仮定			\\
normal:約分・通分 &nops, op				&add,mul:単純な和,積&assuming:一時的仮定\\
combine:公式でまとめる&					&sum,product:数式に対応した和,積&assign:値の確定	\\
collect:次数でまとめる&					&limit:極限&about:仮定の中身	\\
sort:昇べき,降べき&					&	&anames('user'):使用変数名\\
rationalize:分母の有理化\\
convert:形式の変換\\
\hline
\end{tabular}
\end{center}
\label{default}
\end{table}%

このほかにも,solve(解), diff(微分), int(積分),series(級数展開)等は頻繁に数式の導出・変形に登場する.





 

\pagebreak
\section{式の変形に関連したコマンド(Simplify)}
\ChartElementTwo{解説}
\subsection{簡単化(simplify)} 
\begin{MapleInput}
> simplify(exp1,副関係式):
> simplify(3*x+4*x+2*y);
\end{MapleInput}
\begin{MapleOutput}
7\,x+2\,y
\end{MapleOutput}
\begin{MapleInput}
> exp1:=3*sin(x)^3-sin(x)*cos(x)^2; 
> simplify(exp1);
\end{MapleInput}
\begin{MapleOutputGather}
{\it exp1}\, := \,3\, \left( \sin \left( x \right)  \right) ^{3}-\sin \left( x \right)  \left( \cos \left( x \right)  \right) ^{2} \notag \\
- \left( 4\, \left( \cos \left( x \right)  \right) ^{2}-3 \right) \sin \left( x \right) \notag
\end{MapleOutputGather}
\begin{MapleInput}
> simplify(exp1,{cos(x)^2=1-sin(x)^2});
\end{MapleInput}
\begin{MapleOutput}
4\, \left( \sin \left( x \right)  \right) ^{3}-\sin \left( x \right)
\end{MapleOutput}
オプションとしてsizeを指定するとより簡単になる場合がある.
\begin{MapleInput}
> simplify(exp1,size):
\end{MapleInput}

\subsection{展開(expand)} 
\begin{MapleInput}
> expand((x+y)^2);
\end{MapleInput}
\begin{MapleOutput}
{x}^{2}+2\,xy+{y}^{2}
\end{MapleOutput}

\subsection{因数分解(factor)} 
\begin{MapleInput}
> factor(4*x^2-6*x*y+2*y^2);
\end{MapleInput}
\begin{MapleOutput}
2\, \left( 2\,x-y \right)  \left( x-y \right)
\end{MapleOutput}

\subsection{約分・通分(normal)} 
\begin{MapleInput}
> normal((x+y)/(x^2-3*x*y-4*y^2));
\end{MapleInput}
\begin{MapleOutput}
\frac{1}{ x-4\,y }
\end{MapleOutput}
\begin{MapleInput}
> normal(1/x+1/y);
 \end{MapleInput}
\begin{MapleOutput}
{\frac {y+x}{xy}}
\end{MapleOutput}

\subsection{項を変数でまとめる(collect)} 
\begin{MapleInput}
> collect(4*a*x^2-3*y^2/x+6*b*x*y+3*c*y+2*y^2,y);
\end{MapleInput}
\begin{MapleOutput}
\left( -3\,{x}^{-1}+2 \right) {y}^{2}+ \left( 6\,bx+3\,c \right) y+4\,a{x}^{2}
\end{MapleOutput}

\subsection{項を公式でまとめる(combine)} 
\begin{MapleInput}
> combine(sin(x)^2+3*cos(x)^2);
\end{MapleInput}
\begin{MapleOutput}
2+\cos \left( 2\,x \right)
\end{MapleOutput}

\subsection{昇べき,降べき(sort)} 
\begin{MapleInput}
> sort(exp1,[x,y]); 
> sort(exp1, [x],opts);
opts=tdeg(総次数順), plex(辞書式順), ascending(昇順), descending(降順)
\end{MapleInput}

\begin{MapleInput}
> exp1:=x^3+4*x-3*x^2+1: 
> sort(exp1);
\end{MapleInput}
\begin{MapleOutput}
{x}^{3}-3\,{x}^{2}+4\,x+1
\end{MapleOutput}

\begin{MapleInput}
> sort(exp1,[x],ascending);
\end{MapleInput}
\begin{MapleOutput}
1+4\,x-3\,{x}^{2}+{x}^{3}
\end{MapleOutput}

\begin{MapleInput}
> exp2:=x^3-a*x*y+4*x^2+y^2: 
> sort(exp2);
\end{MapleInput}
\begin{MapleOutput}
-axy+{x}^{3}+4\,{x}^{2}+{y}^{2}
\end{MapleOutput}

\begin{MapleInput}
> sort(exp2,[x]);
\end{MapleInput}
\begin{MapleOutput}
{x}^{3}+4\,{x}^{2}-ayx+{y}^{2}
\end{MapleOutput}

\begin{MapleInput}
> sort(exp2,[a,y]); 
> sort(exp2,[a],plex);
\end{MapleInput}
\begin{MapleOutputGather}
-xay+{y}^{2}+{x}^{3}+4\,{x}^{2} \notag \\
-xya+{x}^{3}+4\,{x}^{2}+{y}^{2} \notag
\end{MapleOutputGather}

\subsection{分母の有理化(rationalize)} 

\begin{MapleInput}
> eq1:=(1+sqrt(2))/(1-sqrt(3));
\end{MapleInput}
分母の有理化は,この式を
\begin{MapleOutput}
{\frac {1+\sqrt {2}}{1-\sqrt {3}}}{\frac {1+\sqrt {3}}{1+\sqrt {3}}}
\end{MapleOutput}
とする事に対応します.
\begin{MapleInput}
> rationalize(eq1);
\end{MapleInput}
\begin{MapleOutput}
-1/2\, \left( 1+\sqrt {2} \right)  \left( 1+\sqrt {3} \right) 
\end{MapleOutput}

 
\ChartElementTwo{課題}
1. 以下の式を簡単化せよ.

i) $x^{100}-1$, ii) $x^2-y^2+2x+1$, iii) $(a+b+c)^3-(a^3+b^3+c^3)$

 
\ChartElementTwo{解答例}
\begin{MapleInput}
> factor(x^100-1);
\end{MapleInput}
\begin{MapleOutputGather}
\left( x-1 \right)  \left( 1+{x}^{4}+{x}^{3}+{x}^{2}+x \right)  \left( 1+{x}^{20}+{x}^{15}+{x}^{10}+{x}^{5} \right) \notag \\
\left( 1+x \right)  \left( 1-x+{x}^{2}-{x}^{3}+{x}^{4} \right)  \left( 1-{x}^{5}+{x}^{10}-{x}^{15}+{x}^{20} \right) \notag  \\
\left( 1+{x}^{2} \right)  \left( {x}^{8}-{x}^{6}+{x}^{4}-{x}^{2}+1 \right)  \left( {x}^{40}-{x}^{30}+{x}^{20}-{x}^{10}+1 \right)  \notag 
\end{MapleOutputGather}

\begin{MapleInput}
> factor( x^2-y^2+2*x+1);
\end{MapleInput}
\begin{MapleOutput}
\left( x+1+y \right)  \left( x+1-y \right) 
\end{MapleOutput}

\begin{MapleInput}
> factor((a+b+c)^3-(a^3+b^3+c^3));
\end{MapleInput}
\begin{MapleOutput}
3\, \left( b+c \right)  \left( c+a \right)  \left( a+b \right)
\end{MapleOutput}

 

\pagebreak
\section{式の分割抽出(convert)}
\ChartElementTwo{解説}
\subsection{形式の変換(convert(exp1,opt))}
数式の記述形式を変えるのに頻繁に使う.
\begin{table}[h]
\begin{tabular}{l|l}
\hline
opt&意味 \\ \hline 
polynom&級数を多項式(polynomial)に\\
trig&三角関数(trigonal)に\\
sincos&tanを含まない,sin,cosに\\
exp&指数関数形式に\\
parfrac&部分分数(partial fraction)に\\
rational&浮動小数点数を有理数形式に \\ \hline
\end{tabular}
\end{table}%

\begin{MapleInput}
> s1:=series(sin(x),x,4);
> convert(s1,polynom);
\end{MapleInput}
\begin{MapleOutputGather}
{\it s1}\, := \,x-\frac{1}{6}\,{x}^{3}+O \left( {x}^{4} \right) \notag \\
x-\frac{1}{6}\,{x}^{3} \notag
\end{MapleOutputGather}

\begin{MapleInput}
> convert(sin(x),exp);
\end{MapleInput}
\begin{MapleOutput}
-\frac{1}{2}\,I\left( \exp(Ix)-\exp(-Ix) \right) 
\end{MapleOutput}

\begin{MapleInput}
> convert(sinh(x),exp);
\end{MapleInput}
\begin{MapleOutput}
\frac{1}{2}\,\exp(x)-\frac{1}{2}\,\exp(x)
\end{MapleOutput}

\begin{MapleInput}
> convert(tan(x),sincos);
\end{MapleInput}
\begin{MapleOutput}
{\frac {\sin \left( x \right) }{\cos \left( x \right) }}
\end{MapleOutput}

\begin{MapleInput}
> convert(exp(I*x),trig);
\end{MapleInput}
\begin{MapleOutput}
\cos \left( x \right) +I\sin \left( x \right)
\end{MapleOutput}

\begin{MapleInput}
> convert(1/(x-1)/(x+3),parfrac);
\end{MapleInput}
\begin{MapleOutput}
- \frac{1}{4\left( x+3 \right) }+ \frac{1}{4 \left( x-1 \right) }
\end{MapleOutput}

\begin{MapleInput}
> convert(3.14,rational);
\end{MapleInput}
\begin{MapleOutput}
{\frac {157}{50}}
\end{MapleOutput}

\subsection{左辺,右辺(lhs, rhs)}
それぞれ,左辺,右辺を意味するleft hand side, right hand sideの略
\begin{MapleInput}
> lhs(sin(x)^2=1-1/x);
> rhs(sin(x)^2=1-1/x);
\end{MapleInput}
\begin{MapleOutputGather}
\sin \left( x \right)^{2} \notag \\
1-\frac{1}{x} \notag
\end{MapleOutputGather}

\subsection{分母,分子(denom,numer)} 
それぞれ,分母,分子を意味するdenominator, numeratorの略
\begin{MapleInput}
> numer(a*x/(x+y)^3);
> denom(a*x/(x+y)^3);
\end{MapleInput}
\begin{MapleOutputGather}
ax \notag \\
\left( x+y \right) ^{3} \notag
\end{MapleOutputGather}

\subsection{係数(coeff)} 
\begin{MapleInput}
> coeff(4*a*x^2-3*y^2/x+6*b*x*y+3*c*y+2*y^2,y^2);
\end{MapleInput}
\begin{MapleOutput}
-\frac{3}{x}+2
\end{MapleOutput}

\subsection{要素の取りだし,要素数(op, nops)}
op, nopsはlist配列から要素や要素数を取り出すのに頻繁に使われる.数式を含めた,より一般的な構造に対しても作用させることができる.
\begin{MapleInput}
> op(4*a*x^2-3*y^2/x+6*b*x*y+3*c*y+2*y^2);
\end{MapleInput}
\begin{MapleOutput}
4\,a{x}^{2},\,-{\frac {{3\,y}^{2}}{x}},\,6\,bxy,\,3\,cy,\,2\,{y}^{2}
\end{MapleOutput}

\begin{MapleInput}
> nops(4*a*x^2-3*y^2/x+6*b*x*y+3*c*y+2*y^2); #res: 5
\end{MapleInput}
 
\ChartElementTwo{課題}
\begin{enumerate}
\item 以下の関数をx0まわりで3次までテイラー展開し,得られた関数ともとの関数をプロットせよ.さらに高次まで展開した場合はどう変化するか.

i) $y =\cos \left(x \right),{\it x0} =0$, ii) $y =\ln\left(x \right),{\it x0} =1$, iii) $y =\exp \left(-x \right),{\it x0} =0$
\item $\displaystyle \frac{x+1}{(x-1)(x^2+1)^2}$を部分分数に展開せよ.
\item $\displaystyle \frac{1}{1-x^4} = \frac{a}{x^2+1}+\frac{b}{x+1}+\frac{c}{x-1}$が常に成立する$a, b, c$を定めよ.
\item $\displaystyle \frac{8}{3-\sqrt{5}}-\frac{2}{2+\sqrt{5}}$を簡単化せよ.
\item $\displaystyle x^2+2kx+5-k=0$が重根をもつように$k$を定めよ.
\end{enumerate} 
\ChartElementTwo{解答例}
\begin{enumerate}
\item seriesでTaylor展開した後,convertで多項式(polynom)に変換する.
\begin{MapleInput}
> convert(series(cos(x),x),polynom);
\end{MapleInput}
\begin{MapleOutput}
1-\frac{1}{2}\,{x}^{2}+\frac{1}{24}\,{x}^{4}
\end{MapleOutput}

\begin{MapleInput}
> plot([convert(series(cos(x),x),polynom),cos(x)],x=-Pi..Pi);
\end{MapleInput}
\begin{MapleOutput}
1-\frac{1}{2}\,{x}^{2}+\frac{1}{24}\,{x}^{4}-{\frac {1}{720}}\,{x}^{6}+{\frac {1}{40320}}\,{x}^{8}
\end{MapleOutput}
\MaplePlot{50mm}{./figures/EqManip3plot2d1.eps}
高次展開する場合には,seriesの最後の引数に次数を指定する.デフォルトは6次.関数の一致具合が向上していることに注目.
\begin{MapleInput}
convert(series(cos(x),x,9),polynom);
plot([convert(series(cos(x),x,9),polynom),cos(x)],x=-Pi..Pi);
\end{MapleInput}
\MaplePlot{50mm}{./figures/EqManip3plot2d2.eps}

seriesでの展開で,x=1とするとその周りでの展開になる.
\begin{MapleInput}
> convert(series(ln(x),x=1),polynom);
\end{MapleInput}
\begin{MapleOutput}
x-1-\frac{1}{2}\, \left( x-1 \right) ^{2}+\frac{1}{3}\, \left( x-1 \right) ^{3}-\frac{1}{4}\, \left( x-1 \right) ^{4}+\frac{1}{5}\, \left( x-1 \right) ^{5}
\end{MapleOutput}

\begin{MapleInput}
> plot([convert(series(ln(x),x=1),polynom),ln(x)],x=0..2);
\end{MapleInput}
\MaplePlot{50mm}{./figures/EqManip3plot2d3.eps}

\begin{MapleInput}
> convert(series(exp(-x),x),polynom);
\end{MapleInput}
\begin{MapleOutput}
1-x+\frac{1}{2}\,{x}^{2}-\frac{1}{6}\,{x}^{3}+\frac{1}{24}\,{x}^{4}-{\frac {1}{120}}\,{x}^{5}
\end{MapleOutput}

\begin{MapleInput}
> plot([convert(series(exp(-x),x),polynom),exp(-x)],x=0..2);
\end{MapleInput}
\MaplePlot{50mm}{./figures/EqManip3plot2d4.eps}

\item
\begin{MapleInput}
> restart; eq1:=(x+1)/((x-1)*(x^2+1)^2);
\end{MapleInput}
\begin{MapleOutput}
{\it eq1}\, := \,{\frac {x+1}{ \left( x-1 \right)  \left( {x}^{2}+1 \right) ^{2}}}
\end{MapleOutput}

\begin{MapleInput}
> convert(eq1,parfrac);
\end{MapleInput}
\begin{MapleOutput}
\frac{1}{2 \left(x-1 \right) }-{\frac {x}{ \left( {x}^{2}+1 \right) ^{2}}}+\frac{1}{2}\,{\frac {-x-1}{{x}^{2}+1}}
\end{MapleOutput}

\item
\begin{MapleInput}
> convert(1/(1-x^4),parfrac);
\end{MapleInput}
\begin{MapleOutput}
- \frac{1}{4 \left( x-1 \right)}+ \frac{1}{ 4\,\left(x+1 \right) }+ \frac{1}{2\,\left( {x}^{2}+1 \right)}
\end{MapleOutput}

\item
\begin{MapleInput}
> eq2:=8/(3-sqrt(5))-2/(2+sqrt(5));
\end{MapleInput}
\begin{MapleOutput}
{\it eq2}\, := \frac{8}{3- \sqrt{5}}-\frac{2}{2+ \sqrt{5}}
\end{MapleOutput}
まずはデフォルトの簡単化(simplify).
\begin{MapleInput}
> simplify(eq2); #res: 10
\end{MapleInput}

\item
方程式をたてる.
\begin{MapleInput}
> eq3:=x^2+2*k*x+(5-k);
\end{MapleInput}
\begin{MapleOutput}
{\it eq3}\, := \,{x}^{2}+2\,kx+5-k
\end{MapleOutput}
単純に解を求めて,
\begin{MapleInput}
> sol1:=solve(eq3=0,x);
\end{MapleInput}
\begin{MapleOutput}
{\it sol1}\, := \,-k+ \sqrt{{k}^{2}-5+k},\,-k- \sqrt{{k}^{2}-5+k}
\end{MapleOutput}
それが一致する場合のkを解く.
\begin{MapleInput}
> solve(sol1[1]=sol1[2],k);
\end{MapleInput}
\begin{MapleOutput}
-\frac{1}{2}+\frac{1}{2}\, \sqrt{21},\,-\frac{1}{2}-\frac{1}{2}\, \sqrt{21}
\end{MapleOutput}
別解.まず,係数をcoeffで取り出す.
\begin{MapleInput}
> aa:=coeff(eq3,x^2); #res:1
> bb:=coeff(eq3,x);   #res: 2 k
> cc:=coeff(eq3,x,0); #res: 5 - k
\end{MapleInput}
判別式$D=b^2-4ac$を計算して,$=0$とおいてkについて解く.
\begin{MapleInput}
> solve(bb^2-4*aa*cc=0,k);
\end{MapleInput}
\begin{MapleOutput}
-\frac{1}{2}+\frac{1}{2}\, \sqrt{21},\,-\frac{1}{2}-\frac{1}{2}\, \sqrt{21}
\end{MapleOutput}
\end{enumerate} 

\pagebreak
\section{代入,置換,仮定に関連したコマンド(assume, subs)}
\ChartElementTwo{解説}
\subsection{一時的代入(subs)} 
関係式(x=2)を式(exp1)に一時的に代入した結果を表示.
\begin{MapleInput}
> exp1:=x^2-4*x*y+4; subs(x=2,exp1);
\end{MapleInput}
\begin{MapleOutputGather}
{\it exp1}\, := \,{x}^{2}-4\,xy+4 \notag \\
8-8\,y \notag
\end{MapleOutputGather}
\begin{MapleInput}
> subs({x=a+2,y=sin(x)},exp1);
\end{MapleInput}
\begin{MapleOutput}
\left( a+2 \right) ^{2}-4\, \left( a+2 \right) \sin \left( x \right) +4
\end{MapleOutput}

\subsection{仮定(assume)}
変数になにかの条件を仮定するときに使う.たとえば,根を開くときに,与式が正と仮定すると開かれる.
\begin{MapleInput}
> sqrt(b^2); 
> assume(a>0); sqrt(a^2);
\end{MapleInput}
\begin{MapleOutputGather}
\sqrt{{b}^{2}} \notag \\
a\sim \notag
\end{MapleOutputGather}

\subsection{一時的仮定(assuming)} 
assumeと同じだが,一時的に仮定するときに使われる.
\begin{MapleInput}
> exp1:=x^2-4*x+4; 
> sqrt(exp1);
\end{MapleInput}
\begin{MapleOutputGather}
{\it exp1}\, := \,{x}^{2}-4\,x+4 \notag \\
\sqrt{ \left( -2+x \right) ^{2}} \notag
\end{MapleOutputGather}

\begin{MapleInput}
> sqrt(exp1) assuming x>2;
\end{MapleInput}
\begin{MapleOutput}
-2+x
\end{MapleOutput}

assumeに加えての仮定にadditionallyがある.

\subsection{solveで求めた値の確定(assign)} 
解を求めるsolveをして
\begin{MapleInput}
> x:='x':y:='y': 
> s1:=solve({x-y+1=0,x+y-2=0},{x,y}); 
\end{MapleInput}
\begin{MapleOutput}
{\it s1}\, := \, \left\{ x=\frac{1}{2},y=\frac{3}{2} \right\}
\end{MapleOutput}
このまま,x,yの中身を見ても,
\begin{MapleInput}
> x,y;
\end{MapleInput}
\begin{MapleOutput}
x,\,y
\end{MapleOutput}
代入されていない.s1をassign(確定)すると,
\begin{MapleInput}
> assign(s1);
> x,y;
\end{MapleInput}
\begin{MapleOutput}
\frac{1}{2},\frac{3}{2}
\end{MapleOutput}
と代入される.一時的代入subsと使い分ける.

\subsection{assumeで仮定した内容の確認(about)} 
\begin{MapleInput}
> about(a);
\end{MapleInput}
\begin{MapleError}
Originally a, renamed a~:
  is assumed to be: RealRange(Open(0),infinity)
\end{MapleError}

\subsection{ユーザが定義した変数の確認(anames('user'))} 
\begin{MapleInput}
> anames('user');
\end{MapleInput}
\begin{MapleOutput}
{\it s1},\,y,\,x,\,a
\end{MapleOutput}

\subsection{値の初期化(restart,a:='a')} 
\subsection{連結作用素($||$)} 
前後の変数をくっつけて新たな変数とする.
\begin{MapleInput}
> a||1; #res: a1
> a||b; #res: ab
\end{MapleInput}
プログラムの中で使うとより便利.
\begin{MapleInput}
> for i from 1 to 3 do
    a||i:=i^2; 
  end do;
\end{MapleInput}
\begin{MapleOutputGather}
a1:=1 \notag \\
a2:=4 \notag \\
a3:=9 \notag
\end{MapleOutputGather}

\subsection{for-loopの簡略表記(seq)} 
数列を意味するsequenceの略.
\begin{MapleInput}
> seq(k,k=4..7);
\end{MapleInput}
\begin{MapleOutput}
4,\,5,\,6,\,7
\end{MapleOutput}

\subsection{リスト要素への関数の一括適用(map)} 
\begin{MapleInput}
> f:=x->exp(-x); 
> map(f,[0,1,2,3]);
\end{MapleInput}
\begin{MapleOutputGather}
f\, := \,x\mapsto \exp(-x)  \notag \\
[1,\exp(-1),\exp(-2),\exp(-3)] \notag
\end{MapleOutputGather}

上記の3つを組み合わせると,効率的に式を扱うことができる.
\begin{MapleInput}
> map(sin,[seq(theta||i,i=0..3)]);
\end{MapleInput}
\begin{MapleOutput}
[\sin \left( {\it \theta0} \right) ,\sin \left( {\it \theta1} \right) ,\sin \left( {\it \theta2}\\
\mbox{} \right) ,\sin \left( {\it \theta3} \right) ]
\end{MapleOutput}

\subsection{単純な和,積(add,mul)} 
\begin{MapleInput}
> add(x^i,i=0..3);
\end{MapleInput}
\begin{MapleOutput}
1+x+{x}^{2}+{x}^{3}
\end{MapleOutput}

\begin{MapleInput}
> mul(x^i,i=0..3);
\end{MapleInput}
\begin{MapleOutput}
{x}^{6}
\end{MapleOutput}

\subsection{数式にも対応した和,積(sum,product)} 
\begin{MapleInput}
> add(x^i,i=0..n);
\end{MapleInput}
\begin{MapleError}
Error, unable to execute add
\end{MapleError}
\begin{MapleInput}
> sum(x^i,i=0..n);
\end{MapleInput}
\begin{MapleOutput}
{\frac {{x}^{n+1}}{x-1}}- \frac{1}{x-1}
\end{MapleOutput}

\begin{MapleInput}
> product(x^i,i=0..n);
\end{MapleInput}
\begin{MapleOutput}
{x}^{\frac{1}{2}\, \left( n+1 \right) ^{2}-\frac{1}{2}\,n-\frac{1}{2}}
\end{MapleOutput}

\subsection{極限(limit)} 
\begin{MapleInput}
> limit(exp(-x),x=infinity); #res: 0
\end{MapleInput}

\begin{MapleInput}
> limit(tan(x),x=Pi/2);
\end{MapleInput}
\begin{MapleOutput}
{\it undefined}
\end{MapleOutput}

\begin{MapleInput}
> limit(tan(x),x=Pi/2,left); 
> limit(tan(x),x=Pi/2,complex);
\end{MapleInput}
\begin{MapleOutputGather}
\infty \notag \\
-\infty +\infty \, I\notag
\end{MapleOutputGather}

 

\pagebreak
\section{式の変形の基本(BottomLine)}
\ChartElementTwo{解説}
どうしても解かなければならない課題を前にコマンドリファレンスのあちこちを参照しながら解いていくのが数式処理を修得する最速法である.とびかかる前にちょっとした共通
のコツがある.それをここでは示す.数式処理ソフトでの数式処理とは,数式処理ソフトが『自動的にやって』くれるのではなく,実際に紙と鉛筆で解いていく手順を数式処理ソ
フトに『やらせる』ことであることを肝に銘じよ.

\subsection{鉄則}
Mapleをはじめとする数式処理ソフトの習得にあたって初心者がつまづく共通の過ちを回避する鉄則がある.
\begin{description}
\item[鉄則0:restart をかける]
続けて入力すると前の入力が生きている.違う問題へ移るときや,もう一度入力をし直すときには,restart;を入力して初期状態からはじめる.入力した順番が狂っている場合もある.頭から順にreturnをやり直す.
\item[鉄則1:出力してみる]
多くのテキストではページ数の関係で出力を抑止しているが,初心者が問題を解いていく段階ではデータやグラフをできるだけ多く出力する.最後のコロンをセミコロンに変える,あるいは途中にprint文を入れる.
\item[鉄則2:関数に値を代入してみる]
数値が返ってくるべき時に変数があればどこかで入力をミスっている.plotで以下のようなエラーが出た場合にチェック.
\begin{MapleInput}
> plot(f(x),x);
\end{MapleInput}
\begin{MapleError}
Warning, unable to evaluate the function to numeric values in the region; see
the plotting command's help page to ensure the calling sequence is correct
\end{MapleError}
\item[鉄則3:内側から順に入力する]
長い入力やfor-loopを頭から打ち込んではいけない!! 内側から順に何をしているか解読・確認しながら打ち込む.括弧が合わなかったり,読み飛ばしていたりというエラーが回避できる.
\end{description}

\subsection{具体例:無限積分}
以下に示す積分を実行せよ.
\begin{equation*}
\int _{-\infty }^{\infty }x \exp(-\beta c\,{x}^{2}) \left( 1+\beta g\,{x}^{3} \right) {dx}
\end{equation*}
最新版のMapleでは改良が施されていて,このような複雑な積分も一発で求まる.
\begin{MapleInput}
> f1:=unapply(x*exp(-beta*c*x^2)*(1+beta*g*x^3),x);
\end{MapleInput}
\begin{MapleOutput}
{\it f1}\, := \,x\mapsto x\exp(-\beta c\,{x}^{2}) \left( 1+\beta\,g{x}^{3} \right) 
\end{MapleOutput}

\begin{MapleInput}
> int(f1(x),x=-infinity..infinity);
\end{MapleInput}
\begin{MapleOutput}
\left\{\, \begin {array}{cc} {\displaystyle \frac{3}{4}\,\frac {g \sqrt{\pi }}{\beta\,{c}^{2} \sqrt{c\beta}}}&csgn \left( c\beta \right) =1\\  \infty&otherwise\end {array} \right.
\end{MapleOutput}
ここでは,$c \beta$が正の場合(csgn(beta c)=1)とそれ以外の場合(otherwise)に分けて答えを返している.しかしこのような意図したきれいな結果をいつもMapleが返してくれるとは限らない.これだけしか知らないと,なにかうまくいかないときにお手上げになってしまう.このようなきれいで簡単な結果に行き着く前の,裏でおこなういくつかの予備計算を省略せずに示そう.

先ず鉄則0にしたがってrestartをかけ,関数を定義する.
\begin{MapleInput}
> restart; f1:=unapply(x*exp(-beta*c*x^2)*(1+beta*g*x^3),x);
\end{MapleInput}
\begin{MapleOutput}
{\it f1}\, := \,x\mapsto x \exp(-\beta c\,{x}^{2}) \left( 1+\beta\,g{x}^{3} \right) \end{MapleOutput}
次には鉄則1にしたがって積分する前にどのような関数かプロットしてみる.そのままplotへ投げると怒られる.
\begin{MapleInput}
> plot(f1(x),x=-10..10);
\end{MapleInput}
\begin{MapleError}
Warning, unable to evaluate the function to numeric values in the region; see
the plotting command's help page to ensure the calling sequence is correct
\end{MapleError}
これは鉄則2にあるとおり,数値を代入すれば,
\begin{MapleInput}
> f1(10);
\end{MapleInput}
\begin{MapleOutput}
10\,\exp(-100\,c\beta) \left( 1+1000\,\beta\,g \right)
\end{MapleOutput}

beta,c,gなどのパラメータの値が入っていないためとわかる.適当に値を代入する.
\begin{MapleInput}
> c:=1; g:=0.01; beta:=0.1; #res:1 0.01 0.1
\end{MapleInput}

再度プロットを試みる.
\begin{MapleInput}
> plot(f1(x),x=-10..10);
\end{MapleInput}
\MaplePlot{50mm}{./figures/EqManip5plot2d1.eps}

実際に積分してみる.ここでは,鉄則3にしたがって,式を頭から打ち込むのではなく内側からみていくことが肝要である.これは問題を解いていく時に,思考が必ずたどるであ
ろう順番に相当する.
先ず変数に入れた数値をクリアする.
\begin{MapleInput}
> c:='c': g:='g':beta:='beta':
\end{MapleInput}
不定積分でこの関数が積分できることを確認する.
\begin{MapleInput}
> int(f1(x),x);
\end{MapleInput}
\begin{MapleOutput}
-\frac{1}{2}\,{\frac {1}{{\exp(c\beta\,{x}^{2})}\beta\,c}}+\beta\,g \left( -\frac{1}{2}\,{\frac {{x}^{3}{ \exp(-c\beta\,{x}^{2}) }}{c\beta}}+\frac{3}{2}\, \left( -\frac{1}{2}\,{\frac {x{\exp(-c\beta\,{x}^{2})}}{c\beta}}+\frac{1}{4}\,{\frac { \sqrt{\pi }{erf\left( \sqrt{c\beta}x\right)}}{c\beta\, \sqrt{c\beta}}} \right) {c}^{-1}{\beta}^{-1} \right)
\end{MapleOutput}

次にx=-alpha..alphaの定積分を実行する.これは上記のコマンドに付け足すようにしていく.
\begin{MapleInput}
> int(f1(x),x=-alpha..alpha);
\end{MapleInput}
\begin{MapleOutput}
-\frac{1}{4}\,{\frac {g \left( 4\,{\alpha}^{3}\exp({-c\beta\,{\alpha}^{2}}) \beta\,c \sqrt{c\beta}+6\,\alpha\,\exp({-c\beta\,{\alpha}^{2}}) \sqrt{c\beta}-3\, \sqrt{\pi }{erf\left( \sqrt{c\beta}\alpha\right)}\\
\mbox{} \right) }{\beta\,{c}^{2} \sqrt{c\beta}}}
\end{MapleOutput}
さらに$\alpha \mapsto \infty$としてみる.
\begin{MapleInput}
> limit(int(f1(x),x=-alpha..alpha),alpha=infinity);
\end{MapleInput}
\begin{MapleOutput}
\lim _{\alpha\rightarrow \infty }-\frac{1}{4}\,{\frac {g \left( 4\,{\alpha}^{3}\exp({-c\beta\,{\alpha}^{2}})\beta\,c \sqrt{c\beta}+6\,\alpha\,\exp({-c\beta\,{\alpha}^{2}}) \sqrt{c\beta}-3\, \sqrt{\pi }{erf\left( \sqrt{c\beta}\alpha\right)} \right) \\
\mbox{}}{\beta\,{c}^{2} \sqrt{c\beta}}}
\end{MapleOutput}

ところがこれでは答えを返してくれない.積分した後のそれぞれの項を見ると$\beta \,c>0$を仮定すれば簡単になることが分る.assumeを使って,このような変数の仮定おこ
なう.
\begin{MapleInput}
> assume(beta*c>0);
\end{MapleInput}
結果として最初に出した解答を得る.
\begin{MapleInput}
> limit(int(f1(x),x=-alpha..alpha),alpha=infinity);
\end{MapleInput}
\begin{MapleOutput}
\frac{3}{4}\,{\frac { \sqrt{\pi }g}{\beta \,{c}^{2} \sqrt{\beta \,c}}}
\end{MapleOutput}

\subsection{式のフォローのデフォルト}
Mapleで実際に数式をいじる状況というのは,ほとんどの場合が既知の数式変形のフォローだろう.例えば,論文で「(1)式から(2)式への変形は自明である」とかいう
文章で済ましている変形が本当にあっているのかを確かめたい時.一番単純なやり方は自明と言われた前後の式が一致していることを確かめるだけで十分である.
最も単純な確認法は以下の通り,変形の前後の式を手入力してその差をexpandした結果が0か否かでする.
\begin{MapleInput}
> ex1:=(x-3)^4;
\end{MapleInput}
\begin{MapleOutput}
{\it ex1}\, := \, \left( x-3 \right) ^{4}
\end{MapleOutput}

\begin{MapleInput}
> ex2:=x^4-12*x^3+54*x^2-108*x+81;
\end{MapleInput}
\begin{MapleOutput}
{\it ex2}\, := \,{x}^{4}-12\,{x}^{3}+54\,{x}^{2}-108\,x+81
\end{MapleOutput}

\begin{MapleInput}
> expand(ex1-ex2); #res: 0
\end{MapleInput}

0ならば式の変形は保証されているので,その導出が間違いでなく誤植などもないことが確認できる.ただ,これだけでは変形の哲学や技法が身に付くわけではない.あくまでも
苦し紛れのデフォルトであることは心に留めておくように.
論理値として確かめたいときには,evalbを使う.
\begin{MapleInput}
> evalb(expand(ex1-ex2)=0); #res: true
\end{MapleInput}
 

\pagebreak
\section{センター試験I(CenterExamI)}
\ChartElement{解説}
\paragraph{数式変形実践課題(大学入試センター試験の解法を通して)}
今まで出てきたコマンドを使えば,典型的なセンター試験の問題を解くのも容易である.以下の例題を参照して課題を解いてみよ.使うコマンドは,unapply, solve, diff, expand(展開), factor(因数分解)とsubs(一時的代入)である.expand等の数式変形によく使うコマンドは次節以降で詳しく解説している.subsは以下を参考にせよ.

\subsection{一時的代入(subs)}
代入(:=)が永続的なのに対して,一時的な代入はsubsで行う.
\begin{MapleInput}
> subs(a=1,a+2); #res: 3
\end{MapleInput}
典型的な使い方は,solveで求めた解などを式(equation)として代入しておいて,それをsubsで一時的に当てはめる.
\begin{MapleInput}
> eq1:=a=solve(a+b=0,a); subs(eq1,a+2);
\end{MapleInput}
\begin{MapleOutputGather}
a = -b \notag \\
-b + 2 \notag
\end{MapleOutputGather}

 
\ChartElementTwo{例題}
\subsection{例題:2次関数の頂点}
$a,b$を定数とし, $a <> 0$とする.2次関数
\begin{equation*}
y = a\,x^2-b\,x-a+b\,\cdots(1)
\label{Eq:ExampleCenterExamI-1}
\end{equation*}
のグラフが点 $(-2, 6)$ を通るとする.このとき
\begin{equation*}
b = -a+\fbox{ ア }
\end{equation*}
であり,グラフの頂点の座標を$a$を用いて表すと
\begin{equation*}
\left(\frac{-a+\fbox{ イ }}{\fbox{ ウ }\,a}, -\frac{(\fbox{ エ }\,a- \fbox{ オ })^2}{\fbox{ カ }\,a}\right)
\end{equation*}
である (2008 年度大学入試センター試験数学 I より抜粋).

\subsubsection{解答例}
まず,与えられた2次関数を$f(x)$で定義する.
\begin{MapleInput}
> restart; f:=unapply(a*x^2-b*x-a+b,x);
\end{MapleInput}
\begin{MapleOutput}
 f\, := \,x\mapsto a\,{x}^{2}-b\,x-a+b
\end{MapleOutput}
与えられた点の座標を関数に入れる.
\begin{MapleInput}
> eq1:=f(-2)=6;
\end{MapleInput}
\begin{MapleOutput}
{\it eq1}\, := \,3\,a+3\,b=6
\end{MapleOutput}
これをbについて解く.
\begin{MapleInput}
> eq2:=b=solve(eq1,b);
\end{MapleInput}
\begin{MapleOutput}
{\it eq2}\, := \,b=2-a
\end{MapleOutput}
次は,頂点の座標で傾きが0になることを用いて解いていく.
\begin{MapleInput}
> solve(diff(f(x),x)=0,x);
\end{MapleInput}
\begin{MapleOutput}
\frac{1}{2}\,{\frac {b}{a}}
\end{MapleOutput}
bの値はeq2で求まっているので,それを代入(subs)する.
\begin{MapleInput}
> subs(eq2,solve(diff(f(x),x)=0,x));
\end{MapleInput}
\begin{MapleOutput}
\frac{1}{2}\,{\frac {2-a}{a}}
\end{MapleOutput}
これをx0としてeq3で定義しておく.
\begin{MapleInput}
> eq3:=x0=subs(eq2,solve(diff(f(x),x)=0,x));
\end{MapleInput}
\begin{MapleOutput}
{\it eq3}\, := \,{\it x0}=\frac{1}{2}\,{\frac {2-a}{a}}
\end{MapleOutput}
頂点のy座標は,$f(x0)$で求まる
\begin{MapleInput}
> f(x0);
\end{MapleInput}
\begin{MapleOutput}
a\,{{\it x0}}^{2}-b\,{\it x0}-a+b
\end{MapleOutput}
eq2, eq3で求まっているx0, bを代入する.
\begin{MapleInput}
> eq4:=subs({eq2,eq3},f(x0));
\end{MapleInput}
\begin{MapleOutput}
{\it eq4}\, := \,-\frac{1}{4}\,{\frac { \left( 2-a \right) ^{2}}{a}}-2\,a+2
\end{MapleOutput}
これを因数分解(factor)する.
\begin{MapleInput}
> factor(subs({eq2,eq3},f(x0)));
\end{MapleInput}
\begin{MapleOutput}
-\frac{1}{4}\,{\frac { \left( 3\,a-2 \right) ^{2}}{a}}
\end{MapleOutput}

 
\ChartElementTwo{課題}
\begin{enumerate}
\item
$P = x(x+3)(2x-3)$とする.
また,$a$を定数とする.
$x = a+1$のときの
$P$の値は
\begin{equation*}
2a^3+\fbox{  ア  }a^2+\fbox{  イ  }a-\fbox{  ウ  }
\end{equation*}
である.

$x=a+1$のときの$P$の値と,$x=a$のときの$P$の値が等しいとする.このとき,$a$は
\begin{equation*}
3a^2+\fbox{ エ }a-\fbox{ オ } = 0
\end{equation*}
を満たす.したがって
\begin{equation*}
a = \frac{\fbox{ カキ }\pm \sqrt{\fbox{ クケ }}}{\fbox{ コ }}
\end{equation*}
である.

\item
(例題1.に引き続いて,)
さらに,2次関数(1)のグラフの頂点のy座標が-2であるとする.このとき,aは
\begin{equation*}
\fbox{ キ }\,a^2-\fbox{ クケ }\,a+\fbox{ コ } = 0
\end{equation*}
を満たす.これより,aの値は
\begin{equation*}
a = \fbox{ サ }, \frac{\fbox{ シ }}{\fbox{ ス }}
\end{equation*}
である.
以下,$a = \frac{\fbox{ シ }}{\fbox{ ス }}$であるとする.

このとき,2次関数(1)のグラフの頂点のx座標は\fbox{ セ }であり,(1)のグラフとx軸の2交点のx座標は\fbox{ ソ },\fbox{ タ }である.

また,関数(1)は$0 \leqq x \leqq 9$において

$x$ = \fbox{ チ }のとき,最小値\fbox{ ツテ }をとり, 

$x$ = \fbox{ ト }のとき,最大値$\frac{\fbox{ ナニ }}{\fbox{ ヌ }}$をとる.


(2008 年度大学入試センター試験数学 I より抜粋).

\end{enumerate} 
\ChartElementTwo{解答例}
\begin{enumerate}
\item
$P$を$x$の関数として定義,
\begin{MapleInput}
> restart:
> P:=unapply(x*(x+3)*(2*x-3),x);
\end{MapleInput}
\begin{MapleOutput}
P\, := \,x\mapsto x \left( x+3 \right)  \left( 2\,x-3 \right)
\end{MapleOutput}
$P(a+1)$および$P(a)$を形式的に出してみる.
\begin{MapleInput}
> expand(P(a+1)), expand(P(a));
\end{MapleInput}
\begin{MapleOutput}
2\,{a}^{3}+9\,{a}^{2}+3\,a-4,\,2\,{a}^{3}+3\,{a}^{2}-9\,a
\end{MapleOutput}
2式を差し引く.
\begin{MapleInput}
> eq1:=(expand(P(a+1))-expand(P(a)))/2;
\end{MapleInput}
\begin{MapleOutput}
{\it eq1}\, := \,3\,{a}^{2}+6\,a-2
\end{MapleOutput}
出題にそろえるため2で割っている.その式をeq1として代入し,eq1=0をxについて解く(solve).
\begin{MapleInput}
> sol1:=solve(eq1=0,a);
\end{MapleInput}
\begin{MapleOutput}
{\it sol1}\, := \,-1+\frac{1}{3}\, \sqrt{15},\,-1-\frac{1}{3}\, \sqrt{15}
\end{MapleOutput}

\item
例題のeq3,eq4までを確認
%(あるいは,ここまでは[[EqManip_ExampleCenterExamI]]を打ち込む.)
\begin{MapleInput}
> eq3, eq4;
\end{MapleInput}
\begin{MapleOutput}
{\it x0}=\frac{1}{2}\,{\frac {2-a}{a}},\, -\frac{1}{4}\,{\frac { \left( 2-a \right) ^{2}}{a}}-2\,a+2
\end{MapleOutput}
eq4が頂点の$y$座標の値なので,これから$-2$を引いて展開.
\begin{MapleInput}
expand((eq4-(-2)));
\end{MapleInput}
\begin{MapleOutput}
-\frac{1}{a}+5-\frac{9}{4}\,a
\end{MapleOutput}
これではわかりにくいので,出題にそう形にするため,$-4a$を掛け,eq5とする.
\begin{MapleInput}
> eq5:=expand((eq4+2)*(-4)*a);
\end{MapleInput}
\begin{MapleOutput}
{\it eq5}\, := \,4-20\,a+9\,{a}^{2}
\end{MapleOutput}
これを$a$について解いて(solve)
\begin{MapleInput}
> solve(eq5=0,a);
\end{MapleInput}
\begin{MapleOutput}
2,\,\frac{2}{9}
\end{MapleOutput}
$a=2/9$をeq3に代入して,頂点の$x$座標を出す.
\begin{MapleInput}
> subs(a=2/9,eq3);
\end{MapleInput}
\begin{MapleOutput}
{\it x0}=4
\end{MapleOutput}
$a,b$を$f(x)$に代入して,
\begin{MapleInput}
> eq6:=subs({a=2/9,b=2-2/9},f(x));
\end{MapleInput}
\begin{MapleOutput}
{\it eq6}\, := \,\frac{2}{9}\,{x}^{2}-{\frac {16}{9}}\,x+{\frac {14}{9}}
\end{MapleOutput}
これを解いて,$x$座標の交点を求める.
\begin{MapleInput}
> solve(eq6=0,x);
\end{MapleInput}
\begin{MapleOutput}
7,\,1
\end{MapleOutput}
最大値,最小値を求めるために,今まで求めたパラメータを代入して,plotしてみる.
\begin{MapleInput}
> plot(subs({a=2/9,b=2-2/9},f(x)),x=0..9);
\end{MapleInput}
\MaplePlot{50mm}{./figures/EqManip1plot2d1.eps}
目視で分かるとおり,最小値$x=4$,最大値$x=9$で
\begin{MapleInput}
> subs({a=2/9,b=2-2/9},f(4)), subs({a=2/9,b=2-2/9},f(9));
\end{MapleInput}
\begin{MapleOutput}
-2,\,\frac{32}{9}
\end{MapleOutput}
\end{enumerate}
 

\chapter{描画}
\section{CG(ComputerGraphics)}
\ChartElement{解説}
\subsection{listplot, pointplot}
リスト構造にある離散的なデータはlistplotで表示してくれる.listplotは受け取ったlistの要素をy値に,1から始まる添字をx値にして,デフォルトで
は線でグラフを書く.
\begin{MapleInput}
> T:=[seq(exp(-i),i=0..5)]; 
> listplot(T);
\end{MapleInput}
\begin{MapleOutput}
T\, := \,[1,\exp(-1),\exp(-2),\exp(-3),\exp(-4),\exp(-5)]
\end{MapleOutput}
\MaplePlot{50mm}{./figures/MapleCGplot2d1.eps}

以下のようにoptionをつけるとpointで描く.
\begin{MapleInput}
> listplot(T,style=point):
\end{MapleInput}

それぞれの値の横軸xが1,2,3,..では不都合なときには,2次元のlistlist構造を用意し,[x[i],y[i]]を入れてpointplot関数で表示する
.
\begin{MapleInput}
> T:=[seq([i/2,exp(-i/2)],i=0..6)]; 
> pointplot(T,symbol=circle,symbolsize=20);
\end{MapleInput}
\begin{MapleOutputGather}
T\, := \,[[0,1],[1/2,\exp(-1/2)],[1,\exp(-1)],[3/2,\exp(-3/2)], \notag \\
[2,\exp(-2)],[5/2,\exp(-5/2)],[3,\exp(-3)]]
\end{MapleOutputGather}
\MaplePlot{50mm}{./figures/MapleCGplot2d2.eps}


listplotのように線でつなぎたい時には,以下のようにoptionをつける.
\begin{MapleInput}
> pointplot(T,connect=true):
\end{MapleInput}

\subsection{写像の表示}
ある行列によって点を移動させる写像の様子を示すスクリプトを通して,plottoolsが提供するdisk, arrowの使い方を示す.先ず描画に必要なライブラリーパッケージ(plotsおよびplottools)をwithで読み込んでおく.
\begin{MapleInput}
> restart; with(plots):with(plottools):
\end{MapleInput}
\begin{MapleOutput}
\end{MapleOutput}
行列$A= \left[ \begin {array}{cc} 1&2\\ 2&1\end {array} \right]$
によって点$a_0$(1, 2)が$a_1$(5, 4)に移動するとする(LinearAlgebra参照).
\begin{MapleInput}
> with(LinearAlgebra): A:=Matrix([[1,2],[2,1]]): a0:=Vector([1,2]): a1:=A.a0;
\end{MapleInput}
\begin{MapleOutput}
{\it a1}\, := \, \left[ \begin {array}{c} 5\\ 4\end {array} \right]
\end{MapleOutput}
ベクトルが位置座標を意味するようにlistへ変換(convert)する.
\begin{MapleInput}
> p0:=convert(a0,list):p1:=convert(a1,list):
\end{MapleInput}
位置p0に円(disk)を半径0.2,赤色で描く.同じように位置p1に半径0.2,青色でdiskを描く.
\begin{MapleInput}
> point1:=[disk(p0,0.2,color=red),disk(p1,0.2,color=blue)]:
\end{MapleInput}
もう一つ,p0からp1に向かう矢印(arrow)を適当な大きさで描く.後ろの数字をいじると線の幅や矢印の大きさが変わる.
\begin{MapleInput}
> line1:=arrow(p0,p1,0.05,0.3,0.1):
\end{MapleInput}
これらをまとめて表示(display).このとき,表示範囲を0..6,0..6とする.
\begin{MapleInput}
> display(point1,line1,view=[0..6,0..6],gridlines=true);
\end{MapleInput}
\MaplePlot{50mm}{./figures/MapleCGplot2d3.eps}
$a_0$(1, 2)の赤点が,$a_1$(5, 4)の青点へ移動していることを示している.

\subsection{回転写像}
次に原点周りでの回転の様子を示す.回転の行列.
\begin{MapleInput}
> Matrix([[cos(theta),sin(theta)],[-sin(theta),cos(theta)]]);
\end{MapleInput}
\begin{MapleOutput}
\left[ \begin {array}{cc} \cos \left( \theta \right) &\sin \left( \theta \right) \\ -\sin \left( \theta \right) &\cos \left( \theta \right) \end {array} \right]
\end{MapleOutput}
これを関数のように定義している.
\begin{MapleInput}
> A:=t->Matrix([[cos(t),sin(t)],[-sin(t),cos(t)]]);
\end{MapleInput}
\begin{MapleOutput}
A\, := \,t\mapsto  \left[ \begin {array}{cc} \cos \left( t \right) &\sin \left( t \right) \\ -\sin \left( t \right) &\cos \left( t \right) \end {array} \right]
\end{MapleOutput}
tに回転角(Pi/3)を入れている.
\begin{MapleInput}
> a0:=Vector([3,0]);
> a1:=A(Pi/3).a0;
\end{MapleInput}
\begin{MapleOutputGather}
{\it a0}\, := \, \left[ \begin {array}{c} 3\\ 0\end {array} \right] \notag \\
{\it a1}\, := \, \left[ \begin {array}{c} 3/2\\ -3/2\,\sqrt {3}\end {array} \right] \notag
\end{MapleOutputGather}
表示の仕方は,前節と同じ.
\begin{MapleInput}
> p0:=convert(a0,list):p1:=convert(a1,list):
> point1:=[disk(p0,0.2,color=red),disk(p1,0.2,color=blue)]:
> line1:=arrow(p0,p1,0.05,0.3,0.1):
> display(point1,line1,view=[-4..4,-4..4],gridlines=true);
\end{MapleInput}
\MaplePlot{50mm}{./figures/MapleCGplot2d4.eps}

\subsection{平行投影図の作成}
もう少し複雑な対象物として,ここでは立方体の表示を考える.まず3次元座標を打ち込む.
\begin{MapleInput}
> restart; with(plots): with(plottools): 
  p:=[[0,0,0],[1,0,0],[1,1,0],[0,1,0],
  [0,0,1],[1,0,1],[1,1,1],[0,1,1]]:
\end{MapleInput}
\begin{MapleOutput}
\end{MapleOutput}
次にこれをpointplot3dで簡便に表示.
\begin{MapleInput}
> points:= { seq(p[i],i=1..8) }:
> pointplot3d(points,symbol=circle,symbolsize=40,color=black);
\end{MapleInput}
\MaplePlot{50mm}{./figures/MapleCGplot3d5.eps}
もうすこし見やすいように頂点を結んでおく.たとえば,p[1]とp[2]との間を線で結ぶには,

\begin{MapleInput}
> line(p[1],p[2]);
\end{MapleInput}
とする.それをseqで複数の点間に対して施す.

\begin{MapleInput}
> ll:=[[1,2],[2,3],[3,4],[4,1],[1,5],[2,6],[3,7],[4,8],
> [5,6],[6,7],[7,8],[8,5]]:
> lines:=[seq(line(p[ll[i][1]],p[ll[i][2]]),i=1..nops(ll))]:
> display(lines,scaling=constrained,color=black);
\end{MapleInput}
\MaplePlot{50mm}{./figures/MapleCGplot3d6.eps}

\begin{MapleInput}
> l3:=display(lines,scaling=constrained,color=black):
> p3:=pointplot3d(p,symbol=circle,symbolsize=40,color=black):
> display([p3,l3],scaling=constrained,color=black);
\end{MapleInput}
\MaplePlot{50mm}{./figures/MapleCGplot3d7.eps}


画像をつまんでぐるぐる回してみよ.Mapleではこんな操作は簡単にできるが,よく見ればわかるように,この3次元表示では透視図ではなく,平行投影図といわれるものを書いている.

\subsection{透視図}
透視図のもっとも簡単な変換法は
\begin{MapleInput}
> proj2d:=proc(x,z) 
    local x1,y1; 
    x1:=x[1]*z/(z-x[3]); 
    y1:=x[2]*z/(z-x[3]);
    return [x1,y1]; 
  end proc:
\end{MapleInput}
\begin{MapleOutput}
\end{MapleOutput}
zに視点の距離を入れて,xで座標を受け取って変換した結果を[x1,y1]として返している.この関数を前の表示と組み合わせれば透視図の描画ができる.
\begin{MapleInput}
> z_p:=-8:
  lines:=[seq(line(proj2d(p[ll[i][1]],z_p), proj2d(p[ll[i][2]],z_p)), i=1..nops(ll))]:
  display(lines);
\end{MapleInput}
\MaplePlot{50mm}{./figures/MapleCGplot2d8.eps}

\subsection{Mapleの描画関数の覚書}
mapleにはいくつかの描画レベルに合わせた関数が用意されている.どのような目的にどの関数(パッケージ)を使うかの選択指針として,それぞれがどのような意図で作ら
れ,それらの依存関係は以下の通り.
\begin{description}
\item[描画の下位関数]
plot[structure]にあるPLOT,PLOT3Dデータ構造が一番下でCURVES,POINTS,POLYGONS,TEXTデータを元に絵を描く.
\item[plottoolsパッケージ]
PLOTよりもう少し上位で,グラフィックスの基本形状を生成してくれる関数群.arc, arrow, circle, curve, line, point,sphereなどの関数があり,PLOT構造を吐く.表示にはplots[display]を使う.
\item[plotsパッケージ]
簡単にグラフを書くための道具.たとえばlistplotは,listデータを簡易に表示する事を目的としている.
\end{description}
 

\pagebreak
\section{動画(Animation)}
\ChartElementTwo{解説}
\subsection{animate関数}
plotsパッケージにあるanimate関数を使う.構文は以下の通りで,[]内に動画にしたい関数を定義し,tで時間を変えていく.
\begin{MapleInput}
> with(plots): animate(plot, [sin(x-t),x=0..5*Pi], t=0..10);
\end{MapleInput}
\MaplePlot{50mm}{./figures/MapleCGplot2d9.eps}

\subsection{リストに貯めて,display表示}
おなじ動作を,display関数でオプションとしてinsequence=trueとしても可能.この場合は第一引数に入れるリスト([])に一連の画像を用意し,コマ
送りで表示させる.
\begin{MapleInput}
> tmp:=[]: n:=10: for i from 0 to n do t:=i; tmp:=[op(tmp),
> plot(sin(x-t)+sin(x+t),x=0..5*Pi)]; end do:
> display(tmp,insequence=true);
\end{MapleInput}
\MaplePlot{50mm}{./figures/MapleCGplot2d10.eps}

\subsection{凝った例}
ヘルプにある凝った例.Fという動画のコマを吐く関数を用意する.これを,animate関数から適当な変数を入れて呼び出す.backgroundには動かない絵を指定
することができる.
\begin{MapleInput}
> with(plottools,line): F := proc(t) plots[display](
> line([-2,0],[cos(t)-2,sin(t)],color=blue),
> line([cos(t)-2,sin(t)],[t,sin(t)],color=blue),
> plot(sin(x),x=0..t,view=[-3..7,-5..5]) ); end:
> animate(F,[theta],theta=0..2*Pi, background=plot([cos(t)-2,sin(t),t=0..2*Pi]),
> scaling=constrained,axes=none);
\end{MapleInput}
\begin{MapleOutput}
\end{MapleOutput}
\MaplePlot{50mm}{./figures/MapleCGplot2d11.eps}

\subsection{ファイルへの保存}
animationなどのgif形式のplotを外部ファイルへ出力して表示させるには,以下の一連のコマンドのようにする.
\begin{MapleInput}
> plotsetup(gif,plotoutput=file2): display(tmp,insequence=true);
> plotsetup(default):
\end{MapleInput}
こいつをquicktimeなどに食わせれば,Maple以外のソフトで動画表示が可能となる.3次元図形の標準規格であるvrmlも同じようにして作成することが可能(?vrml;参照).
 



\end{document}
