\begin{enumerate}
\item
$P$を$x$の関数として定義,
\begin{MapleInput}
> restart:
> P:=unapply(x*(x+3)*(2*x-3),x);
\end{MapleInput}
\begin{MapleOutput}
P\, := \,x\mapsto x \left( x+3 \right)  \left( 2\,x-3 \right)
\end{MapleOutput}
$P(a+1)$および$P(a)$を形式的に出してみる.
\begin{MapleInput}
> expand(P(a+1)), expand(P(a));
\end{MapleInput}
\begin{MapleOutput}
2\,{a}^{3}+9\,{a}^{2}+3\,a-4,\,2\,{a}^{3}+3\,{a}^{2}-9\,a
\end{MapleOutput}
2式を差し引く.
\begin{MapleInput}
> eq1:=(expand(P(a+1))-expand(P(a)))/2;
\end{MapleInput}
\begin{MapleOutput}
{\it eq1}\, := \,3\,{a}^{2}+6\,a-2
\end{MapleOutput}
出題にそろえるため2で割っている.その式をeq1として代入し,eq1=0をxについて解く(solve).
\begin{MapleInput}
> sol1:=solve(eq1=0,a);
\end{MapleInput}
\begin{MapleOutput}
{\it sol1}\, := \,-1+\frac{1}{3}\, \sqrt{15},\,-1-\frac{1}{3}\, \sqrt{15}
\end{MapleOutput}

\item
例題のeq3,eq4までを確認
%(あるいは,ここまでは[[EqManip_ExampleCenterExamI]]を打ち込む.)
\begin{MapleInput}
> eq3, eq4;
\end{MapleInput}
\begin{MapleOutput}
{\it x0}=\frac{1}{2}\,{\frac {2-a}{a}},\, -\frac{1}{4}\,{\frac { \left( 2-a \right) ^{2}}{a}}-2\,a+2
\end{MapleOutput}
eq4が頂点の$y$座標の値なので,これから$-2$を引いて展開.
\begin{MapleInput}
expand((eq4-(-2)));
\end{MapleInput}
\begin{MapleOutput}
-\frac{1}{a}+5-\frac{9}{4}\,a
\end{MapleOutput}
これではわかりにくいので,出題にそう形にするため,$-4a$を掛け,eq5とする.
\begin{MapleInput}
> eq5:=expand((eq4+2)*(-4)*a);
\end{MapleInput}
\begin{MapleOutput}
{\it eq5}\, := \,4-20\,a+9\,{a}^{2}
\end{MapleOutput}
これを$a$について解いて(solve)
\begin{MapleInput}
> solve(eq5=0,a);
\end{MapleInput}
\begin{MapleOutput}
2,\,\frac{2}{9}
\end{MapleOutput}
$a=2/9$をeq3に代入して,頂点の$x$座標を出す.
\begin{MapleInput}
> subs(a=2/9,eq3);
\end{MapleInput}
\begin{MapleOutput}
{\it x0}=4
\end{MapleOutput}
$a,b$を$f(x)$に代入して,
\begin{MapleInput}
> eq6:=subs({a=2/9,b=2-2/9},f(x));
\end{MapleInput}
\begin{MapleOutput}
{\it eq6}\, := \,\frac{2}{9}\,{x}^{2}-{\frac {16}{9}}\,x+{\frac {14}{9}}
\end{MapleOutput}
これを解いて,$x$座標の交点を求める.
\begin{MapleInput}
> solve(eq6=0,x);
\end{MapleInput}
\begin{MapleOutput}
7,\,1
\end{MapleOutput}
最大値,最小値を求めるために,今まで求めたパラメータを代入して,plotしてみる.
\begin{MapleInput}
> plot(subs({a=2/9,b=2-2/9},f(x)),x=0..9);
\end{MapleInput}
\MaplePlot{50mm}{./figures/EqManip1plot2d1.eps}
目視で分かるとおり,最小値$x=4$,最大値$x=9$で
\begin{MapleInput}
> subs({a=2/9,b=2-2/9},f(4)), subs({a=2/9,b=2-2/9},f(9));
\end{MapleInput}
\begin{MapleOutput}
-2,\,\frac{32}{9}
\end{MapleOutput}
\end{enumerate}
