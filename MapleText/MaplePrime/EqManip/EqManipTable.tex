数式の変形は,手で直すほうが圧倒的に早くきれいになる場合が多い.しかし,テイラー展開や,複雑な積分公式,三角関数とexp関数の変換などの手間がかかるところを,Mapleは間違いなく変形してくれる.ここで示すコマンドを全て覚える必要は全くない.というか忘れるもの.ここでは,できるだけコンパクトにまとめて,悩んだときに参照できるようにする.初めての人は,ざっと眺めた後,鉄則からじっくりフォローせよ.

\begin{table}[htbp]
\caption{数式処理で頻繁に使うコマンド.}
\begin{center}
\begin{tabular}{llll}
\hline
式の変形&式の分割抽出&省略操作,その他 &代入,置換,仮定\\
\hline
simplify:簡単化 &lhs, rhs:左辺,右辺	&||:連結作用素 &subs:一時的代入	\\
expand:展開		&numer, denom:分子,分母&seq:for-loopの簡易表記&restart,a:='a':初期化\\
factor:因数分解	&coeff:係数				&map:関数の要素への適用&assume:仮定			\\
normal:約分・通分 &nops, op				&add,mul:単純な和,積&assuming:一時的仮定\\
combine:公式でまとめる&					&sum,product:数式に対応した和,積&assign:値の確定	\\
collect:次数でまとめる&					&limit:極限&about:仮定の中身	\\
sort:昇べき,降べき&					&	&anames('user'):使用変数名\\
rationalize:分母の有理化\\
convert:形式の変換\\
\hline
\end{tabular}
\end{center}
\label{default}
\end{table}%

このほかにも,solve(解), diff(微分), int(積分),series(級数展開)等は頻繁に数式の導出・変形に登場する.





