\subsection{一時的代入(subs)} 
関係式(x=2)を式(exp1)に一時的に代入した結果を表示.
\begin{MapleInput}
> exp1:=x^2-4*x*y+4; subs(x=2,exp1);
\end{MapleInput}
\begin{MapleOutputGather}
{\it exp1}\, := \,{x}^{2}-4\,xy+4 \notag \\
8-8\,y \notag
\end{MapleOutputGather}
\begin{MapleInput}
> subs({x=a+2,y=sin(x)},exp1);
\end{MapleInput}
\begin{MapleOutput}
\left( a+2 \right) ^{2}-4\, \left( a+2 \right) \sin \left( x \right) +4
\end{MapleOutput}

\subsection{仮定(assume)}
変数になにかの条件を仮定するときに使う.たとえば,根を開くときに,与式が正と仮定すると開かれる.
\begin{MapleInput}
> sqrt(b^2); 
> assume(a>0); sqrt(a^2);
\end{MapleInput}
\begin{MapleOutputGather}
\sqrt{{b}^{2}} \notag \\
a\sim \notag
\end{MapleOutputGather}

\subsection{一時的仮定(assuming)} 
assumeと同じだが,一時的に仮定するときに使われる.
\begin{MapleInput}
> exp1:=x^2-4*x+4; 
> sqrt(exp1);
\end{MapleInput}
\begin{MapleOutputGather}
{\it exp1}\, := \,{x}^{2}-4\,x+4 \notag \\
\sqrt{ \left( -2+x \right) ^{2}} \notag
\end{MapleOutputGather}

\begin{MapleInput}
> sqrt(exp1) assuming x>2;
\end{MapleInput}
\begin{MapleOutput}
-2+x
\end{MapleOutput}

assumeに加えての仮定にadditionallyがある.

\subsection{solveで求めた値の確定(assign)} 
解を求めるsolveをして
\begin{MapleInput}
> x:='x':y:='y': 
> s1:=solve({x-y+1=0,x+y-2=0},{x,y}); 
\end{MapleInput}
\begin{MapleOutput}
{\it s1}\, := \, \left\{ x=\frac{1}{2},y=\frac{3}{2} \right\}
\end{MapleOutput}
このまま,x,yの中身を見ても,
\begin{MapleInput}
> x,y;
\end{MapleInput}
\begin{MapleOutput}
x,\,y
\end{MapleOutput}
代入されていない.s1をassign(確定)すると,
\begin{MapleInput}
> assign(s1);
> x,y;
\end{MapleInput}
\begin{MapleOutput}
\frac{1}{2},\frac{3}{2}
\end{MapleOutput}
と代入される.一時的代入subsと使い分ける.

\subsection{assumeで仮定した内容の確認(about)} 
\begin{MapleInput}
> about(a);
\end{MapleInput}
\begin{MapleError}
Originally a, renamed a~:
  is assumed to be: RealRange(Open(0),infinity)
\end{MapleError}

\subsection{ユーザが定義した変数の確認(anames('user'))} 
\begin{MapleInput}
> anames('user');
\end{MapleInput}
\begin{MapleOutput}
{\it s1},\,y,\,x,\,a
\end{MapleOutput}

\subsection{値の初期化(restart,a:='a')} 
\subsection{連結作用素($||$)} 
前後の変数をくっつけて新たな変数とする.
\begin{MapleInput}
> a||1; #res: a1
> a||b; #res: ab
\end{MapleInput}
プログラムの中で使うとより便利.
\begin{MapleInput}
> for i from 1 to 3 do
    a||i:=i^2; 
  end do;
\end{MapleInput}
\begin{MapleOutputGather}
a1:=1 \notag \\
a2:=4 \notag \\
a3:=9 \notag
\end{MapleOutputGather}

\subsection{for-loopの簡略表記(seq)} 
数列を意味するsequenceの略.
\begin{MapleInput}
> seq(k,k=4..7);
\end{MapleInput}
\begin{MapleOutput}
4,\,5,\,6,\,7
\end{MapleOutput}

\subsection{リスト要素への関数の一括適用(map)} 
\begin{MapleInput}
> f:=x->exp(-x); 
> map(f,[0,1,2,3]);
\end{MapleInput}
\begin{MapleOutputGather}
f\, := \,x\mapsto \exp(-x)  \notag \\
[1,\exp(-1),\exp(-2),\exp(-3)] \notag
\end{MapleOutputGather}

上記の3つを組み合わせると,効率的に式を扱うことができる.
\begin{MapleInput}
> map(sin,[seq(theta||i,i=0..3)]);
\end{MapleInput}
\begin{MapleOutput}
[\sin \left( {\it \theta0} \right) ,\sin \left( {\it \theta1} \right) ,\sin \left( {\it \theta2}\\
\mbox{} \right) ,\sin \left( {\it \theta3} \right) ]
\end{MapleOutput}

\subsection{単純な和,積(add,mul)} 
\begin{MapleInput}
> add(x^i,i=0..3);
\end{MapleInput}
\begin{MapleOutput}
1+x+{x}^{2}+{x}^{3}
\end{MapleOutput}

\begin{MapleInput}
> mul(x^i,i=0..3);
\end{MapleInput}
\begin{MapleOutput}
{x}^{6}
\end{MapleOutput}

\subsection{数式にも対応した和,積(sum,product)} 
\begin{MapleInput}
> add(x^i,i=0..n);
\end{MapleInput}
\begin{MapleError}
Error, unable to execute add
\end{MapleError}
\begin{MapleInput}
> sum(x^i,i=0..n);
\end{MapleInput}
\begin{MapleOutput}
{\frac {{x}^{n+1}}{x-1}}- \frac{1}{x-1}
\end{MapleOutput}

\begin{MapleInput}
> product(x^i,i=0..n);
\end{MapleInput}
\begin{MapleOutput}
{x}^{\frac{1}{2}\, \left( n+1 \right) ^{2}-\frac{1}{2}\,n-\frac{1}{2}}
\end{MapleOutput}

\subsection{極限(limit)} 
\begin{MapleInput}
> limit(exp(-x),x=infinity); #res: 0
\end{MapleInput}

\begin{MapleInput}
> limit(tan(x),x=Pi/2);
\end{MapleInput}
\begin{MapleOutput}
{\it undefined}
\end{MapleOutput}

\begin{MapleInput}
> limit(tan(x),x=Pi/2,left); 
> limit(tan(x),x=Pi/2,complex);
\end{MapleInput}
\begin{MapleOutputGather}
\infty \notag \\
-\infty +\infty \, I\notag
\end{MapleOutputGather}

