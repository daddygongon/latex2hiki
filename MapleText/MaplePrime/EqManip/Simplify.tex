\subsection{簡単化(simplify)} 
\begin{MapleInput}
> simplify(exp1,副関係式):
> simplify(3*x+4*x+2*y);
\end{MapleInput}
\begin{MapleOutput}
7\,x+2\,y
\end{MapleOutput}
\begin{MapleInput}
> exp1:=3*sin(x)^3-sin(x)*cos(x)^2; 
> simplify(exp1);
\end{MapleInput}
\begin{MapleOutputGather}
{\it exp1}\, := \,3\, \left( \sin \left( x \right)  \right) ^{3}-\sin \left( x \right)  \left( \cos \left( x \right)  \right) ^{2} \notag \\
- \left( 4\, \left( \cos \left( x \right)  \right) ^{2}-3 \right) \sin \left( x \right) \notag
\end{MapleOutputGather}
\begin{MapleInput}
> simplify(exp1,{cos(x)^2=1-sin(x)^2});
\end{MapleInput}
\begin{MapleOutput}
4\, \left( \sin \left( x \right)  \right) ^{3}-\sin \left( x \right)
\end{MapleOutput}
オプションとしてsizeを指定するとより簡単になる場合がある.
\begin{MapleInput}
> simplify(exp1,size):
\end{MapleInput}

\subsection{展開(expand)} 
\begin{MapleInput}
> expand((x+y)^2);
\end{MapleInput}
\begin{MapleOutput}
{x}^{2}+2\,xy+{y}^{2}
\end{MapleOutput}

\subsection{因数分解(factor)} 
\begin{MapleInput}
> factor(4*x^2-6*x*y+2*y^2);
\end{MapleInput}
\begin{MapleOutput}
2\, \left( 2\,x-y \right)  \left( x-y \right)
\end{MapleOutput}

\subsection{約分・通分(normal)} 
\begin{MapleInput}
> normal((x+y)/(x^2-3*x*y-4*y^2));
\end{MapleInput}
\begin{MapleOutput}
\frac{1}{ x-4\,y }
\end{MapleOutput}
\begin{MapleInput}
> normal(1/x+1/y);
 \end{MapleInput}
\begin{MapleOutput}
{\frac {y+x}{xy}}
\end{MapleOutput}

\subsection{項を変数でまとめる(collect)} 
\begin{MapleInput}
> collect(4*a*x^2-3*y^2/x+6*b*x*y+3*c*y+2*y^2,y);
\end{MapleInput}
\begin{MapleOutput}
\left( -3\,{x}^{-1}+2 \right) {y}^{2}+ \left( 6\,bx+3\,c \right) y+4\,a{x}^{2}
\end{MapleOutput}

\subsection{項を公式でまとめる(combine)} 
\begin{MapleInput}
> combine(sin(x)^2+3*cos(x)^2);
\end{MapleInput}
\begin{MapleOutput}
2+\cos \left( 2\,x \right)
\end{MapleOutput}

\subsection{昇べき,降べき(sort)} 
\begin{MapleInput}
> sort(exp1,[x,y]); 
> sort(exp1, [x],opts);
opts=tdeg(総次数順), plex(辞書式順), ascending(昇順), descending(降順)
\end{MapleInput}

\begin{MapleInput}
> exp1:=x^3+4*x-3*x^2+1: 
> sort(exp1);
\end{MapleInput}
\begin{MapleOutput}
{x}^{3}-3\,{x}^{2}+4\,x+1
\end{MapleOutput}

\begin{MapleInput}
> sort(exp1,[x],ascending);
\end{MapleInput}
\begin{MapleOutput}
1+4\,x-3\,{x}^{2}+{x}^{3}
\end{MapleOutput}

\begin{MapleInput}
> exp2:=x^3-a*x*y+4*x^2+y^2: 
> sort(exp2);
\end{MapleInput}
\begin{MapleOutput}
-axy+{x}^{3}+4\,{x}^{2}+{y}^{2}
\end{MapleOutput}

\begin{MapleInput}
> sort(exp2,[x]);
\end{MapleInput}
\begin{MapleOutput}
{x}^{3}+4\,{x}^{2}-ayx+{y}^{2}
\end{MapleOutput}

\begin{MapleInput}
> sort(exp2,[a,y]); 
> sort(exp2,[a],plex);
\end{MapleInput}
\begin{MapleOutputGather}
-xay+{y}^{2}+{x}^{3}+4\,{x}^{2} \notag \\
-xya+{x}^{3}+4\,{x}^{2}+{y}^{2} \notag
\end{MapleOutputGather}

\subsection{分母の有理化(rationalize)} 

\begin{MapleInput}
> eq1:=(1+sqrt(2))/(1-sqrt(3));
\end{MapleInput}
分母の有理化は,この式を
\begin{MapleOutput}
{\frac {1+\sqrt {2}}{1-\sqrt {3}}}{\frac {1+\sqrt {3}}{1+\sqrt {3}}}
\end{MapleOutput}
とする事に対応します.
\begin{MapleInput}
> rationalize(eq1);
\end{MapleInput}
\begin{MapleOutput}
-1/2\, \left( 1+\sqrt {2} \right)  \left( 1+\sqrt {3} \right) 
\end{MapleOutput}

