\subsection{例題:2次関数の頂点}
$a,b$を定数とし, $a <> 0$とする.2次関数
\begin{equation*}
y = a\,x^2-b\,x-a+b\,\cdots(1)
\label{Eq:ExampleCenterExamI-1}
\end{equation*}
のグラフが点 $(-2, 6)$ を通るとする.このとき
\begin{equation*}
b = -a+\fbox{ ア }
\end{equation*}
であり,グラフの頂点の座標を$a$を用いて表すと
\begin{equation*}
\left(\frac{-a+\fbox{ イ }}{\fbox{ ウ }\,a}, -\frac{(\fbox{ エ }\,a- \fbox{ オ })^2}{\fbox{ カ }\,a}\right)
\end{equation*}
である (2008 年度大学入試センター試験数学 I より抜粋).

\subsubsection{解答例}
まず,与えられた2次関数を$f(x)$で定義する.
\begin{MapleInput}
> restart; f:=unapply(a*x^2-b*x-a+b,x);
\end{MapleInput}
\begin{MapleOutput}
 f\, := \,x\mapsto a\,{x}^{2}-b\,x-a+b
\end{MapleOutput}
与えられた点の座標を関数に入れる.
\begin{MapleInput}
> eq1:=f(-2)=6;
\end{MapleInput}
\begin{MapleOutput}
{\it eq1}\, := \,3\,a+3\,b=6
\end{MapleOutput}
これをbについて解く.
\begin{MapleInput}
> eq2:=b=solve(eq1,b);
\end{MapleInput}
\begin{MapleOutput}
{\it eq2}\, := \,b=2-a
\end{MapleOutput}
次は,頂点の座標で傾きが0になることを用いて解いていく.
\begin{MapleInput}
> solve(diff(f(x),x)=0,x);
\end{MapleInput}
\begin{MapleOutput}
\frac{1}{2}\,{\frac {b}{a}}
\end{MapleOutput}
bの値はeq2で求まっているので,それを代入(subs)する.
\begin{MapleInput}
> subs(eq2,solve(diff(f(x),x)=0,x));
\end{MapleInput}
\begin{MapleOutput}
\frac{1}{2}\,{\frac {2-a}{a}}
\end{MapleOutput}
これをx0としてeq3で定義しておく.
\begin{MapleInput}
> eq3:=x0=subs(eq2,solve(diff(f(x),x)=0,x));
\end{MapleInput}
\begin{MapleOutput}
{\it eq3}\, := \,{\it x0}=\frac{1}{2}\,{\frac {2-a}{a}}
\end{MapleOutput}
頂点のy座標は,$f(x0)$で求まる
\begin{MapleInput}
> f(x0);
\end{MapleInput}
\begin{MapleOutput}
a\,{{\it x0}}^{2}-b\,{\it x0}-a+b
\end{MapleOutput}
eq2, eq3で求まっているx0, bを代入する.
\begin{MapleInput}
> eq4:=subs({eq2,eq3},f(x0));
\end{MapleInput}
\begin{MapleOutput}
{\it eq4}\, := \,-\frac{1}{4}\,{\frac { \left( 2-a \right) ^{2}}{a}}-2\,a+2
\end{MapleOutput}
これを因数分解(factor)する.
\begin{MapleInput}
> factor(subs({eq2,eq3},f(x0)));
\end{MapleInput}
\begin{MapleOutput}
-\frac{1}{4}\,{\frac { \left( 3\,a-2 \right) ^{2}}{a}}
\end{MapleOutput}

