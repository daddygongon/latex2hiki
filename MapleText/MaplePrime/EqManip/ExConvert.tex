\begin{enumerate}
\item 以下の関数をx0まわりで3次までテイラー展開し,得られた関数ともとの関数をプロットせよ.さらに高次まで展開した場合はどう変化するか.

i) $y =\cos \left(x \right),{\it x0} =0$, ii) $y =\ln\left(x \right),{\it x0} =1$, iii) $y =\exp \left(-x \right),{\it x0} =0$
\item $\displaystyle \frac{x+1}{(x-1)(x^2+1)^2}$を部分分数に展開せよ.
\item $\displaystyle \frac{1}{1-x^4} = \frac{a}{x^2+1}+\frac{b}{x+1}+\frac{c}{x-1}$が常に成立する$a, b, c$を定めよ.
\item $\displaystyle \frac{8}{3-\sqrt{5}}-\frac{2}{2+\sqrt{5}}$を簡単化せよ.
\item $\displaystyle x^2+2kx+5-k=0$が重根をもつように$k$を定めよ.
\end{enumerate}