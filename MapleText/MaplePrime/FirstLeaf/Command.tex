\subsubsection{コマンドの基本形(command();)}
コマンドは全て次のような構文を取る.
\begin{MapleInput}
> command(引数1,引数2,...);
\end{MapleInput}
あるいは
\begin{MapleInput}
> command(引数1,オプション1,オプション2,...);
\end{MapleInput}
となる.

\begin{enumerate}
\item ()の中の引数やオプションの間はコンマで区切る.
\item 最後の;(セミコロン)は次のコマンドとの区切り記号.
\item セミコロン(;)をコロン(:)に替えるとMapleからの返答が出力されなくなるが,Mapleへの入力は行われている.
\item C言語などの手続き型プログラミング言語の標準的なフォーマットと同じ.
\end{enumerate}

