\subsubsection{間違い(Error)}
いくつかの典型的な間違い.
先ずは,左右の括弧の数が合ってないとき.
\begin{MapleInput}
> plot(sin(x,x=-Pi..Pi);
\end{MapleInput}
\begin{MapleError}
Error, `;` unexpected
\end{MapleError}
正しくは,
\begin{MapleInput}
> plot(sin(x),x=-Pi..Pi);
\end{MapleInput}
です.関数の中に変数が残ったままplotしようとしたとき.
\begin{MapleInput}
> plot(sin(a*x),x=-Pi..Pi);
\end{MapleInput}
\begin{MapleError}
Warning, unable to evaluate the function to numeric values in the region; see
the plotting command's help page to ensure the calling sequence is correct
\end{MapleError}
\MaplePlot{30mm}{./figures/FirstLeafplot2d2.eps}
何も表示されない.xに1を代入しても,sin(a*x)からは数値ではなく記号で答えが返って来ている.plotは関数が数値を返したときしか表示できない.以下のように,変数aのかわりに数値を入れる.
\begin{MapleInput}
> plot(sin(2*x),x=-Pi..Pi);
\end{MapleInput}
\MaplePlot{50mm}{./figures/FirstLeafplot2d3.eps}


