\subsubsection{plot3d}
plot図に対するいくつかの操作は,plotを選んだときにwindowの上部に表示されるリボンのアイコンにあります.同じ操作は,plot3dに与えるoptionsによっても可能です.以下でいくつかのoptionを示します.詳しくは\texttt{?plot3d[option];}を参照.

\begin{enumerate}
\item
\begin{MapleInput}
> plot3d(sin(x)*cos(y),x=-Pi..Pi,y=-Pi..Pi);}
\end{MapleInput}
\MaplePlot{30mm}{./figures/FirstLeafplot3d8.eps}
\item
等高線は
\begin{MapleInput}
> plot3d(sin(x)*cos(y),x=-Pi..Pi,y=-Pi..Pi,style=contour);
\end{MapleInput}
\MaplePlot{30mm}{./figures/FirstLeafplot3d9.eps}

軸の変更は
\begin{MapleInput}
> plot3d(sin(x)*cos(y),x=-Pi..Pi,y=-Pi..Pi,axes=boxed);
\end{MapleInput}
\MaplePlot{30mm}{./figures/FirstLeafplot3d10.eps}

\item
デフォルトの角度もplot3dのoptionで変更することが可能です.
\begin{MapleInput}
> plot3d(sin(x)*cos(y),x=-Pi..Pi,y=-Pi..Pi,orientation=[45,80]);
\end{MapleInput}
\MaplePlot{30mm}{./figures/FirstLeafplot3d11.eps}
\end{enumerate}