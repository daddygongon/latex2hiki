\subsubsection{Equals}
\begin{enumerate}
\item a=3, b=4としてa,bの四則演算をおこなえ.また,べき乗$a^b$を求めよ.
\begin{MapleInput}
> a:=3:
> b:=4:
> a+b;a-b;a*b;a/b;a^b;      #省略
\end{MapleInput}

\item eq1=3*x+4=2*x-2, a=2とした場合のeq1/a,eq1+aを試し,両辺を観察せよ.
\begin{MapleInput}
> eq1:=3*x+4=2*x-2;
> a:=2;
> eq1/a;
> eq1+2;
\end{MapleInput}
\begin{MapleOutputGather}
eq1:=3x+4=2x-2 \notag \\
a:=2 \notag \\
\frac{3}{2}x+2 = x-1 \notag \\
3x+6=2x
\end{MapleOutputGather}

\item 3点(1,2),(-3,4),(-1,1)を通る2次方程式を求めよ.

まず2次関数を定義する.
\begin{MapleInput}
> restart;
> f:=x->a*x^2+b*x+c;
\end{MapleInput}
\begin{MapleOutput}
f := x \mapsto ax^2+bx+c
\end{MapleOutput}
(1,2)を通ることから,f(1)=2が成立.これをeq1として保存.
\begin{MapleInput}
> eq1:=f(1)=2;
\end{MapleInput}
\begin{MapleOutput}
eq1 := a+b+c=2
\end{MapleOutput}
他の点も同様に
\begin{MapleInput}
> eq2:=f(-3)=4;
> eq3:=f(-1)=1;
\end{MapleInput}
\begin{MapleOutputGather}
eq2\, := \,9\,a-3\,b+c=4 \notag\\
eq3\, := \,a-b+c=1 \notag
\end{MapleOutputGather}

この3個の連立方程式から,a,b,cを求めれば解となる.
\begin{MapleInput}
> solve({eq1,eq2,eq3},{a,b,c});
\end{MapleInput}
\begin{MapleOutput}
\left\{ a=1/2,b=1/2,c=1 \right\}
\end{MapleOutput}

\item 方程式$\sin(x+1)-x^2=0$の2つの解をfsolveのヘルプを参照して求めよ.

まず,2つの関数とみなしてプロット.
\begin{MapleInput}
> plot([sin(x+1),x^2],x=-1..1);
\end{MapleInput}
\MaplePlot{30mm}{./figures/Equalsplot2d1.eps}
解が2つあることに注意.与えられた関数値が0となる方程式として定義し,これをsolveでとく.
\begin{MapleInput}
> eq1:=sin(x+1)-x^2=0;
> solve(eq1,x);
\end{MapleInput}
\begin{MapleOutputGather}
eq1 := \,\sin \left( x+1 \right) -{x}^{2}=0 \notag\\
-1+{\it RootOf} \left( -\sin \left( {\it \_Z} \right) +1-2\,{\it \_Z}+{{\it \_Z}}^{2} \right) \notag
\end{MapleOutputGather}
これでは解を求めてくれないので,fsolveで数値解を求める.
\begin{MapleInput}
> fsolve(eq1,x);
\end{MapleInput}
\begin{MapleOutput}
0.9615690350
\end{MapleOutput}
これではxの負にあるもう一つの解がでない.これを解決するには,fsolveでxに初期値を入れて実行する.
\begin{MapleInput}
> fsolve(eq1,x=-1..0);
\end{MapleInput}
\begin{MapleOutput}
-.6137631294
\end{MapleOutput}

\end{enumerate}
