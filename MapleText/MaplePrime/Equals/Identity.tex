\subsection{恒等式(Identity)}
式の変形にも等号が使われる.例えば,
\begin{equation*}
(x-2)^2=x^2-4x+4
\end{equation*}
というのが等号で結ばれている.式の変形とは,変数$x$がどんな値であっても成り立つ恒等的な変形である.

この式変形も,問題としては,
\begin{MapleError}
(x-2)^2を展開(expand)せよ
\end{MapleError}
と与えられるので,そのままMapleコマンドに読み替えて
\begin{MapleInput}
> expand( (x-2)^2 );
\end{MapleInput}
\begin{MapleOutput}
x^2-4x+4
\end{MapleOutput}
とすればよい.因数分解(factor)や微分(diff)・積分(int)も同様に等号で結ばれるが,Mapleには操作を指示する必要がある.詳しくは他の単元で.
