固有値を求める手法として,永年方程式を解くというやり方は回りくどすぎる.少し古めかしいが非対角要素を0にする回転行列を反復的に作用させるJacobi(ヤコビ)法を紹介する.現在認められている最適の方策は,ハウスホルダー(Householder)変換で行列を単純な三重対角化行列に変形してから,反復法で解を追い込んでいくやり方である.Jacobi法は,Householder法ほど万能ではないが,10次程度までの行列には今でも役に立つ.

\subsection{Mapleでみる回転行列}
行列の軸回転の復習をする.対称行列$B$に回転行列$U$を作用すると
\begin{equation}
B.U =  
\left(
\begin{array}{cc}
{a_{1\,1}} & {a_{1\,2}}\\
{a_{2\,1}(={a_{1\,2}})} & {a_{2\,2}}
\end{array}
 \right)
\left( 
\begin{array}{cc}
\cos(\theta) &  -\sin(\theta)\\
\sin(\theta) & \cos(\theta)
\end{array}
 \right) 
\end{equation}

\ifHIKI
\begin{tabular}{|c|}
\hline
         \\ 
\hline
\end{tabular}
\else
\begin{equation*}
\setlength{\unitlength}{1cm}
\begin{picture}(10,3.5)
\put(0,0){\framebox(10,3.5){}}
\end{picture}
\end{equation*}
\fi
となる.回転行列を4x4の行列に
\begin{equation}
U^t B U
\end{equation}
と作用させたときの各要素の様子を以下に示した.
\begin{MapleInput}
> restart:
> n:=4:
> with(LinearAlgebra):
> B:=Matrix(n,n,shape=symmetric,symbol=a);
\end{MapleInput}

\begin{MapleOutput}
B :=  \left[{
	\begin{array}{cccc}
	{a_{1, \,1}} & {a_{1, \,2}} & {a_{1, \,3}} & {a_{1, \,4}} \\
	{a_{1, \,2}} & {a_{2, \,2}} & {a_{2, \,3}} & {a_{2, \,4}} \\
	{a_{1, \,3}} & {a_{2, \,3}} & {a_{3, \,3}} & {a_{3, \,4}} \\
	{a_{1, \,4}} & {a_{2, \,4}} & {a_{3, \,4}} & {a_{4, \,4}}
	\end{array}}
 \right] 
\end{MapleOutput}

\begin{MapleInput}
> U:=Matrix(n,n,[[c,-s,0,0],[s,c,0,0],[0,0,1,0],[0,0,0,1]]);
#U:=Matrix(n,n,[[c,-s],[s,c]]);
\end{MapleInput}
\begin{MapleOutput}
U :=  \left[ 
{\begin{array}{ccrr}
c &  - s & 0 & 0 \\
s & c & 0 & 0 \\
0 & 0 & 1 & 0 \\
0 & 0 & 0 & 1
\end{array}}
 \right] 
\end{MapleOutput}

\begin{MapleInput}
>TT:=Transpose(U).B.U;
\end{MapleInput}

\begin{MapleOutputGather}
\mathit{TT} :=  \\ \notag
{\begin{array}{c}
 \left[ \right.  
  (c\,{a_{1, \,1}} + s\,{a_{1, \,2}})\,c + (c\,{a_{1, \,2}} + s
\,{a_{2, \,2}})\,s\,, \, - (c\,{a_{1, \,1}} + s\,{a_{1, \,2}})\,s
 + (c\,{a_{1, \,2}} + s\,{a_{2, \,2}})\,c\,,  \\
c\,{a_{1, \,3}} + s\,{a_{2, \,3}}\,, \,c\,{a_{1, \,4}} + s\,{a_{2
, \,4}}   \left. \right]  \\
 \left[ \right.  
  ( - s\,{a_{1, \,1}} + c\,{a_{1, \,2}})\,c + ( - s\,{a_{1, \,2
}} + c\,{a_{2, \,2}})\,s\,, \, - ( - s\,{a_{1, \,1}} + c\,{a_{1, 
\,2}})\,s + ( - s\,{a_{1, \,2}} + c\,{a_{2, \,2}})\,c\,,  \\
 - s\,{a_{1, \,3}} + c\,{a_{2, \,3}}\,, \, - s\,{a_{1, \,4}} + c
\,{a_{2, \,4}}   \left. \right] \\
 \left[   c\,{a_{1, \,3}} + s\,{a_{2, \,3}}\,, \, - s\,{a_{1, 
\,3}} + c\,{a_{2, \,3}}\,, \,{a_{3, \,3}}\,, \,{a_{3, \,4}}  
 \right]  \\
 \left[   c\,{a_{1, \,4}} + s\,{a_{2, \,4}}\,, \, - s\,{a_{1, 
\,4}} + c\,{a_{2, \,4}}\,, \,{a_{3, \,4}}\,, \,{a_{4, \,4}}  
 \right] 
 \end{array}}
\end{MapleOutputGather}

\begin{MapleInput}
>expand(TT[1,1]);
expand(TT[2,2]);
expand(TT[1,2]);
expand(TT[2,1]);
\end{MapleInput}
\begin{MapleOutput}
c^{2}\,{a_{1, \,1}} + 2\,c\,s\,{a_{1, \,2}} + s^{2}\,{a_{2, \,2}}
\end{MapleOutput}
\begin{MapleOutput}
s^{2}\,{a_{1, \,1}} - 2\,c\,s\,{a_{1, \,2}} + c^{2}\,{a_{2, \,2}}
\end{MapleOutput}
\begin{MapleOutput}
 - s\,c\,{a_{1, \,1}} - s^{2}\,{a_{1, \,2}} + c^{2}\,{a_{1, \,2}}
 + c\,s\,{a_{2, \,2}}
\end{MapleOutput}
\begin{MapleOutput}
 - s\,c\,{a_{1, \,1}} - s^{2}\,{a_{1, \,2}} + c^{2}\,{a_{1, \,2}}
 + c\,s\,{a_{2, \,2}}
\end{MapleOutput}
この非対角要素を0にする$\theta$は以下のように求まる.
\ifHIKI
\begin{tabular}{|c|}
\hline
         \\ 
\hline
\end{tabular}
\else
\begin{equation*}
\setlength{\unitlength}{1cm}
\begin{picture}(10,3.5)
\put(0,0){\framebox(10,3.5){}}
\end{picture}
\end{equation*}
\fi
このとき注目している$i,j=1,2$以外の要素も変化する.
\begin{MapleInput}
>expand(TT[3,1]);
expand(TT[3,2]);
\end{MapleInput}
\begin{MapleOutput}
c\,{a_{1, \,3}} + s\,{a_{2, \,3}}
\end{MapleOutput}
\begin{MapleOutput}
 - s\,{a_{1, \,3}} + c\,{a_{2, \,3}}
\end{MapleOutput}
これによって一旦0になった要素も値を持つが,なんども繰り返すことによって,徐々に0へ近づいていく.