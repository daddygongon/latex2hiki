一般の数値計算ライブラリーについては,時間の関係で講義ではその能力を紹介するにとどめる.昔の演習で詳しく取り上げていたので,研究や今後のために必要と思うときは,テキストを取りにおいで.

行列の計算は,数値計算の中でも特に利用する機会が多く,また,律速ルーチンとなる可能性が高い.そこで,古くから行列計算の高速ルーチンが開発されてきた.なかでもBLASとLAPACKはフリーながら非常に高速である. 

前回に示した,逆行列を求める単純なLU分解法をC言語でコーディングしたものと,LAPACKのルーチンを比べた場合,1000次元の行列で計測すると
\begin{MapleInput}
>  1000 [dim]     2.5200 [sec] #BOB
>  1000 [dim]     0.4700 [sec] #LAPACK
\end{MapleInput}
となった.用いたPCはMacBook(2GHz Interl Core Duo)であるが,この計算での0.47秒は1.4GFLOPに相当する.07年のMacBook(2GHz Interl Core 2 Duo)ではさらに早くなって
\begin{MapleInput}
bob% gcc -O3 bob.c -o bob
bob% ./bob
1000
 1000 [dim]     1.7543 [sec] #BOB
bob% gcc -O3 lapack.c -llapack -lblas -o lapack
bob% ./lapack
1000
 1000 [dim]     0.1893 [sec] #LAPACK
\end{MapleInput}
で,3.5GFLOPSが出ている.

ライブラリーは世界中の計算機屋さんがよってたかって検証しているので,バグがほとんど無く,また,高速である.初学者はライブラリーを使うべきである.ただし,下のサンプルプログラムの行列生成の違いのように,ブラックボックス化すると思わぬ間違い(ここではFortranとCでの行列の並び順の違いが原因)をしでかすことがあるので,プログラムに組み込む前に必ず小さい次元(サンプルコード)で検証しておくこと.

添付のコードはちょっと長いが時間があればフォローせよ.コンパイルは,OSXでは
\begin{MapleInput}
> gcc -O3 -UPRINT lapack.c -llapack -lblas
\end{MapleInput}
とすればできる.linuxではLAPACK, BLASがインストールされていれば,
\begin{MapleInput}
> #include <vecLib/vecLib.h>
\end{MapleInput}
をコメントアウトして,
\begin{MapleInput}
> gcc -O3 -DPRINT lapack.c -L/usr/local/lib64 -llapack -lblas -lg2c
\end{MapleInput}
などとすればコンパイルできるはず.

\paragraph{リスト: 西谷製lazy逆行列計算プログラム} 
\begin{MapleInput}
#include <stdio.h>
#include <stdlib.h>
#include <math.h>
#include <time.h>

//#undef PRINT
//#define PRINT

void printMatrix(double *a, double *b, long n);
int MatrixInverse(double *a, double *b, long n);

int main(void){
  clock_t start, end;
  int i,j;
  long n;
  double *a,*b;

  scanf("%ld",&n);

  a=(double *)malloc(n*n*sizeof(double));
  b=(double *)malloc(n*sizeof(double));

  for (i=0;i<n;i++){
    for (j=0;j<n;j++){
      a[i*n+j]= 2*(double) random() / RAND_MAX - 1.0;
    }
  }
  for (i=0;i<n;i++){
    b[i]= 2*(double) random() / RAND_MAX - 1.0;
  }
  printMatrix(a,b,n);

  start = clock();
  MatrixInverse(a,b,n);
  end = clock();
  printf("%5d [dim] %10.4f [sec] #BOB\n", 
	 n,(double)(end-start)/CLOCKS_PER_SEC);
  printMatrix(a,b,n);

  free(a);
  free(b);
  return 0;
}

int MatrixInverse(double *a, double *b, long n){
  double *x;
  double pvt=0.00005,am;
  int i,j,k;

  x=(double *)malloc(n*sizeof(double));

  for(i=0;i<n-1;i++){
    if(fabs(a[i*n+i])<pvt){
      printf("Pivot %3d=%10.5f is too small.\n",i,a[i*n+i]);
      return 1;
    }
    for(j=i+1;j<n;j++){
      am=a[j*n+i]/a[i*n+i];
      for(k=0;k<n;k++) a[j*n+k]-=am*a[i*n+k];
      b[j]-=am*b[i];
    }
  }
  //Backward substitution
  for(j=n-1;j>=0;j--){
    x[j]=b[j];
    for(k=j+1;k<n;k++){
      x[j]-=a[j*n+k]*x[k];
    }
    b[j]=x[j]/=a[j*n+j];
  }
  free(x);
  return 0;
}

void printMatrix(double *a, double *b, long n){
  int i,j;
#ifdef PRINT
  printf("\n");
  for (i=0;i<n;i++){
    for (j=0;j<n;j++){
      printf("%10.5f",a[i*n+j]);
    }
    printf(":%10.5f",b[i]);
    printf("\n");
  }
  printf("\n");
#endif
  return;
}
\end{MapleInput}


\paragraph{リスト : LAPACK謹製smart逆行列計算プログラム} 
\begin{MapleInput}
#include <stdio.h>
#include <stdlib.h>
#include <math.h>
#include <time.h>

//#define PRINT
//#undef PRINT

void printMatrix(double *a, double *b, long n);

int main(void){
  clock_t start, end;
  int i,j;
  double *a,*b;
  long n,nrhs=1, lda,ldb, info, *ipiv;

  scanf("%ld",&n);

  a=(double *)malloc(n*n*sizeof(double));
  b=(double *)malloc(n*sizeof(double));
  lda=ldb=n;
  ipiv=(long *)malloc(n*sizeof(long));
  
  for (i=0;i<n;i++){
    for (j=0;j<n;j++){
      a[j*n+i]= 2*(double) random() / RAND_MAX - 1.0;
    }
  }

  for (i=0;i<n;i++){
    b[i]= 2*(double) random() / RAND_MAX - 1.0;
  }
  printMatrix(a,b,n);

  start = clock();
  dgesv_(&n, &nrhs, a, &lda, ipiv, b, &ldb, &info);
  end = clock();
  printf("%5d [dim] %10.4f [sec] #LAPACK\n", 
	 n, (double)(end-start)/CLOCKS_PER_SEC);
  printMatrix(a,b,n);

  free(a);
  free(b);
  free(ipiv);

  return 0;
}

void printMatrix(double *a, double *b, long n){
  int i,j;
#ifdef PRINT
  printf("\n");
  for (i=0;i<n;i++){
    for (j=0;j<n;j++){
      printf("%10.5f",a[i*n+j]);
    }
    printf(":%10.5f",b[i]);
    printf("\n");
  }
  printf("\n");
#endif
  return;
}
\end{MapleInput}



