\begin{quote}
多くの良質なページからリンクされているページはやはり良質なページである
\end{quote}
Googleのpage rankは上のような非常に単純な仮定から成り立っている.ページランクを実際に求めよう.つぎのようなリンクが張られたページを考える.
\MaplePlot{70mm}{./figures/linkstruct.eps}

計算手順は以下の通り\footnote{詳しくは\texttt{http://www.kusastro.kyoto-u.ac.jp/\~baba/wais/pagerank.html}を参照せよ.}.
\begin{enumerate}
\item リンクを再現する隣接行列を作る.ページに番号をつけて,その間が結ばれているi-j要素を1,そうでない要素を0とする.
\item 隣接行列を転置する
\item 列ベクトルの総和が1となるように規格化する.
\item こうして得られた推移確率行列の最大固有値に属する固有ベクトルを求め,適当に規格化する.
\end{enumerate}

\subsection{課題}
\begin{enumerate}
\item 上記手順を参考にして,Mapleでページランクを求めよ.
\item このような問題ではすべての固有値・固有ベクトルを求める必要はなく,最大の固有値を示す固有ベクトルを求めるだけでよい.初期ベクトルを適当に決めて,何度も推移確率行列を掛ける反復法でページランクを求めよ.
\end{enumerate}
\begin{itemize}
\item[隣接行列]
\begin{equation*}
{\it A1}\, := \, \left[ \begin {array}{c|c|c|c|c|c|c|c} 
 &1&2&3&4&5&6&7\\
1&0&1&1&1&1&0&1\\
2&1&0&0&0&0&0&0\\
3& & & & & & & \\
4& & & & & & & \\
5& & & & & & & \\
6& & & & & & & \\
7& & & & & & & 
\end {array} \right] 
\end{equation*}
\item[転置行列]
\begin{equation*}
{Transpose}({\it A1})\, := \, \left[ \begin {array}{c|c|c|c|c|c|c} 
\, \, &\, \, &\, \, &\, \, &\, \, &\, \, &\, \, \\
 & & & & & & \\
 & & & & & & \\
 & & & & & & \\
 & & & & & & \\
 & & & & & & \\
 & & & & & & 
\end {array} \right] 
\end{equation*}
\item[規格化]
\begin{equation*}
\left[ \begin {array}{c|c|c|c|c|c|c} 
\, \, &\, \, &\, \, &\, \, &\, \, &\, \, &\, \, \\
 & & & & & & \\
 & & & & & & \\
 & & & & & & \\
 & & & & & & \\
 & & & & & & \\
 & & & & & & 
\end {array} \right] 
\end{equation*}
\item[遷移]
\begin{equation*}
\left( \begin {array}{ccccccc} 
0 &1 &1/2 &0 &1/4 &1/2 &0 \\
1/5 &0 &1/2 &1/3 &0 &0 &0 \\
1/5 &0 &0 &1/3 &1/4 &0 &0 \\
1/5 &0 &0 &0 &1/4 &0 &0 \\
1/5 &0 &0 &1/3 &0 &1/2 &1 \\
0 &0 &0 &0 &1/4 &0 &0 \\
1/5 &0 &0 &0 &0 &0 &0 
\end {array} \right) 
\left( \begin {array}{c} 
1/7\\ 
1/7\\ 
1/7\\ 
1/7\\ 
1/7\\ 
1/7\\ 
1/7 
\end {array} \right) \, = \, 
\left( \begin {array}{ccccccc} 
\, \, &\, \, &\, \, &\, \, &\, \, &\, \, &\, \, \\
 & & & & & & \\
 & & & & & & \\
 & & & & & & \\
 & & & & & & \\
 & & & & & & \\
 & & & & & & 
\end {array} \right)
\, = \, \left( \begin {array}{c} 
0.32\\ 
0.15\\ 
0.11\\ 
0.06\\ 
0.29\\ 
0.04\\ 
0.03 
\end {array} \right) 
\end{equation*}
\end{itemize}
\pagebreak
