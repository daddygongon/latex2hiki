\subsection{二分法(bisection)}
二分法は領域の端$x_1, x_2$で関数値$f(x_1),f(x_2)$を求め,中間の値を次々に計算して,解を囲い込んでいく方法である.
\begin{MapleInput}
> plot(func(z),z=0..0.8,gridlines=true);
\end{MapleInput}
\MaplePlot{70mm}{./figures/C2_fsolveplot2d2.eps}

\begin{table}[htbp]\begin{center}
\begin{tabular}{|c|c|c|c|}
\hline
$x_1$ & $x_2$ &$f(x_1)$ & $f(x_2)$ \\ \hline
0.0 & 0.8 &      &      \\ \hline
    &      &      &     \\ \hline
    &      &      &     \\ \hline
    &      &      &     \\ \hline
\end{tabular}
\end{center}\label{default}\end{table}%


\subsection{Newton法(あるいはNewton-Raphson法)}
Newton法は最初の点$x_1$から接線をひき,それが$x$軸(y=0)と交わった点を新たな点$x_2$とする.さらにそこでの接線を求めて...

という操作を繰り返しながら解を求める方法である.関数の微分をdf(x)とすると,これらの間には
\ifHIKI %%%%
||               ||
\else %%%%
\begin{equation*}
\setlength{\unitlength}{1cm}
\begin{picture}(10,1.5)
\put(0,0){\framebox(10,1.5){}}
\end{picture}
\end{equation*}
\fi %%%%
という関係が成り立つ.
\begin{MapleInput}
> df:=unapply(diff(func(x),x),x);
\end{MapleInput}
\ifHIKI %%%%
||               ||
\else %%%%
\begin{equation*}
\setlength{\unitlength}{1cm}
\begin{picture}(10,1.5)
\put(0,0){\framebox(10,1.5){}}
\end{picture}
\end{equation*}
\fi %%%%

\begin{MapleInput}
> with(plots):with(plottools): 
> x1:=1.0:x0:=0.0: 
> p:=plot(func(z),z=0..1.1):
> p1:=plot(df(x1)*(z-x1)+func(x1),z=0..1.1,color=blue):
> p2:=[disk([x1,func(x1)],0.02), disk([x0,0],0.02)]:
> display(p,p1,p2,gridlines=true);
\end{MapleInput}
\MaplePlot{70mm}{./figures/C2_fsolveplot2d3.eps}

\begin{table}[h]\begin{center}\begin{tabular}{|l|c|c|}
\hline
$x_1$ &$f(x_1)$ & $df(x_1)$ \\ \hline
1.0 &      &     \\  \hline
       &         &        \\  \hline
       &         &        \\  \hline
\end{tabular}\end{center}\label{default}\end{table}%

