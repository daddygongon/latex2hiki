代数方程式の解f(x)=0を数値的に求めることを考える.標準的な
\begin{center}
二分法(bisection method)とニュートン法(Newton's method)
\end{center}
の考え方と例を説明し,
\begin{center}
収束性(convergency)と安定性(stability)
\end{center}
について議論する.さらに収束判定条件について言及する.


二分法のアイデアは単純.中間値の定理より連続な関数では,関数の符号が変わる二つの変数の間には根が必ず存在する.したがって,この方法は収束性は決して高くはないが,
確実.一方,Newton法は関数の微分を用いて収束性を速めた方法である.しかし,不幸にして収束しない場合や微分に時間がかかる場合があり,初期値や使用対象には注意
を要する.
