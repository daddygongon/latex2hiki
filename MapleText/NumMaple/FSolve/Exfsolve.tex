\begin{enumerate}
\item Newton法の$f(x), df(x)$の関係を示す式を導け.
\item 次の関数 $f(x) = \exp(-x)-2\exp(-2x)$ の解を二分法,Newton法で求めよ.
\item 代数方程式に関する次の課題に答えよ.(2004年度期末試験)
\begin{enumerate}
\item $\exp(-x) = x^2$を二分法およびニュートン法で解け.
\item $n$回目の値$x_n$と小数点以下10桁まで求めた値$x_f=0.7034674225$との差$\Delta x_n$の絶対値(abs)のlogを$n$の関数としてプロットし,その収束性を比較せよ.また,その傾きの違いを両解法の原理から説明せよ.
\end{enumerate}
\item 次の方程式 $f(x) = x^4-x-0.12$ の正数解を二分法で求めよ.(2008年度期末試験)
\item 収束条件がうまく機能しない例を示せ.
\item 割線法は,微分がうまく求まらないような場合に効率がよい,二分法を改良した方法である.二分法では新たな点を元の2点の中点に取っていた.そのかわりに下図に示すごとく,新たな点を元の2点を直線で内挿した点に取る.二分法のコードを少し換えて,割線法のコードを書け.また,収束の様子を二分法,Newton法と比べよ.
\begin{MapleInput}
> func:=x->x^2-4*x+1: x1:=0: x2:=2: f1:=func(x1): f2:=func(x2):
> plot({(z-x1)*(f1-f2)/(x1-x2)+f1,func(z)},z=0..2);
\end{MapleInput}
\MaplePlot{50mm}{./figures/C2_fsolveplot2d6.eps}
\item 次の方程式 $f(x) = \cos(x)-x^2$ の正数解を二分法で求めよ.割線法でも求め,収束性を比べよ.(2009年度期末試験)
\item 次の方程式 $f(x)=x^3-3x+3$の解をニュートン法で求めよ.初期値をそれぞれ$x = -3, x = 2$とした時を比べ,その差について論ぜよ.(2010年度期末試験)
\end{enumerate}
