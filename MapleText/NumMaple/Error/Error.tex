\documentclass[10pt,a4j]{jreport}
\usepackage[dvips]{graphicx,color}
\usepackage{verbatim}
\usepackage{amsmath,amsthm,amssymb}
\topmargin -15mm\oddsidemargin -4mm\evensidemargin\oddsidemargin
\textwidth 170mm\textheight 257mm\columnsep 7mm
\setlength{\fboxrule}{0.2ex}
\setlength{\fboxsep}{0.6ex}

\pagestyle{empty}

\newcommand{\MaplePlot}[2]{{\begin{center}
    \includegraphics[width=#1,clip]{#2}
                     \end{center}
%
} }

\newenvironment{MapleInput}{%
    \color{red}\verbatim
}{%
    \endverbatim
}

\newenvironment{MapleError}{%
    \color{blue}\verbatim
}{%
    \endverbatim
}

\newenvironment{MapleOutput}{%
    \color{blue}\begin{equation*}
}{%
    \end{equation*}
}

\newenvironment{MapleOutputGather}{%
    \color{blue}\gather
}{%
    \endgather
}

\newif\ifHIKI

\begin{document}
\chapter{誤差(Error)}
\section{打ち切り誤差と丸め誤差(Truncation and round off errors)}
\input{TruncationRoundoff.tex}
\section{有効桁数(Significant digits)}
1 ワードの整数の最大値とその2進数表示.
\begin{MapleInput}
> restart;
> 2^(4*8-1)-1;#res: 2147483647

\end{MapleInput}
この整数を2進数で表示するように変換するには,convert(n,binary)を用いて,
\begin{MapleInput}
> convert(2^(4*8-1)-1,binary);  #res: 1111111111111111111111111111111
\end{MapleInput}
となり,31個の1が並んでいることが分かる.1 ワードの整数の最大桁は,$n$ の桁数を戻すコマンドlength(n)を使って,
\begin{MapleInput}
> length(2^(4*8-1)-1); #res: 10
\end{MapleInput}
となり,たかだか10桁程度であることが分かる.一方,64bit の場合の整数の最大桁.
\begin{MapleInput}
> length(2^(8*8-1)-1); #res: 19
\end{MapleInput}
である.
                                    
Maple では多倍長計算するので,通常のプログラミング言語で起こるintの最大数あたりでの奇妙な振る舞いは示さない.

\begin{MapleInput}
> 2147483647+100;  #res: 2147483747
\end{MapleInput}

単精度の浮動小数点数は,仮数部2進数23bit,2倍長実数で52bitである.この有効桁数は以下の通り.
\begin{MapleInput}
> length(2^(23)); #res: 7
> length(2^(52)); #res: 16
\end{MapleInput}

\section{浮動小数点演算による過ち(FloatingPointArithmetic)}
「丸め」にともなって誤差が生じる. CやFortran等の通常のプログラミング言語では「丸める」仕様なのでプログラマーが気をつけなければならない.
\begin{MapleInput}
プログラムリスト : 実数のケタ落ち
#include <stdio.h>

int main(void){
  float a,b,c;
  double x,y,z;

  a=1.23456789;
  printf(" a= %17.10f\n",a);

  b=100.0;
  c=a+b;
  printf("%20.10f %20.10f %20.10f\n",a,b,c);

  x=(float)1.23456789;
  y=(double)100;
  z=x+y;
  printf("%20.12e %20.12e %20.12e\n",x,y,z);
 
  x=(double)1.23456789;
  y=(double)100;
  z=x+y;
  printf("%20.12e %20.12e %20.12e\n",x,y,z);

  return 0;
}
\end{MapleInput}

分かっているつもりでも,よくやる間違い.
\begin{MapleInput}
プログラムリスト : 丸め誤差
#include <stdio.h>

int main(void){
  float x=77777,y=7,y1,z,z1;
  y1=1/y;
  z=x/y;
  z1=x*y1;
  printf("%10.2f %10.2f\n",z,z1);
  if (z!=z1){
    printf("z is not equal to z1.\n");
  }
  printf("Surprising?? \n\n\n\n\n%10.5f %10.5f\n",z,z1);
  return 0; 
}
\end{MapleInput}
これを避けるには,EPSILONという小さな数字を定義しておいて,値の差の絶対値を求めるfabsを使って
\begin{table}[h]\begin{center}\begin{tabular}{|c|}
\hline
\hspace{100mm} \\ 
\\
\\
\hline
\end{tabular}\end{center}\end{table}%

とすべき.このときは数学関数であるfabsを使っているので,
\begin{MapleInput}
> gcc -lm test.c
\end{MapleInput}
とmath libraryを明示的に呼ぶのを忘れないように.
\section{機械精度(Machine epsilon)}
上の例では,浮動小数点数で計算した場合に小さい数の差を区別することができなくなるということを示している.これは,CPUに固有の精度で,機械精度(Machine epsilon)と呼ばれる.つまり,小さい数を足したときにその計算機がその差を認識できなくなる限界ということで,以下のようにして求めることができる.
\begin{MapleInput}
> Digits:=7; 
> e:=evalf(1.0);
> w:=evalf(1.0+e); 
> while (w>1.0) do 
    printf("%-15.10e %-15.10e %-15.10e\n",e,w,evalf(w-1.0)); 
    e:=evalf(e/2.0); 
    w:=evalf(1.0+e); 
  end do:
\end{MapleInput}
\begin{MapleOutputGather}
7 \notag \\
1.0 \notag \\
2.0 \notag
\end{MapleOutputGather}
\begin{MapleError}
1.0000000000e+00 2.0000000000e+00 1.0000000000e+00
5.0000000000e-01 1.5000000000e+00 5.0000000000e-01
2.5000000000e-01 1.2500000000e+00 2.5000000000e-01
1.2500000000e-01 1.1250000000e+00 1.2500000000e-01
6.2500000000e-02 1.0625000000e+00 6.2500000000e-02
3.1250000000e-02 1.0312500000e+00 3.1250000000e-02
1.5625000000e-02 1.0156250000e+00 1.5625000000e-02
7.8125000000e-03 1.0078120000e+00 7.8120000000e-03
3.9062500000e-03 1.0039060000e+00 3.9060000000e-03
1.9531250000e-03 1.0019530000e+00 1.9530000000e-03
9.7656250000e-04 1.0009770000e+00 9.7700000000e-04
4.8828120000e-04 1.0004880000e+00 4.8800000000e-04
2.4414060000e-04 1.0002440000e+00 2.4400000000e-04
1.2207030000e-04 1.0001220000e+00 1.2200000000e-04
6.1035150000e-05 1.0000610000e+00 6.1000000000e-05
3.0517580000e-05 1.0000310000e+00 3.1000000000e-05
1.5258790000e-05 1.0000150000e+00 1.5000000000e-05
7.6293950000e-06 1.0000080000e+00 8.0000000000e-06
3.8146980000e-06 1.0000040000e+00 4.0000000000e-06
1.9073490000e-06 1.0000020000e+00 2.0000000000e-06
9.5367450000e-07 1.0000010000e+00 1.0000000000e-06
\end{MapleError}
\section{桁落ち,情報落ち,積み残し(Cancellation)}
\paragraph{桁落ち(Cancellation)} 
\ifHIKI
\begin{MapleInput}
   0.723657
-  0.723649
------------

\end{MapleInput}
\else
\begin{equation*}
\begin{array}{rr}
  & 0.723657 \\
- & 0.723649\\ \hline
\\
\end{array}
\end{equation*}

\fi

\paragraph{情報落ち(Loss of Information)} 
\ifHIKI
\begin{MapleInput}
   72365.7
-      1.23659
------------

\end{MapleInput}
\else

\begin{equation*}
\begin{array}{rr@{.}l}
  & 72365&7 \\
+& 1&23659\\ \hline
\\
\end{array}
\end{equation*}
\fi

\paragraph{積み残し} 
\ifHIKI
\begin{MapleInput}
   72365.7
-      0.001
------------

\end{MapleInput}
\else
\begin{equation*}
\begin{array}{rr@{.}l}
  & 72365&7 \\
+& 0&001\\ \hline
\\
\end{array}
\end{equation*}
\fi
\section{課題}
\begin{enumerate}
\item 次の項目について答えよ.(2004, 05, 06年度期末試験)
\begin{enumerate}
\item 数値計算の精度を制約するデータ形式とその特徴は何か.
\item 丸め誤差とは何か.
\item 打ち切り誤差とは何か.
\item 安定性とは何か.
\end{enumerate}
\item 10 進数 4 桁の有効桁数をもった計算機になったつもりで,以下の計算をおこなえ. 
 (a) 2718-0.5818 (b) 2718+0.5818 (c)  2718/0.5818 (d) 2718*0.5818
\item 自分の計算機で機械精度がどの位かを確かめよ.Maple スクリプトを参照して, C あるいは Fortran で作成し, 適当に調べよ.
\item (2147483647 + 100) を C あるいは Fortran で試せ.
\item 係数を a = 1,  b = 10000000,  c = 1としたときに, 通常の解の公式を使った解と, 解と係数の関係(下記の記述を参照)を使った解とを出力するプログラムを作成し, 解を比べよ.

2 次方程式 $ax^2+bx+c=0$の
係数$a, b, c$が特殊な値をもつ場合,通常の解の公式 

\begin{equation*}
x = \frac {-b \pm \sqrt{{b}^{2}-4ac}}{2a}
\end{equation*}
にしたがって計算するとケタ落ちによる間違った答えを出す.その特殊な値とは

\begin{equation*}
\sqrt{{b}^{2}-4ac} \approx |b|
\end{equation*}
となる場合である.

ケタ落ちを防ぐには, $b > 0$の場合は,  

\begin{equation*}
x_1 = \frac {-b - \sqrt{{b}^{2}-4ac}}{2a}
\end{equation*}
として,ケタ落ちを起こさずに求め, この解を使って, 解と係数の関係より

\begin{equation*}
x_2 = \frac {c}{a\,  x_1}
\end{equation*}
で求める.$b < 0$ の場合は,解の公式の足し算の方を使って同様に求める.

\end{enumerate}




\end{document}
