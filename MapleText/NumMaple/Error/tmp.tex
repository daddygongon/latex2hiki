%% Created by Maple 15.01, Mac OS X
%% Source Worksheet: C3_Error.mw
%% Generated: Mon Aug 06 10:38:32 JST 2012
\documentclass{article}
\usepackage{maplestd2e}
\def\emptyline{\vspace{12pt}}
\begin{document}
\pagestyle{empty}
\DefineParaStyle{Maple Heading 1}
\DefineParaStyle{Maple Text Output}
\DefineParaStyle{Maple Dash Item}
\DefineParaStyle{Maple Bullet Item}
\DefineParaStyle{Maple Normal}
\DefineParaStyle{Maple Heading 4}
\DefineParaStyle{Maple Heading 3}
\DefineParaStyle{Maple Heading 2}
\DefineParaStyle{Maple Warning}
\DefineParaStyle{Maple Title}
\DefineParaStyle{Maple Error}
\DefineCharStyle{Maple Hyperlink}
\DefineCharStyle{Maple 2D Math}
\DefineCharStyle{Maple Maple Input}
\DefineCharStyle{Maple 2D Output}
\DefineCharStyle{Maple 2D Input}
\section{\begin{center}
\textbf{誤差}\end{center}
}
\begin{maplelatex}\begin{center}
ー数値計算(11/10/7)ー\end{center}
\end{maplelatex}
\begin{maplelatex}\begin{center}
関西学院大学理工学部 西谷滋人\end{center}
\end{maplelatex}
\begin{maplelatex}\begin{flushright}
Copyright @2007-11 by Shigeto R. Nishitani\end{flushright}
\end{maplelatex}
\section{\textbf{誤差は数値計算の肝}}
\begin{maplegroup}
\begin{Maple Normal}{
数値計算のねらいは,できるだけ精確・高速に解を得ることである.\textbf{誤差 (精度) と収束性 (安定性,速度)}が数値計算のキモとなる.前回に説明した収束判定条件による誤差は\textbf{打ち切り誤差 (truncation error)}と呼ばれる.ここでは,誤差のもう一つの代表例である,計算機に特有の\textbf{丸め誤差 (roundoff error)}について見ておこう.}\end{Maple Normal}

\end{maplegroup}
\section{\textbf{整数型と実数型の内部表現}}
\begin{maplegroup}
\begin{Maple Normal}{
計算機は一般に無限精度の計算をおこなっているわけではない.CPU で足し算をおこなう以上,一般的な計算においては CPU が扱う一番効率のいい数の大きさが存在する.これが,32bit の CPU では 1 ワード,4byte(4x8bits) である.1 ワードで表現できる最大整数は,符号に 1bit 必要なので,2(31)-1 となる.実数は以下のような仮数部と指数を取る浮動小数点数で表わされる.}\end{Maple Normal}

\end{maplegroup}
\subsection{\begin{maplegroup}
\begin{mapleinput}
\mapleinline{active}{1d}{}{}
\end{mapleinput}
\end{maplegroup}
\textbf{表 1: 浮動小数点数の内部表現 (IEEE754).}}
\begin{maplegroup}
\begin{Maple Normal}{
s × f × Be-E}\end{Maple Normal}
\begin{Maple Normal}{
s:sign bit(符号ビット:正負の区別を表す)}\end{Maple Normal}
\begin{Maple Normal}{
e:biased exponent(指数部)}\end{Maple Normal}
\begin{Maple Normal}{
f:fraction portion of the number(仮数部)}\end{Maple Normal}

\begin{Maple Normal}{
real(単精度):s=1, e=8, f=23 E=127double precision(倍精度):s=1, e=11, f=52 E=1023}\end{Maple Normal}
\end{maplegroup}
\begin{maplegroup}
\begin{Maple Normal}{
B は base(基底) で通常は 2,E は bias(下駄) と呼ばれ,指数が負となる小数点以下の数を表現するためのもの.演算結果は実際の値から浮動小数点数に変換するための操作「丸め (round-off )」が常に行われ,それに伴って現れる誤差を丸め誤差と呼ぶ.}\end{Maple Normal}

\end{maplegroup}
\section{\textbf{Significant digits(有効桁数)}}
\begin{maplegroup}
\begin{Maple Normal}{
1 ワードの整数の最大値とその 2 進数表示.}\end{Maple Normal}

\textbf{restart;}
\textbf{2\symbol{94}(4*8-1)-1;}
\textbf{convert(2\symbol{94}(4*8-1)-1,binary); \#convert(n,binary):n の 2 進数表示}\mapleresult
\begin{maplelatex}
\mapleinline{inert}{2d}{2147483647}{\[\displaystyle 2147483647\]}
\end{maplelatex}
\mapleresult
\begin{maplelatex}
\mapleinline{inert}{2d}{1111111111111111111111111111111}{\[\displaystyle 1111111111111111111111111111111\]}
\end{maplelatex}
\end{maplegroup}
\begin{maplegroup}
\begin{Maple Normal}{
1 ワードの整数の最大桁.}\end{Maple Normal}

\textbf{length(2\symbol{94}(4*8-1)-1);\#length(n):n の桁数}\mapleresult
\begin{maplelatex}
\mapleinline{inert}{2d}{10}{\[\displaystyle 10\]}
\end{maplelatex}
\end{maplegroup}
\begin{maplegroup}
\begin{Maple Normal}{
64bit の場合の整数の最大桁.}\end{Maple Normal}

\textbf{length(2\symbol{94}(8*8-1)-1);}\mapleresult
\begin{maplelatex}
\mapleinline{inert}{2d}{19}{\[\displaystyle 19\]}
\end{maplelatex}
\end{maplegroup}
\begin{maplegroup}
\begin{Maple Normal}{
Maple では多倍長計算するので,通常のプログラミング言語で起こる int の最大数あたりでの奇妙な振る舞いは示さない.}\end{Maple Normal}

\textbf{2147483647+100;}\mapleresult
\begin{maplelatex}
\mapleinline{inert}{2d}{2147483747}{\[\displaystyle 2147483747\]}
\end{maplelatex}
\end{maplegroup}
\begin{maplegroup}
\begin{Maple Normal}{
単精度の浮動小数点数は,仮数部 2 進数 23bit,2 倍長実数で 52bit である.この有効桁数は以下の通り.}\end{Maple Normal}

\textbf{length(2\symbol{94}(23));}\textbf{length(2\symbol{94}(52));}\mapleresult
\begin{maplelatex}
\mapleinline{inert}{2d}{7}{\[\displaystyle 7\]}
\end{maplelatex}
\mapleresult
\begin{maplelatex}
\mapleinline{inert}{2d}{16}{\[\displaystyle 16\]}
\end{maplelatex}
\end{maplegroup}
\section{\textbf{浮動小数点演算による過ち}}
\begin{maplelatex}\begin{Maple Normal}{
「丸め」にともなって誤差が生じる. CやFortran等の通常のプログラミング言語では「丸める」仕様なのでプログラマーが気をつけなければならない.}\end{Maple Normal}
\end{maplelatex}
\begin{maplegroup}
\begin{Maple Normal}{
プログラムリスト : 実数のケタ落ち}\end{Maple Normal}

\end{maplegroup}
\begin{maplegroup}
\begin{Maple Normal}{
\#include <stdio.h>}\end{Maple Normal}

\begin{Maple Normal}{
}\end{Maple Normal}
\begin{Maple Normal}{
int main(void)\{}\end{Maple Normal}

\begin{Maple Normal}{
float a,b,c;}\end{Maple Normal}

\begin{Maple Normal}{
double x,y,z;}\end{Maple Normal}

\begin{Maple Normal}{
}\end{Maple Normal}
\begin{Maple Normal}{
a=1.23456789;}\end{Maple Normal}

\begin{Maple Normal}{
printf("  a= \%17.10f\symbol{92}n",a);}\end{Maple Normal}

\begin{Maple Normal}{
}\end{Maple Normal}
\begin{Maple Normal}{
b=100.0;}\end{Maple Normal}

\begin{Maple Normal}{
c=a+b;}\end{Maple Normal}

\begin{Maple Normal}{
printf("\%20.10f \%20.10f \%20.10f\symbol{92}n",a,b,c);}\end{Maple Normal}

\begin{Maple Normal}{
}\end{Maple Normal}
\begin{Maple Normal}{
x=(float)1.23456789;}\end{Maple Normal}

\begin{Maple Normal}{
y=(double)100;}\end{Maple Normal}

\begin{Maple Normal}{
z=x+y;}\end{Maple Normal}

\begin{Maple Normal}{
printf("\%20.12e \%20.12e \%20.12e\symbol{92}n",x,y,z);}\end{Maple Normal}

\begin{Maple Normal}{
}\end{Maple Normal}
\begin{Maple Normal}{
x=(double)1.23456789;}\end{Maple Normal}

\begin{Maple Normal}{
y=(double)100;}\end{Maple Normal}

\begin{Maple Normal}{
z=x+y;}\end{Maple Normal}

\begin{Maple Normal}{
printf("\%20.12e \%20.12e \%20.12e\symbol{92}n",x,y,z);}\end{Maple Normal}

\begin{Maple Normal}{
}\end{Maple Normal}
\begin{Maple Normal}{
return 0;|

\}}\end{Maple Normal}

\begin{Maple Normal}{
}\end{Maple Normal}
\end{maplegroup}
\begin{maplegroup}
\begin{Maple Normal}{
分かっているつもりでも,よくやる間違い.}\end{Maple Normal}

\end{maplegroup}
\begin{maplegroup}
\begin{Maple Normal}{
プログラムリスト : 丸め誤差}\end{Maple Normal}

\begin{Maple Normal}{
\#include <stdio.h>}\end{Maple Normal}

\begin{Maple Normal}{
}\end{Maple Normal}
\begin{Maple Normal}{
int main(void)\{}\end{Maple Normal}

\begin{Maple Normal}{
float x=77777,y=7,y1,z,z1;}\end{Maple Normal}

\begin{Maple Normal}{
y1=1/y;}\end{Maple Normal}

\begin{Maple Normal}{
z=x/y;}\end{Maple Normal}

\begin{Maple Normal}{
z1=x*y1;}\end{Maple Normal}

\begin{Maple Normal}{
printf("\%10.2f \%10.2f\symbol{92}n",z,z1);}\end{Maple Normal}

\begin{Maple Normal}{
if (z!=z1)\{}\end{Maple Normal}

\begin{Maple Normal}{
printf("z is not equal to z1.\symbol{92}n");}\end{Maple Normal}

\begin{Maple Normal}{
\}}\end{Maple Normal}

\begin{Maple Normal}{
printf("Surprising?? \symbol{92}n\symbol{92}n\symbol{92}n\symbol{92}n\symbol{92}n\%10.5f \%10.5f\symbol{92}n",z,z1);}\end{Maple Normal}

\begin{Maple Normal}{
return 0;

\}}\end{Maple Normal}
\end{maplegroup}
\begin{maplegroup}
\begin{Maple Normal}{
これを避けるには,EPSILONという小さな数字を定義しておいて,値の差の絶対値を求めるfabsを使って}\end{Maple Normal}

\end{maplegroup}
\begin{maplegroup}
\begin{Maple Normal}{
とすべき.このときは数学関数であるfabsを使っているので,}\end{Maple Normal}

\begin{Maple Normal}{
gcc -lm test.c}\end{Maple Normal}

\begin{Maple Normal}{
とmath libraryを明示的に呼ぶのを忘れないように.}\end{Maple Normal}
\end{maplegroup}
\section{\textbf{機械精度(Machine epsilon)}}
\begin{maplelatex}\begin{Maple Normal}{
上の例では,浮動小数点数で計算した場合に小さい数の差を区別することができなくなるということを示している.これは,CPUに固有の精度で,機械精度(Machine epsilon)と呼ばれる.つまり,小さい数を足したときにその計算機がその差を認識できなくなる限界ということで,以下のようにして求めることができる.}\end{Maple Normal}
\end{maplelatex}
\begin{maplegroup}
\begin{mapleinput}
\mapleinline{active}{1d}{Digits:=7; 
e:=evalf(1.0); 
w:=evalf(1.0+e); 
while (w>1.0) do 
  printf("%-15.10e %-15.10e %-15.10e\symbol{92}n",e,w,evalf(w-1.0)); 
  e:=evalf(e/2.0); 
  w:=evalf(1.0+e); 
end do: 
}{}
\end{mapleinput}
\mapleresult
\begin{maplelatex}
\mapleinline{inert}{2d}{Digits := 7}{\[\displaystyle \]}
\end{maplelatex}
\mapleresult
\begin{maplelatex}
\mapleinline{inert}{2d}{e := 1.0}{\[\displaystyle e\, := \, 1.0\]}
\end{maplelatex}
\mapleresult
\begin{maplelatex}
\mapleinline{inert}{2d}{w := 2.0}{\[\displaystyle w\, := \, 2.0\]}
\end{maplelatex}
\mapleresult
1.0000000000e+00 2.0000000000e+00 1.0000000000e+005.0000000000e-01 1.5000000000e+00 5.0000000000e-012.5000000000e-01 1.2500000000e+00 2.5000000000e-011.2500000000e-01 1.1250000000e+00 1.2500000000e-016.2500000000e-02 1.0625000000e+00 6.2500000000e-023.1250000000e-02 1.0312500000e+00 3.1250000000e-021.5625000000e-02 1.0156250000e+00 1.5625000000e-027.8125000000e-03 1.0078120000e+00 7.8120000000e-033.9062500000e-03 1.0039060000e+00 3.9060000000e-031.9531250000e-03 1.0019530000e+00 1.9530000000e-039.7656250000e-04 1.0009770000e+00 9.7700000000e-044.8828120000e-04 1.0004880000e+00 4.8800000000e-042.4414060000e-04 1.0002440000e+00 2.4400000000e-041.2207030000e-04 1.0001220000e+00 1.2200000000e-046.1035150000e-05 1.0000610000e+00 6.1000000000e-053.0517580000e-05 1.0000310000e+00 3.1000000000e-051.5258790000e-05 1.0000150000e+00 1.5000000000e-057.6293950000e-06 1.0000080000e+00 8.0000000000e-063.8146980000e-06 1.0000040000e+00 4.0000000000e-061.9073490000e-06 1.0000020000e+00 2.0000000000e-069.5367450000e-07 1.0000010000e+00 1.0000000000e-06
\end{maplegroup}
\section{\textbf{桁落ち,情報落ち,積み残し}}
\begin{maplegroup}
\begin{Maple Normal}{
桁落ち(Cancellation)}\end{Maple Normal}

\end{maplegroup}
\begin{maplegroup}
\begin{Maple Normal}{
情報落ち(Loss of Information)}\end{Maple Normal}

\end{maplegroup}
\begin{maplegroup}
\begin{Maple Normal}{
積み残し}\end{Maple Normal}

\end{maplegroup}
\section{\textbf{課題}}
\begin{maplegroup}
次の項目について答えよ.(2004,05,06年度期末試験)
数値計算の精度を制約するデータ形式とその特徴は何か.
丸め誤差とは何か.
打ち切り誤差とは何か.
安定性とは何か.\end{maplegroup}
\begin{maplegroup}
10 進数 4 桁の有効桁数をもった計算機になったつもりで,以下の計算をおこなえ.(a) 2718-0.5818 (b) 2718+0.5818 (c) 2718/0.5818 (d) 2718*0.5818
\end{maplegroup}
\begin{maplegroup}
自分の計算機で機械精度がどの位かを確かめよ.Maple スクリプトを参照して,C あるいは Fortran で作成し,適当に調べよ.
\end{maplegroup}
\begin{maplegroup}
(2147483647 + 100) を C あるいは Fortran で試せ.
\end{maplegroup}
\begin{maplegroup}
係数を\textit{a}= 1,\textit{b}= 10000000,\textit{c}= 1 としたときに,通常の解の公式を使った解と,解と係数の関係(下記の記述を参照)を使った解とを出力するプログラムを作成し,解を比べよ.
2 次方程式\mapleinline{inert}{2d}{Typesetting:-mrow(Typesetting:-msup(Typesetting:-mi("ax", italic = "true", font_style_name = "Ordered List 1", mathvariant = "italic"), Typesetting:-mrow(Typesetting:-mn("2", font_style_name = "Ordered List 1", mathvariant = "normal")), superscriptshift = "0"), Typesetting:-mo("+", font_style_name = "Ordered List 1", mathvariant = "normal", fence = "false", separator = "false", stretchy = "false", symmetric = "false", largeop = "false", movablelimits = "false", accent = "false", lspace = "0.2222222em", rspace = "0.2222222em"))}{$\displaystyle {\it ax} ^{2}+$}
\mapleinline{inert}{2d}{bx+c = 0}{$\displaystyle {\it bx}+c=0$}
の係数a, b, cが特殊な値をもつ場合,通常の解の公式\mapleinline{inert}{2d}{x}{$\displaystyle x$}
=\mapleinline{inert}{2d}{(-b+`&+-`(sqrt(b^2-4*ac)))/(2*a)}{$\displaystyle 1/2\,{\frac {-b+\mbox {{\tt `\&+-`}} \left(  \sqrt{{b}^{2}-4\,{\it ac}} \right) }{a}}$}
にしたがって計算するとケタ落ちによる間違った答えを出す.その特殊な値とはb\symbol{94}2 >> 4acで,\mapleinline{inert}{2d}{Typesetting:-mrow(Typesetting:-msqrt(Typesetting:-mrow(Typesetting:-msup(Typesetting:-mi("b", italic = "true", font_style_name = "2D Math", mathvariant = "italic"), Typesetting:-mrow(Typesetting:-mn("2", font_style_name = "2D Math", mathvariant = "normal")), superscriptshift = "0"), Typesetting:-mo("&minus;", font_style_name = "2D Math", mathvariant = "normal", fence = "false", separator = "false", stretchy = "false", symmetric = "false", largeop = "false", movablelimits = "false", accent = "false", lspace = "0.2222222em", rspace = "0.2222222em"), Typesetting:-mn("4", font_style_name = "2D Math", mathvariant = "normal"), Typesetting:-mo(" ", font_style_name = "2D Math", mathvariant = "normal", fence = "false", separator = "false", stretchy = "false", symmetric = "false", largeop = "false", movablelimits = "false", accent = "false", lspace = "0.0em", rspace = "0.0em"), Typesetting:-mi("ac", italic = "true", font_style_name = "2D Math", mathvariant = "italic"))), Typesetting:-mo("&approx;", font_style_name = "Ordered List 1", mathvariant = "normal", fence = "false", separator = "false", stretchy = "false", symmetric = "false", largeop = "false", movablelimits = "false", accent = "false", lspace = "0.2777778em", rspace = "0.2777778em"))}{$\displaystyle \sqrt{b ^{2}-4\mathop{\rm  }{\it ac} }\esapprox $}
|b|となる場合である.ケタ落ちを防ぐには,b> 0 の場合は,\mapleinline{inert}{2d}{x}{$\displaystyle x$}
\mapleinline{inert}{2d}{Typesetting:-mrow(Typesetting:-msub(Typesetting:-mi(""), Typesetting:-mrow(Typesetting:-mn("1", font_style_name = "Ordered List 1", mathvariant = "normal")), subscriptshift = "0"))}{$\displaystyle _{1}$}
=\mapleinline{inert}{2d}{(-b-sqrt(b^2-4*ac))/(2*a)}{$\displaystyle 1/2\,{\frac {-b- \sqrt{{b}^{2}-4\,{\it ac}}}{a}}$}
ケタ落ちを起こさずに求め,この解を使って,解と係数の関係より\mapleinline{inert}{2d}{x[2] = c/ax[1]}{$\displaystyle x_{{2}}={\frac {c}{{\it ax}_{{1}}}}$}
で求める.b< 0 の場合は,解の公式の足し算の方を使って同様に求める.\end{maplegroup}
\end{document}
