1 ワードの整数の最大値とその2進数表示.
\begin{MapleInput}
> restart;
> 2^(4*8-1)-1;#res: 2147483647

\end{MapleInput}
この整数を2進数で表示するように変換するには,convert(n,binary)を用いて,
\begin{MapleInput}
> convert(2^(4*8-1)-1,binary);  #res: 1111111111111111111111111111111
\end{MapleInput}
となり,31個の1が並んでいることが分かる.1 ワードの整数の最大桁は,$n$ の桁数を戻すコマンドlength(n)を使って,
\begin{MapleInput}
> length(2^(4*8-1)-1); #res: 10
\end{MapleInput}
となり,たかだか10桁程度であることが分かる.一方,64bit の場合の整数の最大桁.
\begin{MapleInput}
> length(2^(8*8-1)-1); #res: 19
\end{MapleInput}
である.
                                    
Maple では多倍長計算するので,通常のプログラミング言語で起こるintの最大数あたりでの奇妙な振る舞いは示さない.

\begin{MapleInput}
> 2147483647+100;  #res: 2147483747
\end{MapleInput}

単精度の浮動小数点数は,仮数部2進数23bit,2倍長実数で52bitである.この有効桁数は以下の通り.
\begin{MapleInput}
> length(2^(23)); #res: 7
> length(2^(52)); #res: 16
\end{MapleInput}
