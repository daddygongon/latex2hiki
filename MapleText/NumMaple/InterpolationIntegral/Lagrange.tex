多項式補間は手続きが簡単であるため,計算間違いが少なく,プログラムとして組むのに適している.しかし,あまり"みとうし"のよい方法とはいえない.その点,Lagrange(ラグランジュ)の内挿公式は見通しがよい.これは
\begin{equation*}
F(x)= \sum_{k=0}^N \frac{\displaystyle \prod_{j \ne k} (x-x_j)}
{\displaystyle \prod_{j \ne k} (x_k-x_j)} y_k
=\sum_{k=0}^N \frac{ \frac{\displaystyle (x-x_0)(x-x_1)\cdots(x-x_N)}{\displaystyle (x-x_k)}}
{\frac{\displaystyle (x_k-x_0)(x_k-x_1)\cdots(x_k-x_N)}{\displaystyle (x_k-x_k)}} y_k
\end{equation*}
と表わされる.数学的に 2つ目の表記は間違っているが,先に割り算を実行すると読み取って欲しい.これは一見複雑に見えるが,単純な発想から出発している.求めたい関数$F(x)$を
\begin{equation*}
F(x) = y_0 L_0(x)+y_1 L_1(x)+y_2 L_2(x)
\end{equation*}
とすると
\begin{equation*}
\begin{array}{ccc}
 L_0(x_0) = 1 &L_0(x_1) = 0 &L_0(x_2) = 0 \\
 L_1(x_0) = 0 &L_1(x_1) = 1 &L_1(x_2) = 0 \\
 L_2(x_0) = 0 &L_2(x_1) = 0 &L_2(x_2) = 1 
\end{array}
\end{equation*}
となるように関数$L_i(x)$を決めればよい.これを以下のようにとればLagrangeの内挿公式となる.
  
\ifHIKI %%%%
||               ||
\else %%%%
\begin{equation*}
\setlength{\unitlength}{1cm}
\begin{picture}(10,10)
\put(0,0){\framebox(10,10){}}
\end{picture}
\end{equation*}
\fi %%%%
