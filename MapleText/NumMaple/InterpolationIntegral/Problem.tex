\begin{enumerate}
\item 補間と近似の違いについて,適切な図を描いて説明せよ.
\item 次の4点
\begin{MapleInput}
x y 
0 1 
1 2
2 3
3 -2
\end{MapleInput}
を通る多項式を以下のそれぞれの手法で求めよ.(a) 逆行列, (b)ラグランジュ補間, (c)ニュートンの差分商公式 
\item
$\tan(5^\circ)=0.08748866355$, 
$\tan(10^\circ)=.1763269807$,
$\tan(15^\circ)=.2679491924$の値を用いて,ラグランジュ補間法により,$\tan(17^\circ)$の値を推定せよ.(2008年度期末試験)
\item exp(0)=1.0, exp(0.1)=1.1052, exp(0.3)=1.3499の値を用いて,ラグランジュ補間法により,exp(0.2)の値を推定せよ.(2009年度期末試験)
\item 次の関数
\begin{equation*}
f(x) = \frac{4}{1+x^2}
\end{equation*}
を$x = 0..1$で数値積分する.
\begin{enumerate}
\item $N$を2,4,8,…256ととり,$N$個の等間隔な区間にわけて中点法で求めよ.(15)
\item 小数点以下10桁まで求めた値3.141592654との差をdXとする.dXと分割数Nとを両対数プロット(loglogplot)して比較せよ(10)
\end{enumerate}
(2008年度期末試験)
\item 次の関数
\begin{equation*}
y = \frac{1}{1+x^2}
\end{equation*}
を$x = 0..1$で等間隔に$N+1$点とり,$N$個の区間にわけて数値積分で求める.$N$を2, 4, 8, 16, 32, 64, 128, 256と取ったときの(a)中点法, (b)台形公式, (c)シンプソン公式それぞれの収束性を比較せよ.

ヒント:Maple script にあるそれぞれの数値積分法を関数 (procedure) に直して,for-loop
で回せば楽.出来なければ,一つ一つ手で変えても OK. 両対数プロット (loglogplot) すると見やすい.
\end{enumerate}