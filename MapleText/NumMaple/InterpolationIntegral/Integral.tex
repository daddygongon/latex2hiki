積分,
\begin{equation*}
I = \int_a^b f(x) dx
\end{equation*}
を求めよう.1次元の数値積分法では連続した領域を細かい短冊に分けて,それぞれの面積を寄せ集めることに相当する.分点の数を N とすると,
\begin{equation*}
\begin{array}{c}
\displaystyle x_i = a+ \frac{b-a}{N} i = a + h \times i \\
\displaystyle h = \frac{b-a}{N}
\end{array}
\end{equation*}
ととれる.そうすると,もっとも単純には,
\begin{equation*}
I_N = \left\{\sum_{i=0}^{N-1} f(x_i)\right\}h =
\left\{\sum_{i=0}^{N-1} f(a+i \times h)\right\}h
\end{equation*}
となる.

\ifHIKI %%%%
||数値積分の模式図.
||
\else %%%%
\begin{table}[htbp]
\caption{数値積分の模式図.}
\begin{center}
\begin{tabular}{c}
\setlength{\unitlength}{1cm}
\begin{picture}(10,10)
\put(0,0){\framebox(10,10){}}
\end{picture}
\end{tabular}
\end{center}
\label{default}
\end{table}%
\fi %%%%

\subsection{中点則 (midpoint rule)}
中点法 (midpoint rule) は,短冊を左端から書くのではなく,真ん中から書くことに対応し,
\begin{equation*}
I_N = \left\{\sum_{i=0}^{N-1}f\left(a+\left(i+\frac{1}{2}\right) \times h\right)\right\}h
\end{equation*}
となる.
\subsection{台形則 (trapezoidal rule)}
さらに短冊の上側を斜めにして,短冊を台形にすれば精度が上がりそうに思う.
その場合は,短冊一枚の面積$S_i$は,
\begin{equation*}
S_i = \frac{f(x_i)+f(x_{i+1})}{2}h
\end{equation*}
で求まる.これを端から端まで加えあわせると,
\begin{equation*}
\displaystyle i_N =\sum _{i=0}^{N-1}S_i =h \left\{ \frac{1}{2} f ( x_0 ) +\sum _{i=1}^{N-1}f ( x_i ) +\frac{1}{2} f \left( x_N \right)  \right\} 
\end{equation*}
が得られる.
\subsection{Simpson(シンプソン)則}
Simpson(シンプソン) 則では,短冊を2次関数,
\begin{equation*}
f(x) = ax^2+bx+c
\end{equation*}
で近似することに対応する.こうすると,
\begin{equation*}
S_i=\int _{x_i}^{x_{i+1}}f ( x )\, {dx}=\int _{x_i}^{x_{i+1}}(ax^2+bx+c)\,{dx}
\end{equation*}
\ifHIKI %%%%
||Simpson則の導出(数式変形).
||
\else %%%%
\begin{equation*}
\setlength{\unitlength}{1cm}
\begin{picture}(10,4.5)
\put(0,0){\framebox(10,4.5){}}
\end{picture}
\end{equation*}
\fi %%%%

\begin{equation*}
\frac{h}{6} \left\{f(x_i)+4f\left(x_i+\frac{h}{2}\right)+f(x_i+h)\right\}
\end{equation*}
となる.これより,
\begin{equation*}
I_N=\frac{h}{6} \left\{ f \left( x_0 \right) +4\,\sum _{i=0}^{N-1}f \left( x_i+\frac{h}{2} \right) +2\,\sum_{i=1}^{N-1}f \left( x_i \right) +f \left( x_N \right)  \right\}
\end{equation*}
として計算できる.ただし,関数値を計算する点の数は台形則などの倍となっている.

教科書によっては,分割数$N$を偶数にして,点を偶数番目 (even) と奇数番目 (odd) に分けて,
\begin{equation*}
I_{{N}}=\frac{h}{3} \left\{ f \left( x_{{0}} \right) +4\,\sum _{i={\it even}}^{N-2}f \left( x_{{i}}+\frac{h}{2} \right) +2\,\sum _{i={\it odd}}^{N-1}f \left( x_{{i}} \right) +f \left( x_{{N}} \right) \right\}
\end{equation*}
としている記述があるが,同じ計算になるので誤解せぬよう.
