単純な2次元データについて補間と近似を考える.補間はたんに点をつなぐことを,近似はある関数にできるだけ近くなるようにフィットすることを言う.補間はIllustratorなどのドロー系ツールで曲線を引くときの,ベジエやスプライン補間の基本となる.本章では補間とそれに密接に関連した積分について述べる.


\ifHIKI
||補間と近似の模式図.
||

\else

\begin{table}[htbp]
\caption{補間と近似の模式図.}
\begin{center}
\begin{tabular}{cc}
補間&近似 \\
\setlength{\unitlength}{1cm}
\begin{picture}(6,6.5)
\put(0,0){\framebox(6,6.5){}}
\end{picture}
&
\setlength{\unitlength}{1cm}
\begin{picture}(6,6.5)
\put(0,0){\framebox(6,6.5){}}
\end{picture}
\end{tabular}
\end{center}
\label{default}
\end{table}%
\fi
