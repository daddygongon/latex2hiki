係数行列$A$とベクトル$b$を足して作られる行列は拡大係数行列と呼ばれます.Mapleでは,これは
\begin{MapleInput}
> <A|b>;
\end{MapleInput}
\begin{MapleOutput}
\left[ \begin {array}{ccc} 2&5&7\\ 4&1&5\end {array} \right]
\end{MapleOutput}
として作られます.ここから行列の掃き出し操作をおこなうには,LUDecompositionというコマンドを使います.
\begin{MapleInput}
> P,L,U:=LUDecomposition(<A|b>);
\end{MapleInput}
\begin{MapleOutput}
P,\,L,\,U\, := \, \left[ \begin {array}{cc} 1&0\\ 0&1\end {array} \right] ,\, \left[ \begin {array}{cc} 1&0\\ 2&1\end {array} \right] ,\, \left[ \begin {array}{ccc} 2&5&7\\ 0&-9&-9\end {array} \right]
\end{MapleOutput}
これは,下三角行列(Lower Triangle Matrix)と上三角行列(Upper Triangle Matrix)に分解(decompose)するコマンドです.$P$行列は置換(permutation)行列を意味します.LUDecompositionだけでは,前進消去が終わっただけの状態です.そこで,後退代入までおこなうには,optionにoutput='R'をつけます.そうすると出力は,
\begin{MapleInput}
> LUDecomposition(<A|b>,output='R');
\end{MapleInput}
\begin{MapleOutput}
\left[ \begin {array}{ccc} 1&0&1\\ 0&1&1\end {array} \right]
\end{MapleOutput}
で,$b$ベクトルの部分が解になっています.
