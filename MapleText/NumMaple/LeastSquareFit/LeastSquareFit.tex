\documentclass[10pt,a4j]{jreport}
\usepackage[dvips]{graphicx,color}
\usepackage{verbatim}
\usepackage{amsmath,amsthm,amssymb}
\topmargin -15mm\oddsidemargin -4mm\evensidemargin\oddsidemargin
\textwidth 170mm\textheight 257mm\columnsep 7mm
\setlength{\fboxrule}{0.2ex}
\setlength{\fboxsep}{0.6ex}

\pagestyle{empty}

\newcommand{\MaplePlot}[2]{{\begin{center}
    \includegraphics[width=#1,clip]{#2}
                     \end{center}
%
} }

\newenvironment{MapleInput}{%
    \color{red}\verbatim
}{%
    \endverbatim
}

\newenvironment{MapleError}{%
    \color{blue}\verbatim
}{%
    \endverbatim
}

\newenvironment{MapleOutput}{%
    \color{blue}\begin{equation*}
}{%
    \end{equation*}
}

\newenvironment{MapleOutputGather}{%
    \color{blue}\gather
}{%
    \endgather
}
\newif\ifHIKI
%\HIKItrue % TRUEの設定
\HIKIfalse % FALSEの設定
\begin{document}
\chapter{線形最小2乗法(LeastSquareFit)}
\section{Mapleによる最小2乗法}
前章では,データに多項式を完全にフィットする補間についてみた.今回は,近似的にフィットする最小二乗法について詳しくみていく.図のようなデータに直線をフィットする場合を考えよう.

\MaplePlot{50mm}{./figures/C8_LeastSquareFitplot2d1.eps}

コマンドleastsquareによるfitting(2変数の例)
\begin{MapleInput}
> restart: X:=[1,2,3,4]: Y:=[0,5,15,24]:
> with(plots):with(linalg):with(stats):
> l1:=pointplot(transpose([X,Y]),symbolsize=30):
> eq_fit:=fit[leastsquare[[x, y], y = a0+a1*x, {a0,a1}]]([X, Y]);
\end{MapleInput}
\begin{MapleOutput}
eq\_fit\, := \,y=-\frac{19}{2}+{\frac {41}{5}}\,x
\end{MapleOutput}
\begin{MapleInput}
> f1:=unapply(rhs(eq_fit),x);
\end{MapleInput}
\begin{MapleOutput}
f1\, := \,x\mapsto -\frac{19}{2}+{\frac {41}{5}}\,x
\end{MapleOutput}
\begin{MapleInput}
> p1:=plot(f1(x),x=0..4):
> display(p1,l1);
\end{MapleInput}
\MaplePlot{50mm}{./figures/C8_LeastSquareFitplot2d2.eps}

\section{最小2乗法の原理}
前章では,データに近似的にフィットする最小二乗法を紹介した.ここでは,フィット式が多項式のような線形関係にない関数の最小二乗法を紹介する.図のようなデータにフィットする場合を考えよう.
\MaplePlot{80mm}{./figures/C9_NonLinearFitplot2d1.eps}
このデータにあてはめるのはローレンツ関数,
\begin{equation*}
F \left(x;\mathbf{a} \right)=a _{1}+\displaystyle \frac{a _{2}}{a _{3}+\left(x -a _{4}\right)^{2}}
\end{equation*}
である.この関数の特徴は,今まで見てきた関数と違いパラメータが線形関係になっていない.誤差関数は,いままでと同様に
\begin{equation*}
\chi ^{2}\left(\mathbf{a} \right)={\sum_i^N }d _{i }^{2}=\sum_i^N \left(F \left(x _{i };a \right)-y _{i }\right)^{2}
\end{equation*}
で,$a={a_0, a_1,..}$をパラメータとして変えた時に最小となる値を求める点もかわらない.しかし,線形の最小二乗法のように微分しても一元の方程式にならず,連立方程式を単に解くだけでは求まらない.

そこで図のような2次関数の最小値を求める場合を考える.最小値の点$a_0$のまわりで,Taylor展開すると,$\mathbf{d,D}$をそれぞれの係数とすると,
\begin{equation*}
\chi^2 \left( \mathbf{a} \right)= \chi^2 \left( \mathbf{a_0}  \right) - \mathbf{d} \left(\mathbf{a}-\mathbf{a_0} \right) +\frac{1}{2} \mathbf{D} \left(\mathbf{a}-\mathbf{a_0} \right)^{2}
\end{equation*}
である.最小の点$a_0$は,微分が$0$になるので,
\begin{equation*}
\mathbf{a _{0}}=\mathbf{a} + \mathbf{D} ^{-1} \times (-\mathbf{d})
\end{equation*}
と予測される.図を参照して上の式を導け.またその意味を考察せよ.
\pagebreak
\MaplePlot{65mm}{./figures/C9_NonLinearFitplot2d2.eps}
\MaplePlot{150mm}{./figures/C9_NonLinearFitplot2d3.eps}

現実には高次項の影響で計算通りにはいかず,単に最小値の近似値を求めるだけである.これは,$ \chi \left(\mathbf{a} \right)  ^{2}$の微分関数の解をNewton法で求める操作に対応する.つまり,この操作を何度も繰り返せばいずれ解がある精度で求まるはず.


\section{$\chi^2$の極小値から(2変数の例)}
\begin{MapleInput}
> restart; X:=[1,2,3,4]: Y:=[0,5,15,24]: f1:=x->a0+a1*x:
  S:=0: 
  for i from 1 to 4 do 
    S:=S+(f1(X[i])-Y[i])^2; 
  end do:
> fS:=unapply(S,(a0,a1));
\end{MapleInput}
\begin{MapleOutput}
{\it fS}\, := \,( {{\it a0},{\it a1}} )\mapsto  \left( {\it a0}+{\it a1} \right) ^{2}+ \left( {\it a0}+2\,{\it a1}-5 \right) ^{2}+ \left( {\it a0}+3\,{\it a1}-15 \right) ^{2}+ \left( {\it a0}+4\,{\it a1}-24 \right) ^{2}
\end{MapleOutput}
\begin{MapleInput}
> expand(fS(a0,a1));
\end{MapleInput}
\begin{MapleOutput}
4\,{{\it a0}}^{2}+20\,{\it a0}\,{\it a1}+30\,{{\it a1}}^{2}-88\,{\it a0}-302\,{\it a1}+826
\end{MapleOutput}

\begin{MapleInput}
> plot3d(fS(a0,a1),a0=-20..20,a1=0..20);
\end{MapleInput}
\MaplePlot{50mm}{./figures/C8_LeastSquareFitplot3d3.eps}


\begin{MapleInput}
> eqs:={diff(expand(S),a0)=0, diff(expand(S),a1)=0};
\end{MapleInput}
\begin{MapleOutput}
{\it eqs}\, := \, \left\{ 8\,{\it a0}+20\,{\it a1}-88=0,20\,{\it a0}+60\,{\it a1}-302=0 \right\}
\end{MapleOutput}

\begin{MapleInput}
> solve(eqs,{a0,a1});
\end{MapleInput}
\begin{MapleOutput}
\left\{ {\it a0}=-\frac{19}{2},{\it a1}={\frac {41}{5}} \right\}
\end{MapleOutput}

\section{正規方程式(Normal Equations)による解}
より一般的な場合の最小二乗法の解法を説明する.先程の例では1次の多項式を近似関数とした.これをより一般的な関数,例えば,$\sin, \cos, \tan, \exp, \sinh$などとする.これを線形につないだ関数を
\begin{equation*}
F \left(x \right)=a _{0}\sin \left(x \right)+a _{1}\cos \left(x \right)+a _{2}\exp \left(-x \right)+a _{3}\sinh \left(x \right)+\cdots ={\sum_{k=1}^{M}}a _{k }X _{k }\left(x \right)
\end{equation*}
ととる.実際には,$X_k(x)$はモデルや,多項式の高次項など論拠のある関数列をとる.これらを基底関数(base functions)と呼ぶ.ここで線形といっているのは,パラメータ$a_k$について線形という意味である.このような,より一般的な基底関数を使っても,$\chi^2$関数は
\begin{equation*}
{\chi}^{2}=\sum _{i=1}^{N} \left( F \left( x_{{i}} \right) -y_{{i}} \right) ^{2}
=\sum _{i=1}^{N} \left( \sum _{k=1}^{M}a_{{k}}X_{{k}} \left( x_{{i}} \right) -y_{{i}} \right) ^{2}
\end{equation*}
と求めることができる.この関数を,$a_k$を変数とする関数とみなす.この関数が最小値を取るのは,$\chi^2$を$M$個の$a_k$で偏微分した式がすべて0となる場合であ
る.これを実際に求めてみると,
\begin{equation*}
\sum _{i=1}^{N} \left( \sum _{j=1}^{M}a_{{j}}X_{{j}} \left( x_{{i}} \right) -y_{{i}} \right) X_{{k}} \left( x_{{i}} \right) =0
\end{equation*}
となる.ここで,$k = 1..M$の$M$個の連立方程式である.この連立方程式を最小二乗法の正規方程式(normal equations)と呼ぶ.

上記の記法のままでは,ややこしいので,行列形式で書き直す.$N \times M$で,各要素を
\begin{equation*}
A_{ij} = X_j(x_i)
\end{equation*}
とする行列$A$を導入する.この行列は,
\begin{equation*}
A=\left[
\begin{array}{cccc}
X_1(x_1) & X_2(x_1) & \cdots & X_M(x_1) \\
\vdots & \vdots & \cdots & \vdots \\
\vdots & \vdots & \cdots & \vdots \\
\vdots & \vdots & \cdots & \vdots \\
X_1(x_N) & X_2(x_N) & \cdots & X_M(x_N) 
\end{array}
\right]
\end{equation*}
となる.これをデザイン行列と呼ぶ.すると先程の正規方程式は,
\begin{equation*}
A^t . A . a = A^t . y
\end{equation*}
で与えられる.$A^t$は行列$A$の転置(transpose)
\begin{equation*}
A^t = A_{ij}^t = A_{ji}
\end{equation*}
を意味し,得られた行列は,$M \times N$である.$a, y$はそれぞれ,
\begin{equation*}
a=\left[
\begin{array}{c}
a_1\\a_2\\\vdots\\a_M
\end{array}
\right],\,
y=\left[
\begin{array}{c}
y_1\\y_2\\\vdots\\y_N
\end{array}
\right]
\end{equation*}
である.

$M = 3, N = 25$として行列の次元だけで表現すると,
\begin{equation*}
\left[
\begin{array}{ccccc}
 &  & \cdots & &\\
\cdots & \cdots &  \cdots & \cdots & \cdots \\
 &  & \cdots & &\\
\end{array}
\right]
\left[
\begin{array}{ccc}
& \vdots &\\
& \vdots &\\
\cdots & \cdots &  \cdots\\
& \vdots &\\
& \vdots &\\
\end{array}
\right]
\left[
\begin{array}{c}
\vdots\\
\vdots\\
\vdots
\end{array}
\right]
=
\left[
\begin{array}{ccccc}
 &  & \cdots & &\\
\cdots & \cdots &  \cdots & \cdots & \cdots \\
 &  & \cdots & &\\
\end{array}
\right]
\left[
\begin{array}{c}
\vdots\\
\vdots\\
\vdots\\
\vdots\\
\vdots
\end{array}
\right]
\end{equation*}
となる.これは少しの計算で$3 \times 3$の逆行列を解く問題に変形できる.

\subsection{Mapleによる具体例}
\begin{MapleInput}
> restart; X:=[1,2,3,4]: Y:=[0,5,15,24]: 
  f1:=x->a[1]+a[2]*x+a[3]*x^2:
  with(LinearAlgebra): Av:=Matrix(1..4,1..3):
  ff:=(x,i)->x^(i-1):
  for i from 1 to 3 do 
    for j from 1 to 4 do
      Av[j,i]:=ff(X[j],i); 
    end do; 
  end do;
  Av;
\end{MapleInput}
\begin{MapleOutput}
\left[ \begin{array}{ccc} 1&1&1\\1&2&4\\1&3&9\\1&4&16\end {array} \right]
\end{MapleOutput}
\begin{MapleInput}
> Ai:=MatrixInverse(Transpose(Av).Av);
\end{MapleInput}
\begin{MapleOutput}
{\it Ai}\, := \, \left[ \begin {array}{ccc} 
{\displaystyle \frac {31}{4}}&-{\displaystyle \frac {27}{4}}&\displaystyle \frac{5}{4}\\
-{\displaystyle \frac {27}{4}}&{\displaystyle \frac {129}{20}}&\displaystyle -\frac{5}{4}\\
\displaystyle \frac{5}{4}&\displaystyle -\frac{5}{4}&\displaystyle \frac{1}{4}
\end {array} \right]
\end{MapleOutput}
\begin{MapleInput}
> b:=Transpose(Av).Vector(Y);
\end{MapleInput}
\begin{MapleOutput}
b\, := \, \left[ \begin {array}{c} 44\\151\\539\end {array} \right]
\end{MapleOutput}

\begin{MapleInput}
> Ai.b;
\end{MapleInput}
\begin{MapleOutput}
\left[ \begin {array}{c}\displaystyle -\frac{9}{2}\\
\displaystyle {\frac {16}{5}}\\
1\end {array} \right]
\end{MapleOutput}

\section{特異値分解(Singular Value Decomposition)による解}
正規方程式を解くときには,少し注意が必要である.正規方程式での共分散行列,特異値分解の導出や標準偏差との関係はNumRecipeを参照せよ.
\begin{MapleInput}
> restart; X:=[1,2,3,4]: Y:=[0,5,15,24]: f1:=x->a[1]+a[2]*x+a[3]*x^2:
> with(LinearAlgebra): Av:=Matrix(1..4,1..3):
> ff:=(x,i)->x^(i-1): 
  for i from 1 to 3 do 
    for j from 1 to 4 do
      Av[j,i]:=ff(X[j],i); 
    end do; 
  end do; 
  Av;
\end{MapleInput}
\begin{MapleOutput}
\left[ \begin {array}{ccc} 1&1&1\\1&2&4\\1&3&9\\1&4&16\end {array} \right] 
\end{MapleOutput}

\begin{MapleInput}
> U,S,Vt:=evalf(SingularValues(Av,output=['U','S','Vt'])):
> DiagonalMatrix(S[1..3],4,3); U.DiagonalMatrix(S[1..3],4,3).Vt:
\end{MapleInput}
\begin{MapleOutput}
\left[ \begin {array}{ccc}  19.6213640200000015&0&0\\0& 1.71206987399999999&0\\0&0& 0.266252879300000022\\0&0&0\end {array} \right]
\end{MapleOutput}

\begin{MapleInput}
> iS:=Vector(3): 
  for i from 1 to 3 do 
    iS[i]:=1/S[i]; 
  end do:
> DiS:=DiagonalMatrix(iS[1..3],3,4);
\end{MapleInput}
\begin{MapleOutput}
{\it DiS}\, := \, \left[ \begin {array}{cccc}  0.05096485642&0&0&0\\0& 0.5840883104&0&0\\0&0& 3.755827928&0\end {array} \right]
\end{MapleOutput}

\begin{MapleInput}
> Transpose(Vt).DiS.(Transpose(U).Vector(Y));
\end{MapleInput}
\begin{MapleOutput}
\left[ \begin {array}{c} - 4.50000000198176498\\ 3.20000000035008324\\ 1.00000000040565196\end {array} \right]
\end{MapleOutput}


\section{2次元曲面へのフィット}
先程の一般化をより発展させると,3次元$(x_i, y_i, z_i)$で提供されるデータへの,2次元平面でのフィットも可能となる.2次元の単純な曲面は,方程式を使って,
\begin{equation*}
F(x, y) = a_1+a_2\,x+a_3\,y+a_4\,xy+a_5\,x^2+a_6\,y^2
\end{equation*}
となる.デザイン行列の$i$行目の要素は,
\begin{equation*}
[1, x_i, y_i, x_i\,y_i, x_i^2, y_i^2]
\end{equation*}
として,それぞれ求める.このデータの変換の様子をMapleスクリプトで詳しく示した.後は,通常の正規方程式を解くようにすれば,このデータを近似する曲面を定めるパラメータ$a_1, a_2, \cdots,a_6$が求まる.最小二乗法はパラメータ$a_k$について線形であればよい.

\subsection{Mapleによる具体例}
実際のデータ解析での例.データの座標をx,y,zで用意して,Mapleの埋め込み関数のleastsquareでfitしている.
\begin{MapleInput}
> with(plots):with(plottools): 
  z:=[0.000046079702088, 0.000029479057275,
  0.000025769637830, 0.000034951410953, 0.000057024385455, 0.000029485453808,
  0.000011519913869, 0.000006442404299, 0.000014252898382, 0.000034951410953,
  0.000025769637773, 0.000006442404242, 0.000000000000057, 0.000006442404242,
  0.000025769637773, 0.000034932221524, 0.000014246501905, 0.000006442404299,
  0.000011519913926, 0.000029479057332, 0.000056973214100, 0.000034932221467,
  0.000025769637773, 0.000029485453808, 0.000046079702031]:
>  x:=[]:
  y:=[]:
  p1:=2:
  for i from -p1 to p1 do
    for j from -p1 to p1 do
      x:=[op(x),i*0.0005];
      y:=[op(y),j*0.0005];
    end do;
  end do;
> with(LinearAlgebra): p2:=convert(Transpose(Matrix([x,y,z])),listlist):
  pp2:=pointplot3d(p2,symbol=circle,symbolsize=30,color=black):
  with(stats): data:=[x,y,z]: 
  fit1:=fit[leastsquare[[t,s,u], 
    u=a1+a2*t+a3*s+a4*t*s+a5*t^2+a6*s^2, 
    {a1,a2,a3,a4,a5,a6}]](data);
\end{MapleInput}
\begin{MapleOutputGather}
{\it fit1}\, := \,u=-{ 8.657142857\times 10^{-13}}- 0.000006396456800\,t+ 0.000006396438400\,s\notag \\ 
- 5.459553587\,ts+ 25.76962838\,{t}^{2}+ 25.76962835\,{s}^{2} \notag 
\end{MapleOutputGather}
\begin{MapleInput}
> f1:=unapply(rhs(fit1),(s,t)):
> pf1:=plot3d(f1(ss,uu),ss=-0.001..0.001,uu=-0.001..0.001,color=gray):
> display(pf1,pp2,axes=boxed);
\end{MapleInput}
\MaplePlot{50mm}{./figures/C8_LeastSquareFitplot3d4.eps}

\subsection{正規方程式による解法}
デザイン行列へのデータ変換
\begin{MapleInput}
> bb:=Vector(25): A:=Matrix(25,6): 
  p1:=2: 
  for i from 1 to 25 do 
    A[i,1]:=1;
    A[i,2]:=x_i; 
    A[i,3]:=y_i; 
    A[i,4]:=x_i*y_i; 
    A[i,5]:=x_i^2; 
    A[i,6]:=y_i^2;
    bb_i:=z_i; 
  end do:
\end{MapleInput}
正規方程式の解
\begin{MapleInput}
> MatrixInverse(Transpose(A).A).(Transpose(A).bb);
\end{MapleInput}
\begin{MapleOutput}
\left[ \begin {array}{c} -{ 9.185257196\times 10^{-13}}\\   - 0.00000639644675999994798\\    0.00000639644220000032532\\   - 5.45955358336000173\\    25.7696284050857187\\    25.7696284050857543\end {array} \right]
\end{MapleOutput}
\section{課題}
\begin{enumerate}
\item 後退代入法で解を求めよ.(2005年度期末類題) 
\begin{equation*}
\begin{array}{rl}
x+4y-3z &= 1 \\
-6y+4z &= 1\\
-\frac{5}{3}z &=  \frac{1}{3}
\end{array}
\end{equation*}

\item 次の行列AをLU分解せよ.
\begin{MapleInput}
> A:=Matrix([[1,4,3],[1,-2,1],[2,-2,-1]]);
\end{MapleInput}
\begin{MapleOutput}
\left[ \begin {array}{ccc} 1&4&3\\ 1&-2&1\\ 2&-2&-1\end {array} \right]
\end{MapleOutput}

\item 次の連立方程式の係数行列をLU分解し,上・下三角行列を求めよ.さらに連立方程式の解を求めよ.(2005年度期末試験) 
\begin{equation*}
\left[ \begin {array}{c} x_{{1}}+3\,x_{{2}}+4\,x_{{3}}+3\,x_{{4}}\\ -2\,x_{{1}}+5\,x_{{2}}+3\,x_{{3}}-3\,x_{{4}}\\ x_{{1}}+3\,x_{{2}}-2\,x_{{3}}+3\,x_{{4}}\\ 3\,x_{{1}}-2\,x_{{2}}+x_{{3}}+4\,x_{{4}}\end {array} \right] = \left[ \begin {array}{c} 1\\ 4\\ -2\\ 3\end {array} \right]
\end{equation*}

\item Jacobi法のプログラムを参照してGauss-Seidel法のプログラムを作れ.Jacobi法と収束性を比べよ.

\item 次の連立方程式の解を求めよ.ただし,pivot操作が必要となる.
\begin{MapleInput}
> with(LinearAlgebra): 
  A:=Matrix([[3,2,2,1],[3,2,3,1],[1,-2,-3,1],[5,3,-2,5]]):
  X:=Vector([w,x,y,z]): 
  b:=Vector([-6,2,-9,2]): 
  A.X=b;
\end{MapleInput}
\begin{MapleOutput}
\left[ \begin {array}{c} 3\,w+2\,x+2\,y+z\\ 3\,w+2\,x+3\,y+z\\ w-2\,x-3\,y+z\\ 5\,w+3\,x-2\,y+5\,z\end {array} \right] = \left[ \begin {array}{c} -6\\ 2\\ -9\\ 2\end {array} \right]
\end{MapleOutput}
\item (おまけ)pivot操作を含めたLU分解のプログラムを作成せよ.上の問題を解き,そのL, U行列および$L^{-1}.b$ベクトルを求めよ.
\end{enumerate}


\end{document}
