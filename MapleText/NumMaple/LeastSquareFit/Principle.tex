もっとも簡単な例で原理を解説する.近似関数として,
\begin{equation*}
F(x) = a_0+a_1\,x
\end{equation*}
という直線近似を考える.もっともらしい関数は$N$点の測定データとの差$d_i = F(x_i)-y_i$を最小にすればよさそうであるが,これはプラスマイナスですぐに消えて不定になる.そこで,
\begin{equation*}
\chi^{2}=\sum_i^N d_i^2=\sum_i^N\left(a_0+a_1\,x_i-y_i\right)^2
\end{equation*}
という関数を考える.この$\chi^2$(カイ二乗)関数が,$a_0, a_1$をパラメータとして変えた時に最小となる$a_0, a_1$を求める.これは,それらの微分がそれぞれ0となる場合である.これは$\chi^2$の和$\sum$(sum)の中身を展開し,
\ifHIKI %%%%
||$\chi^2=$||            ||

\else %%%%

\begin{table}[htbp]
\begin{center}
\begin{tabular}{cc}
$\chi^2=$& \\
&
\setlength{\unitlength}{1cm}
\begin{picture}(10,6.5)
\put(0,0){\framebox(10,6.5){}}
\end{picture}

\end{tabular}
\end{center}
\end{table}%
\fi %%%%

$a_0, a_1$でそれぞれ微分すれば
\ifHIKI %%%%
||$\displaystyle \frac{\partial}{\partial a_0} \chi^2 =$||            ||
||$\displaystyle \frac{\partial}{\partial a_1} \chi^2 =$||            ||
\else %%%%

\begin{table}[htbp]
\begin{center}
\begin{tabular}{cc}
$\displaystyle \frac{\partial}{\partial a_0} \chi^2 =$& 
\setlength{\unitlength}{1cm}
\begin{picture}(10,3.5)
\put(0,0){\framebox(10,3.5){}}
\end{picture}  \\
$\displaystyle \frac{\partial}{\partial a_1} \chi^2 =$&
\setlength{\unitlength}{1cm}
\begin{picture}(10,3.5)
\put(0,0){\framebox(10,3.5){}}
\end{picture}  \\
\end{tabular}
\end{center}
\end{table}%
\fi %%%%

という$a_0, a_1$を未知変数とする2元の連立方程式が得られる.これは前に説明した通り逆行列で解くことができる.
