先程の一般化をより発展させると,3次元$(x_i, y_i, z_i)$で提供されるデータへの,2次元平面でのフィットも可能となる.2次元の単純な曲面は,方程式を使って,
\begin{equation*}
F(x, y) = a_1+a_2\,x+a_3\,y+a_4\,xy+a_5\,x^2+a_6\,y^2
\end{equation*}
となる.デザイン行列の$i$行目の要素は,
\begin{equation*}
[1, x_i, y_i, x_i\,y_i, x_i^2, y_i^2]
\end{equation*}
として,それぞれ求める.このデータの変換の様子をMapleスクリプトで詳しく示した.後は,通常の正規方程式を解くようにすれば,このデータを近似する曲面を定めるパラメータ$a_1, a_2, \cdots,a_6$が求まる.最小二乗法はパラメータ$a_k$について線形であればよい.

\subsection{Mapleによる具体例}
実際のデータ解析での例.データの座標をx,y,zで用意して,Mapleの埋め込み関数のleastsquareでfitしている.
\begin{MapleInput}
> with(plots):with(plottools): 
  z:=[0.000046079702088, 0.000029479057275,
  0.000025769637830, 0.000034951410953, 0.000057024385455, 0.000029485453808,
  0.000011519913869, 0.000006442404299, 0.000014252898382, 0.000034951410953,
  0.000025769637773, 0.000006442404242, 0.000000000000057, 0.000006442404242,
  0.000025769637773, 0.000034932221524, 0.000014246501905, 0.000006442404299,
  0.000011519913926, 0.000029479057332, 0.000056973214100, 0.000034932221467,
  0.000025769637773, 0.000029485453808, 0.000046079702031]:
>  x:=[]:
  y:=[]:
  p1:=2:
  for i from -p1 to p1 do
    for j from -p1 to p1 do
      x:=[op(x),i*0.0005];
      y:=[op(y),j*0.0005];
    end do;
  end do;
> with(LinearAlgebra): p2:=convert(Transpose(Matrix([x,y,z])),listlist):
  pp2:=pointplot3d(p2,symbol=circle,symbolsize=30,color=black):
  with(stats): data:=[x,y,z]: 
  fit1:=fit[leastsquare[[t,s,u], 
    u=a1+a2*t+a3*s+a4*t*s+a5*t^2+a6*s^2, 
    {a1,a2,a3,a4,a5,a6}]](data);
\end{MapleInput}
\begin{MapleOutputGather}
{\it fit1}\, := \,u=-{ 8.657142857\times 10^{-13}}- 0.000006396456800\,t+ 0.000006396438400\,s\notag \\ 
- 5.459553587\,ts+ 25.76962838\,{t}^{2}+ 25.76962835\,{s}^{2} \notag 
\end{MapleOutputGather}
\begin{MapleInput}
> f1:=unapply(rhs(fit1),(s,t)):
> pf1:=plot3d(f1(ss,uu),ss=-0.001..0.001,uu=-0.001..0.001,color=gray):
> display(pf1,pp2,axes=boxed);
\end{MapleInput}
\MaplePlot{50mm}{./figures/C8_LeastSquareFitplot3d4.eps}

\subsection{正規方程式による解法}
デザイン行列へのデータ変換
\begin{MapleInput}
> bb:=Vector(25): A:=Matrix(25,6): 
  p1:=2: 
  for i from 1 to 25 do 
    A[i,1]:=1;
    A[i,2]:=x_i; 
    A[i,3]:=y_i; 
    A[i,4]:=x_i*y_i; 
    A[i,5]:=x_i^2; 
    A[i,6]:=y_i^2;
    bb_i:=z_i; 
  end do:
\end{MapleInput}
正規方程式の解
\begin{MapleInput}
> MatrixInverse(Transpose(A).A).(Transpose(A).bb);
\end{MapleInput}
\begin{MapleOutput}
\left[ \begin {array}{c} -{ 9.185257196\times 10^{-13}}\\   - 0.00000639644675999994798\\    0.00000639644220000032532\\   - 5.45955358336000173\\    25.7696284050857187\\    25.7696284050857543\end {array} \right]
\end{MapleOutput}