\documentclass[10pt,a4j]{jreport}
\usepackage[dvips]{graphicx,color}
\usepackage{verbatim}
\usepackage{amsmath,amsthm,amssymb}
\topmargin -15mm\oddsidemargin -4mm\evensidemargin\oddsidemargin
\textwidth 170mm\textheight 257mm\columnsep 7mm
\setlength{\fboxrule}{0.2ex}
\setlength{\fboxsep}{0.6ex}

\pagestyle{empty}

\newcommand{\MaplePlot}[2]{{\begin{center}
    \includegraphics[width=#1,clip]{#2}
                     \end{center}
%
} }

\newenvironment{MapleInput}{%
    \color{red}\verbatim
}{%
    \endverbatim
}

\newenvironment{MapleError}{%
    \color{blue}\verbatim
}{%
    \endverbatim
}

\newenvironment{MapleOutput}{%
    \color{blue}\begin{equation*}
}{%
    \end{equation*}
}

\newenvironment{MapleOutputGather}{%
    \color{blue}\gather
}{%
    \endgather
}
\newif\ifHIKI
%\HIKItrue % TRUEの設定
\HIKIfalse % FALSEの設定
\begin{document}
\chapter{線形代数--逆行列(LAMatrixInverse)}
\section{行列計算の概要}
単純な2次元データについて補間と近似を考える.補間はたんに点をつなぐことを,近似はある関数にできるだけ近くなるようにフィットすることを言う.補間はIllustratorなどのドロー系ツールで曲線を引くときの,ベジエやスプライン補間の基本となる.本章では補間とそれに密接に関連した積分について述べる.


\ifHIKI
||補間と近似の模式図.
||

\else

\begin{table}[htbp]
\caption{補間と近似の模式図.}
\begin{center}
\begin{tabular}{cc}
補間&近似 \\
\setlength{\unitlength}{1cm}
\begin{picture}(6,6.5)
\put(0,0){\framebox(6,6.5){}}
\end{picture}
&
\setlength{\unitlength}{1cm}
\begin{picture}(6,6.5)
\put(0,0){\framebox(6,6.5){}}
\end{picture}
\end{tabular}
\end{center}
\label{default}
\end{table}%
\fi

\section{ガウス消去法による連立一次方程式の解}
逆行列は連立一次方程式を解くことと等価である.すなわち,$A$を行列,$x$を未知数ベクトル,$b$を数値ベクトルとすると,
\begin{equation*}
\begin{array}{rl} Ax &= b \\
A^{-1}Ax &= A^{-1}b \\
x &= A^{-1}b 
\end{array}
\end{equation*}
である.未知数の少ない連立一次方程式では,適当に組み合わせて未知数を消していけばいいが,未知数が多くなってしまうと破綻する.未知数の多い多元連立一次方程式で,ルーチン的に解を求めていく方法がガウス消去法で,前進消去と後退代入という2つの操作からなる.

後退代入(Backward substitution)による解の求め方を先ず見よう.たとえば,
\begin{equation*}
\begin{array}{rl}
x+y-2z & = -4 \\
-3y+3z & = 9\\
-z & = -2
\end{array}
\end{equation*}
では,下から順番に$z\rightarrow y\rightarrow x$と適宜代入することによって,簡単に解を求めることが出来る.係数で作る行列でこのような形をした上三角行列にする操作を前進消去あるいはガウスの消去法(Gaussian elimination)という.下三角行列L(lower triangular matrix)と上三角行列U(upper triangular matrix)の積に分解する操作
\begin{equation*}
A = L.U
\end{equation*}
をLU分解(LU decomposition)という.例えば先に示した上三角行列を係数とする連立方程式は,
\begin{equation*}
\begin{array}{rl}
x+y-2z&=-4 \\
x-2y+z&=5 \\
2x-2y-z&=2
\end{array}
\end{equation*}
を変形することで得られる.この変形を示せ.
\ifHIKI
\begin{tabular}{|c|}
\hline
         \\ 
\hline
\end{tabular}
\else
\begin{equation*}
\setlength{\unitlength}{1cm}
\begin{picture}(10,3.5)
\put(0,0){\framebox(10,3.5){}}
\end{picture}
\end{equation*}
\fi


  

\section{MapleによるLU分解}
係数行列$A$とベクトル$b$を足して作られる行列は拡大係数行列と呼ばれます.Mapleでは,これは
\begin{MapleInput}
> <A|b>;
\end{MapleInput}
\begin{MapleOutput}
\left[ \begin {array}{ccc} 2&5&7\\ 4&1&5\end {array} \right]
\end{MapleOutput}
として作られます.ここから行列の掃き出し操作をおこなうには,LUDecompositionというコマンドを使います.
\begin{MapleInput}
> P,L,U:=LUDecomposition(<A|b>);
\end{MapleInput}
\begin{MapleOutput}
P,\,L,\,U\, := \, \left[ \begin {array}{cc} 1&0\\ 0&1\end {array} \right] ,\, \left[ \begin {array}{cc} 1&0\\ 2&1\end {array} \right] ,\, \left[ \begin {array}{ccc} 2&5&7\\ 0&-9&-9\end {array} \right]
\end{MapleOutput}
これは,下三角行列(Lower Triangle Matrix)と上三角行列(Upper Triangle Matrix)に分解(decompose)するコマンドです.$P$行列は置換(permutation)行列を意味します.LUDecompositionだけでは,前進消去が終わっただけの状態です.そこで,後退代入までおこなうには,optionにoutput='R'をつけます.そうすると出力は,
\begin{MapleInput}
> LUDecomposition(<A|b>,output='R');
\end{MapleInput}
\begin{MapleOutput}
\left[ \begin {array}{ccc} 1&0&1\\ 0&1&1\end {array} \right]
\end{MapleOutput}
で,$b$ベクトルの部分が解になっています.

\section{LU分解のコード}
LU分解すれば線形方程式の解が容易に求まることは理解できると思う.具体的に$A$をLU分解する行列(消去行列と称す)T1,T2の係数は次のようにして求められる.
\begin{MapleInput}
> A0:=Matrix([[1,1,-2],[1,-2,1],[2,-2,-1]]): 
  b0:=Vector([-4,5,2]):
  A:=Matrix(A0): B:=Vector(b0): n:=3: 
  L:=Matrix(array(1..n,1..n,identity)): 
  for i from 1 to n do #i行目
    T[i]:=Matrix(array(1..n,1..n,identity)): 
                            #i番目の消去行列を作る
    for j from i+1 to n do 
      am:=A[j,i]/A[i,i];    #i行の要素を使って,i+1行目の先頭を消す係数を求める
      T[i][j,i]:=-am;       #i番目の消去行列に要素を入れる
      L[j,i]:=am;           #LTMの要素
      for k from 1 to n do
        A[j,k]:=A[j,k]-am*A[i,k]; #もとの行列をUTMにしていく
      end do; 
      B[j]:=B[j]-B[i]*am;   #数値ベクトルも操作
    end do; 
  end do:
\end{MapleInput}
\begin{MapleOutput}
\end{MapleOutput}

上のコードによって得られた消去行列.
\begin{MapleInput}
> T[1]; T[2];
\end{MapleInput}
\begin{MapleOutputGather}
\left[ \begin {array}{ccc} 1&0&0\\ -1&1&0\\ -2&0&1\end {array} \right] \notag \\
\left[ \begin {array}{ccc} 1&0&0\\ 0&1&0\\ 0&-4/3&1\end {array} \right] \notag 
\end{MapleOutputGather}
これを実際に元の行列$A0$に作用させると,UTMが求められる.
\begin{MapleInput}
> U:=T[2].T[1].A0;
\end{MapleInput}
\begin{MapleOutput}
U\, := \, \left[ \begin {array}{ccc} 1&1&-2\\ 0&-3&3\\ 0&0&-1\end {array} \right]
\end{MapleOutput}
求められたLTM, UTMを掛けると
\begin{MapleInput}
> L.U;
\end{MapleInput}
\begin{MapleOutput}
\left[ \begin {array}{ccc} 1&1&-2\\ 1&-2&1\\ 2&-2&-1\end {array} \right]
\end{MapleOutput}
元の行列を得られる.L,Aに求めたい行列が入っていることを確認.
\begin{MapleInput}
> L;A;
\end{MapleInput}
\begin{MapleOutputGather}
\left[ \begin {array}{ccc} 1&0&0\\ 1&1&0\\ 2&4/3&1\end {array} \right] \notag \\
\left[ \begin {array}{ccc} 1&1&-2\\ 0&-3&3\\ 0&0&-1\end {array} \right]  \notag 
\end{MapleOutputGather}
数値ベクトルも期待通り変換されている.
\begin{MapleInput}
> B;
\end{MapleInput}
\begin{MapleOutput}
\left[ \begin {array}{c} -4\\ 9\\ -2\end {array} \right]
\end{MapleOutput}


\section{ピボット操作}
ガウス消去法で困るのは,割ろうとした対角要素が0の場合である.しかし,この場合にも,方程式の順序を,行列の行と右辺の値をペアにして入れ替えれば解決する.この割る
ほうの要素をピボット要素あるいはピボット(pivot,バスケの軸足を動かさずにくるくる回すやつ)と呼ぶ.この操作は,変数の並びを変えたわけではなく,単に方程式の
順番を変更する操作に相当する.

さらに対角要素の数値が厳密に0でなくとも,極端に0に近づいた場合にも,その数で割った数値が大きくなり他の数との差を取ると以前に示した情報落ちの可能性が出てくる.
この現象を防ぐためには,絶対値が最大のピボットを選んで行の入れ替えを毎回おこなうといい結果が得られることが知られている.

MapleのLUDecompositionコマンドをこのような行列に適用すると,置換行列(permutation
matrix)Pが単位行列ではなく,ピボット操作に対応した行列となる.P.A=L.Uとなることに注意.
\section{反復法による連立方程式の解}
以下のような連立方程式を
\begin{equation*}
\left[ \begin {array}{c} 5\,x+y+z+u\\ x+3\,y+z+u\\ x-2\,y-9\,z+u\\ x+3\,y-2\,z+5\,u\end {array} \right] = \left[ \begin {array}{c} -6\\ 2\\ -7\\ 3\end {array} \right]
\end{equation*}
形式的に解くと
\begin{equation*}
x=\frac{-6-(y+z+u)}{5}
\end{equation*}
となる.他の未知数も,
\ifHIKI
||y=||         ||
||z=||         ||
||u=||         ||
\else
\begin{equation*}
\setlength{\unitlength}{1cm}
\begin{picture}(5,4.5)
\put(0,3.0){y=\framebox(4,1.3){}}
\put(0,1.5){z=\framebox(4,1.3){}}
\put(0,0.0){u=\framebox(4,1.3){}}
\end{picture}
\end{equation*}
\fi
となる.適当に初期値($x_0,y_0,z_0,u_0$)をとり,下側の方程式に代入すると,得られた出力($x_1,y_1,z_1,u_1$)はより正解に近い値となる.これを繰り返すことによって正解が得られる.これをヤコビ(Jacobi)法と呼び,係数行列の対角要素が非対角要素にくらべて大きいときに適用できる.多くの現実の問題ではこの状況が成り立っている.

Gauss-Seidel法はJacobi法の高速版である.$n$番目の解の組が得られた後に一度に次の解の組に入れ替えるのではなく,得られた解を順次改良した解として使っていく.これにより,収束が早まる.以下にはヤコビ法のコードを示した.x1[i]の配列を変数に換えるだけで,Gauss-Seidel法となる.
\begin{MapleInput}
> AA:=Matrix([[5,1,1,1],[1,3,1,1],[1,-2,-9,1],[1,3,-2,5]]):
  b:=Vector([-6,2,-7,3]): n:=4; 
  x0:=[0,0,0,0]: x1:=[0,0,0,0]: 
  for iter from 1 to 20 do
    for i from 1 to n do
      x1[i]:=b[i]; 
      for j from 1 to n do
        x1[i]:=x1[i]-AA[i,j]*x0[j];
      end do:
      x1[i]:=x1[i]+AA[i,i]*x0[i];
      x1[i]:=x1[i]/AA[i,i];
    end do:
    x0:=evalf(x1);
    print(iter,x0);
  end do:
\end{MapleInput}
\begin{MapleError}
                                      4
          1, [-1.200000000, 0.6666666667, 0.7777777778, 0.6000000000]
          2, [-1.608888889, 0.6074074073, 0.5629629630, 0.7511111112]
          3, [-1.584296296, 0.7649382717, 0.5474897119, 0.7825185186]
          4, [-1.618989300, 0.7514293553, 0.5187050756, 0.6768921810]
          5, [-1.589405322, 0.8077973477, 0.5061160189, 0.6804222770]
          6, [-1.598867129, 0.8009556753, 0.4972691400, 0.6356490634]
          7, [-1.586774776, 0.8219829753, 0.4927633981, 0.6381076766]
          8, [-1.590570810, 0.8186345670, 0.4897074389, 0.6212705292]
          9, [-1.585922507, 0.8265309473, 0.4881589539, 0.6228163974]
         10, [-1.587501260, 0.8249823853, 0.4870924439, 0.6165295146]
         11, [-1.585720869, 0.8279597673, 0.4865626093, 0.6173477984]
         12, [-1.586374035, 0.8272701537, 0.4861897104, 0.6149933572]
         13, [-1.585690644, 0.8283969890, 0.4860087794, 0.6153885990]
         14, [-1.585958873, 0.8280977553, 0.4858782197, 0.6145034472]
         15, [-1.585695884, 0.8285257353, 0.4858165626, 0.6146844092]
         16, [-1.585805341, 0.8283983040, 0.4857707838, 0.6143503606]
         17, [-1.585703890, 0.8285613990, 0.4857498236, 0.6144303994]
         18, [-1.585748324, 0.8285078890, 0.4857337457, 0.6143038680]
         19, [-1.585709101, 0.8285702367, 0.4857266407, 0.6143384296]
         20, [-1.585727061, 0.8285480103, 0.4857209840, 0.6142903344]
\end{MapleError}
\section{課題}
\begin{enumerate}
\item 補間と近似の違いについて,適切な図を描いて説明せよ.
\item 次の4点
\begin{MapleInput}
x y 
0 1 
1 2
2 3
3 -2
\end{MapleInput}
を通る多項式を以下のそれぞれの手法で求めよ.(a) 逆行列, (b)ラグランジュ補間, (c)ニュートンの差分商公式 
\item
$\tan(5^\circ)=0.08748866355$, 
$\tan(10^\circ)=.1763269807$,
$\tan(15^\circ)=.2679491924$の値を用いて,ラグランジュ補間法により,$\tan(17^\circ)$の値を推定せよ.(2008年度期末試験)
\item exp(0)=1.0, exp(0.1)=1.1052, exp(0.3)=1.3499の値を用いて,ラグランジュ補間法により,exp(0.2)の値を推定せよ.(2009年度期末試験)
\item 次の関数
\begin{equation*}
f(x) = \frac{4}{1+x^2}
\end{equation*}
を$x = 0..1$で数値積分する.
\begin{enumerate}
\item $N$を2,4,8,…256ととり,$N$個の等間隔な区間にわけて中点法で求めよ.(15)
\item 小数点以下10桁まで求めた値3.141592654との差をdXとする.dXと分割数Nとを両対数プロット(loglogplot)して比較せよ(10)
\end{enumerate}
(2008年度期末試験)
\item 次の関数
\begin{equation*}
y = \frac{1}{1+x^2}
\end{equation*}
を$x = 0..1$で等間隔に$N+1$点とり,$N$個の区間にわけて数値積分で求める.$N$を2, 4, 8, 16, 32, 64, 128, 256と取ったときの(a)中点法, (b)台形公式, (c)シンプソン公式それぞれの収束性を比較せよ.

ヒント:Maple script にあるそれぞれの数値積分法を関数 (procedure) に直して,for-loop
で回せば楽.出来なければ,一つ一つ手で変えても OK. 両対数プロット (loglogplot) すると見やすい.
\end{enumerate}
\section{解答例}
4. Jacobi法のプログラムを参照してGauss-Seidel法のプログラムを作れ.Jacobi法と収束性を比べよ.
\begin{MapleInput}
#Gauss-Seidel
AA:=Matrix([[5,1,1,1],[1,3,1,1],[1,-2,-9,1],[1,3,-2,5]]):
b:=Vector([-6,2,-7,3]):
n:=4;
x0:=[0,0,0,0]:
x1:=[0,0,0,0]:
for iter from 1 to 20 do
for i from 1 to n do
  x1[i]:=b[i];
  for j from 1 to n do
    x1[i]:=x1[i]-AA[i,j]*x0[j];
  end do:
  x1[i]:=x1[i]+AA[i,i]*x0[i];
  x1[i]:=x1[i]/AA[i,i];
  x0:=evalf(x1);  #change here from ...
end do:
print(iter,x0);
end do:
\end{MapleInput}
\begin{MapleError}
                               4
   1, [-1.200000000, 1.066666667, 0.4074074073, 0.3629629628]
  2, [-1.567407407, 0.9323456790, 0.4367626887, 0.5287791494]
  3, [-1.579577503, 0.8713452217, 0.4673901337, 0.5800644210]
  4, [-1.583759955, 0.8454351333, 0.4783815777, 0.6008435420]
  5, [-1.584932051, 0.8352356437, 0.4828266893, 0.6089756998]
  6, [-1.585407607, 0.8312017393, 0.4845738460, 0.6121900162]
  7, [-1.585593120, 0.8296097527, 0.4852641546, 0.6134584342]
  8, [-1.585666468, 0.8289812930, 0.4855365978, 0.6139591570]
  9, [-1.585695410, 0.8287332183, 0.4856441456, 0.6141568092]
  10, [-1.585706835, 0.8286352933, 0.4856865986, 0.6142348304]
  11, [-1.585711344, 0.8285966383, 0.4857033566, 0.6142656284]
  12, [-1.585713125, 0.8285813800, 0.4857099714, 0.6142777856]
  13, [-1.585713827, 0.8285753567, 0.4857125829, 0.6142825846]
  14, [-1.585714105, 0.8285729793, 0.4857136134, 0.6142844788]
  15, [-1.585714214, 0.8285720407, 0.4857140204, 0.6142852266]
  16, [-1.585714258, 0.8285716703, 0.4857141809, 0.6142855218]
  17, [-1.585714275, 0.8285715240, 0.4857142443, 0.6142856384]
  18, [-1.585714281, 0.8285714660, 0.4857142694, 0.6142856844]
  19, [-1.585714284, 0.8285714433, 0.4857142792, 0.6142857024]
  20, [-1.585714285, 0.8285714343, 0.4857142831, 0.6142857096]
\end{MapleError}











\end{document}
