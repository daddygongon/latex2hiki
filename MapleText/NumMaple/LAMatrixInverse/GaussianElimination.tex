逆行列は連立一次方程式を解くことと等価である.すなわち,$A$を行列,$x$を未知数ベクトル,$b$を数値ベクトルとすると,
\begin{equation*}
\begin{array}{rl} Ax &= b \\
A^{-1}Ax &= A^{-1}b \\
x &= A^{-1}b 
\end{array}
\end{equation*}
である.未知数の少ない連立一次方程式では,適当に組み合わせて未知数を消していけばいいが,未知数が多くなってしまうと破綻する.未知数の多い多元連立一次方程式で,ルーチン的に解を求めていく方法がガウス消去法で,前進消去と後退代入という2つの操作からなる.

後退代入(Backward substitution)による解の求め方を先ず見よう.たとえば,
\begin{equation*}
\begin{array}{rl}
x+y-2z & = -4 \\
-3y+3z & = 9\\
-z & = -2
\end{array}
\end{equation*}
では,下から順番に$z\rightarrow y\rightarrow x$と適宜代入することによって,簡単に解を求めることが出来る.係数で作る行列でこのような形をした上三角行列にする操作を前進消去あるいはガウスの消去法(Gaussian elimination)という.下三角行列L(lower triangular matrix)と上三角行列U(upper triangular matrix)の積に分解する操作
\begin{equation*}
A = L.U
\end{equation*}
をLU分解(LU decomposition)という.例えば先に示した上三角行列を係数とする連立方程式は,
\begin{equation*}
\begin{array}{rl}
x+y-2z&=-4 \\
x-2y+z&=5 \\
2x-2y-z&=2
\end{array}
\end{equation*}
を変形することで得られる.この変形を示せ.
\ifHIKI
\begin{tabular}{|c|}
\hline
         \\ 
\hline
\end{tabular}
\else
\begin{equation*}
\setlength{\unitlength}{1cm}
\begin{picture}(10,3.5)
\put(0,0){\framebox(10,3.5){}}
\end{picture}
\end{equation*}
\fi


  
