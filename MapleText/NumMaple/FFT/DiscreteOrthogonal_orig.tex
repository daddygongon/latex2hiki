ところが,実際に積分していては,時間がかかりすぎる.直交関数系の選点直交性を使うとより簡単になる.

\paragraph{直交関数系の選点直交性}
\begin{quotation}
直交多項式は,
\begin{equation*}
\varphi _{{n}} \left( x \right) =0 \,at\,x_1,\,x_2,\,\cdots x_{{n}}
\end{equation*}
である.$n-1$以下の次数$m,\,l$では,
\begin{equation*}
\sum _{i=1}^{n}\phi_{{l}} \left( x_{{i}} \right) \varphi _{{m}} \left( x_{{i}} \right) =\delta_{{{\it ml}}}C_{{l}}
\end{equation*}
が成り立つ.これは,直交関係と違い積分でないことに注意.証明は略.
\end{quotation}
これを使えば,この先程の直交関数展開
\begin{equation*}
F \left( x \right) =\sum _{l=1}^{N}a_{{l}}\varphi _{{l}} \left( x \right)
\end{equation*}
の両辺に$\varphi _{{m}} \left( x_{{i}} \right)$を掛けて$i$について和をとれば,
\begin{gather}
\sum _{i=1}^{n} F \left(x _{i }\right)\phi _{m }\left(x _{i }\right) =
\sum _{i=1}^{n}  \sum _{l=1}^{N}a_{{l}}\varphi _{{l}} \left( x_{{i}} \right) \varphi _{{m}} \left( x_{{i}} \right)  \notag \\
=\sum _{l=1}^{N}  a_{{l}}\sum _{i=1}^{n}\varphi _{{l}} \left( x_{{i}} \right) \varphi _{{m}} \left( x_{{i}} \right) \notag \\
=\sum _{l=1}^{N}a_{{l}}\delta_{{{\it ml}}}C_{{m}}=a_{{m}}C_{{m}} \notag
\end{gather}
となる.つまり,
\begin{equation*}
a_{{m}}=\frac{1}{C_{{m}}} {\sum _{i=1}^{n}F \left( x_{{i}} \right) \varphi _{{m}} \left( x_{{i}} \right) }
\end{equation*}
となり,単純な関数の代入とかけ算で係数が決定される.

\subsection{選点直交性を用いた結果}
\begin{MapleInput}
> KK:=4; N:=2^KK;L:=1-0;
> for k from 0 to N-1 do 
    c[k]:=evalf(sum(F(i*L/N)*exp(-I*2*Pi*k*i/N),i=0..N-1));
  end do;
\end{MapleInput}
\begin{MapleError}
c_0:=0. 
c_1:=-2.000000000 + 10.05467898 I 
c_2:=0. 
c_3:=-2.000000000 + 2.993211524 I 
c_4:=0. 
c_5:=-2.000000001 + 1.336357276 I 
c_6:=0. 
c_7:=-2.000000001 + 0.3978247331 I 
c_8:=0. 
c_9:=-2.000000001 - 0.3978247331 I 
c_10:=0. 
c_11:=-2.000000001 - 1.336357276 I 
c_12:=0. 
c_13:=-2.000000000 - 2.993211524 I 
c_14:=0. 
c_15:=-2.000000000 - 10.05467898 I
\end{MapleError}

\begin{MapleInput}
> F1:=unapply(sum(evalf(c[i]*exp(I*2*Pi*i/L*x)/N),i=0..(N/2-1))+
> sum(evalf(c[N-i]*exp(-I*2*Pi*i/L*x)/N),i=1..(N/2-1)),x):
> plot({Re(F1(x)),F(x)},x=0..1);
\end{MapleInput}
\MaplePlot{50mm}{./figures/C10_FFTplot2d13.eps}

