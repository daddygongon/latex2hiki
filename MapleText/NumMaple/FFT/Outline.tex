このように便利なFFTであるが,どのような理屈で導かれるのか? Fourier変換法は,この課題だけでも何回ものコマ数が必要なほどの内容を含んでいる.ここでは,
その基本となる考え方(のひとつ)だけを提示する.
\begin{enumerate}
\item 関数の内挿で導入した基底関数を直交関数系でとる.ところが,展開係数を逆行列で求める手法では計算が破綻.
\item 直交関係からの積分による係数決定.
\item 選点直交性による計算の簡素化.
\item 高速フーリエ変換アルゴリズムによる高速化.
\end{enumerate}
