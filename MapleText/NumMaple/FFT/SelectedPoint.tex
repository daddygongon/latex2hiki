選点直交性による計算の簡素化
ところが,実際に積分していては,時間がかかりすぎる.直交関数系の選点直交性を使うとより簡単になる.
直交関数系の選点直交性
直交多項式は,
`&varphi;`[n](x) = (0*at)*x and (0*at)*x = x[1], x[2], `&ctdot;`(x[n])
である.n-1以下の次数m, lでは,
sum(phi[l](x[i])*`&varphi;`[m](x[i]), i = 1 .. n) = delta[ml]*C[l]

が成り立つ.これは,直交関係と違い積分でないことに注意.証明は略.
これを使えば,この先程の直交関数展開
F(x) = sum(a[l]*`&varphi;`[l](x), l = 1 .. N)
の両辺に`&varphi;`[m](x[i])を掛けてiについて和をとれば,

"(&sum;)F(x[i])phi[m](x[i])="sum(sum(a[l]*`&varphi;`[l](x[i])*`&varphi;`[m](x[i]
), l = 1 .. N), i = 1 .. n)
=sum(a[l]*(sum(`&varphi;`[l](x[i])*`&varphi;`[m](x[i]), i = 1 .. n)), l = 1 ..
N)
=sum(a[l]*delta[ml]*C[m], l = 1 .. N) = a[m]*C[m]
となる.つまり,
a[m] = (sum(F(x[i])*`&varphi;`[m](x[i]), i = 1 .. n))/C[m]
となり,単純な関数の代入とかけ算で係数が決定される.
選点直交性を用いた結果
> KK:=4; N:=2^KK;L:=1-0;
                                      4
                                      16
                                      1
> for k from 0 to N-1 do c[k]:=evalf(sum(F(i*L/N)*exp(-I*2*Pi*k*i/N),i=0..N-1));
> end do;
                                      0.
                         -2.000000000 + 10.05467898 I
                                      0.
                         -2.000000000 + 2.993211524 I
                                      0.
                         -2.000000001 + 1.336357276 I
                                      0.
                        -2.000000001 + 0.3978247331 I
                                      0.
                        -2.000000001 - 0.3978247331 I
                                      0.
                         -2.000000001 - 1.336357276 I
                                      0.
                         -2.000000000 - 2.993211524 I
                                      0.
                         -2.000000000 - 10.05467898 I
> F1:=unapply(sum(evalf(c[i]*exp(I*2*Pi*i/L*x)/N),i=0..(N/2-1))+
> sum(evalf(c[N-i]*exp(-I*2*Pi*i/L*x)/N),i=1..(N/2-1)),x):
> plot({Re(F1(x)),F(x)},x=0..1);

