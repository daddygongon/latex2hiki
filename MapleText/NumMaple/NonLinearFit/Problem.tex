\begin{enumerate}
\item 一山ピークへのフィット

以下の256個のデータ
\begin{MapleInput}
> ndata:=256; f1:=t->subs({a1=10,a2=40000,a3=380,a4=128},a1+a2/(a3+(t-a4)^2) );
> T:=[seq(f1(i)*(0.6+0.8*evalf(rand()/10^12)),i=1..ndata)]:
> f:=t->a1+a2/(a3+(t-a4)^2);
\end{MapleInput}
で近似したときのパラメータa1,a2,a3,a4を求めよ.ただし,パラメータの初期値は,ある程度近い値にしないと収束しない.
\item 二山ピークのフィット
以下のように作成したデータ
\begin{MapleInput}
> ndata:=256; f1:=t->subs({a=10,b=40000,c=380,d=128},a+b/(c+(t-d)^2) );
> f2:=t->subs({a=10,b=40000,c=380,e=90},a+b/(c+(t-e)^2) );
> T:=[seq((f1(i)+f2(i))*(0.6+0.2*evalf(rand()/10^12)),i=1..ndata)]:
\end{MapleInput}
を
\begin{MapleInput}
> f:=t->a1+a2/(a3+(t-a4)^2)+a2/(a3+(t-a5)^2);
\end{MapleInput}
\begin{MapleOutput}
f\, := \,t\mapsto {\it a1}+{\frac {{\it a2}}{{\it a3}+ \left( t-{\it a4} \right) ^{2}}}+{\frac {{\it a2}}{{\it a3}+ \left( t-{\it a5} \right) ^{2}}}
\end{MapleOutput}
で近似したときのパラメータを求めよ.
\begin{MapleInput}
> l1:=listplot(T): display(l1);
\end{MapleInput}
\MaplePlot{50mm}{./figures/C9_NonLinearFitplot2d8.eps}

\end{enumerate}
