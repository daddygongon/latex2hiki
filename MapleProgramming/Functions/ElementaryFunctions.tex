\subsection{初等関数(ElementaryFunctions)}
\paragraph{四則演算とevalf}
四則演算は"+-*/".割り切れない割り算は分数のまま表示される.
\begin{MapleInput}
> 3/4;
\end{MapleInput}
\begin{MapleOutput}
\frac{3}{4}
\end{MapleOutput}
強制的に数値(浮動小数点数)で出力するにはevalfを用いる.
\begin{MapleInput}
> evalf(3/4);
\end{MapleInput}
\begin{MapleOutput}
0.7500000000
\end{MapleOutput}
\paragraph{多項式関数(polynom)}
かけ算も省略せずに打ち込む必要がある.またベキ乗は\verb="^"=である.
\begin{MapleInput}
> 3*x^2-4*x+3;
\end{MapleInput}
\begin{MapleOutput}
3x^2-4x+3
\end{MapleOutput}
\paragraph{平方根(sqrt)}
平方根はsquare rootを略したsqrtを使う.
\begin{MapleInput}
> sqrt(2);
\end{MapleInput}
\begin{MapleOutput}
\sqrt{2}
\end{MapleOutput}

\paragraph{三角関数(trigonal)}
sin, cosなどの三角関数はラジアンで入力する.ただし, $\sin^2x$などは
\begin{MapleInput}
> sin^2 x;
\end{MapleInput}
\begin{MapleError}
Error, missing operator or `;`
\end{MapleError}
ではだめで,
\begin{MapleInput}
> sin(x)^2;
\end{MapleInput}
\begin{MapleOutput}
\sin^2x
\end{MapleOutput}
と省略せずに打ち込まねばならない.三角関数でよく使う定数$\pi$は"Pi"と入力する.Mapleは大文字と小文字を区別するので注意.

ラジアン(radian)に度(degree)から変換するには以下のようにする.
\begin{MapleInput}
> convert(90*degrees, radians);
  convert(1/6*Pi,degrees);
\end{MapleInput}
\begin{MapleOutput}
\frac{1}{2}\pi
\end{MapleOutput}
\begin{MapleOutput}
30\,\, degrees
\end{MapleOutput}
\paragraph{その他の関数(inifnc)}
その他の初等関数やよく使われる超越関数など,Mapleの起動時に用意されている関数のリストは,
\begin{MapleInput}
> ?inifnc;
\end{MapleInput}
で得られる.
