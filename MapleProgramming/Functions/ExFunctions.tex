\subsubsection{関数についての課題}
\begin{enumerate}
\item evalfのヘルプを参照して,Piを1000桁まで表示せよ.
\item 正接関数(tan)とその逆関数arctanをx=-Pi/2..Pi/2,y=-Pi..Pi,scaling=constrainedで同時にプロットせよ.
\item 対数関数はln(x)で与えられる.ヘルプを参照しながら次の値を求めよ.
\begin{equation*}
\log_{10}1000, \log_{2}\frac{1}{16}, \log_{\sqrt{5}}125
\end{equation*}
\item 次の関数は$y=2^x$とどのような位置関係にあるかx=-5..5,y=-5..5で同時にプロットして観察せよ.
\begin{equation*}
y = -2^x, y = (1/2)^x,y = -(1/2)^x
\end{equation*}
\item 指数関数はexpで与えられる.$\mbox{e}^x$と$\log x$関数および$y = x$を同時にx=-5..5,y=-5..5でplotせよ.またそれらの関数の位置関係を述べよ.
\item 階乗関数factorialに3を代入して何を求める関数か予測せよ.ヘルプを参照し,よりなじみの深い表記を試してみよ.
\end{enumerate}
