\subsection{入力領域と注意点(ShiftEnter)}
Mapleを起動すると赤いプロンプトがともっている.ここに命令(コマンド)を打ち込んでMapleの計算部に『こちらの意志』を伝えて動作させる.例えば,
\begin{MapleInput}
> plot(sin(x),x); 
\end{MapleInput}
と入力し,
\ifHIKI
[ Enter ]
\else
\fbox{enter}
\fi
を入力してみよ.sin関数がプロットされる.
\MaplePlot{50mm}{./figures/FirstLeafplot2d1.eps}
\begin{enumerate}
\item 赤い領域のどこにカーソルがあっても
\ifHIKI
[ Enter ]
\else
\fbox{enter}
\fi
を入れれば,そのブロックごとMapleに命令として渡される.テキストでは
\ifHIKI
[ Enter ]
\else
\fbox{enter}
\fi
を省略している.
\item テキストの修正は普通のワープロソフトと同じ.
\item 命令の入力ではなく,\textbf{改行だけをいれたいときはshift+enterを入れる.}
\item 命令は,enterを入れた順に解釈されるのであって,テキストの上下とは関係ない.
\end{enumerate}
