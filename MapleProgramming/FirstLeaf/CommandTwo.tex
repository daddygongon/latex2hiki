\subsection{命令コマンドの基本形(command();)}
命令コマンドは全て次のような構造を取る.
\begin{MapleInput}
> command(引数1,引数2,...);
\end{MapleInput}
あるいは
\begin{MapleInput}
> command(引数1,オプション1,オプション2,...);
\end{MapleInput}
となる.

\begin{enumerate}
\item ()の中の引数やオプションの間はコンマで区切る.
\item 最後の;(セミコロン)は次のコマンドとの区切り記号.
\item セミコロン(;)をコロン(:)に替えるとMapleからの返答が出力されなくなるが,Mapleへの入力は行われている.
\item C言語などの手続き型プログラミング言語の標準的なフォーマットと同じ.
\end{enumerate}

命令コマンドを,英語の命令文と解釈してもよい.たとえば,
\begin{MapleError}
sin(x)をxについて0からpiまでplotせよ.
\end{MapleError}
という日本語を英語に訳すと,
\begin{MapleError}
plot sin(x) with x from 0 to Pi.
\end{MapleError}
となる.この英語をMaple語に訳して
\begin{MapleInput}
> plot(sin(x),x=0..Pi);
\end{MapleInput}
となったとみなせる.英語文法のVerb (動詞), Object (目的語)を当てはめると,Mapleへの命令は,
\begin{MapleInput}
> Verb(Object, その他の修飾);
\end{MapleInput}
である.英文でピリオドを忘れるなと中学時代に言われたのと同じく,Mapleでセミコロンを忘れぬように.

