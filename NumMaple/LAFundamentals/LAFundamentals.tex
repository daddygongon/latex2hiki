\documentclass[10pt,a4j]{jreport}
\usepackage[dvips]{graphicx,color}
\usepackage{verbatim}
\usepackage{tabularx}
\usepackage{amsmath,amsthm,amssymb}
\topmargin -15mm\oddsidemargin -4mm\evensidemargin\oddsidemargin
\textwidth 170mm\textheight 257mm\columnsep 7mm
\setlength{\fboxrule}{0.2ex}
\setlength{\fboxsep}{0.6ex}

\pagestyle{empty}

\newcommand{\MaplePlot}[2]{{\begin{center}
    \includegraphics[width=#1,clip]{#2}
                     \end{center}
%
} }

\newenvironment{MapleInput}{%
    \color{red}\verbatim
}{%
    \endverbatim
}

\newenvironment{MapleError}{%
    \color{blue}\verbatim
}{%
    \endverbatim
}

\newenvironment{MapleOutput}{%
    \color{blue}\begin{equation*}
}{%
    \end{equation*}
}

\newenvironment{MapleOutputGather}{%
    \color{blue}\gather
}{%
    \endgather
}
\newif\ifHIKI
%\HIKItrue % TRUEの設定
\HIKIfalse % FALSEの設定
\begin{document}
\chapter{線形代数--写像(LAFundamentals)}
\section{行列と連立方程式}
大学の理系で必修なのは微積分と線形代数です.線形代数というと逆行列と固有値の計算がすぐに思い浮かぶでしょう.計算がややこしくてそれだけでいやになります.でも,行列の計算法は一連の手順で記述できるので,Mapleでは微積分とおなじように一個のコマンドで片が付きます.それが3x3以上でも同じです.問題はその意味です.ここでは,線形代数の計算がMapleを使えばどれほど簡単にできるかを示すと共に,線形代数の基本となる概念についてスクリプトと描画を使って,直観的に理解することを目的とします.

先ずは連立方程式から入っていきます.中学の時に
\begin{equation*}
4x = 2
\end{equation*}
というのを解きますよね.一般的には
\begin{equation*}
\begin {array}{rl}
ax &= b \\
x &= b/a
\end {array}
\end{equation*}
と書けるというのは皆さんご存知のはず.これと同じようにして連立方程式を書こうというのが逆行列の基本.つまり
\begin{equation*}
\begin {array}{rrl}
2x\, + &5y &=7 \\
4x\, + &y &=5
\end {array}
\end{equation*}
という連立方程式は,係数から作られる2x2行列を係数行列$A$,左辺の値で作るベクトルを$b$として,
\begin{equation*}
\begin {array}{rll}
Ax &= b & \\
x &= b/A &= A^{-1}b
\end {array}
\end{equation*}
としたいわけです.

実際にMapleでやってみましょう.行列は英語でMatrixです.
\begin{MapleInput}
> restart: A:=Matrix([[2,5],[4,1]]); 
\end{MapleInput}
\begin{MapleOutput}
A\, := \, \left[ \begin {array}{cc} 2&5\\ 4&1\end {array} \right]
\end{MapleOutput}
こうして行列を作ります.

\begin{MapleInput}
> b:=Vector([7,5]); #(2)ベクトルは英語でVectorです.これで縦ベクトルができます.
\end{MapleInput}
\begin{MapleOutput}
b\, := \, \left[ \begin {array}{c} 7\\ 5\end {array} \right]
\end{MapleOutput}
線形代数はlinear algebraと言います.withでLinearAlgebraというライブラリーパッケージを読み込んでおきます.
\begin{MapleInput}
> with(LinearAlgebra): 
\end{MapleInput}
逆行列はmatrix inverseと言います.
\begin{MapleInput}
> x0:=MatrixInverse(A).b;
\end{MapleInput}
行列AのMatrixInverseを求めて,ベクトルbに掛けています.
\begin{MapleOutput}
{\it x0}\, := \, \left[ \begin {array}{c} 1\\ 1\end {array} \right] 
\end{MapleOutput}
と簡単に求めることができます.

\section{掃き出し}
線形代数の計算にはあらかじめ関数パッケージ(LinearAlgebra)を呼び出しておく.
\begin{MapleInput}
> with(LinearAlgebra):
\end{MapleInput}

\subsection{行列の基本操作}
行列の掃き出しに必要となる行列の基本操作はRowOperation, ColumnOperationを参照.

\subsection{掃き出し法,LU分解(LUDecomposition)}
掃き出し法の計算は,LUDecompositionでおこなう.まず拡大係数行列を作る.
\begin{MapleInput}
> A1:=<1,2;3,4>; b:=<2,3>; <A1|b>;
\end{MapleInput}
\begin{MapleOutputGather}
{\it A1}\, := \, \left[ \begin {array}{cc} 1&2\\ 3&4\end {array} \right] \notag \\
b\, := \, \left[ \begin {array}{c} 2\\ 3\end {array} \right] \notag \\
\left[ \begin {array}{ccc} 1&2&2\\ 3&4&3\end {array} \right]  \notag
\end{MapleOutputGather}
これにLU分解をかける.それぞれP(permutation,置換), L(lower triangle,下三角), U(upper
triangle,上三角)行列に代入している.
\begin{MapleInput}
> P,L,U:=LUDecomposition(<A1|b>);
\end{MapleInput}
\begin{MapleOutput}
P,\,L,\,U\, := \, \left[ \begin {array}{cc} 1&0\\ 0&1\end {array} \right] ,\, \left[ \begin {array}{cc} 1&0\\ 3&1\end {array} \right] ,\, \left[ \begin {array}{ccc} 1&2&2\\ 0&-2&-3\end {array} \right] 
\end{MapleOutput}
さらに被約階段行列(row reduced echelonmatrix;後退代入までおこなって,解まで求めた状態)を求めるには,output='R'を指定する.
\begin{MapleInput}
> LUDecomposition(<A1|b>, output='R');
\end{MapleInput}
\begin{MapleOutput}
\left[ \begin {array}{ccc} 1&0&-1\\ 0&1&3/2\end {array} \right]
\end{MapleOutput}

\subsection{階数(Rank)}
行列の性質の中でも特に重要な階数(Rank)は次のコマンドで求められる.
\begin{MapleInput}
> Rank(A1);
\end{MapleInput}
\begin{MapleOutput}
2
\end{MapleOutput}
\section{写像}
次に,これを2次元上のグラフで見てみましょう.先ず描画に必要なライブラリーパッケージ(plotsおよびplottools)をwithで読み込んでおきます.
\begin{MapleInput}
> with(plots):with(plottools): 
\end{MapleInput}
ベクトルは,位置座標を意味するようにlistへ変換(convert)しておきます.
\begin{MapleInput}
> p0:=convert(x0,list); p1:=convert(b,list); 
\end{MapleInput}
位置p0に円(disk)を半径0.2,赤色で描きます.同じように位置p1に半径0.2,青色でdiskを描きます.もう一つ,p0からp1に向かう矢印(arrow)
を適当な大きさで描きます.後ろの数字をいじると線の幅や矢印の大きさが変わります.
\begin{MapleInput}
> point1:=[disk(p0,0.2,color=red), disk(p1,0.2,color=blue)]:
> line1:=arrow(p0,p1,.05,.3,.1 ):
\end{MapleInput}

これらをまとめて表示(display)します.このとき,表示範囲を-8..8,-8..8とします.
\begin{MapleInput}
> display(point1,line1,view=[-2..8,-2..8],gridlines=true);
\end{MapleInput}
\MaplePlot{70mm}{./figures/LAFundamentalsplot2d1.eps}

逆行列は
\begin{MapleInput}
> MatrixInverse(A);
\end{MapleInput}
\begin{MapleOutput}
\left[ \begin {array}{cc} -1/18&5/18\\ 2/9&-1/9\end {array} \right]
%\left[  \begin {array}{cc} \displaystyle -\frac{1}{18}&  \displaystyle \frac{5}{18}\\
% \displaystyle  \frac{2}{9}&  \displaystyle -\frac{1}{9}\end {array} \right]
\end{MapleOutput}
で求まります.先ほどの矢印を逆に青から赤へたどる変換になっています.これが,連立方程式を解く様子をグラフで示しています.つまり,行列Aで示される変換によって求まる青点で示したベクトルb(7,5)を指す元の赤点を捜すというものです.答えは(1,1)となります.

では,元の赤点をもう少しいろいろ取って,行列Aでどのような点へ写されるかを見てみましょう.
\begin{MapleInput}
> N:=30:point2:=[]:line2:=[]: 
  for k from 0 to N-1 do
    x0:=Vector([sin(2*Pi*k/N),cos(2*Pi*k/N)]); 
    x1:=A.x0; 
    p0:=convert(x0,list);
    p1:=convert(x1,list); 
    point2:=[op(point2),disk(p0,0.05,color=red)];
    point2:=[op(point2),disk(p1,0.05,color=blue)]; 
    line2:=[op(line2),line(p0,p1)];
  end do:
\end{MapleInput}
N:=30で分割した円周上の点をx0で求めて,point2にその円とそれのA.x0を,line2にはその2点を結ぶline(線)を足しています.
使っているコマンドは,先ほどの描画とほぼ同じです.ただし,Mapleスクリプトに特有のidiom(熟語)を使っています.この基本形をとり出すと,
\begin{MapleInput}
> list1:=[]; 
  for k from 0 to 2 do 
    list1:=[op(list1),k]; 
  end do; 
  list1;
\end{MapleInput}
\begin{MapleOutputGather}
[] \notag \\
[0] \notag \\
[0, 1] \notag \\
[0, 1, 2] \notag \\
[0, 1, 2] \notag
\end{MapleOutputGather}
となります.for-loopでkを0から4まで回し,list1に次々と値を追加していくというテクです.

できあがりの次の図を見てください.
\begin{MapleInput}
> d:=6: display(point2,line2,view=[-d..d,-d..d]);
\end{MapleInput}
\MaplePlot{100mm}{./figures/LAFundamentalsplot2d2.eps}

何やっているか分かります? 中心の赤点で示される円が,青点で示される楕円へ写されていることが分かるでしょうか.

線形代数の講義で,写像を示すときによく使われるポンチ絵を現実の空間で示すとこのようになります.ポンチ絵では,赤で示した$V$空間が青で示した$W$空間へ行列$A$によって写像され,それぞれの要素$v$が$w$へ移されると意図しています.
\MaplePlot{60mm}{./figures/Projection3-4.eps}

\section{固有ベクトルの幾何学的意味}
$A$を対称正方行列,$x$をベクトルとしたときに,
\begin{equation}
Ax = \lambda x
\label{Eq:Eigen}
\end{equation}
の解,$\lambda$を固有値,$x$を固有ベクトルという.$x$がゼロベクトルではない意味のある解は特性方程式det$(A-\lambda E)=0$が成り立つときにのみ得られる.

まずMapleで特性方程式を解いてみる.

\begin{MapleInput}
> restart;
> with(LinearAlgebra):with(plots):with(plottools):
> A:=Matrix(1..2,1..2,[[3,2/3],[2/3,2]]);
\end{MapleInput}

\begin{MapleOutput}
A\, := \, \left[ \begin {array}{cc} 3&2/3\\2/3&2\end {array} \right]
\end{MapleOutput}

\begin{MapleInput}
> EE:=Matrix([[1,0],[0,1]]):
  A-lambda.EE;
\end{MapleInput}

\ifHIKI
\begin{tabular}{|c|}
\hline
         \\ 
\hline
\end{tabular}
\else
\begin{equation*}
\setlength{\unitlength}{1cm}
\begin{picture}(10,3.5)
\put(0,0){\framebox(10,3.5){}}
\end{picture}
\end{equation*}
\fi

\begin{MapleInput}
> eq2:=Determinant(A-lambda.EE);
\end{MapleInput}

\ifHIKI
\begin{tabular}{|c|}
\hline
         \\ 
\hline
\end{tabular}
\else
\begin{equation*}
\setlength{\unitlength}{1cm}
\begin{picture}(10,3.5)
\put(0,0){\framebox(10,3.5){}}
\end{picture}
\end{equation*}
\fi

\begin{MapleOutput}
{\it eq2}\, := \,{\frac {50}{9}}-5\,\lambda+{\lambda}^{2}
\end{MapleOutput}

\begin{MapleInput}
> solve(eq2=0,lambda);
\end{MapleInput}
\begin{MapleOutput}
10/3,\,5/3
\end{MapleOutput}

固有値を求めるコマンドEigenvectorsを適用すると,固有値と固有ベクトルが求まる.ここで,固有ベクトルは行列の列(Column)ベクトルに入っている.
\begin{MapleInput}
> lambda,V:=Eigenvectors(A);
\end{MapleInput}
\begin{MapleOutput}
\lambda,\,V\, := \, \left[ \begin {array}{c} 10/3\\5/3\end {array} \right] ,\, \left[ \begin {array}{cc} 2&-1/2\\1&1\end {array} \right] 
\end{MapleOutput}
得られた固有ベクトルは規格化されているわけではない.

行列の列を取り出すコマンドColumnを用いて,方程式(\ref{Eq:Eigen})が成り立っていることを確認する.
\begin{MapleInput}
> lambda[1].Column(V,1)=A.Column(V,1);
\end{MapleInput}
\begin{MapleOutput}
\displaystyle
\left[ \begin {array}{c} 20/3\\
10/3\end {array} \right] = 
\left[ \begin {array}{c} 20/3\\
10/3\end {array} \right] 
\end{MapleOutput}
一般的な規格化は,コマンドNormalize(vector,Euclidean)によっておこなう.
\begin{MapleInput}
> Normalize(Column(v,1),Euclidean);
\end{MapleInput}


\section{行列式の幾何学的意味}
\input{Determinant.tex}
\section{行列式が0の写像}
では,行列式が0になるというのはどういう状態でしょう? 次のような行列を考えてみましょう.
\begin{MapleInput}
> A:=Matrix([[2,1],[4,2]]);
\end{MapleInput}
\begin{MapleOutput}
\end{MapleOutput}
この行列式は
\begin{MapleInput}
> Determinant(A);
\end{MapleInput}
\begin{MapleOutput}
0
\end{MapleOutput}
です.この変換行列で,上と同じように写像の様子を表示させてみましょう.
\begin{MapleInput}
> N:=30:point2:=[]:line2:=[]: 
  for k from 0 to N-1 do
    x0:=Vector([sin(2*Pi*k/N),cos(2*Pi*k/N)]); x1:=A.x0; p0:=convert(x0,list);
    p1:=convert(x1,list);
    point2:=[op(point2),disk(p0,0.05,color=red),disk(p1,0.05,color=blue)];
    line2:=[op(line2),line(p0,p1)]; 
  end:
> d:=6: display(point2,line2,view=[-d..d,-d..d]);
\end{MapleInput}
\MaplePlot{80mm}{./figures/LAFundamentalsplot2d5.eps}

わかります?

今回の移動先の青点は直線となっています.つまり,determinantが0ということは,変換すると面積がつぶれるという事を意味しています.平面がひとつ次元を落として線になるということです.

次に,この行列の表わす写像によって原点(0,0)に写される元の座標を求めてみます.連立方程式に戻してみると
\begin{MapleInput}
> A.Vector([x,y])=Vector([0,0]);
\end{MapleInput}
\begin{MapleOutput}
\left[ \begin {array}{c} 2\,x+y\\ 4\,x+2\,y\end {array} \right] = \left[ \begin {array}{c} 0\\ 0\end {array} \right]
\end{MapleOutput}
となります.とよく見ると,1行目も2行目もおなじ式になっています.2次元正方行列で,行列式が0の時には必ずこういう形になり,直線の式となります.これを表示すると
\begin{MapleInput}
> plot([-2*x,-2*x+1,-2*x-1],x=-4..4,y=-4..4);
> plot([2*x],x=-4..4,y=-4..4,color=blue);
\end{MapleInput}
\MaplePlot{100mm}{./figures/LAFundamentalsKernelImage.eps}
左図の赤線となります.この直線上の全ての点が[0,0]へ写されることを確認してください.また,緑の線上の点は全て[1,2]へ写されることが確認できます.
\begin{MapleInput}
> A.Vector([-1,2]);
\end{MapleInput}
\begin{MapleOutput}
\left[ \begin {array}{c} 0\\ 0\end {array} \right]
\end{MapleOutput}

こうしてすべて調べていけば,左の平面上のすべて点は右の青の直線上へ写されることが分かります.今まで見てきた円と楕円とはまったく違った写像が,行列式が0の行列では起こっていることが分かると思います.右の青線を行列Aによる像(Image, Im$A$と表記),左の赤線,つまり写像によって[0,0]へ写される集合を核(Kernel, Ker$A$と表記)と呼びます.

これをポンチ絵で描くと,次の通りです.
\ifHIKI

||像(Image) || 核(Kernel) 
||{{attach_view(Projection1.png,LAFundamentals)}}||{{attach_view(Projection2.png,LAFundamentals)}}

\else

\begin{center}
\begin{tabularx}{100mm}{|X|X|}
\hline
像(Image) & 核(Kernel) \\
\MaplePlot{40mm}{./figures/Projection1.eps}&
\MaplePlot{40mm}{./figures/Projection2.eps}\\
\hline
\end{tabularx}
\end{center}

\fi
\section{全単射}
行列$A$による写像を$f$として,赤点に限らず元の点の集合を$V$, 移った先の点の集合を$W$とすると,
\begin{equation*}
f: V \rightarrow W
\end{equation*}
と表記されます.$v,w$を$V,W$の要素としたとき,異なる$v$が異なる$w$に写されることを単射,全ての$w$に対応する$v$がある写像を全射と言います.全単射,つまり全射でかつ単射,だと要素は一対一に対応します.先ほどのAは全射でもなく,単射でもない例です.

行列式が0の場合の写像は単射ではありません.このとき,逆写像が作れそうにありません.これを連立方程式に戻して考えましょう.もともと,
\begin{equation*}
v = A^{-1} w
\end{equation*}
の解$v$は点$w$が写像$A$によってどこから写されてきたかという意味を持ちます.逆写像が作れない場合は,連立方程式の解はパラメータをひとつ持った複数の解(直線)となります.これが係数行列の行列式が0の場合に,連立方程式の解が不定となる,あるいは像がつぶれるという関係です.

行列の次元が高い場合には,いろいろなつぶれかたをします.行列の階数と次元は
\begin{MapleInput}
> Rank(A); 
  Dimension(A);
\end{MapleInput}
\begin{MapleOutputGather}
1 \notag \\
2, 2 \notag
\end{MapleOutputGather}
で求まります.

Aをm行n列の行列とするとき,
\ifHIKI
""Rank(''A'') = Dimension (Im ''A'')

""Dimension (Ker ''A'') = ''n'' - Rank(''A'') 

\else

\begin{center}
Rank({\it A}) = Dimension (Im {\it A}) \\
Dimension ({\rm Ker} {\it A}) = {\it n} - Rank({\it A}) 
\end{center}

\fi
が成立し,これを次元定理といいます.
全射と単射の関係は,下の表のような一変数の方程式での解の性質の拡張と捉えることができます.

\begin{table}[htbp]
\caption{代数方程式$a x =b$の解の存在性.}
\begin{center}
\begin{tabular}{|c|l|l|}
\hline
一意&$a<>0$ &$x=b/a$ \\
不定&$a=0, b=0$ &解は無数\\ 
不能&$a=0, b<>0$ &解は存在しない\\
\hline
\end{tabular}
\end{center}
\label{default}
\end{table}%

\ifHIKI

||m x n行列A||全射でない(Im $A < m$), 値域上にあるときのみ解が存在||全射(Im $A =m$), 解は必ず存在
||単射でない(Ker $A <> 0$), 解は複数 ||{{attach_view(Projection3-1.png,LAFundamentals)}}||{{attach_view(Projection3-2.png,LAFundamentals)}}
||単射(Ker $A = 0$), 解はひとつ||{{attach_view(Projection3-3.png,LAFundamentals)}}||{{attach_view(Projection3-4.png,LAFundamentals)}}

\else

\begin{table}[htbp]
\caption{連立方程式$A x =b$の解の存在性.}
\begin{center}
\begin{tabularx}{150mm}{|X|X|X|}
\hline
$m \times n$行列$A$ &全射でない(Im $A < m$), 値域上にあるときのみ解が存在 &全射(Im$ A =m$),解は必ず存在 \\
\hline
単射でない(Ker $A <> 0$), 解は複数 & 
\MaplePlot{40mm}{./figures/Projection3-1.eps}&
\MaplePlot{40mm}{./figures/Projection3-2.eps}\\
\hline
単射(Ker $A = 0$), 解はひとつ &
\MaplePlot{40mm}{./figures/Projection3-3.eps} &
\MaplePlot{40mm}{./figures/Projection3-4.eps}\\
\hline
\end{tabularx}
\end{center}
\label{default}
\end{table}%

\fi


\section{課題}
\begin{enumerate}
\item 後退代入法で解を求めよ.(2005年度期末類題) 
\begin{equation*}
\begin{array}{rl}
x+4y-3z &= 1 \\
-6y+4z &= 1\\
-\frac{5}{3}z &=  \frac{1}{3}
\end{array}
\end{equation*}

\item 次の行列AをLU分解せよ.
\begin{MapleInput}
> A:=Matrix([[1,4,3],[1,-2,1],[2,-2,-1]]);
\end{MapleInput}
\begin{MapleOutput}
\left[ \begin {array}{ccc} 1&4&3\\ 1&-2&1\\ 2&-2&-1\end {array} \right]
\end{MapleOutput}

\item 次の連立方程式の係数行列をLU分解し,上・下三角行列を求めよ.さらに連立方程式の解を求めよ.(2005年度期末試験) 
\begin{equation*}
\left[ \begin {array}{c} x_{{1}}+3\,x_{{2}}+4\,x_{{3}}+3\,x_{{4}}\\ -2\,x_{{1}}+5\,x_{{2}}+3\,x_{{3}}-3\,x_{{4}}\\ x_{{1}}+3\,x_{{2}}-2\,x_{{3}}+3\,x_{{4}}\\ 3\,x_{{1}}-2\,x_{{2}}+x_{{3}}+4\,x_{{4}}\end {array} \right] = \left[ \begin {array}{c} 1\\ 4\\ -2\\ 3\end {array} \right]
\end{equation*}

\item Jacobi法のプログラムを参照してGauss-Seidel法のプログラムを作れ.Jacobi法と収束性を比べよ.

\item 次の連立方程式の解を求めよ.ただし,pivot操作が必要となる.
\begin{MapleInput}
> with(LinearAlgebra): 
  A:=Matrix([[3,2,2,1],[3,2,3,1],[1,-2,-3,1],[5,3,-2,5]]):
  X:=Vector([w,x,y,z]): 
  b:=Vector([-6,2,-9,2]): 
  A.X=b;
\end{MapleInput}
\begin{MapleOutput}
\left[ \begin {array}{c} 3\,w+2\,x+2\,y+z\\ 3\,w+2\,x+3\,y+z\\ w-2\,x-3\,y+z\\ 5\,w+3\,x-2\,y+5\,z\end {array} \right] = \left[ \begin {array}{c} -6\\ 2\\ -9\\ 2\end {array} \right]
\end{MapleOutput}
\item (おまけ)pivot操作を含めたLU分解のプログラムを作成せよ.上の問題を解き,そのL, U行列および$L^{-1}.b$ベクトルを求めよ.
\end{enumerate}


\end{document}
