\subsection{二分法(bisection)}
\begin{MapleInput}
> x1:=0: x2:=0.8: 
  f1:=func(x1): f2:=func(x2): 
  for i from 1 to 5 do 
    x:=(x1+x2)/2;
    f:=func(x); 
    if f*f1>=0.0 then 
      x1:=x; f1:=f; 
    else 
      x2:=x; f2:=f; 
    end if;
    printf("%20.15f, %20.15f\n",x,f); 
  end do:
\end{MapleInput}
\begin{MapleError}
0.400000000000000, -0.440000000000000 
0.200000000000000,  0.240000000000000
0.300000000000000, -0.110000000000000 
0.250000000000000,  0.062500000000000
0.275000000000000, -0.024375000000000
\end{MapleError}

\subsection{Newton法(あるいはNewton-Raphson法)}

\begin{MapleInput}
> dfunc:=unapply(diff(func(z),z),z);
\end{MapleInput}
\begin{MapleOutput}
{\it dfunc} := z\mapsto 2 z - 4
\end{MapleOutput}
\begin{MapleInput}
> x:=1: f:=func(x): 
  printf("%15.10f, %+24.25f\n",x,f); 
  for i from 1 to 5 do
    x:=x-f/dfunc(x); 
    f:=func(x); 
    printf("%15.10f, %+24.25f\n",x,f); 
  end do:
\end{MapleInput}
\begin{MapleError}
1.0000000000, -2.0000000000000000000000000 
0.0000000000, +1.0000000000000000000000000 
0.2500000000, +0.0625000000000000000000000
0.2678571429, +0.0003188775510000000000000 
0.2679491900, +0.0000000084726737970000000 
0.2679491924, +0.0000000000000000059821834
\end{MapleError}

以下のようにDigitsを変更すれば,Mapleでは浮動小数点演算の有効数字を変えることができる.
\begin{MapleInput}
> Digits:=40;
\end{MapleInput}
\begin{MapleOutput}
40
\end{MapleOutput}

