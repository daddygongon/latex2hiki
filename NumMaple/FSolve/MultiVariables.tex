2変数の関数では,解を求める一般的な手法は無い.この様子は実際に2変数の関数で構成される面の様子をみれば納得されよう.
\begin{MapleInput}
> restart;
> f:=(x,y)->4*x+2*y-6*x*y; g:=(x,y)->10*x-2*y+1;
\end{MapleInput}
\begin{MapleOutputGather}
f\, := \,( {x,y} )\mapsto 4\,x+2\,y-6\,xy \notag \\
g\, := \,( {x,y} )\mapsto 10\,x-2\,y+1 \notag
\end{MapleOutputGather}
\begin{MapleInput}
> p1:=plot3d({f(x,y)},x=-2..2,y=-2..2,color=red):
  p2:=plot3d({g(x,y)},x=-2..2,y=-2..2,color=blue):
  p3:=plot3d({0},x=-2..2,y=-2..2,color=gray):
  with(plots):
  display([p1,p2,p3],axes=boxed,orientation=[-150,70]);
\end{MapleInput}
\MaplePlot{60mm}{./figures/C2_fsolveplot3d5.eps}

解のある程度近くからは,Newton法で効率良く求められる.
\begin{MapleInput}
> fsolve({f(x,y)=0,g(x,y)=0},{x,y});
\end{MapleInput}
\begin{MapleOutput}
\left\{ x=- 0.07540291160,y= 0.1229854420 \right\}
\end{MapleOutput}
