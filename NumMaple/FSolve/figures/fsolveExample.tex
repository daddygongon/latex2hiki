%% Created by Maple 15.01, Mac OS X
%% Source Worksheet: fsolveExample.mw
%% Generated: Sun Aug 05 10:47:34 JST 2012
\documentclass{article}
\usepackage{maplestd2e}
\def\emptyline{\vspace{12pt}}
\begin{document}
\pagestyle{empty}
\DefineParaStyle{Maple Heading 1}
\DefineParaStyle{Maple Text Output}
\DefineParaStyle{Maple Dash Item}
\DefineParaStyle{Maple Bullet Item}
\DefineParaStyle{Maple Normal}
\DefineParaStyle{Maple Heading 4}
\DefineParaStyle{Maple Heading 3}
\DefineParaStyle{Maple Heading 2}
\DefineParaStyle{Maple Warning}
\DefineParaStyle{Maple Title}
\DefineParaStyle{Maple Error}
\DefineCharStyle{Maple Hyperlink}
\DefineCharStyle{Maple 2D Math}
\DefineCharStyle{Maple Maple Input}
\DefineCharStyle{Maple 2D Output}
\DefineCharStyle{Maple 2D Input}
\begin{maplegroup}
\begin{mapleinput}
\mapleinline{active}{1d}{restart;
Digits:=40;
func:=unapply(exp(-x)-x\symbol{94}2,x);
}{}
\end{mapleinput}
\mapleresult
\begin{maplelatex}
\mapleinline{inert}{2d}{Digits := 40}{\[\displaystyle \]}
\end{maplelatex}
\mapleresult
\begin{maplelatex}
\mapleinline{inert}{2d}{func := proc (x) options operator, arrow; exp(-x)-x^2 end proc}{\[\displaystyle {\it func}\, := \,x\mapsto {{\rm e}^{-x}}-{x}^{2}\]}
\end{maplelatex}
\end{maplegroup}
\begin{maplegroup}
\begin{mapleinput}
\mapleinline{active}{1d}{plot(func(x),x=-5..5,y=-10..10);
}{}
\end{mapleinput}
\mapleresult
\mapleplot{fsolveExampleplot2d1.eps}
\end{maplegroup}
\begin{maplegroup}
\begin{mapleinput}
\mapleinline{active}{1d}{x0:=fsolve(func(x)=0,x);
}{}
\end{mapleinput}
\mapleresult
\begin{maplelatex}
\mapleinline{inert}{2d}{x0 := .7034674224983916520498186018599021303429}{\[\displaystyle {\it x0}\, := \, 0.7034674224983916520498186018599021303429\]}
\end{maplelatex}
\end{maplegroup}
\begin{maplegroup}
\begin{mapleinput}
\mapleinline{active}{1d}{x1:=0: x2:=0.8: 
res1:=[]:
  f1:=func(x1): f2:=func(x2): 
  for i from 1 to 20 do 
    x:=(x1+x2)/2;
    f:=func(x); 
    if f*f1>=0.0 then 
      x1:=x; f1:=f; 
    else 
      x2:=x; f2:=f; 
    end if;
    printf("%20.15f, %20.15f\symbol{92}n",x,f); 
    res1:=[op(res1),[i,abs(x-x0)]]:
  end do:
}{}
\end{mapleinput}
\mapleresult
0.400000000000000,    0.5103200460356390.600000000000000,    0.1888116360940260.700000000000000,    0.0065853037914100.750000000000000,   -0.0901334472589850.725000000000000,   -0.0413004310446380.712500000000000,   -0.0172396279240940.706250000000000,   -0.0052977381001470.703125000000000,    0.0006511313011990.704687500000000,   -0.0023214653417450.703906250000000,   -0.0008347076237050.703515625000000,   -0.0000916733268540.703320312500000,    0.0002797576939280.703417968750000,    0.0000940493604560.703466796875000,    0.0000011898110600.703491210937500,   -0.0000452413093290.703479003906250,   -0.0000220256369930.703472900390625,   -0.0000104178849310.703469848632812,   -0.0000046140299270.703468322753906,   -0.0000017121076810.703467559814453,   -0.000000261147873
\end{maplegroup}
\begin{maplegroup}
\begin{mapleinput}
\mapleinline{active}{1d}{with(plots):
l1:=logplot(res1);
}{}
\end{mapleinput}
\mapleresult
\begin{maplelatex}
\mapleinline{inert}{2d}{l1 := PLOT(`...`)}{\[\displaystyle {\it l1}\, := \,{\it PLOT} \left( \mbox {{\tt `...`}} \right) \]}
\end{maplelatex}
\end{maplegroup}
\begin{maplegroup}
\begin{mapleinput}
\mapleinline{active}{1d}{dfunc:=unapply(diff(func(z),z),z);
}{}
\end{mapleinput}
\mapleresult
\begin{maplelatex}
\mapleinline{inert}{2d}{dfunc := proc (z) options operator, arrow; -exp(-z)-2*z end proc}{\[\displaystyle {\it dfunc}\, := \,z\mapsto -{{\rm e}^{-z}}-2\,z\]}
\end{maplelatex}
\end{maplegroup}
\begin{maplegroup}
\begin{mapleinput}
\mapleinline{active}{1d}{x0;
}{}
\end{mapleinput}
\mapleresult
\begin{maplelatex}
\mapleinline{inert}{2d}{.7034674224983916520498186018599021303429}{\[\displaystyle  0.7034674224983916520498186018599021303429\]}
\end{maplelatex}
\end{maplegroup}
\begin{maplegroup}
\begin{mapleinput}
\mapleinline{active}{1d}{x:=1.0: f:=func(x):
printf("%15.10f, %+24.25f\symbol{92}n",x,f); 
res2:=[[1,abs(x-x0)]]:
for i from 2 to 5 do
    x:=x-f/dfunc(x); 
    f:=func(x); 
    printf("%15.10f, %+24.25f\symbol{92}n",x,f); 
    res2:=[op(res2),[i,abs(x-x0)]]:
  end do:
}{}
\end{mapleinput}
\mapleresult
1.0000000000, -0.63212055882855767840447620.7330436052, -0.05690844800402540746845760.7038077863, -0.00064739153874650147619730.7034674683, -0.00000008716603056242310970.7034674225, -0.0000000000000015809178420
\end{maplegroup}
\begin{maplegroup}
\begin{mapleinput}
\mapleinline{active}{1d}{l2:=logplot(res2);}{}
\end{mapleinput}
\mapleresult
\begin{maplelatex}
\mapleinline{inert}{2d}{l2 := PLOT(`...`)}{\[\displaystyle {\it l2}\, := \,{\it PLOT} \left( \mbox {{\tt `...`}} \right) \]}
\end{maplelatex}
\end{maplegroup}
\begin{maplegroup}
\begin{mapleinput}
\mapleinline{active}{1d}{display(l1,l2);
}{}
\end{mapleinput}
\mapleresult
\mapleplot{fsolveExampleplot2d2.eps}
\end{maplegroup}
\begin{maplegroup}
\begin{mapleinput}
\mapleinline{active}{1d}{res1[1];
}{}
\end{mapleinput}
\mapleresult
\begin{maplelatex}
\mapleinline{inert}{2d}{[1, .3034674224983916520498186018599021303429]}{\[\displaystyle [1, 0.3034674224983916520498186018599021303429\\
\mbox{}]\]}
\end{maplelatex}
\end{maplegroup}
\begin{maplegroup}
\begin{mapleinput}
\mapleinline{active}{1d}{}{}
\end{mapleinput}
\end{maplegroup}
\end{document}
