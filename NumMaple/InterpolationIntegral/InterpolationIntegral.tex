\documentclass[10pt,a4j]{jreport}
\usepackage[dvips]{graphicx,color}
\usepackage{verbatim}
\usepackage{amsmath,amsthm,amssymb}
\topmargin -15mm\oddsidemargin -4mm\evensidemargin\oddsidemargin
\textwidth 170mm\textheight 257mm\columnsep 7mm
\setlength{\fboxrule}{0.2ex}
\setlength{\fboxsep}{0.6ex}

\pagestyle{empty}

\newcommand{\MaplePlot}[2]{{\begin{center}
    \includegraphics[width=#1,clip]{#2}
                     \end{center}
%
} }

\newenvironment{MapleInput}{%
    \color{red}\verbatim
}{%
    \endverbatim
}

\newenvironment{MapleError}{%
    \color{blue}\verbatim
}{%
    \endverbatim
}

\newenvironment{MapleOutput}{%
    \color{blue}\begin{equation*}
}{%
    \end{equation*}
}

\newenvironment{MapleOutputGather}{%
    \color{blue}\gather
}{%
    \endgather
}
\newif\ifHIKI
%\HIKItrue % TRUEの設定
\HIKIfalse % FALSEの設定
\begin{document}
\chapter{補間(interpolation)と数値積分(Integral)}
\section{概要:補間と近似}
単純な2次元データについて補間と近似を考える.補間はたんに点をつなぐことを,近似はある関数にできるだけ近くなるようにフィットすることを言う.補間はIllustratorなどのドロー系ツールで曲線を引くときの,ベジエやスプライン補間の基本となる.本章では補間とそれに密接に関連した積分について述べる.


\ifHIKI
||補間と近似の模式図.
||

\else

\begin{table}[htbp]
\caption{補間と近似の模式図.}
\begin{center}
\begin{tabular}{cc}
補間&近似 \\
\setlength{\unitlength}{1cm}
\begin{picture}(6,6.5)
\put(0,0){\framebox(6,6.5){}}
\end{picture}
&
\setlength{\unitlength}{1cm}
\begin{picture}(6,6.5)
\put(0,0){\framebox(6,6.5){}}
\end{picture}
\end{tabular}
\end{center}
\label{default}
\end{table}%
\fi

\section{多項式補間(polynomial interpolation)}
データを単純に多項式で補間する方法を先ず示そう.$N+1$点をN次の多項式でつなぐ.この場合の補間関数は,
\begin{equation*}
F \left(x \right)={\sum_{i=0}^{N} } a _{i }x ^{i }=a_{0}+a_{1}x +a_{2}x^{2}+\cdots +a_{N}x^{N}
\end{equation*}
である.データの点を$(x_{i},\,y_{i}),i=0..N$とすると
\begin{equation*}
\begin{array}{cl}
a _{0}+a _{1}x _{0}+a _{2}x _{0}^{2}+\cdots +a _{N }x _{0}^{N }& =y _{0}\\
a _{0}+a _{1}x _{1}+a _{2}x _{1}^{2}+\cdots +a _{N }x _{1}^{N }& =y _{1}\\
\vdots& \\
a _{0}+a _{1}x _{N}+a _{2}x _{N}^{2}+\cdots +a _{N }x _{N}^{N }& =y _{N}
\end{array}
\end{equation*}
が,係数 $a_i$を未知数と見なした線形の連立方程式となっている.係数行列は
\begin{equation*}
A=\left[
\begin{array}{ccccc}
1&x_0&x_0^2&\cdots&x_0^N \\
1&x_1&x_1^2&\cdots&x_1^N \\
\vdots& & & & \vdots \\
1&x_N&x_N^2&\cdots&x_N^N 
\end{array} \right]
\end{equation*}
となる.$a_i$と$y_i$をそれぞれベクトルとみなすと
\ifHIKI %%%%
||               ||
\else %%%%
\begin{equation*}
\setlength{\unitlength}{1cm}
\begin{picture}(10,3.5)
\put(0,0){\framebox(10,3.5){}}
\end{picture}
\end{equation*}
\fi %%%%
により未知数ベクトル$a_i$が求まる.これは単純に,前に紹介した Gauss の消去法や LU 分解で解ける.

\subsection{Mapleによる多項式補間の実例}
\begin{MapleInput}
> restart; X:=[0,1,2,3]: Y:=[1,2,3,-2]:
> with(LinearAlgebra):
> list1:=[X,Y];
\end{MapleInput}
\begin{MapleOutput}
{\it list1}\, := \,[[0,1,2,3],[1,2,3,-2]]
\end{MapleOutput}
\begin{MapleInput}
> with(plots):
  l1p:=listplot(Transpose(Matrix(list1))):
  display(l1p);
\end{MapleInput}
\MaplePlot{50mm}{./figures/C7_InterpolationIntegralplot2d1.eps}

\begin{MapleInput}
> A:=Matrix(4,4): 
  for i from 1 to 4 do 
    for j from 1 to 4 do 
      A[i,j]:=X[i]^(j-1);
    end do; 
  end do:
  A;
\end{MapleInput}
\begin{MapleOutput}
\left[ \begin {array}{cccc} 1&0&0&0\\ 1&1&1&1\\ 1&2&4&8\\ 1&3&9&27\end {array} \right]
\end{MapleOutput}

\begin{MapleInput}
> a1:=MatrixInverse(A).Vector(Y);
\end{MapleInput}
\begin{MapleOutput}
{\it a1}\, := \, \left[ \begin {array}{c} 1\\ -1\\ 3\\ -1\end {array} \right]
\end{MapleOutput}
\begin{MapleInput}
> f1:=unapply(add(a1[ii]*x^(ii-1),ii=1..4),x);
\end{MapleInput}
\begin{MapleOutput}
{\it f1}\, := \,x\mapsto 1-x+3\,{x}^{2}-{x}^{3}
\end{MapleOutput}
\begin{MapleInput}
> f1p:=plot(f1(x),x=0..3): 
  l1p:=listplot(Transpose(Matrix(list1))):
  display(f1p,l1p);
\end{MapleInput}
\MaplePlot{50mm}{./figures/C7_InterpolationIntegralplot2d2.eps}

\section{Lagrange(ラグランジュ) の内挿公式}
多項式補間は手続きが簡単であるため,計算間違いが少なく,プログラムとして組むのに適している.しかし,あまり"みとうし"のよい方法とはいえない.その点,Lagrange(ラグランジュ)の内挿公式は見通しがよい.これは
\begin{equation*}
F(x)= \sum_{k=0}^N \frac{\displaystyle \prod_{j \ne k} (x-x_j)}
{\displaystyle \prod_{j \ne k} (x_k-x_j)} y_k
=\sum_{k=0}^N \frac{ \frac{\displaystyle (x-x_0)(x-x_1)\cdots(x-x_N)}{\displaystyle (x-x_k)}}
{\frac{\displaystyle (x_k-x_0)(x_k-x_1)\cdots(x_k-x_N)}{\displaystyle (x_k-x_k)}} y_k
\end{equation*}
と表わされる.数学的に 2つ目の表記は間違っているが,先に割り算を実行すると読み取って欲しい.これは一見複雑に見えるが,単純な発想から出発している.求めたい関数$F(x)$を
\begin{equation*}
F(x) = y_0 L_0(x)+y_1 L_1(x)+y_2 L_2(x)
\end{equation*}
とすると
\begin{equation*}
\begin{array}{ccc}
 L_0(x_0) = 1 &L_0(x_1) = 0 &L_0(x_2) = 0 \\
 L_1(x_0) = 0 &L_1(x_1) = 1 &L_1(x_2) = 0 \\
 L_2(x_0) = 0 &L_2(x_1) = 0 &L_2(x_2) = 1 
\end{array}
\end{equation*}
となるように関数$L_i(x)$を決めればよい.これを以下のようにとればLagrangeの内挿公式となる.
  
\ifHIKI %%%%
||               ||
\else %%%%
\begin{equation*}
\setlength{\unitlength}{1cm}
\begin{picture}(10,10)
\put(0,0){\framebox(10,10){}}
\end{picture}
\end{equation*}
\fi %%%%

\section{Newton(ニュートン) の差分商公式}
もう一つ有名なNewton(ニュートン) の内挿公式は,
\begin{equation*}
\begin{array}{rc}
F (x )&=F (x _{0})+
(x -x _{0})f _{1}\lfloor x_0,x_1\rfloor+
(x -x _{0})(x -x _{1})
f _{2}\lfloor x_0,x_1,x_2\rfloor + \\
& \cdots + \displaystyle \prod_{i=0}^{n-1} (x-x_i) \, 
f_n \lfloor x_0,x_1,\cdots,x_n \rfloor
\end{array}
\end{equation*}
となる.ここで$f_i \lfloor\, \rfloor$ は次のような関数を意味していて,
\begin{equation*}
\begin{array}{rl}
f _{1}\lfloor x_0,x_1\rfloor &= \displaystyle \frac{y_1-y_0}{x_1-x_0} \\
f _{2}\lfloor x_0,x_1,x_2\rfloor &= \displaystyle \frac{f _{1}\lfloor x_1,x_2\rfloor-
f _{1}\lfloor x_0,x_1\rfloor}{x_2-x_0} \\
\vdots & \\
f _{n}\lfloor x_0,x_1,\cdots,x_n\rfloor &= \displaystyle \frac{f_{n-1}\lfloor x_1,x_2\cdots,x_{n}\rfloor-
f _{n-1}\lfloor x_0,x_1,\cdots,x_{n-1}\rfloor}{x_n-x_0} \\
\end{array}
\end{equation*}
差分商と呼ばれる.得られた多項式は,Lagrange の内挿公式で得られたものと当然一致する.Newtonの内挿公式の利点は,新たなデータ点が増えたときに,新たな項を加えるだけで,内挿式が得られる点である.

\subsection{Newton補間と多項式補間の一致の検証}
関数$F(x)$を$x$の多項式として展開.その時の,係数の取るべき値と,差分商で得られる値が一致.
\begin{MapleInput}
> restart: F:=x->f0+(x-x0)*f1p+(x-x0)*(x-x1)*f2p;
\end{MapleInput}
\begin{MapleOutput}
F\, := \,x\mapsto f0+ ( x-x0 ) f1p+ ( x-x0 )  ( x-x1 ) f2p
\end{MapleOutput}
\begin{MapleInput}
> F(x1); 
  sf1p:=solve(F(x1)=f1,f1p);
\end{MapleInput}
\begin{MapleOutputGather}
f0+ ( x1-x0 ) f1p \notag \\
\displaystyle sf1p\, := \,{\frac {f0-f1}{-x1+x0}} \notag
\end{MapleOutputGather}
f20の取るべき値の導出
\begin{MapleInput}
> sf2p:=solve(F(x2)=f2,f2p); 
  fac_f2p:=factor(subs(f1p=sf1p,sf2p));
\end{MapleInput}
\begin{MapleOutputGather}
sf2p\, := \displaystyle -{\frac {f0+f1p\,x2-f1p\,x0-f2}{ ( -x2+x0 ) 
 ( -x2+x1 ) }} \notag \\
\displaystyle {\it fac\_f2p}\, := \displaystyle {\frac {f0\,x1-x2\,f0+x2\,f1-x0\,f1-f2\,x1+f2\,x0}{ ( -x1+x0 )  ( -x2+x0 )  ( -x2+x1 ) }} \notag
\end{MapleOutputGather}
ニュートンの差分商公式を変形
\begin{MapleInput}
> ff11:=(f0-f1)/(x0-x1); 
  ff12:=(f1-f2)/(x1-x2); 
  ff2:=(ff11-ff12)/(x0-x2);
  fac_newton:=factor(ff2);
\end{MapleInput}
\begin{MapleOutputGather}
ff11:= {\displaystyle \frac {f0-f1}{-x1+x0}} \notag \\
ff12 := {\displaystyle \frac {f1-f2}{-x2+x1}} \notag \\
ff2 :=  \frac{ {\displaystyle \frac {f0-f1}{-x1+x0}}-{\displaystyle \frac {f1-f2}{-x2+x1}} }{-x2+x0 }\notag \\
{\it fac\_newton} := {\displaystyle \frac {f0\,x1-x2\,f0+x2\,f1-x0\,f1-f2\,x1+f2\,x0}{ ( -x1+x0 )  ( -x2+x0 )  ( -x2+x1 ) }} \notag 
\end{MapleOutputGather}

二式が等しいかどうかをevalbで判定
\begin{MapleInput}
> evalb(fac_f2p=fac_newton);
\end{MapleInput}
\begin{MapleOutput}
true
\end{MapleOutput}

\section{数値積分 (Numerical integration)}
\documentclass[10pt,a4j]{jreport}
\usepackage[dvips]{graphicx,color}
\usepackage{verbatim}
\usepackage{amsmath,amsthm,amssymb}
\topmargin -15mm\oddsidemargin -4mm\evensidemargin\oddsidemargin
\textwidth 170mm\textheight 257mm\columnsep 7mm
\setlength{\fboxrule}{0.2ex}
\setlength{\fboxsep}{0.6ex}

\pagestyle{empty}

\newcommand{\MaplePlot}[2]{{\begin{center}
    \includegraphics[width=#1,clip]{#2}
                     \end{center}
%
} }

\newenvironment{MapleInput}{%
    \color{red}\verbatim
}{%
    \endverbatim
}

\newenvironment{MapleError}{%
    \color{blue}\verbatim
}{%
    \endverbatim
}

\newenvironment{MapleOutput}{%
    \color{blue}\begin{equation*}
}{%
    \end{equation*}
}

\newenvironment{MapleOutputGather}{%
    \color{blue}\gather
}{%
    \endgather
}
\newcommand{\ChartElement}[1]{{
	\color{magenta}\begin{flushleft}$\left[\left[\left[\textbf{\large #1}\right]\right]\right]$
	\end{flushleft}\vspace{-10mm}
} }

\newcommand{\ChartElementTwo}[1]{{
	\color{magenta}\begin{flushleft}$\left[\left[\left[\textbf{\large #1}\right]\right]\right]$
	\end{flushleft}
} }
\begin{document}
\pagebreak
\section{積分(int)}
\ChartElement{解説}
\subsection{単純な積分(int)}
単純な不定積分.
\begin{MapleInput}
> int(ln(x),x);	#res: x ln(x) - x
\end{MapleInput}
定積分を実行するには,積分変数の範囲を指定する.
\begin{MapleInput}
> int(sin(x),x=-Pi..0);	#res: -2
\end{MapleInput}
特異点をもつ場合にも適切に積分結果を求めてくれる.
\begin{MapleInput}
> int(1/sqrt(x*(2-x)),x=0..2); #res: pi
\end{MapleInput}
無限区間(infinity)における定積分も同様に計算してくれる.
\begin{MapleInput}
> int(1/(x^2+4),x=-infinity..infinity); #res: 1/2 pi
\end{MapleInput}
部分積分法や置換積分法を用いる必要のある複雑な積分も一発で求まる.
\begin{MapleInput}
> eq:=sqrt(4-x^2);int(eq,x);
\end{MapleInput}
\begin{MapleOutputGather}
 {\it eq}\, := \, \sqrt{4-{x}^{2}} \notag \\
 \frac{1}{2}\,x \sqrt{4-{x}^{2}}+2\,\arcsin \left( 1/2\,x \right) \notag
\end{MapleOutputGather}
数学の公式集に載っているような積分も同じコマンドで求まる.
\begin{MapleInput}
> eq2:=exp(-x^2);int(eq2,x=0..zz);
\end{MapleInput}
\begin{MapleOutputGather}
{\it eq2}\, := \,\exp({-{x}^{2}}) \notag \\
\frac{1}{2}\, \sqrt{\pi }\, \mbox{erf} \left(zz\right) \notag
\end{MapleOutputGather} 
\subsection{studentパッケージいろいろ}
ちょっとぐらい難しい積分も,Mapleは単純にintコマンドだけで実行してくれる.しかし,時には,途中の計算法である部分積分,置換積分,部分分数展開が必要になる.このような計算はstudentパッケージに用意さている.
\begin{MapleInput}
> with(student):
\end{MapleInput}
\paragraph{部分積分(integration by parts)} 
\begin{MapleInput}
> intparts(Int(x*exp(x),x),x);
\end{MapleInput}
\begin{MapleOutput}
x\exp(x)-\int \exp(x){dx}
\end{MapleOutput}
\paragraph{置換(change of variables)による積分} 
\begin{MapleInput}
> Int((cos(x)+1)^3*sin(x), x);
> changevar(cos(x)+1=u, Int((cos(x)+1)^3*sin(x), x=a..b), u);
> changevar(cos(x)+1=u, int((cos(x)+1)^3*sin(x), x), u);
\end{MapleInput}
\begin{MapleOutputGather}
\int \left( \cos \left( x \right) +1 \right) ^{3}\sin \left( x \right) {dx} \notag \\
\int _{\cos \left( a \right) +1}^{\cos \left( b \right) +1}-{u}^{3}{du} \notag \\
 -\frac{1}{4}\,{u}^{4} \notag \\
\end{MapleOutputGather}
\paragraph{部分分数(partial fraction)展開による積分}
部分分数(partial fraction)展開による積分では,convertコマンドを用いる.
\begin{MapleInput}
> pf1:=convert(1/(1+x^3),parfrac,x);
  int(pf1,x);
\end{MapleInput}
\begin{MapleOutputGather}
 {\it pf1}\, := \frac{1}{3}\,{\frac {-x+2}{{x}^{2}-x+1}}+ \frac{1}{3\,\left( x+1 \right) } \notag \\
 -\frac{1}{6}\,\ln  \left( {x}^{2}-x+1 \right) +\frac{1}{3}\, \sqrt{3}\arctan \left( 1/3\, \left( 2\,x-1 \right)  \sqrt{3} \right) +\frac{1}{3}\,\ln  \left( x+1 \right) \notag
\end{MapleOutputGather}

 
\ChartElementTwo{課題}
\begin{enumerate}
\item 不定積分:次の不定積分を求めよ.

i) $\int 4\,x+3{dx}$
,ii)$\displaystyle \int  \frac{1}{ 1+\mbox{e}^{x} }{dx}$
,iii)$\displaystyle \int  \frac{1}{ \mbox{e}^{-x}+\mbox{e}^{x} }{dx}$
,iv)$\displaystyle \int  \sqrt{1-{x}^{2}}{dx}$


\item 定積分:次の定積分を求めよ.

i)$\displaystyle \int _{0}^{\pi } \sin x{dx}$
,ii)$\displaystyle \int _{0}^{1} \arctan x{dx}$
,iii)$\displaystyle \int _{-2}^{2} \frac{1}{ \sqrt{4-{x}^{2}} }{dx}$
,iv)$\displaystyle \int _{0}^{1} \frac{1}{ {x}^{2}+x+1 }{dx}$

\item (発展課題,重積分)次の2重積分を求めよ.
\begin{equation*}
\int \int_{D} \sqrt{x^2+y^2}dxdy\,\, D:0\leq y \leq x \leq 1
\end{equation*}
\end{enumerate} 
\ChartElementTwo{解答例}
\begin{enumerate}

\item 
\begin{MapleInput}
> int(4*x+3,x);
> int( 1/(1+exp(x)),x);
> int(1/(exp(-x)+exp(x)),x);
> int(sqrt(1-x^2),x);
\end{MapleInput}
\begin{MapleOutputGather}
\displaystyle 2\,{x}^{2}+3\,x  \notag \\
\displaystyle -\ln  \left( 1+ \mbox{e}^x \right) +\ln  \left( \mbox{e}^x \right)  \notag \\
\displaystyle \arctan \left( \mbox{e}^x \right)  \notag \\
\displaystyle \frac{1}{2}\,x \sqrt{1-{x}^{2}}+\frac{1}{2}\,\arcsin \left( x \right)  \notag
\end{MapleOutputGather}

\item
\begin{MapleInput}
> int(sin(x),x=0..Pi);
> int(arctan(x),x=0..1);
> int(1/(sqrt(4-x^(2))),x=-2..2);
> int(1/(x^2+x+1),x=0..1);
\end{MapleInput}
\begin{MapleOutputGather}
\displaystyle 2   \notag \\
\displaystyle \frac{1}{4}\,\pi -\frac{1}{2}\,\ln  \left( 2 \right)   \notag \\
\displaystyle \pi   \notag \\
\displaystyle \frac{1}{9}\,\pi \, \sqrt{3}  \notag
\end{MapleOutputGather}

\item
\begin{MapleInput}
> with(plots):
> inequal({x-y>=0,x>=0,x<=1,y>=0},x=-0.5..1.5,y=-0.5..1.5,optionsexcluded=(color=white));
\end{MapleInput}

\MaplePlot{40mm}{./figures/Intplot2d1.eps}

\begin{MapleInput}
> f:=unapply(sqrt(x^2+y^2),(x,y)):
> plot3d(f(x,y),x=0..1,y=0..1,axes=box);
\end{MapleInput}

\MaplePlot{40mm}{./figures/Intplot3d2.eps}
\begin{MapleInput}
> int(int(f(x,y),y=0..x),x=0..1);
\end{MapleInput}
\begin{MapleOutput}
\displaystyle \frac{1}{6} \sqrt{2}+\frac{1}{6}\,\ln  \left( 1+ \sqrt{2} \right) 
\end{MapleOutput}

\end{enumerate} 

\end{document}

\section{数値積分のコード}
次の積分を例に,Mapleのコードを示す.
\begin{equation*}
\int_0^1 \frac{4}{1+x^2} \, dx
\end{equation*}
先ずは問題が与えられたらできるだけMapleで解いてしまう.答えをあらかじめ知っておくと間違いを見つけるのが容易.プロットしてみる.
\begin{MapleInput}
> restart; 
  f1:=x->4/(1+x^2); 
  plot(f1(x),x=0..5);
\end{MapleInput}
\begin{MapleOutput}
{\it f1}\, := \,x\mapsto \frac{4}{1+{x}^{2}}
\end{MapleOutput}
\MaplePlot{50mm}{./figures/C7_InterpolationIntegralplot2d3.eps}
Mapleで解いてみる.
\begin{MapleInput}
>int(f1(x),x=0..1);
\end{MapleInput}
\begin{MapleOutput}
\pi
\end{MapleOutput}
えっと思うかも知れないが,
\begin{MapleInput}
>int(1/(1+x^2),x);
\end{MapleInput}
\begin{MapleOutput}
arctan(x)
\end{MapleOutput}
となるので,納得できるでしょう.

具体的にMapleでコードを示す.先ずは初期設定.
\begin{MapleInput}
>N:=8: x0:=0: xn:=1: Digits:=20:
\end{MapleInput}

\paragraph{Midpoint rule(中点法)} 
\begin{MapleInput}
> h:=(xn-x0)/N: S:=0: 
  for i from 0 to N-1 do 
    xi:=x0+(i+1/2)*h; 
    dS:=h*f1(xi);
    S:=S+dS; 
  end do: 
  evalf(S);
\end{MapleInput}
\begin{MapleOutput}
3.1428947295916887799
\end{MapleOutput}

\paragraph{Trapezoidal rule(台形公式)} 
\begin{MapleInput}
> h:=(xn-x0)/N: S:=f1(x0)/2: 
  for i from 1 to N-1 do 
    xi:=x0+i*h; 
    dS:=f1(xi);
    S:=S+dS; 
  end do: 
  S:=S+f1(xn)/2: 
  evalf(h*S);
\end{MapleInput}
\begin{MapleOutput}
3.1389884944910890093
\end{MapleOutput}

\paragraph{Simpson's rule(シンプソンの公式)} 
\begin{MapleInput}
> M:=N/2: h:=(xn-x0)/(2*M): Seven:=0: Sodd:=0: 
  for i from 1 to 2*M-1 by 2 do
    xi:=x0+i*h; 
    Sodd:=Sodd+f1(xi); 
  end do: 
  for i from 2 to 2*M-1 by 2 do
    xi:=x0+i*h; 
    Seven:=Seven+f1(xi); 
  end do:
  evalf(h*(f1(x0)+4*Sodd+2*Seven+f1(xn))/3);
\end{MapleInput}
\begin{MapleOutput}
3.1415925024587069144
\end{MapleOutput}


\section{課題}
\begin{enumerate}
\item 補間と近似の違いについて,適切な図を描いて説明せよ.
\item 次の4点
\begin{MapleInput}
x y 
0 1 
1 2
2 3
3 -2
\end{MapleInput}
を通る多項式を以下のそれぞれの手法で求めよ.(a) 逆行列, (b)ラグランジュ補間, (c)ニュートンの差分商公式 
\item
$\tan(5^\circ)=0.08748866355$, 
$\tan(10^\circ)=.1763269807$,
$\tan(15^\circ)=.2679491924$の値を用いて,ラグランジュ補間法により,$\tan(17^\circ)$の値を推定せよ.(2008年度期末試験)
\item exp(0)=1.0, exp(0.1)=1.1052, exp(0.3)=1.3499の値を用いて,ラグランジュ補間法により,exp(0.2)の値を推定せよ.(2009年度期末試験)
\item 次の関数
\begin{equation*}
f(x) = \frac{4}{1+x^2}
\end{equation*}
を$x = 0..1$で数値積分する.
\begin{enumerate}
\item $N$を2,4,8,…256ととり,$N$個の等間隔な区間にわけて中点法で求めよ.(15)
\item 小数点以下10桁まで求めた値3.141592654との差をdXとする.dXと分割数Nとを両対数プロット(loglogplot)して比較せよ(10)
\end{enumerate}
(2008年度期末試験)
\item 次の関数
\begin{equation*}
y = \frac{1}{1+x^2}
\end{equation*}
を$x = 0..1$で等間隔に$N+1$点とり,$N$個の区間にわけて数値積分で求める.$N$を2, 4, 8, 16, 32, 64, 128, 256と取ったときの(a)中点法, (b)台形公式, (c)シンプソン公式それぞれの収束性を比較せよ.

ヒント:Maple script にあるそれぞれの数値積分法を関数 (procedure) に直して,for-loop
で回せば楽.出来なければ,一つ一つ手で変えても OK. 両対数プロット (loglogplot) すると見やすい.
\end{enumerate}

\end{document}
