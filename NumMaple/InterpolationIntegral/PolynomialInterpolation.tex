データを単純に多項式で補間する方法を先ず示そう.$N+1$点をN次の多項式でつなぐ.この場合の補間関数は,
\begin{equation*}
F \left(x \right)={\sum_{i=0}^{N} } a _{i }x ^{i }=a_{0}+a_{1}x +a_{2}x^{2}+\cdots +a_{N}x^{N}
\end{equation*}
である.データの点を$(x_{i},\,y_{i}),i=0..N$とすると
\begin{equation*}
\begin{array}{cl}
a _{0}+a _{1}x _{0}+a _{2}x _{0}^{2}+\cdots +a _{N }x _{0}^{N }& =y _{0}\\
a _{0}+a _{1}x _{1}+a _{2}x _{1}^{2}+\cdots +a _{N }x _{1}^{N }& =y _{1}\\
\vdots& \\
a _{0}+a _{1}x _{N}+a _{2}x _{N}^{2}+\cdots +a _{N }x _{N}^{N }& =y _{N}
\end{array}
\end{equation*}
が,係数 $a_i$を未知数と見なした線形の連立方程式となっている.係数行列は
\begin{equation*}
A=\left[
\begin{array}{ccccc}
1&x_0&x_0^2&\cdots&x_0^N \\
1&x_1&x_1^2&\cdots&x_1^N \\
\vdots& & & & \vdots \\
1&x_N&x_N^2&\cdots&x_N^N 
\end{array} \right]
\end{equation*}
となる.$a_i$と$y_i$をそれぞれベクトルとみなすと
\ifHIKI %%%%
||               ||
\else %%%%
\begin{equation*}
\setlength{\unitlength}{1cm}
\begin{picture}(10,3.5)
\put(0,0){\framebox(10,3.5){}}
\end{picture}
\end{equation*}
\fi %%%%
により未知数ベクトル$a_i$が求まる.これは単純に,前に紹介した Gauss の消去法や LU 分解で解ける.

\subsection{Mapleによる多項式補間の実例}
\begin{MapleInput}
> restart; X:=[0,1,2,3]: Y:=[1,2,3,-2]:
> with(LinearAlgebra):
> list1:=[X,Y];
\end{MapleInput}
\begin{MapleOutput}
{\it list1}\, := \,[[0,1,2,3],[1,2,3,-2]]
\end{MapleOutput}
\begin{MapleInput}
> with(plots):
  l1p:=listplot(Transpose(Matrix(list1))):
  display(l1p);
\end{MapleInput}
\MaplePlot{50mm}{./figures/C7_InterpolationIntegralplot2d1.eps}

\begin{MapleInput}
> A:=Matrix(4,4): 
  for i from 1 to 4 do 
    for j from 1 to 4 do 
      A[i,j]:=X[i]^(j-1);
    end do; 
  end do:
  A;
\end{MapleInput}
\begin{MapleOutput}
\left[ \begin {array}{cccc} 1&0&0&0\\ 1&1&1&1\\ 1&2&4&8\\ 1&3&9&27\end {array} \right]
\end{MapleOutput}

\begin{MapleInput}
> a1:=MatrixInverse(A).Vector(Y);
\end{MapleInput}
\begin{MapleOutput}
{\it a1}\, := \, \left[ \begin {array}{c} 1\\ -1\\ 3\\ -1\end {array} \right]
\end{MapleOutput}
\begin{MapleInput}
> f1:=unapply(add(a1[ii]*x^(ii-1),ii=1..4),x);
\end{MapleInput}
\begin{MapleOutput}
{\it f1}\, := \,x\mapsto 1-x+3\,{x}^{2}-{x}^{3}
\end{MapleOutput}
\begin{MapleInput}
> f1p:=plot(f1(x),x=0..3): 
  l1p:=listplot(Transpose(Matrix(list1))):
  display(f1p,l1p);
\end{MapleInput}
\MaplePlot{50mm}{./figures/C7_InterpolationIntegralplot2d2.eps}
