Fast Fourier Transformation(FFT)高速フーリエ変換(あるいはデジタル(離散)フーリエ変換(DFT))は,周波数分解やフィルターを初め,画像処理などの多くの分野で使われている.基本となる考え方は,直交基底による関数の内挿法である.最初にその応用例を見た後,どのような理屈でFFTが動いているかを解説する.

\subsection{周波数分解}
はじめの例は,周波数分解.先ずは,非整合な波を二つ用意しておく.
\begin{MapleInput}
> restart:
  funcs:=[sin(i/13),sin(i/2)];
  #funcs:=[sin(i*2),2*sin(i/2)];
  plot(funcs,i=0..300);
\end{MapleInput}
\MaplePlot{50mm}{./figures/C10_FFTplot2d1.eps}
これを重ねあわせた波を作る.
\begin{MapleInput}
> data1:=[]:
  for i from 1 to 256 do
    data1:=[op(data1),evalf(funcs[1]+funcs[2]))]; 
  end do:
  with(plots): 
  listplot(data1);
\end{MapleInput}
\MaplePlot{50mm}{./figures/C10_FFTplot2d2.eps}
ゆっくり変化する波に,激しく変化する波が重なっていることが読み取れる.これにFFTを掛ける
\begin{MapleInput}
> X:=array(data1): 
  Y:=array(1..256,sparse):
  FFT(8,X,Y);
\end{MapleInput}
\begin{MapleOutput}
256
\end{MapleOutput}
その強さを求めて,周波数で表示すると,
\begin{MapleInput}
> Data2:=[seq([i,sqrt(X[i]^2+Y[i]^2)],i=1..128)]:
  plot(Data2);
\end{MapleInput}
\MaplePlot{50mm}{./figures/C10_FFTplot2d3.eps}
もとの2つの周波数に対応するところにピークができているのが確認できる.広がりは,誤差のせい.
logplotでも良い.

\subsection{高周波フィルター}
次の例は,高周波フィルター.たとえば次のようなローレンツ関数を考える.
\begin{MapleInput}
> restart;
  f1:=t->subs(a=10,b=40000,c=380,d=128,a+b/(c+(t-d)^2));
\end{MapleInput}
\begin{MapleOutput}
{\it f1}\, := \,t\mapsto 10+\displaystyle \frac{40000}{380+ \left( t-128 \right) ^{2} }
\end{MapleOutput}

\begin{MapleInput}
> plot(f1(t),t=1..256);
\end{MapleInput}
\MaplePlot{50mm}{./figures/C10_FFTplot2d4.eps}

これにノイズがのると,次のようになる.
\begin{MapleInput}
> T:=[seq(f1(i)*(0.6+0.8*evalf(rand()/10^12)),i=1..256)]:
  #T:=[seq(evalf(rand()/10^12),i=1..256)]:  #これはホワイトノイズ
  #T:=[seq(f1(i),i=1..256)]: #これは元の関数そのまま
  with(plots):
  listplot(T);
\end{MapleInput}
\MaplePlot{50mm}{./figures/C10_FFTplot2d5.eps}
これに高周波フィルターを掛けるとノイズが消えるが,その様子を示そう.先ずは,FFTを掛ける.
\begin{MapleInput}
> Idata:=array([seq(0,i=1..256)]):
  Rdata:=convert(T,array):
  FFT(8,Rdata,Idata);
\end{MapleInput}
\begin{MapleOutput}
256
\end{MapleOutput}
これは次のような強度分布をもっている.
\begin{MapleInput}
> Adata:=[seq([i,sqrt(Idata[i]^2+Rdata[i]^2)],i=1..128)]:
> logplot(Adata);
\end{MapleInput}
\MaplePlot{70mm}{./figures/C10_FFTplot2d6.eps}
低周波の部分に,ゆっくりとした変化を表す成分が固まっている.次のような三角フィルターを用意する.これは,低周波ほど影響を大きくするフィルター.
\begin{MapleInput}
> filter:=x->piecewise(x>=0 and x<=20,(1-x/20)): #三角フィルター
  #filter:=x->piecewise(x>=0 and x<=20,1); #方形フィルター
  plot(filter(x),x=0..40);
\end{MapleInput}
\MaplePlot{30mm}{./figures/C10_FFTplot2d7.eps}
これとデータを各点で掛けあわせる事によって,フィルターを通したことになる.
\begin{MapleInput}
> FRdata:=array([seq(Rdata[i]*filter(i),i=1..256)]):
> FIdata:=array([seq(Idata[i]*filter(i),i=1..256)]):
\end{MapleInput}
先ほどと同様に表示すると
\begin{MapleInput}
> Bdata:=[seq([i,sqrt(FIdata[i]^2+FRdata[i]^2)],i=1..128)]:
> logplot(Adata);
\end{MapleInput}
\MaplePlot{70mm}{./figures/C10_FFTplot3d6.eps}
$i=20$以上の領域がフィルターによってちょん切られていることが確認できる.これを逆フーリエ変換する.
\begin{MapleInput}
> iFFT(8,FRdata,FIdata);
\end{MapleInput}
\begin{MapleOutput}
256
\end{MapleOutput}
これを表示すると,
\begin{MapleInput}
> listplot(FRdata);
\end{MapleInput}
\MaplePlot{50mm}{./figures/C10_FFTplot2d8.eps}
となる.ノイズが取り除かれているのが確認できる.元の関数に加えたホワイトノイズにFFTを掛ければ分かるが,全周波数域にわたって均質に広がった関数となる.これを三角フィルターなどで高周波成分をカットすることで,ノイズが取り除かれていくのが理解されよう.
