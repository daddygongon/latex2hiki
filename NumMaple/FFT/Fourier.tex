一連の関数系による関数の内挿は,基底関数を$\varphi_n(x)$として
\begin{equation*}
F(x) = \sum^N_{n=1}a_n \varphi_n(x)
\end{equation*}
で得られることを見た.
Fourier変換では基底関数として$\varphi _{{n}} \left( x \right) =\sin \left( 2\,\pi {\it nx} \right) ,\,\cos \left( 2\,\pi {\it nx} \right)$をとる.関数の内挿法で示したように,この$x_i$での値$f_i, i=1 \cdots M$と,近似の次数($N$)とでつくる係数行列,
\begin{equation*}
A=\left[ \begin{array}{cccc}
\varphi_0(x_0)&\varphi_1(x_0)& \cdots &\varphi_N(x_0) \\
\vdots & \vdots & \vdots & \vdots \\
\varphi_0(x_M)&\varphi_1(x_M)& \cdots &\varphi_N(x_M) 
\end{array}\right]
\end{equation*}
を求めて,係数$a_i$とデータ点$f_i$をそれぞれベクトルと考えると,
\begin{equation*}
\bm{A}.\bm{a} = \bm{f}
\end{equation*}
から,通常の逆行列を求める手法で係数を決定することもできる.しかし,この強引な方法はデータ数,関数の次数が多い,フーリエ変換が対象としようとする問題では破綻する.もっといい方法が必要で,それが直交関数系では存在する.
