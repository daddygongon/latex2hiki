関数の直交関係は,
\begin{equation}
\int _{a}^{b}\varphi _{{n}} \left( x \right) \varphi _{{m}} \left( x \right) {dx}=\delta_{{{\it mn}}}C_{{n}}= 
\left\{\begin{array}{lr}
C_m & at\, n=m \\
0 & at\, n\neq m
\end{array}\right.
\label{Eq:Orthogonal}
\end{equation}
である.定数$C_m$は,$\sin,\cos$の三角関数系では次の通り.
\begin{MapleInput}
> plot([sin(x),sin(3*x)],x=0..2*Pi);
\end{MapleInput}
\MaplePlot{50mm}{./figures/C10_FFTplot2d9.eps}

\begin{MapleInput}
> plot([sin(x)*sin(3*x)],x=0..2*Pi, color=black);
\end{MapleInput}
\MaplePlot{50mm}{./figures/C10_FFTplot2d10.eps}

\begin{MapleInput}
> int(sin(x)*sin(3*x),x=0..2*Pi);
\end{MapleInput}
\begin{MapleOutput}
0
\end{MapleOutput}
\begin{MapleInput}
> for i from 1 to 3 do for j from 1 to 3 do S:=int(sin(i*x)*sin(j*x),x=0..2*Pi);
> print(i,j,S); end do; end do:
\end{MapleInput}
\begin{MapleOutputGather}
1, 1, \pi \notag \\
1, 2, 0 \notag \\
1, 3, 0 \notag \\
2, 1, 0 \notag \\
2, 2, \pi \notag \\
2, 3, 0 \notag \\
3, 1, 0 \notag \\
3, 2, 0 \notag \\
3, 3, \pi \notag
\end{MapleOutputGather}
\begin{equation*}
\int _{a}^{b}F \left( x \right) \varphi _{{m}} \left( x \right) {dx}
\end{equation*}
を考える.先程の\ref{Eq:Orthogonal}式をいれると
\begin{equation}
\int _{a }^{b }F \left(x \right)\varphi _{m }\left(x \right) dx =\int _{a }^{b }{\sum^N_{n=1} }a _{n }\varphi _{n }\left(x \right)\varphi _{m }\left(x \right)d x = 
\left\{\begin{array}{lr}
a_m C_m & at\, n=m \\
0 & at\, n\neq m
\end{array}\right.
\end{equation}
となる.こうして,係数$a_n$が
\begin{equation*}
a_{{n}}=\frac {1}{C_n}\int _{a}^{b} F \left( x \right) \varphi _{{n}} \left( x \right) {dx}
\end{equation*}
で決定できる.
