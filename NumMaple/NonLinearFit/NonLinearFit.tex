\documentclass[10pt,a4j]{jreport}
\usepackage[dvips]{graphicx,color}
\usepackage{tabularx}
\usepackage{verbatim}
\usepackage{amsmath,amsthm,amssymb}
\usepackage{bm}
\topmargin -15mm\oddsidemargin -4mm\evensidemargin\oddsidemargin
\textwidth 170mm\textheight 257mm\columnsep 7mm
\setlength{\fboxrule}{0.2ex}
\setlength{\fboxsep}{0.6ex}

\pagestyle{empty}

\newcommand{\MaplePlot}[2]{{\begin{center}
    \includegraphics[width=#1,clip]{#2}
                     \end{center}
%
} }

\newenvironment{MapleInput}{%
    \color{red}\verbatim
}{%
    \endverbatim
}

\newenvironment{MapleError}{%
    \color{blue}\verbatim
}{%
    \endverbatim
}

\newenvironment{MapleOutput}{%
    \color{blue}\begin{equation*}
}{%
    \end{equation*}
}

\newenvironment{MapleOutputGather}{%
    \color{blue}\gather
}{%
    \endgather
}
\newif\ifHIKI
%\HIKItrue % TRUEの設定
\HIKIfalse % FALSEの設定
\begin{document}
\chapter{非線形最小2乗法(NonLinearFit)}
\section{非線形最小2乗法の原理}
もっとも簡単な例で原理を解説する.近似関数として,
\begin{equation*}
F(x) = a_0+a_1\,x
\end{equation*}
という直線近似を考える.もっともらしい関数は$N$点の測定データとの差$d_i = F(x_i)-y_i$を最小にすればよさそうであるが,これはプラスマイナスですぐに消えて不定になる.そこで,
\begin{equation*}
\chi^{2}=\sum_i^N d_i^2=\sum_i^N\left(a_0+a_1\,x_i-y_i\right)^2
\end{equation*}
という関数を考える.この$\chi^2$(カイ二乗)関数が,$a_0, a_1$をパラメータとして変えた時に最小となる$a_0, a_1$を求める.これは,それらの微分がそれぞれ0となる場合である.これは$\chi^2$の和$\sum$(sum)の中身を展開し,
\ifHIKI %%%%
||$\chi^2=$||            ||

\else %%%%

\begin{table}[htbp]
\begin{center}
\begin{tabular}{cc}
$\chi^2=$& \\
&
\setlength{\unitlength}{1cm}
\begin{picture}(10,6.5)
\put(0,0){\framebox(10,6.5){}}
\end{picture}

\end{tabular}
\end{center}
\end{table}%
\fi %%%%

$a_0, a_1$でそれぞれ微分すれば
\ifHIKI %%%%
||$\displaystyle \frac{\partial}{\partial a_0} \chi^2 =$||            ||
||$\displaystyle \frac{\partial}{\partial a_1} \chi^2 =$||            ||
\else %%%%

\begin{table}[htbp]
\begin{center}
\begin{tabular}{cc}
$\displaystyle \frac{\partial}{\partial a_0} \chi^2 =$& 
\setlength{\unitlength}{1cm}
\begin{picture}(10,3.5)
\put(0,0){\framebox(10,3.5){}}
\end{picture}  \\
$\displaystyle \frac{\partial}{\partial a_1} \chi^2 =$&
\setlength{\unitlength}{1cm}
\begin{picture}(10,3.5)
\put(0,0){\framebox(10,3.5){}}
\end{picture}  \\
\end{tabular}
\end{center}
\end{table}%
\fi %%%%

という$a_0, a_1$を未知変数とする2元の連立方程式が得られる.これは前に説明した通り逆行列で解くことができる.

\section{具体的な手順}

パラメータの初期値を
\begin{equation*}
a_{{0}}+\Delta\,a,\,b_{{0}}+\Delta\,b,\,c_{{0}}+\Delta\,c,\,d_{{0}}+\Delta\,d
\end{equation*}
とする.このとき関数$f$を真値$a_0, b_0, c_0, d_0$のまわりでテイラー展開し,高次項を無視すると
\begin{equation*}
\Delta\,f=f \left( a_{{0}}+\Delta\,a_{{1}},b_{{0}}+\Delta\,b_{{1}},c_{{0}}+\Delta\,c_{{1}},d_{{0}}+\Delta\,d_{{1}} \right) -f \left( a_{{0}},b_{{0}},c_{{0}},d_{{0}} \right)
\end{equation*}

\begin{equation*}
=\left(\frac{\partial }{\partial a }f \right)_{0}\Delta a _{1}+\left(\frac{\partial }{\partial b }f \right)_{0}\Delta b _{1}+\left(\frac{\partial }{\partial c }f \right)_{0}\Delta c _{1}+\left(\frac{\partial }{\partial d }f \right)_{0}\Delta d _{1}
\end{equation*}
となる.

課題でつくったデータはt = 1からt = 256までの時刻に対応したデータ点$f_{1},\,f_{2},\,\cdots  f_{256}$とする.各測定値とモデル関数から予想される値との差$\Delta f_1,\Delta f_2,\cdots,\Delta f_{256}$は,
\begin{equation}
\left(\begin{array}{c}\Delta f _{1} \\\Delta f _{2} \\ \vdots \\\Delta f _{256} \\\end{array}\right)=J \left(\begin{array}{c}\Delta a _{1} \\\Delta b _{1} \\\Delta c _{1} \\\Delta d _{1} \\\end{array}\right)
\end{equation}
となる.ここで$J$はヤコビ行列と呼ばれる行列で,4列256行
\begin{equation}
\displaystyle J =\left(\begin{array}{cccc}\left(\frac{\partial }{\partial a }f \right)_{1} & \left(\frac{\partial }{\partial b }f \right)_{1} & \left(\frac{\partial }{\partial c }f \right)_{1} & \left(\frac{\partial }{\partial d }f \right)_{1} \\ \vdots & \vdots  &  \vdots & \vdots  \\\left(\frac{\partial }{\partial a }f \right)_{256} & \left(\frac{\partial }{\partial b }f \right)_{256} & \left(\frac{\partial }{\partial c }f \right)_{256} & \left(\frac{\partial }{\partial d }f \right)_{256} \\\end{array}\right)
\end{equation}
である.このような矩形行列の逆行列は転置行列$J^T$を用いて,`
\begin{equation}
J ^{-1}=\left(J ^{T }J \right)^{-1}J ^{T }
\end{equation}
と表わされる.したがって,真値からのずれは
\begin{equation}
\left(\begin{array}{c}\Delta a_2 \\\Delta b_2 \\\Delta c_2 \\\Delta d_2 \\\end{array}\right)
=\left(J ^{T }J \right)^{-1}J ^{T }
\left(\begin{array}{c}\Delta f _{1} \\\Delta f _{2} \\ \vdots \\\Delta f _{256} \\\end{array}\right)
\end{equation}
で求められる.理想的には$(\Delta a_2,\,\Delta b_2,\,\Delta c_2,\,\Delta d_2)$は$(\Delta a,\,\Delta b,\,\Delta c,\,\Delta d)$に一致するはずだが,測定誤差と高次項のために一致しない.初期値に比べ,より真値に近づくだけ.そこで,新たに得られたパラメータの組を新たな初期値に用いて,より良いパラメータに近付けていくという操作を繰り返す.新たに得られたパラメータと前のパラメータとの差がある誤差以下になったところで計算を打ち切り,フィッティングの終了となる.

\section{Mapleによる解法の指針}
線形代数計算のためにサブパッケージとしてLinearAlgebraを呼びだしておく.
\begin{MapleInput}
> restart; 
  with(plots): 
  with(LinearAlgebra):
\end{MapleInput}


データを読み込む.
\begin{MapleInput}
> ndata:=8: 
  f1:=t->subs({a1=1,a2=10,a3=1,a4=4},a1+a2/(a3+(t-a4)^2) );
\end{MapleInput}
\begin{MapleOutput}
{\it f1}\, := \,t\mapsto 1+10\, \left( 1+ \left( t-4 \right) ^{2} \right) ^{-1}
\end{MapleOutput}
データの表示
\begin{MapleInput}
> T:=[seq(f1(i),i=1..ndata)]:
  listplot(T); 
  l1:=listplot(T):
\end{MapleInput}
\MaplePlot{80mm}{./figures/C9_NonLinearFitplot2d4.eps}


ローレンツ型の関数を仮定し,関数として定義.
\begin{MapleInput}
> f:=t->a1+a2/(a3+(t-a4)^2); nparam:=4:
\end{MapleInput}
\begin{MapleOutput}
f\, := \,t\mapsto {\it a1}+{\frac {{\it a2}}{{\it a3}+ \left( t-{\it a4} \right) ^{2}}}\end{MapleOutput}
ヤコビアンの中の微分を新たな関数として定義.
\begin{MapleInput}
> for i from 1 to nparam do 
    dfda||i:=unapply(diff(f(x),a||i),x); 
  end do;
\end{MapleInput}
\begin{MapleOutputGather}
{\it dfda1}\, := \,x\mapsto 1 \notag \\
{\it dfda2}\, := \,x\mapsto  \left( {\it a3}+ \left( x-{\it a4} \right) ^{2} \right) ^{-1}
\notag \\
{\it dfda3}\, := \,x\mapsto -{\frac {{\it a2}}{ \left( {\it a3}+ \left( x-{\it a4} \right) ^{2} \right) ^{2}}} \notag \\
{\it dfda4}\, := \,x\mapsto -{\frac {{\it a2}\, \left( -2\,x+2\,{\it a4} \right) }{ \left( {\it a3}+ \left( x-{\it a4} \right) ^{2} \right) ^{2}}}  \notag
\end{MapleOutputGather}
ここで,"$||$"は連結作用素とよばれるMapleのコマンドで,$dfda||1 \mapsto dfda1$と連結する.
初期値を仮定して,データとともに関数を表示.
\begin{MapleInput}
> g1:=Vector([1,8,1,4.5]): 
  guess1:={}: 
  for i from 1 to nparam do
    guess1:={op(guess1),a||i=g1[i]}; 
  end do: 
  guess1;
\end{MapleInput}
\begin{MapleOutput}
\left\{ {\it a1}=1,{\it a2}=8,{\it a3}=1,{\it a4}= 4.5\right\}
\end{MapleOutput}

\begin{MapleInput}
> p1:=plot(subs(guess1,f(x)),x=1..ndata): 
  display(l1,p1);  
\end{MapleInput}
\MaplePlot{50mm}{./figures/C9_NonLinearFitplot2d5.eps}
見やすいように,小数点以下を3桁表示に制限する.
\begin{MapleInput}
> interface(displayprecision=3):
> df:=Vector([seq(subs(guess1,T[i]-f(i)),i=1..ndata)]);
\end{MapleInput}
\begin{MapleOutput}
{\it df}\, := \, \left[ \begin {array}{c}  0.396\\  0.897\\  2.538\\  3.600\\ - 1.400\\ - 0.462\\ - 0.103\\ - 0.016\end {array} \right]
\end{MapleOutput}

\begin{MapleInput}
> Jac:=Matrix(ndata,nparam): 
  for i from 1 to ndata do 
    for j from 1 to nparam do
      Jac[i,j]:=evalf(subs(guess1,dfda||j(i))); 
    end do: 
  end do:
  Jac;
\end{MapleInput}
\begin{MapleOutput}
\left[ \begin {array}{cccc}  1.0& 0.075&- 0.046&- 0.319\\  1.0& 0.138&- 0.152&- 0.761\\  1.0& 0.308&- 0.757&- 2.272\\  1.0& 0.800&- 5.120&- 5.120\\  1.0& 0.800&- 5.120& 5.120\\  1.0& 0.308&- 0.757& 2.272\\  1.0& 0.138&- 0.152& 0.761\\  1.0& 0.075&- 0.046& 0.319\end {array} \right]
\end{MapleOutput}

\begin{MapleInput}
> tJac:=(MatrixInverse(Transpose(Jac).Jac)).Transpose(Jac);
\end{MapleInput}
\begin{MapleOutput}
{\it tJac}\, := \, \left[ \begin {array}{cccccccc}  0.565& 0.249&- 0.354& 0.040& 0.040&- 0.354& 0.249& 0.565\\ - 2.954&- 0.506& 4.012&- 0.552&- 0.552& 4.012&- 0.506&- 2.954\\ - 0.352&- 0.029& 0.557&- 0.176&- 0.176& 0.557&- 0.029&- 0.352\\ - 0.005&- 0.012&- 0.035&- 0.080& 0.080& 0.035& 0.012& 0.005\end {array} \right] 
\end{MapleOutput}

\begin{MapleInput}
> g2:=tJac.df; 
  g1:=g1+g2;
\end{MapleInput}
\begin{MapleOutputGather}
{\it g2}\, := \, \left[ \begin {array}{c} - 0.235\\  5.592\\  0.613\\ - 0.520\end {array} \right]  \notag \\
{\it g1}\, := \, \left[ \begin {array}{c}  0.765\\  13.592\\  1.613\\  3.980\end {array} \right] \notag
\end{MapleOutputGather}
これをまたもとの近似値(guess)に入れ直して表示させると以下のようになる.カーブがデータに近づいているのが確認できよう.この操作をずれが十分小さくなるまで繰
り返す.
\begin{MapleInput}
> guess1:={seq(a||i=g1[i],i=1..nparam)};
  p1:=plot(subs(guess1,f(x)),x=1..ndata):
  display(l1,p1);
\end{MapleInput}
\begin{MapleOutput}
guess1:=\{a1=0.765, a2=13.592, a3=1.613, a4=3.980\}
\end{MapleOutput}
\MaplePlot{50mm}{./figures/C9_NonLinearFitplot2d6.eps}

4回ほど繰り返すと以下の通り,いい値に収束している.
\begin{MapleOutput}
guess1:=\{a1 = 1.006, a2 = 9.926, a3 = .989, a4 = 4.000\}
\end{MapleOutput}


\section{Gauss-Newton法に関するメモ}
このGauss-Newton法と呼ばれる非線形最小二乗法は線形問題から拡張した方法として論理的に簡明であり,広く使われている.しかし,収束性は高くなく,むしろ発散しやすいので注意が必要.2次の項を無視するのでなく,うまく見積もる方法を用いたのがLevenberg-Marquardt法である.明快な解説がNumerical Recipes in C(C 言語による数値計算のレシピ)WilliamH.Press 他著,技術評論社1993にある.

\section{課題}
\begin{enumerate}
\item 補間と近似の違いについて,適切な図を描いて説明せよ.
\item 次の4点
\begin{MapleInput}
x y 
0 1 
1 2
2 3
3 -2
\end{MapleInput}
を通る多項式を以下のそれぞれの手法で求めよ.(a) 逆行列, (b)ラグランジュ補間, (c)ニュートンの差分商公式 
\item
$\tan(5^\circ)=0.08748866355$, 
$\tan(10^\circ)=.1763269807$,
$\tan(15^\circ)=.2679491924$の値を用いて,ラグランジュ補間法により,$\tan(17^\circ)$の値を推定せよ.(2008年度期末試験)
\item exp(0)=1.0, exp(0.1)=1.1052, exp(0.3)=1.3499の値を用いて,ラグランジュ補間法により,exp(0.2)の値を推定せよ.(2009年度期末試験)
\item 次の関数
\begin{equation*}
f(x) = \frac{4}{1+x^2}
\end{equation*}
を$x = 0..1$で数値積分する.
\begin{enumerate}
\item $N$を2,4,8,…256ととり,$N$個の等間隔な区間にわけて中点法で求めよ.(15)
\item 小数点以下10桁まで求めた値3.141592654との差をdXとする.dXと分割数Nとを両対数プロット(loglogplot)して比較せよ(10)
\end{enumerate}
(2008年度期末試験)
\item 次の関数
\begin{equation*}
y = \frac{1}{1+x^2}
\end{equation*}
を$x = 0..1$で等間隔に$N+1$点とり,$N$個の区間にわけて数値積分で求める.$N$を2, 4, 8, 16, 32, 64, 128, 256と取ったときの(a)中点法, (b)台形公式, (c)シンプソン公式それぞれの収束性を比較せよ.

ヒント:Maple script にあるそれぞれの数値積分法を関数 (procedure) に直して,for-loop
で回せば楽.出来なければ,一つ一つ手で変えても OK. 両対数プロット (loglogplot) すると見やすい.
\end{enumerate}
\section{解答例}
2. ふた山ピークへのフィット.
\begin{MapleInput}
> restart; with(plots): with(LinearAlgebra):
> f1:=t->subs({a=10,b=40000,c=380,d=128},a+b/(c+(t-d)^2) );
> f2:=t->subs({a=10,b=40000,c=380,e=90},a+b/(c+(t-e)^2) );
> T:=[seq((f1(i)+f2(i))*(0.6+0.2*evalf(rand()/10^12)),i=1..256)]:
\end{MapleInput}
\begin{MapleOutputGather}
{\it f1}\, := \,t\mapsto 10+40000\, \left( 380+ \left( t-128 \right) ^{2} \right) ^{-1} \notag \\
{\it f2}\, := \,t\mapsto 10+40000\, \left( 380+ \left( t-90 \right) ^{2} \right) ^{-1} \notag
\end{MapleOutputGather}
\begin{MapleInput}
> l1:=listplot(T):
> f:=t->a1+a2/(a3+(t-a4)^2)+a2/(a3+(t-a5)^2); 
  nparam:=5:
\end{MapleInput}
\begin{MapleOutput}
f\, := \,t\mapsto {\it a1}+{\frac {{\it a2}}{{\it a3}+ \left( t-{\it a4} \right) ^{2}}}+{\frac {{\it a2}}{{\it a3}+ \left( t-{\it a5} \right) ^{2}}}
\end{MapleOutput}
\begin{MapleInput}
> for i from 1 to nparam do 
    dfda||i:=unapply(diff(f(x),a||i),x); 
  end do;
\end{MapleInput}
\begin{MapleOutputGather}
{\it dfda1}\, := \,x\mapsto 1 \notag \\
{\it dfda2}\, := \,x\mapsto  \left( {\it a3}+ \left( x-{\it a4} \right) ^{2} \right) ^{-1}+ \left( {\it a3}+ \left( x-{\it a5} \right) ^{2} \right) ^{-1} \notag \\
{\it dfda3}\, := \,x\mapsto -{\frac {{\it a2}}{ \left( {\it a3}+ \left( x-{\it a4} \right) ^{2} \right) ^{2}}}-{\frac {{\it a2}}{ \left( {\it a3}+ \left( x-{\it a5} \right) ^{2} \right) ^{2}}} \notag \\
{\it dfda4}\, := \,x\mapsto -{\frac {{\it a2}\, \left( -2\,x+2\,{\it a4} \right) }{ \left( {\it a3}+ \left( x-{\it a4} \right) ^{2} \right) ^{2}}} \notag \\
{\it dfda5}\, := \,x\mapsto -{\frac {{\it a2}\, \left( -2\,x+2\,{\it a5} \right) }{ \left( {\it a3}+ \left( x-{\it a5} \right) ^{2} \right) ^{2}}} \notag
\end{MapleOutputGather}

\begin{MapleInput}
> g1:=Vector([10,1200,10,125,90]);
\end{MapleInput}
\begin{MapleOutput}
{\it g1}\, := \, \left[ \begin {array}{c} 10\\ 1200\\ 10\\ 125\\ 90\end {array} \right] 
\end{MapleOutput}
\begin{MapleInput}
> guess1:={seq(a||i=g1[i],i=1..nparam)};
\end{MapleInput}
\begin{MapleOutput}
guess1 := \{a1 = 10, a2 = 1200, a3 = 10, a4 = 125, a5 = 90\}
\end{MapleOutput}

\begin{MapleInput}
> p1:=plot(subs(guess1,f(x)),x=1..256): 
  display(l1);
\end{MapleInput}
\MaplePlot{50mm}{./figures/C9_NonLinearFitplot2d9.eps}

\begin{MapleInput}
> df:=Vector([seq(subs(guess1,T[i]-f(i)),i=1..256)]):
  Jac:=Matrix(1..256,1..nparam,sparse):
  for i from 1 to 256 do 
    for j from 1 to nparam do
      Jac[i,j]:=evalf(subs(guess1,dfda||j(i))); 
    end do:
  end do:
  tJac:=(MatrixInverse(Transpose(Jac).Jac)).Transpose(Jac):
  g2:=tJac.df; g1:=g1+g2;
\end{MapleInput}
\begin{MapleOutputGather}
{\it g2}\, := \, \left[ \begin {array}{c} - 0.390553882992161205\\  1584.55290636967129\\  24.9577909601538366\\ - 0.0472041829705451138\\ - 0.00719532042503852940\end {array} \right]  \notag \\
{\it g1}\, := \, \left[ \begin {array}{c}  13.6348019182603064\\  29567.3667677707381\\  410.545681677467769\\  128.512734548828887\\  90.9223109918718678\end {array} \right] \notag
\end{MapleOutputGather}

\begin{MapleInput}
> guess1:={seq(a||i=g1[i],i=1..nparam)};
  p1:=plot(subs(guess1,f(x)),x=1..256):
  display(l1,p1);
\end{MapleInput}
\begin{MapleOutput}
guess1 := \{a1 = 30.251, a2 = 3854.136, a3 = 39.571, a4 = 124.800, a5 = 89.960\}
\end{MapleOutput}
\MaplePlot{50mm}{./figures/C9_NonLinearFitplot2d10.eps}
何回か繰り返せば,データ点に近づいてくるはず.


\end{document}
