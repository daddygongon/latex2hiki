
パラメータの初期値を
\begin{equation*}
a_{{0}}+\Delta\,a,\,b_{{0}}+\Delta\,b,\,c_{{0}}+\Delta\,c,\,d_{{0}}+\Delta\,d
\end{equation*}
とする.このとき関数$f$を真値$a_0, b_0, c_0, d_0$のまわりでテイラー展開し,高次項を無視すると
\begin{equation*}
\Delta\,f=f \left( a_{{0}}+\Delta\,a_{{1}},b_{{0}}+\Delta\,b_{{1}},c_{{0}}+\Delta\,c_{{1}},d_{{0}}+\Delta\,d_{{1}} \right) -f \left( a_{{0}},b_{{0}},c_{{0}},d_{{0}} \right)
\end{equation*}

\begin{equation*}
=\left(\frac{\partial }{\partial a }f \right)_{0}\Delta a _{1}+\left(\frac{\partial }{\partial b }f \right)_{0}\Delta b _{1}+\left(\frac{\partial }{\partial c }f \right)_{0}\Delta c _{1}+\left(\frac{\partial }{\partial d }f \right)_{0}\Delta d _{1}
\end{equation*}
となる.

課題でつくったデータはt = 1からt = 256までの時刻に対応したデータ点$f_{1},\,f_{2},\,\cdots  f_{256}$とする.各測定値とモデル関数から予想される値との差$\Delta f_1,\Delta f_2,\cdots,\Delta f_{256}$は,
\begin{equation}
\left(\begin{array}{c}\Delta f _{1} \\\Delta f _{2} \\ \vdots \\\Delta f _{256} \\\end{array}\right)=J \left(\begin{array}{c}\Delta a _{1} \\\Delta b _{1} \\\Delta c _{1} \\\Delta d _{1} \\\end{array}\right)
\end{equation}
となる.ここで$J$はヤコビ行列と呼ばれる行列で,4列256行
\begin{equation}
\displaystyle J =\left(\begin{array}{cccc}\left(\frac{\partial }{\partial a }f \right)_{1} & \left(\frac{\partial }{\partial b }f \right)_{1} & \left(\frac{\partial }{\partial c }f \right)_{1} & \left(\frac{\partial }{\partial d }f \right)_{1} \\ \vdots & \vdots  &  \vdots & \vdots  \\\left(\frac{\partial }{\partial a }f \right)_{256} & \left(\frac{\partial }{\partial b }f \right)_{256} & \left(\frac{\partial }{\partial c }f \right)_{256} & \left(\frac{\partial }{\partial d }f \right)_{256} \\\end{array}\right)
\end{equation}
である.このような矩形行列の逆行列は転置行列$J^T$を用いて,`
\begin{equation}
J ^{-1}=\left(J ^{T }J \right)^{-1}J ^{T }
\end{equation}
と表わされる.したがって,真値からのずれは
\begin{equation}
\left(\begin{array}{c}\Delta a_2 \\\Delta b_2 \\\Delta c_2 \\\Delta d_2 \\\end{array}\right)
=\left(J ^{T }J \right)^{-1}J ^{T }
\left(\begin{array}{c}\Delta f _{1} \\\Delta f _{2} \\ \vdots \\\Delta f _{256} \\\end{array}\right)
\end{equation}
で求められる.理想的には$(\Delta a_2,\,\Delta b_2,\,\Delta c_2,\,\Delta d_2)$は$(\Delta a,\,\Delta b,\,\Delta c,\,\Delta d)$に一致するはずだが,測定誤差と高次項のために一致しない.初期値に比べ,より真値に近づくだけ.そこで,新たに得られたパラメータの組を新たな初期値に用いて,より良いパラメータに近付けていくという操作を繰り返す.新たに得られたパラメータと前のパラメータとの差がある誤差以下になったところで計算を打ち切り,フィッティングの終了となる.
