$A$を対称正方行列,$x$をベクトルとしたときに,
\begin{equation}
Ax = \lambda x
\label{Eq:Eigen}
\end{equation}
の解,$\lambda$を固有値,$x$を固有ベクトルという.$x$がゼロベクトルではない意味のある解は特性方程式det$(A-\lambda E)=0$が成り立つときにのみ得られる.

まずMapleで特性方程式を解いてみる.

\begin{MapleInput}
> restart;
> with(LinearAlgebra):with(plots):with(plottools):
> A:=Matrix(1..2,1..2,[[3,2/3],[2/3,2]]);
\end{MapleInput}

\begin{MapleOutput}
A\, := \, \left[ \begin {array}{cc} 3&2/3\\2/3&2\end {array} \right]
\end{MapleOutput}

\begin{MapleInput}
> EE:=Matrix([[1,0],[0,1]]):
  A-lambda.EE;
\end{MapleInput}

\ifHIKI
\begin{tabular}{|c|}
\hline
         \\ 
\hline
\end{tabular}
\else
\begin{equation*}
\setlength{\unitlength}{1cm}
\begin{picture}(10,3.5)
\put(0,0){\framebox(10,3.5){}}
\end{picture}
\end{equation*}
\fi

\begin{MapleInput}
> eq2:=Determinant(A-lambda.EE);
\end{MapleInput}

\ifHIKI
\begin{tabular}{|c|}
\hline
         \\ 
\hline
\end{tabular}
\else
\begin{equation*}
\setlength{\unitlength}{1cm}
\begin{picture}(10,3.5)
\put(0,0){\framebox(10,3.5){}}
\end{picture}
\end{equation*}
\fi

\begin{MapleOutput}
{\it eq2}\, := \,{\frac {50}{9}}-5\,\lambda+{\lambda}^{2}
\end{MapleOutput}

\begin{MapleInput}
> solve(eq2=0,lambda);
\end{MapleInput}
\begin{MapleOutput}
10/3,\,5/3
\end{MapleOutput}

固有値を求めるコマンドEigenvectorsを適用すると,固有値と固有ベクトルが求まる.ここで,固有ベクトルは行列の列(Column)ベクトルに入っている.
\begin{MapleInput}
> lambda,V:=Eigenvectors(A);
\end{MapleInput}
\begin{MapleOutput}
\lambda,\,V\, := \, \left[ \begin {array}{c} 10/3\\5/3\end {array} \right] ,\, \left[ \begin {array}{cc} 2&-1/2\\1&1\end {array} \right] 
\end{MapleOutput}
得られた固有ベクトルは規格化されているわけではない.

行列の列を取り出すコマンドColumnを用いて,方程式(\ref{Eq:Eigen})が成り立っていることを確認する.
\begin{MapleInput}
> lambda[1].Column(V,1)=A.Column(V,1);
\end{MapleInput}
\begin{MapleOutput}
\displaystyle
\left[ \begin {array}{c} 20/3\\
10/3\end {array} \right] = 
\left[ \begin {array}{c} 20/3\\
10/3\end {array} \right] 
\end{MapleOutput}
一般的な規格化は,コマンドNormalize(vector,Euclidean)によっておこなう.
\begin{MapleInput}
> Normalize(Column(v,1),Euclidean);
\end{MapleInput}

