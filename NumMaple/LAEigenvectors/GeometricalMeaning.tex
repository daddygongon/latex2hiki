次に,固有値の幾何学的な意味を2次元行列で確認しておこう.ある点$x_0$に対称正方行列$A$を作用すると,$x_1$に移動する.これを原点を中心とする円上の点に次々に作用させ,移動前後の点を結ぶ.

\begin{MapleInput}
> restart;
  with(LinearAlgebra):with(plots):with(plottools):
  A:=Matrix(1..2,1..2,[[3,2/3],[2/3,2]]):
\end{MapleInput}


\begin{MapleInput}
> N:=30:p1:=[]:l1:=[]:
for k from 0 to N-1 do
  x0:=Vector([sin(2*Pi*k/N),cos(2*Pi*k/N)]);
  x1:=MatrixVectorMultiply(A,x0);
  p1:=[op(p1),pointplot({x0,x1})];
  l1:=[op(l1),line( evalf(convert(x0,list)),evalf(convert(x1,list)) )];
end do:

> n:=4;
  display(p1,l1,view=[-n..n,-n..n]);
\end{MapleInput}
\MaplePlot{70mm}{./figures/EigenVectors3plot1.eps}

真ん中の円状の領域が,外側の楕円状の領域に写像されている様子が示されている.この図の中に固有ベクトル,固有値が隠れている.どこかわかる?

