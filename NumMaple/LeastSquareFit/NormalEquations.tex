より一般的な場合の最小二乗法の解法を説明する.先程の例では1次の多項式を近似関数とした.これをより一般的な関数,例えば,$\sin, \cos, \tan, \exp, \sinh$などとする.これを線形につないだ関数を
\begin{equation*}
F \left(x \right)=a _{0}\sin \left(x \right)+a _{1}\cos \left(x \right)+a _{2}\exp \left(-x \right)+a _{3}\sinh \left(x \right)+\cdots ={\sum_{k=1}^{M}}a _{k }X _{k }\left(x \right)
\end{equation*}
ととる.実際には,$X_k(x)$はモデルや,多項式の高次項など論拠のある関数列をとる.これらを基底関数(base functions)と呼ぶ.ここで線形といっているのは,パラメータ$a_k$について線形という意味である.このような,より一般的な基底関数を使っても,$\chi^2$関数は
\begin{equation*}
{\chi}^{2}=\sum _{i=1}^{N} \left( F \left( x_{{i}} \right) -y_{{i}} \right) ^{2}
=\sum _{i=1}^{N} \left( \sum _{k=1}^{M}a_{{k}}X_{{k}} \left( x_{{i}} \right) -y_{{i}} \right) ^{2}
\end{equation*}
と求めることができる.この関数を,$a_k$を変数とする関数とみなす.この関数が最小値を取るのは,$\chi^2$を$M$個の$a_k$で偏微分した式がすべて0となる場合であ
る.これを実際に求めてみると,
\begin{equation*}
\sum _{i=1}^{N} \left( \sum _{j=1}^{M}a_{{j}}X_{{j}} \left( x_{{i}} \right) -y_{{i}} \right) X_{{k}} \left( x_{{i}} \right) =0
\end{equation*}
となる.ここで,$k = 1..M$の$M$個の連立方程式である.この連立方程式を最小二乗法の正規方程式(normal equations)と呼ぶ.

上記の記法のままでは,ややこしいので,行列形式で書き直す.$N \times M$で,各要素を
\begin{equation*}
A_{ij} = X_j(x_i)
\end{equation*}
とする行列$A$を導入する.この行列は,
\begin{equation*}
A=\left[
\begin{array}{cccc}
X_1(x_1) & X_2(x_1) & \cdots & X_M(x_1) \\
\vdots & \vdots & \cdots & \vdots \\
\vdots & \vdots & \cdots & \vdots \\
\vdots & \vdots & \cdots & \vdots \\
X_1(x_N) & X_2(x_N) & \cdots & X_M(x_N) 
\end{array}
\right]
\end{equation*}
となる.これをデザイン行列と呼ぶ.すると先程の正規方程式は,
\begin{equation*}
A^t . A . a = A^t . y
\end{equation*}
で与えられる.$A^t$は行列$A$の転置(transpose)
\begin{equation*}
A^t = A_{ij}^t = A_{ji}
\end{equation*}
を意味し,得られた行列は,$M \times N$である.$a, y$はそれぞれ,
\begin{equation*}
a=\left[
\begin{array}{c}
a_1\\a_2\\\vdots\\a_M
\end{array}
\right],\,
y=\left[
\begin{array}{c}
y_1\\y_2\\\vdots\\y_N
\end{array}
\right]
\end{equation*}
である.

$M = 3, N = 25$として行列の次元だけで表現すると,
\begin{equation*}
\left[
\begin{array}{ccccc}
 &  & \cdots & &\\
\cdots & \cdots &  \cdots & \cdots & \cdots \\
 &  & \cdots & &\\
\end{array}
\right]
\left[
\begin{array}{ccc}
& \vdots &\\
& \vdots &\\
\cdots & \cdots &  \cdots\\
& \vdots &\\
& \vdots &\\
\end{array}
\right]
\left[
\begin{array}{c}
\vdots\\
\vdots\\
\vdots
\end{array}
\right]
=
\left[
\begin{array}{ccccc}
 &  & \cdots & &\\
\cdots & \cdots &  \cdots & \cdots & \cdots \\
 &  & \cdots & &\\
\end{array}
\right]
\left[
\begin{array}{c}
\vdots\\
\vdots\\
\vdots\\
\vdots\\
\vdots
\end{array}
\right]
\end{equation*}
となる.これは少しの計算で$3 \times 3$の逆行列を解く問題に変形できる.

\subsection{Mapleによる具体例}
\begin{MapleInput}
> restart; X:=[1,2,3,4]: Y:=[0,5,15,24]: 
  f1:=x->a[1]+a[2]*x+a[3]*x^2:
  with(LinearAlgebra): Av:=Matrix(1..4,1..3):
  ff:=(x,i)->x^(i-1):
  for i from 1 to 3 do 
    for j from 1 to 4 do
      Av[j,i]:=ff(X[j],i); 
    end do; 
  end do;
  Av;
\end{MapleInput}
\begin{MapleOutput}
\left[ \begin{array}{ccc} 1&1&1\\1&2&4\\1&3&9\\1&4&16\end {array} \right]
\end{MapleOutput}
\begin{MapleInput}
> Ai:=MatrixInverse(Transpose(Av).Av);
\end{MapleInput}
\begin{MapleOutput}
{\it Ai}\, := \, \left[ \begin {array}{ccc} 
{\displaystyle \frac {31}{4}}&-{\displaystyle \frac {27}{4}}&\displaystyle \frac{5}{4}\\
-{\displaystyle \frac {27}{4}}&{\displaystyle \frac {129}{20}}&\displaystyle -\frac{5}{4}\\
\displaystyle \frac{5}{4}&\displaystyle -\frac{5}{4}&\displaystyle \frac{1}{4}
\end {array} \right]
\end{MapleOutput}
\begin{MapleInput}
> b:=Transpose(Av).Vector(Y);
\end{MapleInput}
\begin{MapleOutput}
b\, := \, \left[ \begin {array}{c} 44\\151\\539\end {array} \right]
\end{MapleOutput}

\begin{MapleInput}
> Ai.b;
\end{MapleInput}
\begin{MapleOutput}
\left[ \begin {array}{c}\displaystyle -\frac{9}{2}\\
\displaystyle {\frac {16}{5}}\\
1\end {array} \right]
\end{MapleOutput}
