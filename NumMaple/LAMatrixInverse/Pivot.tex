ガウス消去法で困るのは,割ろうとした対角要素が0の場合である.しかし,この場合にも,方程式の順序を,行列の行と右辺の値をペアにして入れ替えれば解決する.この割る
ほうの要素をピボット要素あるいはピボット(pivot,バスケの軸足を動かさずにくるくる回すやつ)と呼ぶ.この操作は,変数の並びを変えたわけではなく,単に方程式の
順番を変更する操作に相当する.

さらに対角要素の数値が厳密に0でなくとも,極端に0に近づいた場合にも,その数で割った数値が大きくなり他の数との差を取ると以前に示した情報落ちの可能性が出てくる.
この現象を防ぐためには,絶対値が最大のピボットを選んで行の入れ替えを毎回おこなうといい結果が得られることが知られている.

MapleのLUDecompositionコマンドをこのような行列に適用すると,置換行列(permutation
matrix)Pが単位行列ではなく,ピボット操作に対応した行列となる.P.A=L.Uとなることに注意.