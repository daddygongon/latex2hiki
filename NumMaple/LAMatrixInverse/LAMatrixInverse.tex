\documentclass[10pt,a4j]{jreport}
\usepackage[dvips]{graphicx,color}
\usepackage{verbatim}
\usepackage{amsmath,amsthm,amssymb}
\topmargin -15mm\oddsidemargin -4mm\evensidemargin\oddsidemargin
\textwidth 170mm\textheight 257mm\columnsep 7mm
\setlength{\fboxrule}{0.2ex}
\setlength{\fboxsep}{0.6ex}

\pagestyle{empty}

\newcommand{\MaplePlot}[2]{{\begin{center}
    \includegraphics[width=#1,clip]{#2}
                     \end{center}
%
} }

\newenvironment{MapleInput}{%
    \color{red}\verbatim
}{%
    \endverbatim
}

\newenvironment{MapleError}{%
    \color{blue}\verbatim
}{%
    \endverbatim
}

\newenvironment{MapleOutput}{%
    \color{blue}\begin{equation*}
}{%
    \end{equation*}
}

\newenvironment{MapleOutputGather}{%
    \color{blue}\gather
}{%
    \endgather
}
\newif\ifHIKI
%\HIKItrue % TRUEの設定
\HIKIfalse % FALSEの設定
\begin{document}
\chapter{線形代数--逆行列(LAMatrixInverse)}
\section{行列計算の概要}
数値計算の中心課題の一つである,行列に関する演算について見ていく.多次元,大規模な行列に対する効率のよい計算法が多数開発されており,多くの既存のライブラリが用意
されている.本章ではそれらの中心をなす,逆行列(matrix inverse)と固有値(Eigen
values)に関して具体的な計算方法を示す.現実的な問題には既存のライブラリを使うのが上策であるが,それでも基礎となる原理の理解や,ちょっとした計算,ライブラ
リの結果の検証に使えるルーチンを示す.

逆行列は連立一次方程式を解くことと等価である.ルーチン的なやり方にガウスの消去法がある.これは上三角行列になれば代入を適宜おこなうことで解が容易に求まることを利
用する.さらに,初期値から始めて次々に解に近づけていく反復法がある.この代表例であるJacobi(ヤコビ)法と,収束性を高めたGauss-Seidel(ガウス-
ザイデル)法を紹介する.

上記の手法をより高速にした修正コレスキー分解と共役傾斜(共役勾配)法があるが,少し複雑になるので割愛する.必要ならばNumRecipeを読め.
\section{ガウス消去法による連立一次方程式の解}
逆行列は連立一次方程式を解くことと等価である.すなわち,$A$を行列,$x$を未知数ベクトル,$b$を数値ベクトルとすると,
\begin{equation*}
\begin{array}{rl} Ax &= b \\
A^{-1}Ax &= A^{-1}b \\
x &= A^{-1}b 
\end{array}
\end{equation*}
である.未知数の少ない連立一次方程式では,適当に組み合わせて未知数を消していけばいいが,未知数が多くなってしまうと破綻する.未知数の多い多元連立一次方程式で,ルーチン的に解を求めていく方法がガウス消去法で,前進消去と後退代入という2つの操作からなる.

後退代入(Backward substitution)による解の求め方を先ず見よう.たとえば,
\begin{equation*}
\begin{array}{rl}
x+y-2z & = -4 \\
-3y+3z & = 9\\
-z & = -2
\end{array}
\end{equation*}
では,下から順番に$z\rightarrow y\rightarrow x$と適宜代入することによって,簡単に解を求めることが出来る.係数で作る行列でこのような形をした上三角行列にする操作を前進消去あるいはガウスの消去法(Gaussian elimination)という.下三角行列L(lower triangular matrix)と上三角行列U(upper triangular matrix)の積に分解する操作
\begin{equation*}
A = L.U
\end{equation*}
をLU分解(LU decomposition)という.例えば先に示した上三角行列を係数とする連立方程式は,
\begin{equation*}
\begin{array}{rl}
x+y-2z&=-4 \\
x-2y+z&=5 \\
2x-2y-z&=2
\end{array}
\end{equation*}
を変形することで得られる.この変形を示せ.
\ifHIKI
\begin{tabular}{|c|}
\hline
         \\ 
\hline
\end{tabular}
\else
\begin{equation*}
\setlength{\unitlength}{1cm}
\begin{picture}(10,3.5)
\put(0,0){\framebox(10,3.5){}}
\end{picture}
\end{equation*}
\fi


  

\section{MapleによるLU分解}
線形代数の計算にはあらかじめ関数パッケージ(LinearAlgebra)を呼び出しておく.
\begin{MapleInput}
> with(LinearAlgebra):
\end{MapleInput}

\subsection{行列の基本操作}
行列の掃き出しに必要となる行列の基本操作はRowOperation, ColumnOperationを参照.

\subsection{掃き出し法,LU分解(LUDecomposition)}
掃き出し法の計算は,LUDecompositionでおこなう.まず拡大係数行列を作る.
\begin{MapleInput}
> A1:=<1,2;3,4>; b:=<2,3>; <A1|b>;
\end{MapleInput}
\begin{MapleOutputGather}
{\it A1}\, := \, \left[ \begin {array}{cc} 1&2\\ 3&4\end {array} \right] \notag \\
b\, := \, \left[ \begin {array}{c} 2\\ 3\end {array} \right] \notag \\
\left[ \begin {array}{ccc} 1&2&2\\ 3&4&3\end {array} \right]  \notag
\end{MapleOutputGather}
これにLU分解をかける.それぞれP(permutation,置換), L(lower triangle,下三角), U(upper
triangle,上三角)行列に代入している.
\begin{MapleInput}
> P,L,U:=LUDecomposition(<A1|b>);
\end{MapleInput}
\begin{MapleOutput}
P,\,L,\,U\, := \, \left[ \begin {array}{cc} 1&0\\ 0&1\end {array} \right] ,\, \left[ \begin {array}{cc} 1&0\\ 3&1\end {array} \right] ,\, \left[ \begin {array}{ccc} 1&2&2\\ 0&-2&-3\end {array} \right] 
\end{MapleOutput}
さらに被約階段行列(row reduced echelonmatrix;後退代入までおこなって,解まで求めた状態)を求めるには,output='R'を指定する.
\begin{MapleInput}
> LUDecomposition(<A1|b>, output='R');
\end{MapleInput}
\begin{MapleOutput}
\left[ \begin {array}{ccc} 1&0&-1\\ 0&1&3/2\end {array} \right]
\end{MapleOutput}

\subsection{階数(Rank)}
行列の性質の中でも特に重要な階数(Rank)は次のコマンドで求められる.
\begin{MapleInput}
> Rank(A1);
\end{MapleInput}
\begin{MapleOutput}
2
\end{MapleOutput}
\section{LU分解のコード}
LU分解すれば線形方程式の解が容易に求まることは理解できると思う.具体的に$A$をLU分解する行列(消去行列と称す)T1,T2の係数は次のようにして求められる.
\begin{MapleInput}
> A0:=Matrix([[1,1,-2],[1,-2,1],[2,-2,-1]]): 
  b0:=Vector([-4,5,2]):
  A:=Matrix(A0): B:=Vector(b0): n:=3: 
  L:=Matrix(array(1..n,1..n,identity)): 
  for i from 1 to n do #i行目
    T[i]:=Matrix(array(1..n,1..n,identity)): 
                            #i番目の消去行列を作る
    for j from i+1 to n do 
      am:=A[j,i]/A[i,i];    #i行の要素を使って,i+1行目の先頭を消す係数を求める
      T[i][j,i]:=-am;       #i番目の消去行列に要素を入れる
      L[j,i]:=am;           #LTMの要素
      for k from 1 to n do
        A[j,k]:=A[j,k]-am*A[i,k]; #もとの行列をUTMにしていく
      end do; 
      B[j]:=B[j]-B[i]*am;   #数値ベクトルも操作
    end do; 
  end do:
\end{MapleInput}
\begin{MapleOutput}
\end{MapleOutput}

上のコードによって得られた消去行列.
\begin{MapleInput}
> T[1]; T[2];
\end{MapleInput}
\begin{MapleOutputGather}
\left[ \begin {array}{ccc} 1&0&0\\ -1&1&0\\ -2&0&1\end {array} \right] \notag \\
\left[ \begin {array}{ccc} 1&0&0\\ 0&1&0\\ 0&-4/3&1\end {array} \right] \notag 
\end{MapleOutputGather}
これを実際に元の行列$A0$に作用させると,UTMが求められる.
\begin{MapleInput}
> U:=T[2].T[1].A0;
\end{MapleInput}
\begin{MapleOutput}
U\, := \, \left[ \begin {array}{ccc} 1&1&-2\\ 0&-3&3\\ 0&0&-1\end {array} \right]
\end{MapleOutput}
求められたLTM, UTMを掛けると
\begin{MapleInput}
> L.U;
\end{MapleInput}
\begin{MapleOutput}
\left[ \begin {array}{ccc} 1&1&-2\\ 1&-2&1\\ 2&-2&-1\end {array} \right]
\end{MapleOutput}
元の行列を得られる.L,Aに求めたい行列が入っていることを確認.
\begin{MapleInput}
> L;A;
\end{MapleInput}
\begin{MapleOutputGather}
\left[ \begin {array}{ccc} 1&0&0\\ 1&1&0\\ 2&4/3&1\end {array} \right] \notag \\
\left[ \begin {array}{ccc} 1&1&-2\\ 0&-3&3\\ 0&0&-1\end {array} \right]  \notag 
\end{MapleOutputGather}
数値ベクトルも期待通り変換されている.
\begin{MapleInput}
> B;
\end{MapleInput}
\begin{MapleOutput}
\left[ \begin {array}{c} -4\\ 9\\ -2\end {array} \right]
\end{MapleOutput}


\section{ピボット操作}
\input{Pivot.tex}
\section{反復法による連立方程式の解}
\input{JacobiGaussSeidel.tex}
\section{課題}
\begin{enumerate}
\item 後退代入法で解を求めよ.(2005年度期末類題) 
\begin{equation*}
\begin{array}{rl}
x+4y-3z &= 1 \\
-6y+4z &= 1\\
-\frac{5}{3}z &=  \frac{1}{3}
\end{array}
\end{equation*}

\item 次の行列AをLU分解せよ.
\begin{MapleInput}
> A:=Matrix([[1,4,3],[1,-2,1],[2,-2,-1]]);
\end{MapleInput}
\begin{MapleOutput}
\left[ \begin {array}{ccc} 1&4&3\\ 1&-2&1\\ 2&-2&-1\end {array} \right]
\end{MapleOutput}

\item 次の連立方程式の係数行列をLU分解し,上・下三角行列を求めよ.さらに連立方程式の解を求めよ.(2005年度期末試験) 
\begin{equation*}
\left[ \begin {array}{c} x_{{1}}+3\,x_{{2}}+4\,x_{{3}}+3\,x_{{4}}\\ -2\,x_{{1}}+5\,x_{{2}}+3\,x_{{3}}-3\,x_{{4}}\\ x_{{1}}+3\,x_{{2}}-2\,x_{{3}}+3\,x_{{4}}\\ 3\,x_{{1}}-2\,x_{{2}}+x_{{3}}+4\,x_{{4}}\end {array} \right] = \left[ \begin {array}{c} 1\\ 4\\ -2\\ 3\end {array} \right]
\end{equation*}

\item Jacobi法のプログラムを参照してGauss-Seidel法のプログラムを作れ.Jacobi法と収束性を比べよ.

\item 次の連立方程式の解を求めよ.ただし,pivot操作が必要となる.
\begin{MapleInput}
> with(LinearAlgebra): 
  A:=Matrix([[3,2,2,1],[3,2,3,1],[1,-2,-3,1],[5,3,-2,5]]):
  X:=Vector([w,x,y,z]): 
  b:=Vector([-6,2,-9,2]): 
  A.X=b;
\end{MapleInput}
\begin{MapleOutput}
\left[ \begin {array}{c} 3\,w+2\,x+2\,y+z\\ 3\,w+2\,x+3\,y+z\\ w-2\,x-3\,y+z\\ 5\,w+3\,x-2\,y+5\,z\end {array} \right] = \left[ \begin {array}{c} -6\\ 2\\ -9\\ 2\end {array} \right]
\end{MapleOutput}
\item (おまけ)pivot操作を含めたLU分解のプログラムを作成せよ.上の問題を解き,そのL, U行列および$L^{-1}.b$ベクトルを求めよ.
\end{enumerate}
\section{解答例}
4. Jacobi法のプログラムを参照してGauss-Seidel法のプログラムを作れ.Jacobi法と収束性を比べよ.
\begin{MapleInput}
#Gauss-Seidel
AA:=Matrix([[5,1,1,1],[1,3,1,1],[1,-2,-9,1],[1,3,-2,5]]):
b:=Vector([-6,2,-7,3]):
n:=4;
x0:=[0,0,0,0]:
x1:=[0,0,0,0]:
for iter from 1 to 20 do
for i from 1 to n do
  x1[i]:=b[i];
  for j from 1 to n do
    x1[i]:=x1[i]-AA[i,j]*x0[j];
  end do:
  x1[i]:=x1[i]+AA[i,i]*x0[i];
  x1[i]:=x1[i]/AA[i,i];
  x0:=evalf(x1);  #change here from ...
end do:
print(iter,x0);
end do:
\end{MapleInput}
\begin{MapleError}
                               4
   1, [-1.200000000, 1.066666667, 0.4074074073, 0.3629629628]
  2, [-1.567407407, 0.9323456790, 0.4367626887, 0.5287791494]
  3, [-1.579577503, 0.8713452217, 0.4673901337, 0.5800644210]
  4, [-1.583759955, 0.8454351333, 0.4783815777, 0.6008435420]
  5, [-1.584932051, 0.8352356437, 0.4828266893, 0.6089756998]
  6, [-1.585407607, 0.8312017393, 0.4845738460, 0.6121900162]
  7, [-1.585593120, 0.8296097527, 0.4852641546, 0.6134584342]
  8, [-1.585666468, 0.8289812930, 0.4855365978, 0.6139591570]
  9, [-1.585695410, 0.8287332183, 0.4856441456, 0.6141568092]
  10, [-1.585706835, 0.8286352933, 0.4856865986, 0.6142348304]
  11, [-1.585711344, 0.8285966383, 0.4857033566, 0.6142656284]
  12, [-1.585713125, 0.8285813800, 0.4857099714, 0.6142777856]
  13, [-1.585713827, 0.8285753567, 0.4857125829, 0.6142825846]
  14, [-1.585714105, 0.8285729793, 0.4857136134, 0.6142844788]
  15, [-1.585714214, 0.8285720407, 0.4857140204, 0.6142852266]
  16, [-1.585714258, 0.8285716703, 0.4857141809, 0.6142855218]
  17, [-1.585714275, 0.8285715240, 0.4857142443, 0.6142856384]
  18, [-1.585714281, 0.8285714660, 0.4857142694, 0.6142856844]
  19, [-1.585714284, 0.8285714433, 0.4857142792, 0.6142857024]
  20, [-1.585714285, 0.8285714343, 0.4857142831, 0.6142857096]
\end{MapleError}











\end{document}
