\begin{enumerate}
\item 後退代入法で解を求めよ.(2005年度期末類題) 
\begin{equation*}
\begin{array}{rl}
x+4y-3z &= 1 \\
-6y+4z &= 1\\
-\frac{5}{3}z &=  \frac{1}{3}
\end{array}
\end{equation*}

\item 次の行列AをLU分解せよ.
\begin{MapleInput}
> A:=Matrix([[1,4,3],[1,-2,1],[2,-2,-1]]);
\end{MapleInput}
\begin{MapleOutput}
\left[ \begin {array}{ccc} 1&4&3\\ 1&-2&1\\ 2&-2&-1\end {array} \right]
\end{MapleOutput}

\item 次の連立方程式の係数行列をLU分解し,上・下三角行列を求めよ.さらに連立方程式の解を求めよ.(2005年度期末試験) 
\begin{equation*}
\left[ \begin {array}{c} x_{{1}}+3\,x_{{2}}+4\,x_{{3}}+3\,x_{{4}}\\ -2\,x_{{1}}+5\,x_{{2}}+3\,x_{{3}}-3\,x_{{4}}\\ x_{{1}}+3\,x_{{2}}-2\,x_{{3}}+3\,x_{{4}}\\ 3\,x_{{1}}-2\,x_{{2}}+x_{{3}}+4\,x_{{4}}\end {array} \right] = \left[ \begin {array}{c} 1\\ 4\\ -2\\ 3\end {array} \right]
\end{equation*}

\item Jacobi法のプログラムを参照してGauss-Seidel法のプログラムを作れ.Jacobi法と収束性を比べよ.

\item 次の連立方程式の解を求めよ.ただし,pivot操作が必要となる.
\begin{MapleInput}
> with(LinearAlgebra): 
  A:=Matrix([[3,2,2,1],[3,2,3,1],[1,-2,-3,1],[5,3,-2,5]]):
  X:=Vector([w,x,y,z]): 
  b:=Vector([-6,2,-9,2]): 
  A.X=b;
\end{MapleInput}
\begin{MapleOutput}
\left[ \begin {array}{c} 3\,w+2\,x+2\,y+z\\ 3\,w+2\,x+3\,y+z\\ w-2\,x-3\,y+z\\ 5\,w+3\,x-2\,y+5\,z\end {array} \right] = \left[ \begin {array}{c} -6\\ 2\\ -9\\ 2\end {array} \right]
\end{MapleOutput}
\item (おまけ)pivot操作を含めたLU分解のプログラムを作成せよ.上の問題を解き,そのL, U行列および$L^{-1}.b$ベクトルを求めよ.
\end{enumerate}