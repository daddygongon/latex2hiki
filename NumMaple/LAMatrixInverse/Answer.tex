4. Jacobi法のプログラムを参照してGauss-Seidel法のプログラムを作れ.Jacobi法と収束性を比べよ.
\begin{MapleInput}
#Gauss-Seidel
AA:=Matrix([[5,1,1,1],[1,3,1,1],[1,-2,-9,1],[1,3,-2,5]]):
b:=Vector([-6,2,-7,3]):
n:=4;
x0:=[0,0,0,0]:
x1:=[0,0,0,0]:
for iter from 1 to 20 do
for i from 1 to n do
  x1[i]:=b[i];
  for j from 1 to n do
    x1[i]:=x1[i]-AA[i,j]*x0[j];
  end do:
  x1[i]:=x1[i]+AA[i,i]*x0[i];
  x1[i]:=x1[i]/AA[i,i];
  x0:=evalf(x1);  #change here from ...
end do:
print(iter,x0);
end do:
\end{MapleInput}
\begin{MapleError}
                               4
   1, [-1.200000000, 1.066666667, 0.4074074073, 0.3629629628]
  2, [-1.567407407, 0.9323456790, 0.4367626887, 0.5287791494]
  3, [-1.579577503, 0.8713452217, 0.4673901337, 0.5800644210]
  4, [-1.583759955, 0.8454351333, 0.4783815777, 0.6008435420]
  5, [-1.584932051, 0.8352356437, 0.4828266893, 0.6089756998]
  6, [-1.585407607, 0.8312017393, 0.4845738460, 0.6121900162]
  7, [-1.585593120, 0.8296097527, 0.4852641546, 0.6134584342]
  8, [-1.585666468, 0.8289812930, 0.4855365978, 0.6139591570]
  9, [-1.585695410, 0.8287332183, 0.4856441456, 0.6141568092]
  10, [-1.585706835, 0.8286352933, 0.4856865986, 0.6142348304]
  11, [-1.585711344, 0.8285966383, 0.4857033566, 0.6142656284]
  12, [-1.585713125, 0.8285813800, 0.4857099714, 0.6142777856]
  13, [-1.585713827, 0.8285753567, 0.4857125829, 0.6142825846]
  14, [-1.585714105, 0.8285729793, 0.4857136134, 0.6142844788]
  15, [-1.585714214, 0.8285720407, 0.4857140204, 0.6142852266]
  16, [-1.585714258, 0.8285716703, 0.4857141809, 0.6142855218]
  17, [-1.585714275, 0.8285715240, 0.4857142443, 0.6142856384]
  18, [-1.585714281, 0.8285714660, 0.4857142694, 0.6142856844]
  19, [-1.585714284, 0.8285714433, 0.4857142792, 0.6142857024]
  20, [-1.585714285, 0.8285714343, 0.4857142831, 0.6142857096]
\end{MapleError}

