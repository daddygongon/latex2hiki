数値計算の中心課題の一つである,行列に関する演算について見ていく.多次元,大規模な行列に対する効率のよい計算法が多数開発されており,多くの既存のライブラリが用意
されている.本章ではそれらの中心をなす,逆行列(matrix inverse)と固有値(Eigen
values)に関して具体的な計算方法を示す.現実的な問題には既存のライブラリを使うのが上策であるが,それでも基礎となる原理の理解や,ちょっとした計算,ライブラ
リの結果の検証に使えるルーチンを示す.

逆行列は連立一次方程式を解くことと等価である.ルーチン的なやり方にガウスの消去法がある.これは上三角行列になれば代入を適宜おこなうことで解が容易に求まることを利
用する.さらに,初期値から始めて次々に解に近づけていく反復法がある.この代表例であるJacobi(ヤコビ)法と,収束性を高めたGauss-Seidel(ガウス-
ザイデル)法を紹介する.

上記の手法をより高速にした修正コレスキー分解と共役傾斜(共役勾配)法があるが,少し複雑になるので割愛する.必要ならばNumRecipeを読め.