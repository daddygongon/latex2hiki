上の例では,浮動小数点数で計算した場合に小さい数の差を区別することができなくなるということを示している.これは,CPUに固有の精度で,機械精度(Machine epsilon)と呼ばれる.つまり,小さい数を足したときにその計算機がその差を認識できなくなる限界ということで,以下のようにして求めることができる.
\begin{MapleInput}
> Digits:=7; 
> e:=evalf(1.0);
> w:=evalf(1.0+e); 
> while (w>1.0) do 
    printf("%-15.10e %-15.10e %-15.10e\n",e,w,evalf(w-1.0)); 
    e:=evalf(e/2.0); 
    w:=evalf(1.0+e); 
  end do:
\end{MapleInput}
\begin{MapleOutputGather}
7 \notag \\
1.0 \notag \\
2.0 \notag
\end{MapleOutputGather}
\begin{MapleError}
1.0000000000e+00 2.0000000000e+00 1.0000000000e+00
5.0000000000e-01 1.5000000000e+00 5.0000000000e-01
2.5000000000e-01 1.2500000000e+00 2.5000000000e-01
1.2500000000e-01 1.1250000000e+00 1.2500000000e-01
6.2500000000e-02 1.0625000000e+00 6.2500000000e-02
3.1250000000e-02 1.0312500000e+00 3.1250000000e-02
1.5625000000e-02 1.0156250000e+00 1.5625000000e-02
7.8125000000e-03 1.0078120000e+00 7.8120000000e-03
3.9062500000e-03 1.0039060000e+00 3.9060000000e-03
1.9531250000e-03 1.0019530000e+00 1.9530000000e-03
9.7656250000e-04 1.0009770000e+00 9.7700000000e-04
4.8828120000e-04 1.0004880000e+00 4.8800000000e-04
2.4414060000e-04 1.0002440000e+00 2.4400000000e-04
1.2207030000e-04 1.0001220000e+00 1.2200000000e-04
6.1035150000e-05 1.0000610000e+00 6.1000000000e-05
3.0517580000e-05 1.0000310000e+00 3.1000000000e-05
1.5258790000e-05 1.0000150000e+00 1.5000000000e-05
7.6293950000e-06 1.0000080000e+00 8.0000000000e-06
3.8146980000e-06 1.0000040000e+00 4.0000000000e-06
1.9073490000e-06 1.0000020000e+00 2.0000000000e-06
9.5367450000e-07 1.0000010000e+00 1.0000000000e-06
\end{MapleError}