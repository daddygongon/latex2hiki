「丸め」にともなって誤差が生じる. CやFortran等の通常のプログラミング言語では「丸める」仕様なのでプログラマーが気をつけなければならない.
\begin{MapleInput}
プログラムリスト : 実数のケタ落ち
#include <stdio.h>

int main(void){
  float a,b,c;
  double x,y,z;

  a=1.23456789;
  printf(" a= %17.10f\n",a);

  b=100.0;
  c=a+b;
  printf("%20.10f %20.10f %20.10f\n",a,b,c);

  x=(float)1.23456789;
  y=(double)100;
  z=x+y;
  printf("%20.12e %20.12e %20.12e\n",x,y,z);
 
  x=(double)1.23456789;
  y=(double)100;
  z=x+y;
  printf("%20.12e %20.12e %20.12e\n",x,y,z);

  return 0;
}
\end{MapleInput}

分かっているつもりでも,よくやる間違い.
\begin{MapleInput}
プログラムリスト : 丸め誤差
#include <stdio.h>

int main(void){
  float x=77777,y=7,y1,z,z1;
  y1=1/y;
  z=x/y;
  z1=x*y1;
  printf("%10.2f %10.2f\n",z,z1);
  if (z!=z1){
    printf("z is not equal to z1.\n");
  }
  printf("Surprising?? \n\n\n\n\n%10.5f %10.5f\n",z,z1);
  return 0; 
}
\end{MapleInput}
これを避けるには,EPSILONという小さな数字を定義しておいて,値の差の絶対値を求めるfabsを使って
\begin{table}[h]\begin{center}\begin{tabular}{|c|}
\hline
\hspace{100mm} \\ 
\\
\\
\hline
\end{tabular}\end{center}\end{table}%

とすべき.このときは数学関数であるfabsを使っているので,
\begin{MapleInput}
> gcc -lm test.c
\end{MapleInput}
とmath libraryを明示的に呼ぶのを忘れないように.